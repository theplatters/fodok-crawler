\subsection*{Begutachtung von Publikationen und Herausgebertätigkeiten}
\begin{enumerate}
	\item Hager, Theresa  PLOS Sustainability and Transformation (Fachzeitschrift oder Schriftenreihe)  2025-12-22
	\item Porak, Laura  International Review of Economics (Fachzeitschrift oder Schriftenreihe)  2025-11-04
	\item Porak, Laura  Review of Evolutionary Political Economy (Fachzeitschrift oder Schriftenreihe)  2025-11-01
	\item Hornykewycz, Anna  Journal of Institutional Economics (JOIE) (Fachzeitschrift oder Schriftenreihe)  2025-04-30
	\item Theine, Hendrik  Degrowth Journal (Fachzeitschrift oder Schriftenreihe)  2025-01-01
	\item Theine, Hendrik  Communication \& Society (Fachzeitschrift oder Schriftenreihe)  2025-01-01
	\item Theine, Hendrik  Momentum Quarterly (Fachzeitschrift oder Schriftenreihe)  2025-01-01
	\item Theine, Hendrik  Journalism (Fachzeitschrift oder Schriftenreihe)  2025-01-01
	\item Theine, Hendrik  Competition \& Change (Fachzeitschrift oder Schriftenreihe)  2025-01-01
	\item Theine, Hendrik  Wirtschaft und Gesellschaft -- WuG (Fachzeitschrift oder Schriftenreihe)  2025-01-01
	\item Theine, Hendrik  Journalism \& Mass Communication Quarterly (Fachzeitschrift oder Schriftenreihe)  2025-01-01
	\item Hornykewycz, Anna  New Political Economy (Fachzeitschrift oder Schriftenreihe)  2025-01-01
	\item Hornykewycz, Anna  The Annals of Regional Science (Fachzeitschrift oder Schriftenreihe)  2025-01-01
	\item Bäuerle, Lukas  Energy Research and Social Science (Fachzeitschrift oder Schriftenreihe)  2025-01-01
	\item Bäuerle, Lukas  Sozioökonomische Bildung und Wissenschaft (Fachzeitschrift oder Schriftenreihe)  2024-10-01
	\item Bäuerle, Lukas  Schäffer \& Poeschel (Verlag)  2024-10-01
	\item Hornykewycz, Anna  Global Networks (GN) (Fachzeitschrift oder Schriftenreihe)  2024-09-24
	\item Porak, Laura  Conflict as Catalyst of Trust. A New Research Agenda for International Political Sociology (Fachzeitschrift oder Schriftenreihe)  2024-02-19
	\item Rath, Johanna  Guest Editor for Review of Evolutionary Political Economy (REPE) Special Issue on \glqq Power in Economics without Power in Economics?\grqq{} (Fachzeitschrift oder Schriftenreihe)  2024-01-01
	\item Hornykewycz, Anna  Guest Editor for Review of Evolutionary Political Economy (REPE) Special Issue on \glqq Power in Economics without Power in Economics?\grqq{} (Fachzeitschrift oder Schriftenreihe)  2024-01-01
	\item Eder, Julia Theresa  Mitglied Redaktion Journal für Entwicklungspolitik (Fachzeitschrift oder Schriftenreihe)  2024-01-01
	\item   Heterodox Economics Directory (Fachzeitschrift oder Schriftenreihe)  2024-01-01
	\item Kapeller, Jakob  Heterodox Economics Directory (Fachzeitschrift oder Schriftenreihe)  2024-01-01
	\item   European Journal of Communication (Fachzeitschrift oder Schriftenreihe)  2024-01-01
	\item Theine, Hendrik  European Journal of Communication (Fachzeitschrift oder Schriftenreihe)  2024-01-01
	\item   International Journal of Communication (Fachzeitschrift oder Schriftenreihe)  2024-01-01
	\item Theine, Hendrik  International Journal of Communication (Fachzeitschrift oder Schriftenreihe)  2024-01-01
	\item   Review of Evolutionary Political Economy (Fachzeitschrift oder Schriftenreihe)  2023-12-01
	\item Rath, Johanna  Review of Evolutionary Political Economy (Fachzeitschrift oder Schriftenreihe)  2023-12-01
	\item Hornykewycz, Anna  Review of Evolutionary Political Economy (Fachzeitschrift oder Schriftenreihe)  2023-12-01
	\item Hornykewycz, Anna  Journal of Institutional Economics (JOIE) (Fachzeitschrift oder Schriftenreihe)  2023-11-15
	\item Pühringer, Stephan  Review of Evolutionary Political Economy (Fachzeitschrift oder Schriftenreihe)  2023-11-01
	\item Porak, Laura  International Political Sociology (Fachzeitschrift oder Schriftenreihe)  2023-11-01
	\item Aistleitner, Matthias  Journal of Institutional Economics (Fachzeitschrift oder Schriftenreihe)  2023-11-01
	\item Kapeller, Jakob  European Journal of the History of Economic Thought (Fachzeitschrift oder Schriftenreihe)  2023-10-01
	\item Kapeller, Jakob  Review of Politics and Public Policy in Emerging Economies (Fachzeitschrift oder Schriftenreihe)  2023-08-01
	\item Kapeller, Jakob  Journal of Economic Integration (Fachzeitschrift oder Schriftenreihe)  2023-08-01
	\item Pühringer, Stephan  Global Governance (Fachzeitschrift oder Schriftenreihe)  2023-08-01
	\item Hager, Theresa  European Journal of the History of Economic Thought (Fachzeitschrift oder Schriftenreihe)  2023-06-01
	\item Kapeller, Jakob  Review of Politics and Public Policy in Emerging Economies (Fachzeitschrift oder Schriftenreihe)  2023-06-01
	\item Aistleitner, Matthias  Research Evaluation (Fachzeitschrift oder Schriftenreihe)  2023-05-01
	\item Kapeller, Jakob  Cambridge Journal of Economics (Fachzeitschrift oder Schriftenreihe)  2023-05-01
	\item Kapeller, Jakob  Expert Systems With Applications (Fachzeitschrift oder Schriftenreihe)  2023-04-01
	\item Kapeller, Jakob  Review of International Political Economy (Fachzeitschrift oder Schriftenreihe)  2023-04-01
	\item Kapeller, Jakob  Journal of Philosophical Economics (Fachzeitschrift oder Schriftenreihe)  2023-02-01
	\item Kapeller, Jakob  Journal of Economic Integration (Fachzeitschrift oder Schriftenreihe)  2023-01-01
	\item   Zeitschrift für Politikwissenschaft (Fachzeitschrift oder Schriftenreihe)  2023-01-01
	\item Theine, Hendrik  Zeitschrift für Politikwissenschaft (Fachzeitschrift oder Schriftenreihe)  2023-01-01
	\item   Zeitschrift für Politikwissenschaft (Fachzeitschrift oder Schriftenreihe)  2023-01-01
	\item   Zeitschrift für Politikwissenschaft (Fachzeitschrift oder Schriftenreihe)  2023-01-01
	\item Kapeller, Jakob  European Journal of the History of Economic Thought (Fachzeitschrift oder Schriftenreihe)  2022-11-01
	\item Pühringer, Stephan  Serendipities -- Journal for the Sociology and History of the Social Sciences (Fachzeitschrift oder Schriftenreihe)  2022-11-01
	\item Pühringer, Stephan  Review of Evolutionary Political Economy (Fachzeitschrift oder Schriftenreihe)  2022-11-01
	\item Gräbner-Radkowitsch, Claudius  Journal of Artificial Societies and Social Simulation (JASSS) (Fachzeitschrift oder Schriftenreihe)  2022-10-01
	\item Gräbner-Radkowitsch, Claudius  Journal of Artificial Societies and Social Simulation (Fachzeitschrift oder Schriftenreihe)  2022-07-01
	\item Gräbner-Radkowitsch, Claudius  Journal of Economic Dynamics and Control (Fachzeitschrift oder Schriftenreihe)  2022-07-01
	\item Pühringer, Stephan  Open Education Studies (Fachzeitschrift oder Schriftenreihe)  2022-07-01
	\item Pühringer, Stephan  Politics and Governance (Fachzeitschrift oder Schriftenreihe)  2022-06-01
	\item Kapeller, Jakob  Review of International Political Economy (Fachzeitschrift oder Schriftenreihe)  2022-05-02
	\item Kapeller, Jakob  Zeitschrift für Wirtschafts- und Unternehmensethik (Fachzeitschrift oder Schriftenreihe)  2022-05-01
	\item Kapeller, Jakob  Social Studies of Science (Fachzeitschrift oder Schriftenreihe)  2022-04-01
	\item Kapeller, Jakob  Expert Systems With Applications (Fachzeitschrift oder Schriftenreihe)  2022-04-01
	\item Kapeller, Jakob  Review of Political Economy (Fachzeitschrift oder Schriftenreihe)  2022-03-01
	\item Kapeller, Jakob  Cambridge Journal of Economics (Fachzeitschrift oder Schriftenreihe)  2022-03-01
	\item Gräbner-Radkowitsch, Claudius  Research Policy (Fachzeitschrift oder Schriftenreihe)  2022-01-01
	\item Gräbner-Radkowitsch, Claudius  Socio-Economic Review (Fachzeitschrift oder Schriftenreihe)  2022-01-01
	\item Pühringer, Stephan  PSL Quaterly Review (Fachzeitschrift oder Schriftenreihe)  2022-01-01
	\item Gräbner-Radkowitsch, Claudius  Journal of Economic Interaction and Coordination (Fachzeitschrift oder Schriftenreihe)  2022-01-01
	\item Hornykewycz, Anna  International Journal of Pluralism and Economics Education (IJPEE) (Fachzeitschrift oder Schriftenreihe)  2022-01-01
	\item Gräbner-Radkowitsch, Claudius  Economic Modelling (Fachzeitschrift oder Schriftenreihe)  2022-01-01
	\item Kapeller, Jakob  Competition \& Change (Fachzeitschrift oder Schriftenreihe)  2022-01-01
	\item Porak, Laura  Pluralism and Economics Education (Fachzeitschrift oder Schriftenreihe)  2022-01-01
	\item Theine, Hendrik  Pluralism and Economics Education (Fachzeitschrift oder Schriftenreihe)  2022-01-01
	\item Hirte, Katrin  Zeitschrift für Politikwissenschaft (Fachzeitschrift oder Schriftenreihe)  2021-11-01
	\item Gräbner-Radkowitsch, Claudius  Journal of Economic Dynamics and Control (Fachzeitschrift oder Schriftenreihe)  2021-10-01
	\item Gräbner-Radkowitsch, Claudius  Economic Thought (Fachzeitschrift oder Schriftenreihe)  2021-10-01
	\item Pühringer, Stephan  Zeitschrift für Wirtschafts- und Unternehmensethik (Fachzeitschrift oder Schriftenreihe)  2021-10-01
	\item Pühringer, Stephan  Socio-Economic Review (Fachzeitschrift oder Schriftenreihe)  2021-07-01
	\item Gräbner-Radkowitsch, Claudius  Socio-Economic Review (Fachzeitschrift oder Schriftenreihe)  2021-06-01
	\item Gräbner-Radkowitsch, Claudius  Journal of Common Market Studies (Fachzeitschrift oder Schriftenreihe)  2021-06-01
	\item Gräbner-Radkowitsch, Claudius  Journal of the Asia Pacific Economy (Fachzeitschrift oder Schriftenreihe)  2021-05-01
	\item Gräbner-Radkowitsch, Claudius  Journal of Economic Methodology (Fachzeitschrift oder Schriftenreihe)  2021-05-01
	\item Gräbner-Radkowitsch, Claudius  Regional Studies (Fachzeitschrift oder Schriftenreihe)  2021-04-01
	\item Gräbner-Radkowitsch, Claudius  European Journal for Philosophy of Science (Fachzeitschrift oder Schriftenreihe)  2021-04-01
	\item Gräbner-Radkowitsch, Claudius  Journal of Common Market Studies (Fachzeitschrift oder Schriftenreihe)  2021-03-01
	\item Pühringer, Stephan  Socio-Economic Review (Fachzeitschrift oder Schriftenreihe)  2021-02-01
	\item Pühringer, Stephan  Palgrave Communications (Fachzeitschrift oder Schriftenreihe)  2021-02-01
	\item Pühringer, Stephan  Third World Quarterly (Fachzeitschrift oder Schriftenreihe)  2021-02-01
	\item Schütz, Bernhard  Wirtschaft und Gesellschaft (Fachzeitschrift oder Schriftenreihe)  2021-02-01
	\item Pühringer, Stephan  Constellations (Fachzeitschrift oder Schriftenreihe)  2021-01-01
	\item Pühringer, Stephan  Interdisciplinary Political Studies (Fachzeitschrift oder Schriftenreihe)  2021-01-01
	\item Kapeller, Jakob  Economic and Regional Studies (Fachzeitschrift oder Schriftenreihe)  2020-11-01
	\item Kapeller, Jakob  Journal of Economics and Management (Fachzeitschrift oder Schriftenreihe)  2020-10-01
	\item Schütz, Bernhard  Journal of Common Market Studies (Fachzeitschrift oder Schriftenreihe)  2020-10-01
	\item Schütz, Bernhard  European Journal of Economics and Economic Policies (Fachzeitschrift oder Schriftenreihe)  2020-10-01
	\item Schütz, Bernhard  Momentum Quarterly (Fachzeitschrift oder Schriftenreihe)  2020-08-01
	\item Kapeller, Jakob  European Journal of the History of Economic Thought (Fachzeitschrift oder Schriftenreihe)  2020-05-01
	\item Pühringer, Stephan  International Journal of Pluralism and Economics Education (IJPEE) (Fachzeitschrift oder Schriftenreihe)  2020-05-01
	\item Kapeller, Jakob  Review of Evolutionary Political Economy (Fachzeitschrift oder Schriftenreihe)  2020-01-01
	\item Schütz, Bernhard  Wirtschaft und Gesellschaft (Fachzeitschrift oder Schriftenreihe)  2019-11-01
	\item Pühringer, Stephan  International Journal of Pluralism and Economics Education (IJPEE) (Fachzeitschrift oder Schriftenreihe)  2019-09-01
	\item Schütz, Bernhard  Metroeconomica (Fachzeitschrift oder Schriftenreihe)  2019-09-01
	\item Kapeller, Jakob  Cambridge Journal of Economics (Fachzeitschrift oder Schriftenreihe)  2019-08-01
	\item Schütz, Bernhard  Momentum Quarterly (Fachzeitschrift oder Schriftenreihe)  2019-06-01
	\item Kapeller, Jakob  Review of Politics and Public Policy in Emerging Economies (Fachzeitschrift oder Schriftenreihe)  2019-05-01
	\item Schütz, Bernhard  Review of Income and Wealth (Fachzeitschrift oder Schriftenreihe)  2019-05-01
	\item Kapeller, Jakob  Socio-Economic Review (Fachzeitschrift oder Schriftenreihe)  2019-04-01
	\item   Change and persistence in contemporary economics (Fachzeitschrift oder Schriftenreihe)  2019-04-01
	\item Kapeller, Jakob  Change and persistence in contemporary economics (Fachzeitschrift oder Schriftenreihe)  2019-04-01
	\item Pühringer, Stephan  Journal for Discourse Studies (Fachzeitschrift oder Schriftenreihe)  2019-03-01
	\item Pühringer, Stephan  Critical Policy Studies (Fachzeitschrift oder Schriftenreihe)  2019-02-01
	\item Rath, Johanna  Heterodox Economics Newsletter (Fachzeitschrift oder Schriftenreihe)  2019-01-01
	\item Kapeller, Jakob  Journal of Economic Behavior and Organization (Fachzeitschrift oder Schriftenreihe)  2019-01-01
	\item Kapeller, Jakob  PSL Quarterly Review (Fachzeitschrift oder Schriftenreihe)  2019-01-01
	\item Schütz, Bernhard  Empirica (Fachzeitschrift oder Schriftenreihe)  2018-12-01
	\item   International Journal of Computational Economics and Econometrics, The dynamics of and on networks (Fachzeitschrift oder Schriftenreihe)  2018-12-01
	\item   International Journal of Computational Economics and Econometrics, The dynamics of and on networks (Fachzeitschrift oder Schriftenreihe)  2018-12-01
	\item   International Journal of Computational Economics and Econometrics, The dynamics of and on networks (Fachzeitschrift oder Schriftenreihe)  2018-12-01
	\item Gräbner-Radkowitsch, Claudius  International Journal of Computational Economics and Econometrics, The dynamics of and on networks (Fachzeitschrift oder Schriftenreihe)  2018-12-01
	\item Pühringer, Stephan  Journal of Philosophical Economics (Fachzeitschrift oder Schriftenreihe)  2018-11-01
	\item Hirte, Katrin  Constellations (Fachzeitschrift oder Schriftenreihe)  2018-10-01
	\item Pühringer, Stephan  International Journal for Social Economics (Fachzeitschrift oder Schriftenreihe)  2018-06-01
	\item Schütz, Bernhard  Momentum Quarterly (Fachzeitschrift oder Schriftenreihe)  2018-05-01
	\item Kapeller, Jakob  Forum for Social Economics (Fachzeitschrift oder Schriftenreihe)  2018-01-01
	\item Pühringer, Stephan  Critical Discourse Studies (Fachzeitschrift oder Schriftenreihe)  2017-12-01
	\item Hirte, Katrin  Journal of Language and Politics (Fachzeitschrift oder Schriftenreihe)  2017-11-01
	\item Pühringer, Stephan  Journal of Language and Politics (Fachzeitschrift oder Schriftenreihe)  2017-11-01
	\item Schütz, Bernhard  Wirtschaft und Gesellschaft (Fachzeitschrift oder Schriftenreihe)  2017-08-01
	\item Pühringer, Stephan  Policy and Society (Fachzeitschrift oder Schriftenreihe)  2017-06-01
	\item Schütz, Bernhard  Journal of Economic Surveys (Fachzeitschrift oder Schriftenreihe)  2017-02-01
	\item Schütz, Bernhard  Empirica (Fachzeitschrift oder Schriftenreihe)  2016-09-01
	\item Pühringer, Stephan  Momentum Quarterly (Fachzeitschrift oder Schriftenreihe)  2016-08-01
	\item Schütz, Bernhard  Momentum Quarterly (Fachzeitschrift oder Schriftenreihe)  2016-06-01
	\item Kapeller, Jakob  Reviewer Egon-Matzner-Award for Socio Economics (Veranstaltung) Reviewer Egon-Matzner-Award for Socio Economics 2014-01-01
	\item Kapeller, Jakob  Heterodox Economics Newsletter (Fachzeitschrift oder Schriftenreihe)  2013-01-01
	\item Kapeller, Jakob  Heterodox Economics Newsletter (Fachzeitschrift oder Schriftenreihe)  2013-01-01
\end{enumerate}
\subsection*{Teilnahme an oder Organisation einer Veranstaltung}
\begin{enumerate}
	\item Hager, Theresa  Jubiläumsfeier: 15 Jahre ICAE Jubiläumsfeier: 15 Jahre ICAE 2025-12-12
	\item Terhorst, Carlotta  Jubiläumsfeier: 15 Jahre ICAE Jubiläumsfeier: 15 Jahre ICAE 2025-12-12
	\item Pühringer, Stephan  Jubiläumsfeier: 15 Jahre ICAE Jubiläumsfeier: 15 Jahre ICAE 2025-12-12
	\item Hornykewycz, Anna  Jubiläumsfeier: 15 Jahre ICAE Jubiläumsfeier: 15 Jahre ICAE 2025-12-12
	\item Hager, Theresa  Jubiläumsfeier: 15 Jahre ICAE Jubiläumsfeier: 15 Jahre ICAE 2025-12-12
	\item Cserjan, Lukas  Jubiläumsfeier: 15 Jahre ICAE Jubiläumsfeier: 15 Jahre ICAE 2025-12-12
	\item Hofmann, Julia  Mapping the one percentDie unsichtbare Spitze Mapping the one percentDie unsichtbare Spitze 2025-11-21
	\item Pühringer, Stephan  Mapping the one percentDie unsichtbare Spitze Mapping the one percentDie unsichtbare Spitze 2025-11-21
	\item Hager, Theresa  Session Research Area K \glqq Gender Economics and Social Identity" @36th Annual Conference of European Association for Ecolutionary Political Economy (EAEPE) Session Research Area K "Gender Economics and Social Identity\grqq{} @36th Annual Conference of European Association for Ecolutionary Political Economy (EAEPE) 2025-09-25
	\item   Special Sessions on \glqq Pluralism in Economics and Postcolonial Political Economy" at the 2025 EAEPE Annual Conference Special Sessions on "Pluralism in Economics and Postcolonial Political Economy\grqq{} at the 2025 EAEPE Annual Conference 2025-09-24
	\item Gräbner-Radkowitsch, Claudius  Special Sessions on \glqq Pluralism in Economics and Postcolonial Political Economy" at the 2025 EAEPE Annual Conference Special Sessions on "Pluralism in Economics and Postcolonial Political Economy\grqq{} at the 2025 EAEPE Annual Conference 2025-09-24
	\item Hager, Theresa  Special Session \glqq Global and National Studies of Socio-Ecological Transformation" @36th Annual Conference of the European Association of Evolutionary Political Economy (EAEPE) Special Session "Global and National Studies of Socio-Ecological Transformation\grqq{} @36th Annual Conference of the European Association of Evolutionary Political Economy (EAEPE) 2025-09-24
	\item Hornykewycz, Anna  Special Session \glqq Global and National Studies of Socio-Ecological Transformation" @36th Annual Conference of the European Association of Evolutionary Political Economy (EAEPE) Special Session "Global and National Studies of Socio-Ecological Transformation\grqq{} @36th Annual Conference of the European Association of Evolutionary Political Economy (EAEPE) 2025-09-24
	\item Cserjan, Lukas  Special Session \glqq Global and National Studies of Socio-Ecological Transformation" @36th Annual Conference of the European Association of Evolutionary Political Economy (EAEPE) Special Session "Global and National Studies of Socio-Ecological Transformation\grqq{} @36th Annual Conference of the European Association of Evolutionary Political Economy (EAEPE) 2025-09-24
	\item Hager, Theresa  Special Session \glqq Global and National Studies of Socio-Ecological Transformation" @36th Annual Conference of the European Association of Evolutionary Political Economy (EAEPE) Special Session "Global and National Studies of Socio-Ecological Transformation\grqq{} @36th Annual Conference of the European Association of Evolutionary Political Economy (EAEPE) 2025-09-24
	\item Rath, Johanna  YSI \& EAEPE Pre-Conference for Young Scholars 2025: Pluralist Perspectives on Digital Technologies and Societal Transformation YSI \& EAEPE Pre-Conference for Young Scholars 2025: Pluralist Perspectives on Digital Technologies and Societal Transformation 2025-09-23
	\item Hornykewycz, Anna  YSI \& EAEPE Pre-Conference for Young Scholars 2025: Pluralist Perspectives on Digital Technologies and Societal Transformation YSI \& EAEPE Pre-Conference for Young Scholars 2025: Pluralist Perspectives on Digital Technologies and Societal Transformation 2025-09-23
	\item   YSI \& EAEPE Pre-Conference for Young Scholars 2025: Pluralist Perspectives on Digital Technologies and Societal Transformation YSI \& EAEPE Pre-Conference for Young Scholars 2025: Pluralist Perspectives on Digital Technologies and Societal Transformation 2025-09-23
	\item Terhorst, Carlotta  Spaces of Contestation: Transgressing Policies and Practices of Eco-Social Transformation Spaces of Contestation: Transgressing Policies and Practices of Eco-Social Transformation 2025-09-18
	\item Hager, Theresa  Moderation of Keynote given by Prof. Erin Cech: \glqq Who’s Afraid of DEI? How the Moral Commandments of Work Anchor Opposition to Workplace Equity Efforts" @ SASE's 37th Annual Conference 2025 Moderation of Keynote given by Prof. Erin Cech: "Who’s Afraid of DEI? How the Moral Commandments of Work Anchor Opposition to Workplace Equity Efforts\grqq{} @ SASE's 37th Annual Conference 2025 2025-07-11
	\item Hager, Theresa  Organization of the Open Assembly of the Women and Gender+ Forum @ SASE's 37th Annual Conference 2025 Organization of the Open Assembly of the Women and Gender+ Forum @ SASE's 37th Annual Conference 2025 2025-07-10
	\item Bäuerle, Lukas  Discourse Net Conference: Discourse and the imaginaries of past, present and future societies: media and representations of (inter)national (dis)orders Discourse Net Conference: Discourse and the imaginaries of past, present and future societies: media and representations of (inter)national (dis)orders 2025-07-08
	\item Pühringer, Stephan  Discourse Net Conference: Discourse and the imaginaries of past, present and future societies: media and representations of (inter)national (dis)orders Discourse Net Conference: Discourse and the imaginaries of past, present and future societies: media and representations of (inter)national (dis)orders 2025-07-08
	\item Hornykewycz, Anna  Akademie für den Sozialen und Ökologischen Umbau Akademie für den Sozialen und Ökologischen Umbau 2025-06-11
	\item Eder, Julia Theresa  Akademie für den Sozialen und Ökologischen Umbau Akademie für den Sozialen und Ökologischen Umbau 2025-06-11
	\item Cserjan, Lukas  Akademie für den Sozialen und Ökologischen Umbau Akademie für den Sozialen und Ökologischen Umbau 2025-06-11
	\item Pühringer, Stephan  Akademie für den Sozialen und Ökologischen Umbau Akademie für den Sozialen und Ökologischen Umbau 2025-06-11
	\item Eder, Julia Theresa  Akademie für den Sozialen und Ökologischen Umbau Akademie für den Sozialen und Ökologischen Umbau 2025-06-10
	\item Hager, Theresa  Akademie für den Sozialen und Ökologischen Umbau Akademie für den Sozialen und Ökologischen Umbau 2025-06-10
	\item Pühringer, Stephan  Akademie für den Sozialen und Ökologischen Umbau Akademie für den Sozialen und Ökologischen Umbau 2025-06-10
	\item Cserjan, Lukas  Akademie für den Sozialen und Ökologischen Umbau Akademie für den Sozialen und Ökologischen Umbau 2025-06-10
	\item Terhorst, Carlotta  Akademie für den Sozialen und Ökologischen Umbau Akademie für den Sozialen und Ökologischen Umbau 2025-06-10
	\item Verita, Carlotta  Akademie für den Sozialen und Ökologischen Umbau Akademie für den Sozialen und Ökologischen Umbau 2025-06-10
	\item Hornykewycz, Anna  Akademie für den Sozialen und Ökologischen Umbau Akademie für den Sozialen und Ökologischen Umbau 2025-06-10
	\item Eder, Julia Theresa  Akademie für den Sozialen und Ökologischen Umbau Akademie für den Sozialen und Ökologischen Umbau 2025-06-10
	\item Pühringer, Stephan  Akademie für den Sozialen und Ökologischen Umbau Akademie für den Sozialen und Ökologischen Umbau 2025-06-10
	\item Hager, Theresa  SASE WAG online event: Empowering Caregivers in Academia: An Online Exchange SASE WAG online event: Empowering Caregivers in Academia: An Online Exchange 2025-04-10
	\item Terhorst, Carlotta  Early Career Writing \& Research Skills Workshop in the Energy Justice Space Early Career Writing \& Research Skills Workshop in the Energy Justice Space 2025-04-09
	\item Eder, Julia Theresa  Konferenz \glqq Rohstoffpolitik gerecht gestalten" Konferenz "Rohstoffpolitik gerecht gestalten\grqq{} 2025-01-30
	\item Eder, Julia Theresa  Panel Discussion \glqq Greening the European Economy at the Expense of the Global South? Insights from Raw Material Exporting Countries" Panel Discussion "Greening the European Economy at the Expense of the Global South? Insights from Raw Material Exporting Countries\grqq{} 2025-01-29
	\item Wiendl-Stark, Patricia  Sustainable Transformation Lunch Sustainable Transformation Lunch 2024-12-05
	\item Pühringer, Stephan  Sustainable Transformation Lunch Sustainable Transformation Lunch 2024-12-05
	\item Gegenhuber, Thomas  Sustainable Transformation Lunch Sustainable Transformation Lunch 2024-12-05
	\item   Who owns the future? Democratic economic planning on multiple levels for a just social-ecological transformation Who owns the future? Democratic economic planning on multiple levels for a just social-ecological transformation 2024-11-21
	\item Pühringer, Stephan  Who owns the future? Democratic economic planning on multiple levels for a just social-ecological transformation Who owns the future? Democratic economic planning on multiple levels for a just social-ecological transformation 2024-11-21
	\item Porak, Laura  Who owns the future? Democratic economic planning on multiple levels for a just social-ecological transformation Who owns the future? Democratic economic planning on multiple levels for a just social-ecological transformation 2024-11-21
	\item   Who owns the future? Democratic economic planning on multiple levels for a just social-ecological transformation Who owns the future? Democratic economic planning on multiple levels for a just social-ecological transformation 2024-11-21
	\item   Utopias are dead. Long live Utopias! Utopias are dead. Long live Utopias! 2024-11-14
	\item   Utopias are dead. Long live Utopias! Utopias are dead. Long live Utopias! 2024-11-14
	\item Bäuerle, Lukas  Utopias are dead. Long live Utopias! Utopias are dead. Long live Utopias! 2024-11-14
	\item Schäfer, Kristina  Lunch Lecture mit Walter Pfeil: Prekäre Wissenschaft durch rechtliche Rahmenbedingungen. Lunch Lecture mit Walter Pfeil: Prekäre Wissenschaft durch rechtliche Rahmenbedingungen. 2024-11-14
	\item Rath, Johanna  Lunch Lecture mit Walter Pfeil: Prekäre Wissenschaft durch rechtliche Rahmenbedingungen. Lunch Lecture mit Walter Pfeil: Prekäre Wissenschaft durch rechtliche Rahmenbedingungen. 2024-11-14
	\item Pühringer, Stephan  Lunch Lecture mit Walter Pfeil: Prekäre Wissenschaft durch rechtliche Rahmenbedingungen. Lunch Lecture mit Walter Pfeil: Prekäre Wissenschaft durch rechtliche Rahmenbedingungen. 2024-11-14
	\item Hornykewycz, Anna  Lunch Lecture mit Walter Pfeil: Prekäre Wissenschaft durch rechtliche Rahmenbedingungen. Lunch Lecture mit Walter Pfeil: Prekäre Wissenschaft durch rechtliche Rahmenbedingungen. 2024-11-14
	\item Heine, Frederic  Lunch Lecture mit Walter Pfeil: Prekäre Wissenschaft durch rechtliche Rahmenbedingungen. Lunch Lecture mit Walter Pfeil: Prekäre Wissenschaft durch rechtliche Rahmenbedingungen. 2024-11-14
	\item Hager, Theresa  Lunch Lecture mit Walter Pfeil: Prekäre Wissenschaft durch rechtliche Rahmenbedingungen. Lunch Lecture mit Walter Pfeil: Prekäre Wissenschaft durch rechtliche Rahmenbedingungen. 2024-11-14
	\item Deindl, Raphael  Lunch Lecture mit Walter Pfeil: Prekäre Wissenschaft durch rechtliche Rahmenbedingungen. Lunch Lecture mit Walter Pfeil: Prekäre Wissenschaft durch rechtliche Rahmenbedingungen. 2024-11-14
	\item Aistleitner, Matthias  Lunch Lecture mit Walter Pfeil: Prekäre Wissenschaft durch rechtliche Rahmenbedingungen. Lunch Lecture mit Walter Pfeil: Prekäre Wissenschaft durch rechtliche Rahmenbedingungen. 2024-11-14
	\item   A Just and Green Transformation in a Changing World A Just and Green Transformation in a Changing World 2024-09-04
	\item Hornykewycz, Anna  A Just and Green Transformation in a Changing World A Just and Green Transformation in a Changing World 2024-09-04
	\item Cserjan, Lukas  A Just and Green Transformation in a Changing World A Just and Green Transformation in a Changing World 2024-09-04
	\item Rath, Johanna  YSI \& EAEPE Pre-Conference for Young Scholars 2024: Change in the Capitalist Order and its Constraints YSI \& EAEPE Pre-Conference for Young Scholars 2024: Change in the Capitalist Order and its Constraints 2024-09-03
	\item Hornykewycz, Anna  YSI \& EAEPE Pre-Conference for Young Scholars 2024: Change in the Capitalist Order and its Constraints YSI \& EAEPE Pre-Conference for Young Scholars 2024: Change in the Capitalist Order and its Constraints 2024-09-03
	\item   YSI \& EAEPE Pre-Conference for Young Scholars 2024: Change in the Capitalist Order and its Constraints YSI \& EAEPE Pre-Conference for Young Scholars 2024: Change in the Capitalist Order and its Constraints 2024-09-03
	\item Hager, Theresa  SASE Women and Gender Forum: General Assembly SASE Women and Gender Forum: General Assembly 2024-06-27
	\item Hager, Theresa  SASE: Book Publishing Lunch SASE: Book Publishing Lunch 2024-06-27
	\item Schäfer, Kristina  Podiumsdiskussion zum Thema \glqq Mehr als Geld: Was braucht gute Wissenschaft?" Podiumsdiskussion zum Thema "Mehr als Geld: Was braucht gute Wissenschaft?\grqq{} 2024-03-14
	\item Rath, Johanna  Podiumsdiskussion zum Thema \glqq Mehr als Geld: Was braucht gute Wissenschaft?" Podiumsdiskussion zum Thema "Mehr als Geld: Was braucht gute Wissenschaft?\grqq{} 2024-03-14
	\item Pühringer, Stephan  Podiumsdiskussion zum Thema \glqq Mehr als Geld: Was braucht gute Wissenschaft?" Podiumsdiskussion zum Thema "Mehr als Geld: Was braucht gute Wissenschaft?\grqq{} 2024-03-14
	\item Hornykewycz, Anna  Podiumsdiskussion zum Thema \glqq Mehr als Geld: Was braucht gute Wissenschaft?" Podiumsdiskussion zum Thema "Mehr als Geld: Was braucht gute Wissenschaft?\grqq{} 2024-03-14
	\item Heine, Frederic  Podiumsdiskussion zum Thema \glqq Mehr als Geld: Was braucht gute Wissenschaft?" Podiumsdiskussion zum Thema "Mehr als Geld: Was braucht gute Wissenschaft?\grqq{} 2024-03-14
	\item Deindl, Raphael  Podiumsdiskussion zum Thema \glqq Mehr als Geld: Was braucht gute Wissenschaft?" Podiumsdiskussion zum Thema "Mehr als Geld: Was braucht gute Wissenschaft?\grqq{} 2024-03-14
	\item Aistleitner, Matthias  Podiumsdiskussion zum Thema \glqq Mehr als Geld: Was braucht gute Wissenschaft?" Podiumsdiskussion zum Thema "Mehr als Geld: Was braucht gute Wissenschaft?\grqq{} 2024-03-14
	\item Hornykewycz, Anna  Coordinator YSI (Young Scholar Initiative), Working Group Sustainability, Institute of New Economic Thinking Coordinator YSI (Young Scholar Initiative), Working Group Sustainability, Institute of New Economic Thinking 2024-01-01
	\item Rath, Johanna  Coordinator YSI (Young Scholar Initiative), Working Group Philosophy of Economics, Institute of New Economic Thinking Coordinator YSI (Young Scholar Initiative), Working Group Philosophy of Economics, Institute of New Economic Thinking 2024-01-01
	\item Pühringer, Stephan  Hegemonie in Bildung und Wissenschaft Hegemonie in Bildung und Wissenschaft 2023-10-19
	\item Rath, Johanna  EAEPE pre-conference: Power of economics without power in economics? EAEPE pre-conference: Power of economics without power in economics? 2023-09-12
	\item Hornykewycz, Anna  EAEPE pre-conference: Power of economics without power in economics? EAEPE pre-conference: Power of economics without power in economics? 2023-09-12
	\item Hager, Theresa  SASE WAG online event: How are gender inequalities in academia enabled? The case of gender-based violence, career norms and hiring practices SASE WAG online event: How are gender inequalities in academia enabled? The case of gender-based violence, career norms and hiring practices 2023-07-12
	\item Hager, Theresa  SASE WAG online event: Gender in the field of socio-economics – current and future perspectives SASE WAG online event: Gender in the field of socio-economics – current and future perspectives 2023-05-24
	\item Pühringer, Stephan  Exzellent prekär. Diese Uni wollen wir nicht mehr! Exzellent prekär. Diese Uni wollen wir nicht mehr! 2023-04-22
	\item Hager, Theresa  Exzellent prekär. Diese Uni wollen wir nicht mehr! Exzellent prekär. Diese Uni wollen wir nicht mehr! 2023-04-22
	\item Schäfer, Kristina  Vollversammlung des Mittelbaus der JKU Linz Vollversammlung des Mittelbaus der JKU Linz 2023-03-16
	\item Rath, Johanna  Vollversammlung des Mittelbaus der JKU Linz Vollversammlung des Mittelbaus der JKU Linz 2023-03-16
	\item Pühringer, Stephan  Vollversammlung des Mittelbaus der JKU Linz Vollversammlung des Mittelbaus der JKU Linz 2023-03-16
	\item Hornykewycz, Anna  Vollversammlung des Mittelbaus der JKU Linz Vollversammlung des Mittelbaus der JKU Linz 2023-03-16
	\item Heine, Frederic  Vollversammlung des Mittelbaus der JKU Linz Vollversammlung des Mittelbaus der JKU Linz 2023-03-16
	\item Hager, Theresa  Vollversammlung des Mittelbaus der JKU Linz Vollversammlung des Mittelbaus der JKU Linz 2023-03-16
	\item Deindl, Raphael  Vollversammlung des Mittelbaus der JKU Linz Vollversammlung des Mittelbaus der JKU Linz 2023-03-16
	\item Aistleitner, Matthias  Vollversammlung des Mittelbaus der JKU Linz Vollversammlung des Mittelbaus der JKU Linz 2023-03-16
	\item Pühringer, Stephan  Workshop: Verzerrte Expertise: Zur (Frei)Handelsdebatte in den Wirtschaftswissenschaften Workshop: Verzerrte Expertise: Zur (Frei)Handelsdebatte in den Wirtschaftswissenschaften 2022-11-12
	\item Hager, Theresa  Workshop: Verzerrte Expertise: Zur (Frei)Handelsdebatte in den Wirtschaftswissenschaften Workshop: Verzerrte Expertise: Zur (Frei)Handelsdebatte in den Wirtschaftswissenschaften 2022-11-12
	\item   Back to the future? Pluralist economics coming of age in an era of multiple crises Back to the future? Pluralist economics coming of age in an era of multiple crises 2022-10-10
	\item Porak, Laura  Back to the future? Pluralist economics coming of age in an era of multiple crises Back to the future? Pluralist economics coming of age in an era of multiple crises 2022-10-10
	\item Rath, Johanna  Mission Competition: Transdisziplinäre Perspektiven auf Wettbewerbs-gesellschaften Mission Competition: Transdisziplinäre Perspektiven auf Wettbewerbs-gesellschaften 2022-09-28
	\item Pühringer, Stephan  Mission Competition: Transdisziplinäre Perspektiven auf Wettbewerbs-gesellschaften Mission Competition: Transdisziplinäre Perspektiven auf Wettbewerbs-gesellschaften 2022-09-28
	\item Porak, Laura  Mission Competition: Transdisziplinäre Perspektiven auf Wettbewerbs-gesellschaften Mission Competition: Transdisziplinäre Perspektiven auf Wettbewerbs-gesellschaften 2022-09-28
	\item Hager, Theresa  Mission Competition: Transdisziplinäre Perspektiven auf Wettbewerbs-gesellschaften Mission Competition: Transdisziplinäre Perspektiven auf Wettbewerbs-gesellschaften 2022-09-28
	\item Gräbner-Radkowitsch, Claudius  Mission Competition: Transdisziplinäre Perspektiven auf Wettbewerbs-gesellschaften Mission Competition: Transdisziplinäre Perspektiven auf Wettbewerbs-gesellschaften 2022-09-28
	\item Aistleitner, Matthias  Mission Competition: Transdisziplinäre Perspektiven auf Wettbewerbs-gesellschaften Mission Competition: Transdisziplinäre Perspektiven auf Wettbewerbs-gesellschaften 2022-09-28
	\item Pühringer, Stephan  Spatial competition as a mean for coordination or control? Discourses, institutions and everyday practices Spatial competition as a mean for coordination or control? Discourses, institutions and everyday practices 2022-09-07
	\item Gräbner-Radkowitsch, Claudius  Spatial competition as a mean for coordination or control? Discourses, institutions and everyday practices Spatial competition as a mean for coordination or control? Discourses, institutions and everyday practices 2022-09-07
	\item   Rankings and the Structure of the economic sciences: promoting excellence, preserving academic quality, or constructing hierarchies and exclusions? Rankings and the Structure of the economic sciences: promoting excellence, preserving academic quality, or constructing hierarchies and exclusions? 2022-07-20
	\item   Rankings and the Structure of the economic sciences: promoting excellence, preserving academic quality, or constructing hierarchies and exclusions? Rankings and the Structure of the economic sciences: promoting excellence, preserving academic quality, or constructing hierarchies and exclusions? 2022-07-20
	\item Pühringer, Stephan  Rankings and the Structure of the economic sciences: promoting excellence, preserving academic quality, or constructing hierarchies and exclusions? Rankings and the Structure of the economic sciences: promoting excellence, preserving academic quality, or constructing hierarchies and exclusions? 2022-07-20
	\item   Competitiveness and Space. Mini Conference at SASE Annual Meeting Competitiveness and Space. Mini Conference at SASE Annual Meeting 2022-07-09
	\item   Competitiveness and Space. Mini Conference at SASE Annual Meeting Competitiveness and Space. Mini Conference at SASE Annual Meeting 2022-07-09
	\item Pühringer, Stephan  Competitiveness and Space. Mini Conference at SASE Annual Meeting Competitiveness and Space. Mini Conference at SASE Annual Meeting 2022-07-09
	\item Gräbner-Radkowitsch, Claudius  Competitiveness and Space. Mini Conference at SASE Annual Meeting Competitiveness and Space. Mini Conference at SASE Annual Meeting 2022-07-09
	\item Altreiter, Carina  Competitiveness and Space. Mini Conference at SASE Annual Meeting Competitiveness and Space. Mini Conference at SASE Annual Meeting 2022-07-09
	\item Pühringer, Stephan  Workshop „Freihandel“ – wer profitiert? Workshop „Freihandel“ – wer profitiert? 2021-11-30
	\item Kaps, Klemens  Workshop „Freihandel“ – wer profitiert? Workshop „Freihandel“ – wer profitiert? 2021-11-30
	\item Fischer, Karin  Workshop „Freihandel“ – wer profitiert? Workshop „Freihandel“ – wer profitiert? 2021-11-30
	\item Eder, Julia Theresa  Workshop „Freihandel“ – wer profitiert? Workshop „Freihandel“ – wer profitiert? 2021-11-30
	\item Pühringer, Stephan  Workshop: Diskursanalyse in den Sozial- und Wirtschaftswissenschaften Workshop: Diskursanalyse in den Sozial- und Wirtschaftswissenschaften 2021-10-20
	\item Porak, Laura  Workshop: Diskursanalyse in den Sozial- und Wirtschaftswissenschaften Workshop: Diskursanalyse in den Sozial- und Wirtschaftswissenschaften 2021-10-20
	\item   Workshop Economics of knowledge creation and utilization in regions Workshop Economics of knowledge creation and utilization in regions 2021-10-01
	\item   Workshop Economics of knowledge creation and utilization in regions Workshop Economics of knowledge creation and utilization in regions 2021-10-01
	\item   Workshop Economics of knowledge creation and utilization in regions Workshop Economics of knowledge creation and utilization in regions 2021-10-01
	\item Gräbner-Radkowitsch, Claudius  Workshop Economics of knowledge creation and utilization in regions Workshop Economics of knowledge creation and utilization in regions 2021-10-01
	\item Schütz, Bernhard  Festakt 10 Jahre ICAE und 70. Geburtstag von Walter Ötsch Festakt 10 Jahre ICAE und 70. Geburtstag von Walter Ötsch 2020-09-25
	\item Rath, Johanna  Festakt 10 Jahre ICAE und 70. Geburtstag von Walter Ötsch Festakt 10 Jahre ICAE und 70. Geburtstag von Walter Ötsch 2020-09-25
	\item Pühringer, Stephan  Festakt 10 Jahre ICAE und 70. Geburtstag von Walter Ötsch Festakt 10 Jahre ICAE und 70. Geburtstag von Walter Ötsch 2020-09-25
	\item Kapeller, Jakob  Festakt 10 Jahre ICAE und 70. Geburtstag von Walter Ötsch Festakt 10 Jahre ICAE und 70. Geburtstag von Walter Ötsch 2020-09-25
	\item Hirte, Katrin  Festakt 10 Jahre ICAE und 70. Geburtstag von Walter Ötsch Festakt 10 Jahre ICAE und 70. Geburtstag von Walter Ötsch 2020-09-25
	\item Pühringer, Stephan  ICAE Research Seminar ICAE Research Seminar 2019-10-10
	\item Kapeller, Jakob  ICAE Research Seminar ICAE Research Seminar 2019-10-10
	\item Hornykewycz, Anna  ICAE Research Seminar ICAE Research Seminar 2019-10-10
	\item Gräbner-Radkowitsch, Claudius  ICAE Research Seminar ICAE Research Seminar 2019-10-10
	\item Pühringer, Stephan  Understanding economic history for shaping the future Understanding economic history for shaping the future 2019-10-01
	\item Gräbner-Radkowitsch, Claudius  Summer School Course: The Political Economy of Economic Complexity: Models, Data and Policy Summer School Course: The Political Economy of Economic Complexity: Models, Data and Policy 2019-09-11
	\item   Summer School Course: Complexity Economics -- Theory and Computational Methods Summer School Course: Complexity Economics -- Theory and Computational Methods 2019-08-09
	\item Gräbner-Radkowitsch, Claudius  Summer School Course: Complexity Economics -- Theory and Computational Methods Summer School Course: Complexity Economics -- Theory and Computational Methods 2019-08-09
	\item Rath, Johanna  Wettbewerb infrage -- Podcast-Reihe Wettbewerb infrage -- Podcast-Reihe 2019-05-01
	\item   Wettbewerb infrage -- Podcast-Reihe Wettbewerb infrage -- Podcast-Reihe 2019-05-01
	\item   Wettbewerb infrage -- Podcast-Reihe Wettbewerb infrage -- Podcast-Reihe 2019-05-01
	\item   Forms of Power in Economics: New perspectives for the Social Studies of Economics between networks, discourses and fields Forms of Power in Economics: New perspectives for the Social Studies of Economics between networks, discourses and fields 2018-12-06
	\item Pühringer, Stephan  Forms of Power in Economics: New perspectives for the Social Studies of Economics between networks, discourses and fields Forms of Power in Economics: New perspectives for the Social Studies of Economics between networks, discourses and fields 2018-12-06
	\item   Forms of Power in Economics: New perspectives for the Social Studies of Economics between networks, discourses and fields Forms of Power in Economics: New perspectives for the Social Studies of Economics between networks, discourses and fields 2018-12-06
	\item Pühringer, Stephan  Power Structures in Economics Power Structures in Economics 2018-11-16
	\item   Power Structures in Economics Power Structures in Economics 2018-11-16
	\item   Trade runner 2049: Complexity, development, and the future of the economy Trade runner 2049: Complexity, development, and the future of the economy 2018-09-26
	\item Gräbner-Radkowitsch, Claudius  Trade runner 2049: Complexity, development, and the future of the economy Trade runner 2049: Complexity, development, and the future of the economy 2018-09-26
	\item   History and Content of Heterodox Economics History and Content of Heterodox Economics 2018-09-06
	\item   History and Content of Heterodox Economics History and Content of Heterodox Economics 2018-09-06
	\item Gräbner-Radkowitsch, Claudius  History and Content of Heterodox Economics History and Content of Heterodox Economics 2018-09-06
	\item   Summer School Course: Complexity Economics -- Theory and Computational Methods Summer School Course: Complexity Economics -- Theory and Computational Methods 2018-08-03
	\item Gräbner-Radkowitsch, Claudius  Summer School Course: Complexity Economics -- Theory and Computational Methods Summer School Course: Complexity Economics -- Theory and Computational Methods 2018-08-03
	\item Grimm, Christian  9. Jahrestagung des ICAE: Der (wirtschafts-)politische Einfluss des Ordoliberalismus 9. Jahrestagung des ICAE: Der (wirtschafts-)politische Einfluss des Ordoliberalismus 2017-10-09
	\item Beyer, Karl  9. Jahrestagung des ICAE: Der (wirtschafts-)politische Einfluss des Ordoliberalismus 9. Jahrestagung des ICAE: Der (wirtschafts-)politische Einfluss des Ordoliberalismus 2017-10-09
	\item Pühringer, Stephan  9. Jahrestagung des ICAE: Der (wirtschafts-)politische Einfluss des Ordoliberalismus 9. Jahrestagung des ICAE: Der (wirtschafts-)politische Einfluss des Ordoliberalismus 2017-10-09
	\item Ötsch, Walter  9. Jahrestagung des ICAE: Der (wirtschafts-)politische Einfluss des Ordoliberalismus 9. Jahrestagung des ICAE: Der (wirtschafts-)politische Einfluss des Ordoliberalismus 2017-10-09
	\item   Summer School Course: Complexity Economics -- Theory and Computational Methods Summer School Course: Complexity Economics -- Theory and Computational Methods 2017-08-04
	\item Gräbner-Radkowitsch, Claudius  Summer School Course: Complexity Economics -- Theory and Computational Methods Summer School Course: Complexity Economics -- Theory and Computational Methods 2017-08-04
	\item Springholz, Florian  8. Sommerakademie des ICAE mit dem Titel: Gemeinsam oder getrennt? Europa zwischen Integration und Polarisierung 8. Sommerakademie des ICAE mit dem Titel: Gemeinsam oder getrennt? Europa zwischen Integration und Polarisierung 2017-05-19
	\item Schütz, Bernhard  8. Sommerakademie des ICAE mit dem Titel: Gemeinsam oder getrennt? Europa zwischen Integration und Polarisierung 8. Sommerakademie des ICAE mit dem Titel: Gemeinsam oder getrennt? Europa zwischen Integration und Polarisierung 2017-05-19
	\item Kapeller, Jakob  8. Sommerakademie des ICAE mit dem Titel: Gemeinsam oder getrennt? Europa zwischen Integration und Polarisierung 8. Sommerakademie des ICAE mit dem Titel: Gemeinsam oder getrennt? Europa zwischen Integration und Polarisierung 2017-05-19
	\item Hirte, Katrin  8. Sommerakademie des ICAE mit dem Titel: Gemeinsam oder getrennt? Europa zwischen Integration und Polarisierung 8. Sommerakademie des ICAE mit dem Titel: Gemeinsam oder getrennt? Europa zwischen Integration und Polarisierung 2017-05-19
	\item Heimberger, Philipp  8. Sommerakademie des ICAE mit dem Titel: Gemeinsam oder getrennt? Europa zwischen Integration und Polarisierung 8. Sommerakademie des ICAE mit dem Titel: Gemeinsam oder getrennt? Europa zwischen Integration und Polarisierung 2017-05-19
	\item Grimm, Christian  8. Sommerakademie des ICAE mit dem Titel: Gemeinsam oder getrennt? Europa zwischen Integration und Polarisierung 8. Sommerakademie des ICAE mit dem Titel: Gemeinsam oder getrennt? Europa zwischen Integration und Polarisierung 2017-05-19
	\item Beyer, Karl  8. Sommerakademie des ICAE mit dem Titel: Gemeinsam oder getrennt? Europa zwischen Integration und Polarisierung 8. Sommerakademie des ICAE mit dem Titel: Gemeinsam oder getrennt? Europa zwischen Integration und Polarisierung 2017-05-19
	\item Aistleitner, Matthias  8. Sommerakademie des ICAE mit dem Titel: Gemeinsam oder getrennt? Europa zwischen Integration und Polarisierung 8. Sommerakademie des ICAE mit dem Titel: Gemeinsam oder getrennt? Europa zwischen Integration und Polarisierung 2017-05-19
	\item Kapeller, Jakob  International Conference: A Great Transformation? Global Perspectives on Contemporary Capitalisms International Conference: A Great Transformation? Global Perspectives on Contemporary Capitalisms 2017-01-09
	\item Grimm, Christian  International Conference: A Great Transformation? Global Perspectives on Contemporary Capitalisms International Conference: A Great Transformation? Global Perspectives on Contemporary Capitalisms 2017-01-09
	\item Fischer, Karin  International Conference: A Great Transformation? Global Perspectives on Contemporary Capitalisms International Conference: A Great Transformation? Global Perspectives on Contemporary Capitalisms 2017-01-09
	\item Decieux, Fabienne  International Conference: A Great Transformation? Global Perspectives on Contemporary Capitalisms International Conference: A Great Transformation? Global Perspectives on Contemporary Capitalisms 2017-01-09
	\item Beyer, Karl  International Conference: A Great Transformation? Global Perspectives on Contemporary Capitalisms International Conference: A Great Transformation? Global Perspectives on Contemporary Capitalisms 2017-01-09
	\item Aulenbacher, Brigitte  International Conference: A Great Transformation? Global Perspectives on Contemporary Capitalisms International Conference: A Great Transformation? Global Perspectives on Contemporary Capitalisms 2017-01-09
	\item Atzmüller, Roland  International Conference: A Great Transformation? Global Perspectives on Contemporary Capitalisms International Conference: A Great Transformation? Global Perspectives on Contemporary Capitalisms 2017-01-09
	\item Aistleitner, Matthias  International Conference: A Great Transformation? Global Perspectives on Contemporary Capitalisms International Conference: A Great Transformation? Global Perspectives on Contemporary Capitalisms 2017-01-09
	\item Springholz, Florian  7. Sommerakademie des ICAE mit dem Titel: Globalisierung und Konzernmacht – Zur Moral des Profits im 21. Jahrhundert 7. Sommerakademie des ICAE mit dem Titel: Globalisierung und Konzernmacht – Zur Moral des Profits im 21. Jahrhundert 2016-06-03
	\item Schütz, Bernhard  7. Sommerakademie des ICAE mit dem Titel: Globalisierung und Konzernmacht – Zur Moral des Profits im 21. Jahrhundert 7. Sommerakademie des ICAE mit dem Titel: Globalisierung und Konzernmacht – Zur Moral des Profits im 21. Jahrhundert 2016-06-03
	\item Pühringer, Stephan  7. Sommerakademie des ICAE mit dem Titel: Globalisierung und Konzernmacht – Zur Moral des Profits im 21. Jahrhundert 7. Sommerakademie des ICAE mit dem Titel: Globalisierung und Konzernmacht – Zur Moral des Profits im 21. Jahrhundert 2016-06-03
	\item Kapeller, Jakob  7. Sommerakademie des ICAE mit dem Titel: Globalisierung und Konzernmacht – Zur Moral des Profits im 21. Jahrhundert 7. Sommerakademie des ICAE mit dem Titel: Globalisierung und Konzernmacht – Zur Moral des Profits im 21. Jahrhundert 2016-06-03
	\item   7. Sommerakademie des ICAE mit dem Titel: Globalisierung und Konzernmacht – Zur Moral des Profits im 21. Jahrhundert 7. Sommerakademie des ICAE mit dem Titel: Globalisierung und Konzernmacht – Zur Moral des Profits im 21. Jahrhundert 2016-06-03
	\item Hirte, Katrin  7. Sommerakademie des ICAE mit dem Titel: Globalisierung und Konzernmacht – Zur Moral des Profits im 21. Jahrhundert 7. Sommerakademie des ICAE mit dem Titel: Globalisierung und Konzernmacht – Zur Moral des Profits im 21. Jahrhundert 2016-06-03
	\item Heimberger, Philipp  7. Sommerakademie des ICAE mit dem Titel: Globalisierung und Konzernmacht – Zur Moral des Profits im 21. Jahrhundert 7. Sommerakademie des ICAE mit dem Titel: Globalisierung und Konzernmacht – Zur Moral des Profits im 21. Jahrhundert 2016-06-03
	\item Grimm, Christian  7. Sommerakademie des ICAE mit dem Titel: Globalisierung und Konzernmacht – Zur Moral des Profits im 21. Jahrhundert 7. Sommerakademie des ICAE mit dem Titel: Globalisierung und Konzernmacht – Zur Moral des Profits im 21. Jahrhundert 2016-06-03
	\item Beyer, Karl  7. Sommerakademie des ICAE mit dem Titel: Globalisierung und Konzernmacht – Zur Moral des Profits im 21. Jahrhundert 7. Sommerakademie des ICAE mit dem Titel: Globalisierung und Konzernmacht – Zur Moral des Profits im 21. Jahrhundert 2016-06-03
	\item Aistleitner, Matthias  7. Sommerakademie des ICAE mit dem Titel: Globalisierung und Konzernmacht – Zur Moral des Profits im 21. Jahrhundert 7. Sommerakademie des ICAE mit dem Titel: Globalisierung und Konzernmacht – Zur Moral des Profits im 21. Jahrhundert 2016-06-03
	\item Pühringer, Stephan  7. Jahrestagung des ICAE: Ökonomie! Welche Ökonomie? Zu Stand und Status der Wirtschaftswissenschaft 7. Jahrestagung des ICAE: Ökonomie! Welche Ökonomie? Zu Stand und Status der Wirtschaftswissenschaft 2015-12-04
	\item Ötsch, Walter  7. Jahrestagung des ICAE: Ökonomie! Welche Ökonomie? Zu Stand und Status der Wirtschaftswissenschaft 7. Jahrestagung des ICAE: Ökonomie! Welche Ökonomie? Zu Stand und Status der Wirtschaftswissenschaft 2015-12-04
	\item Kapeller, Jakob  7. Jahrestagung des ICAE: Ökonomie! Welche Ökonomie? Zu Stand und Status der Wirtschaftswissenschaft 7. Jahrestagung des ICAE: Ökonomie! Welche Ökonomie? Zu Stand und Status der Wirtschaftswissenschaft 2015-12-04
	\item Hirte, Katrin  7. Jahrestagung des ICAE: Ökonomie! Welche Ökonomie? Zu Stand und Status der Wirtschaftswissenschaft 7. Jahrestagung des ICAE: Ökonomie! Welche Ökonomie? Zu Stand und Status der Wirtschaftswissenschaft 2015-12-04
	\item Heimberger, Philipp  7. Jahrestagung des ICAE: Ökonomie! Welche Ökonomie? Zu Stand und Status der Wirtschaftswissenschaft 7. Jahrestagung des ICAE: Ökonomie! Welche Ökonomie? Zu Stand und Status der Wirtschaftswissenschaft 2015-12-04
	\item Grimm, Christian  7. Jahrestagung des ICAE: Ökonomie! Welche Ökonomie? Zu Stand und Status der Wirtschaftswissenschaft 7. Jahrestagung des ICAE: Ökonomie! Welche Ökonomie? Zu Stand und Status der Wirtschaftswissenschaft 2015-12-04
	\item Bräutigam, Lars  7. Jahrestagung des ICAE: Ökonomie! Welche Ökonomie? Zu Stand und Status der Wirtschaftswissenschaft 7. Jahrestagung des ICAE: Ökonomie! Welche Ökonomie? Zu Stand und Status der Wirtschaftswissenschaft 2015-12-04
	\item Beyer, Karl  7. Jahrestagung des ICAE: Ökonomie! Welche Ökonomie? Zu Stand und Status der Wirtschaftswissenschaft 7. Jahrestagung des ICAE: Ökonomie! Welche Ökonomie? Zu Stand und Status der Wirtschaftswissenschaft 2015-12-04
	\item Aistleitner, Matthias  7. Jahrestagung des ICAE: Ökonomie! Welche Ökonomie? Zu Stand und Status der Wirtschaftswissenschaft 7. Jahrestagung des ICAE: Ökonomie! Welche Ökonomie? Zu Stand und Status der Wirtschaftswissenschaft 2015-12-04
	\item   Introduction to Complexity Economics Introduction to Complexity Economics 2015-09-16
	\item Gräbner-Radkowitsch, Claudius  Introduction to Complexity Economics Introduction to Complexity Economics 2015-09-16
	\item   Introduction to Complexity Economics Introduction to Complexity Economics 2015-09-16
	\item Pühringer, Stephan  6. Sommerakademie des ICAE mit dem Titel: Kapitalismus und Gerechtigkeit -- Die Rolle der Ungleichheit im 21. Jahrhundert 6. Sommerakademie des ICAE mit dem Titel: Kapitalismus und Gerechtigkeit -- Die Rolle der Ungleichheit im 21. Jahrhundert 2015-06-19
	\item Ötsch, Walter  6. Sommerakademie des ICAE mit dem Titel: Kapitalismus und Gerechtigkeit -- Die Rolle der Ungleichheit im 21. Jahrhundert 6. Sommerakademie des ICAE mit dem Titel: Kapitalismus und Gerechtigkeit -- Die Rolle der Ungleichheit im 21. Jahrhundert 2015-06-19
	\item Kapeller, Jakob  6. Sommerakademie des ICAE mit dem Titel: Kapitalismus und Gerechtigkeit -- Die Rolle der Ungleichheit im 21. Jahrhundert 6. Sommerakademie des ICAE mit dem Titel: Kapitalismus und Gerechtigkeit -- Die Rolle der Ungleichheit im 21. Jahrhundert 2015-06-19
	\item Hirte, Katrin  6. Sommerakademie des ICAE mit dem Titel: Kapitalismus und Gerechtigkeit -- Die Rolle der Ungleichheit im 21. Jahrhundert 6. Sommerakademie des ICAE mit dem Titel: Kapitalismus und Gerechtigkeit -- Die Rolle der Ungleichheit im 21. Jahrhundert 2015-06-19
	\item Grimm, Christian  6. Sommerakademie des ICAE mit dem Titel: Kapitalismus und Gerechtigkeit -- Die Rolle der Ungleichheit im 21. Jahrhundert 6. Sommerakademie des ICAE mit dem Titel: Kapitalismus und Gerechtigkeit -- Die Rolle der Ungleichheit im 21. Jahrhundert 2015-06-19
	\item Griesser, Markus  6. Sommerakademie des ICAE mit dem Titel: Kapitalismus und Gerechtigkeit -- Die Rolle der Ungleichheit im 21. Jahrhundert 6. Sommerakademie des ICAE mit dem Titel: Kapitalismus und Gerechtigkeit -- Die Rolle der Ungleichheit im 21. Jahrhundert 2015-06-19
	\item Bräutigam, Lars  6. Sommerakademie des ICAE mit dem Titel: Kapitalismus und Gerechtigkeit -- Die Rolle der Ungleichheit im 21. Jahrhundert 6. Sommerakademie des ICAE mit dem Titel: Kapitalismus und Gerechtigkeit -- Die Rolle der Ungleichheit im 21. Jahrhundert 2015-06-19
	\item Beyer, Karl  6. Sommerakademie des ICAE mit dem Titel: Kapitalismus und Gerechtigkeit -- Die Rolle der Ungleichheit im 21. Jahrhundert 6. Sommerakademie des ICAE mit dem Titel: Kapitalismus und Gerechtigkeit -- Die Rolle der Ungleichheit im 21. Jahrhundert 2015-06-19
	\item Pühringer, Stephan  6. Jahrestagung des ICAE: Markt! Welcher Markt? Der interdisziplinäre Diskurs um Märkte und Marktwirtschaft 6. Jahrestagung des ICAE: Markt! Welcher Markt? Der interdisziplinäre Diskurs um Märkte und Marktwirtschaft 2014-12-11
	\item Ötsch, Walter  6. Jahrestagung des ICAE: Markt! Welcher Markt? Der interdisziplinäre Diskurs um Märkte und Marktwirtschaft 6. Jahrestagung des ICAE: Markt! Welcher Markt? Der interdisziplinäre Diskurs um Märkte und Marktwirtschaft 2014-12-11
	\item Hirte, Katrin  6. Jahrestagung des ICAE: Markt! Welcher Markt? Der interdisziplinäre Diskurs um Märkte und Marktwirtschaft 6. Jahrestagung des ICAE: Markt! Welcher Markt? Der interdisziplinäre Diskurs um Märkte und Marktwirtschaft 2014-12-11
	\item Griesser, Markus  6. Jahrestagung des ICAE: Markt! Welcher Markt? Der interdisziplinäre Diskurs um Märkte und Marktwirtschaft 6. Jahrestagung des ICAE: Markt! Welcher Markt? Der interdisziplinäre Diskurs um Märkte und Marktwirtschaft 2014-12-11
	\item Bräutigam, Lars  6. Jahrestagung des ICAE: Markt! Welcher Markt? Der interdisziplinäre Diskurs um Märkte und Marktwirtschaft 6. Jahrestagung des ICAE: Markt! Welcher Markt? Der interdisziplinäre Diskurs um Märkte und Marktwirtschaft 2014-12-11
	\item Beyer, Karl  6. Jahrestagung des ICAE: Markt! Welcher Markt? Der interdisziplinäre Diskurs um Märkte und Marktwirtschaft 6. Jahrestagung des ICAE: Markt! Welcher Markt? Der interdisziplinäre Diskurs um Märkte und Marktwirtschaft 2014-12-11
	\item   Einführung in die Komplexitätsökonomik Einführung in die Komplexitätsökonomik 2014-10-08
	\item Gräbner-Radkowitsch, Claudius  Einführung in die Komplexitätsökonomik Einführung in die Komplexitätsökonomik 2014-10-08
	\item   Einführung in die Komplexitätsökonomik Einführung in die Komplexitätsökonomik 2014-10-08
	\item Pühringer, Stephan  5. Sommerakademie des ICAE mit dem Titel: Arbeit ohne Perspektive? Zwischen Ausbeutung, Arbeitslosigkeit und Selbstbestimmung 5. Sommerakademie des ICAE mit dem Titel: Arbeit ohne Perspektive? Zwischen Ausbeutung, Arbeitslosigkeit und Selbstbestimmung 2014-06-13
	\item Ötsch, Walter  5. Sommerakademie des ICAE mit dem Titel: Arbeit ohne Perspektive? Zwischen Ausbeutung, Arbeitslosigkeit und Selbstbestimmung 5. Sommerakademie des ICAE mit dem Titel: Arbeit ohne Perspektive? Zwischen Ausbeutung, Arbeitslosigkeit und Selbstbestimmung 2014-06-13
	\item Nordmann, Jürgen  5. Sommerakademie des ICAE mit dem Titel: Arbeit ohne Perspektive? Zwischen Ausbeutung, Arbeitslosigkeit und Selbstbestimmung 5. Sommerakademie des ICAE mit dem Titel: Arbeit ohne Perspektive? Zwischen Ausbeutung, Arbeitslosigkeit und Selbstbestimmung 2014-06-13
	\item Hirte, Katrin  5. Sommerakademie des ICAE mit dem Titel: Arbeit ohne Perspektive? Zwischen Ausbeutung, Arbeitslosigkeit und Selbstbestimmung 5. Sommerakademie des ICAE mit dem Titel: Arbeit ohne Perspektive? Zwischen Ausbeutung, Arbeitslosigkeit und Selbstbestimmung 2014-06-13
	\item Bräutigam, Lars  5. Sommerakademie des ICAE mit dem Titel: Arbeit ohne Perspektive? Zwischen Ausbeutung, Arbeitslosigkeit und Selbstbestimmung 5. Sommerakademie des ICAE mit dem Titel: Arbeit ohne Perspektive? Zwischen Ausbeutung, Arbeitslosigkeit und Selbstbestimmung 2014-06-13
	\item Beyer, Karl  5. Sommerakademie des ICAE mit dem Titel: Arbeit ohne Perspektive? Zwischen Ausbeutung, Arbeitslosigkeit und Selbstbestimmung 5. Sommerakademie des ICAE mit dem Titel: Arbeit ohne Perspektive? Zwischen Ausbeutung, Arbeitslosigkeit und Selbstbestimmung 2014-06-13
	\item Pühringer, Stephan  5. Jahrestagung des ICAE: Wissen! Welches Wissen? Wahrheit, Theorie und Glauben in der ökonomischen Theorie 5. Jahrestagung des ICAE: Wissen! Welches Wissen? Wahrheit, Theorie und Glauben in der ökonomischen Theorie 2013-12-12
	\item Ötsch, Walter  5. Jahrestagung des ICAE: Wissen! Welches Wissen? Wahrheit, Theorie und Glauben in der ökonomischen Theorie 5. Jahrestagung des ICAE: Wissen! Welches Wissen? Wahrheit, Theorie und Glauben in der ökonomischen Theorie 2013-12-12
	\item Hirte, Katrin  5. Jahrestagung des ICAE: Wissen! Welches Wissen? Wahrheit, Theorie und Glauben in der ökonomischen Theorie 5. Jahrestagung des ICAE: Wissen! Welches Wissen? Wahrheit, Theorie und Glauben in der ökonomischen Theorie 2013-12-12
	\item Bräutigam, Lars  5. Jahrestagung des ICAE: Wissen! Welches Wissen? Wahrheit, Theorie und Glauben in der ökonomischen Theorie 5. Jahrestagung des ICAE: Wissen! Welches Wissen? Wahrheit, Theorie und Glauben in der ökonomischen Theorie 2013-12-12
	\item Beyer, Karl  5. Jahrestagung des ICAE: Wissen! Welches Wissen? Wahrheit, Theorie und Glauben in der ökonomischen Theorie 5. Jahrestagung des ICAE: Wissen! Welches Wissen? Wahrheit, Theorie und Glauben in der ökonomischen Theorie 2013-12-12
	\item Nordmann, Jürgen  Wissen und Krise Wissen und Krise 2013-09-26
	\item Hirte, Katrin  Wissen und Krise Wissen und Krise 2013-09-26
	\item Pühringer, Stephan  4. Sommerakademie des ICAE zum Thema Wissen schaf(f)t Gesellschaft. Neutrale Wissenschaft? Objektive ExpertInnen? Unabhängige Medien? 4. Sommerakademie des ICAE zum Thema Wissen schaf(f)t Gesellschaft. Neutrale Wissenschaft? Objektive ExpertInnen? Unabhängige Medien? 2013-06-07
	\item Ötsch, Walter  4. Sommerakademie des ICAE zum Thema Wissen schaf(f)t Gesellschaft. Neutrale Wissenschaft? Objektive ExpertInnen? Unabhängige Medien? 4. Sommerakademie des ICAE zum Thema Wissen schaf(f)t Gesellschaft. Neutrale Wissenschaft? Objektive ExpertInnen? Unabhängige Medien? 2013-06-07
	\item Nordmann, Jürgen  4. Sommerakademie des ICAE zum Thema Wissen schaf(f)t Gesellschaft. Neutrale Wissenschaft? Objektive ExpertInnen? Unabhängige Medien? 4. Sommerakademie des ICAE zum Thema Wissen schaf(f)t Gesellschaft. Neutrale Wissenschaft? Objektive ExpertInnen? Unabhängige Medien? 2013-06-07
	\item Hirte, Katrin  4. Sommerakademie des ICAE zum Thema Wissen schaf(f)t Gesellschaft. Neutrale Wissenschaft? Objektive ExpertInnen? Unabhängige Medien? 4. Sommerakademie des ICAE zum Thema Wissen schaf(f)t Gesellschaft. Neutrale Wissenschaft? Objektive ExpertInnen? Unabhängige Medien? 2013-06-07
	\item Beyer, Karl  4. Sommerakademie des ICAE zum Thema Wissen schaf(f)t Gesellschaft. Neutrale Wissenschaft? Objektive ExpertInnen? Unabhängige Medien? 4. Sommerakademie des ICAE zum Thema Wissen schaf(f)t Gesellschaft. Neutrale Wissenschaft? Objektive ExpertInnen? Unabhängige Medien? 2013-06-07
	\item Pühringer, Stephan  4. Jahrestagung des ICAE: The political Economy of Offshore Jurisdiction 4. Jahrestagung des ICAE: The political Economy of Offshore Jurisdiction 2012-11-29
	\item Hirte, Katrin  4. Jahrestagung des ICAE: The political Economy of Offshore Jurisdiction 4. Jahrestagung des ICAE: The political Economy of Offshore Jurisdiction 2012-11-29
	\item Bräutigam, Lars  4. Jahrestagung des ICAE: The political Economy of Offshore Jurisdiction 4. Jahrestagung des ICAE: The political Economy of Offshore Jurisdiction 2012-11-29
	\item Beyer, Karl  4. Jahrestagung des ICAE: The political Economy of Offshore Jurisdiction 4. Jahrestagung des ICAE: The political Economy of Offshore Jurisdiction 2012-11-29
	\item Ötsch, Walter  4. Jahrestagung des ICAE: The political Economy of Offshore Jurisdiction 4. Jahrestagung des ICAE: The political Economy of Offshore Jurisdiction 2012-11-29
	\item Pühringer, Stephan  3. Sommerakademie: Kritik, Einmischung, Protest. Politisches Handeln in der heutigen Zeit 3. Sommerakademie: Kritik, Einmischung, Protest. Politisches Handeln in der heutigen Zeit 2012-06-07
	\item Plaimer, Wolfgang  3. Sommerakademie: Kritik, Einmischung, Protest. Politisches Handeln in der heutigen Zeit 3. Sommerakademie: Kritik, Einmischung, Protest. Politisches Handeln in der heutigen Zeit 2012-06-07
	\item Ötsch, Walter  3. Sommerakademie: Kritik, Einmischung, Protest. Politisches Handeln in der heutigen Zeit 3. Sommerakademie: Kritik, Einmischung, Protest. Politisches Handeln in der heutigen Zeit 2012-06-07
	\item Nordmann, Jürgen  3. Sommerakademie: Kritik, Einmischung, Protest. Politisches Handeln in der heutigen Zeit 3. Sommerakademie: Kritik, Einmischung, Protest. Politisches Handeln in der heutigen Zeit 2012-06-07
	\item Hirte, Katrin  3. Sommerakademie: Kritik, Einmischung, Protest. Politisches Handeln in der heutigen Zeit 3. Sommerakademie: Kritik, Einmischung, Protest. Politisches Handeln in der heutigen Zeit 2012-06-07
	\item Pühringer, Stephan  3. Jahrestagung des ICAE: Demokratie! Welche Demokratie? Postdemokratie kritisch hinterfragt 3. Jahrestagung des ICAE: Demokratie! Welche Demokratie? Postdemokratie kritisch hinterfragt 2011-12-01
	\item Plaimer, Wolfgang  3. Jahrestagung des ICAE: Demokratie! Welche Demokratie? Postdemokratie kritisch hinterfragt 3. Jahrestagung des ICAE: Demokratie! Welche Demokratie? Postdemokratie kritisch hinterfragt 2011-12-01
	\item Hirte, Katrin  3. Jahrestagung des ICAE: Demokratie! Welche Demokratie? Postdemokratie kritisch hinterfragt 3. Jahrestagung des ICAE: Demokratie! Welche Demokratie? Postdemokratie kritisch hinterfragt 2011-12-01
	\item Ötsch, Walter  3. Jahrestagung des ICAE: Demokratie! Welche Demokratie? Postdemokratie kritisch hinterfragt 3. Jahrestagung des ICAE: Demokratie! Welche Demokratie? Postdemokratie kritisch hinterfragt 2011-12-01
	\item Nordmann, Jürgen  3. Jahrestagung des ICAE: Demokratie! Welche Demokratie? Postdemokratie kritisch hinterfragt 3. Jahrestagung des ICAE: Demokratie! Welche Demokratie? Postdemokratie kritisch hinterfragt 2011-12-01
	\item Pühringer, Stephan  2. Sommerakademie: Verborgenes Geld, Verheimlichte Macht, Verachtete Arbeit. Offene Geheimnisse des Kapitalismus 2. Sommerakademie: Verborgenes Geld, Verheimlichte Macht, Verachtete Arbeit. Offene Geheimnisse des Kapitalismus 2011-06-23
	\item Plaimer, Wolfgang  2. Sommerakademie: Verborgenes Geld, Verheimlichte Macht, Verachtete Arbeit. Offene Geheimnisse des Kapitalismus 2. Sommerakademie: Verborgenes Geld, Verheimlichte Macht, Verachtete Arbeit. Offene Geheimnisse des Kapitalismus 2011-06-23
	\item Ötsch, Walter  2. Sommerakademie: Verborgenes Geld, Verheimlichte Macht, Verachtete Arbeit. Offene Geheimnisse des Kapitalismus 2. Sommerakademie: Verborgenes Geld, Verheimlichte Macht, Verachtete Arbeit. Offene Geheimnisse des Kapitalismus 2011-06-23
	\item Nordmann, Jürgen  2. Sommerakademie: Verborgenes Geld, Verheimlichte Macht, Verachtete Arbeit. Offene Geheimnisse des Kapitalismus 2. Sommerakademie: Verborgenes Geld, Verheimlichte Macht, Verachtete Arbeit. Offene Geheimnisse des Kapitalismus 2011-06-23
	\item Hirte, Katrin  2. Sommerakademie: Verborgenes Geld, Verheimlichte Macht, Verachtete Arbeit. Offene Geheimnisse des Kapitalismus 2. Sommerakademie: Verborgenes Geld, Verheimlichte Macht, Verachtete Arbeit. Offene Geheimnisse des Kapitalismus 2011-06-23
	\item Nordmann, Jürgen  2. Jahrestagung des ICAE: Gesellschaft! Welche Gesellschaft? 2. Jahrestagung des ICAE: Gesellschaft! Welche Gesellschaft? 2010-12-02
	\item Hirte, Katrin  2. Jahrestagung des ICAE: Gesellschaft! Welche Gesellschaft? 2. Jahrestagung des ICAE: Gesellschaft! Welche Gesellschaft? 2010-12-02
	\item Ötsch, Walter  2. Jahrestagung des ICAE: Gesellschaft! Welche Gesellschaft? 2. Jahrestagung des ICAE: Gesellschaft! Welche Gesellschaft? 2010-12-02
	\item Ötsch, Walter  1. Sommerakademie: Macht, Eliten, Medien. Wie unsere Gesellschaft funktioniert 1. Sommerakademie: Macht, Eliten, Medien. Wie unsere Gesellschaft funktioniert 2010-06-03
	\item Nordmann, Jürgen  1. Sommerakademie: Macht, Eliten, Medien. Wie unsere Gesellschaft funktioniert 1. Sommerakademie: Macht, Eliten, Medien. Wie unsere Gesellschaft funktioniert 2010-06-03
	\item Hirte, Katrin  1. Sommerakademie: Macht, Eliten, Medien. Wie unsere Gesellschaft funktioniert 1. Sommerakademie: Macht, Eliten, Medien. Wie unsere Gesellschaft funktioniert 2010-06-03
	\item Nordmann, Jürgen  1. Jahrestagung des ICAE: Krise! Welche Krise? 1. Jahrestagung des ICAE: Krise! Welche Krise? 2009-12-03
	\item Hirte, Katrin  1. Jahrestagung des ICAE: Krise! Welche Krise? 1. Jahrestagung des ICAE: Krise! Welche Krise? 2009-12-03
	\item Ötsch, Walter  1. Jahrestagung des ICAE: Krise! Welche Krise? 1. Jahrestagung des ICAE: Krise! Welche Krise? 2009-12-03
\end{enumerate}
\subsection*{Vortrag oder Präsentation}
\begin{enumerate}
	\item Cserjan, Lukas  Mobilitätswende Produzieren! Was kann Deutschland von österreichischer Infrastrukturpolitik lernen? Labour Environmentalism: Krise, Perspektiven und Strategien 2025-12-02
	\item Porak, Laura  Mobilitätswende Produzieren! Was kann Deutschland von österreichischer Infrastrukturpolitik lernen? Labour Environmentalism: Krise, Perspektiven und Strategien 2025-12-02
	\item Hieselmayr, Sophie  Netzwerke der Superreichen in Österreich Die unsichtbare Spitze. Neue Ansätze zur Analyse wirtschaftlicher Eliten und Vermögensungleichheit 2025-11-21
	\item Pühringer, Stephan  Netzwerke der Superreichen in Österreich Die unsichtbare Spitze. Neue Ansätze zur Analyse wirtschaftlicher Eliten und Vermögensungleichheit 2025-11-21
	\item Cserjan, Lukas  Netzwerke der Superreichen in Österreich Die unsichtbare Spitze. Neue Ansätze zur Analyse wirtschaftlicher Eliten und Vermögensungleichheit 2025-11-21
	\item Aistleitner, Matthias  Netzwerke der Superreichen in Österreich Die unsichtbare Spitze. Neue Ansätze zur Analyse wirtschaftlicher Eliten und Vermögensungleichheit 2025-11-21
	\item Hager, Theresa  Die Macht der Rankings (in) der Wettbewerbswissenschaft Open Access-Tagung 2025 2025-11-13
	\item Pühringer, Stephan  Die Macht der Rankings (in) der Wettbewerbswissenschaft Open Access-Tagung 2025 2025-11-13
	\item Pühringer, Stephan  Ökonomische und politische Machtkonzentration in Reichennetzwerken Kurt Rothschild Preis Workshop 2025-11-06
	\item Kapeller, Jakob  Vermögenskonzentration als gesellschaftliche Herausforderung  2025-11-01
	\item Kapeller, Jakob  Model-Platonism on Economics – a Popperian view on Mainstream Economics  2025-11-01
	\item Pühringer, Stephan  Kapitalmacht in der Praxis: Lobbyismus, Spenden und Medien: Wie Besitzverhältnisse und Informationssubstitute Medien beeinflussen  2025-10-28
	\item Stäudelmayr, Alexander  Kapitalmacht in der Praxis: Lobbyismus, Spenden und Medien: Wie Besitzverhältnisse und Informationssubstitute Medien beeinflussen  2025-10-28
	\item Aistleitner, Matthias  Barriers to Socio-Ecological Transition in the Face of Structural Global Inequalities 29th FMM Conference 2025-10-24
	\item Hager, Theresa  Barriers to Socio-Ecological Transition in the Face of Structural Global Inequalities 29th FMM Conference 2025-10-24
	\item Cserjan, Lukas  Barriers to Socio-Ecological Transition in the Face of Structural Global Inequalities 29th FMM Conference 2025-10-24
	\item Scharnreitner, Franz  Barriers to Socio-Ecological Transition in the Face of Structural Global Inequalities 29th FMM Conference 2025-10-24
	\item Hornykewycz, Anna  Barriers to Socio-Ecological Transition in the Face of Structural Global Inequalities 29th FMM Conference 2025-10-24
	\item Pühringer, Stephan  Netzwerke der Macht und des Überreichtums  2025-10-09
	\item Pühringer, Stephan  Netzwerke der Macht und Superreichtum in Österreich Netzwerke der Macht und Superreichtum in Österreich 2025-10-09
	\item Theine, Hendrik  Giving Back Wealth: Analysis of the Communication and Media Reception of the \glqq Good Council for Redistribution\grqq{} (De)legitimized Wealth 2025-10-08
	\item Verita, Carlotta  Giving Back Wealth: Analysis of the Communication and Media Reception of the \glqq Good Council for Redistribution\grqq{} (De)legitimized Wealth 2025-10-08
	\item Stäudelmayr, Alexander  Giving Back Wealth: Analysis of the Communication and Media Reception of the \glqq Good Council for Redistribution\grqq{} (De)legitimized Wealth 2025-10-08
	\item Bäuerle, Lukas  Zentrale Herausforderungen des deutschen New Economy Ökosystems Strategy Platform New Economy Space Deutschland 2025-10-07
	\item Porak, Laura  Mobilitätswende Produzieren -- Konversionspotenziale für nachhaltige Mobilität  2025-10-03
	\item Pühringer, Stephan  Socio-Ecological Transformation: Still a blindspot in Economics Young Economists Conference 2025 2025-10-03
	\item Aistleitner, Matthias  Barriers to Socio-Ecological Transition in the Face of Structural Global Inequalities Young Economists Conference 2025 2025-10-02
	\item Hager, Theresa  Barriers to Socio-Ecological Transition in the Face of Structural Global Inequalities Young Economists Conference 2025 2025-10-02
	\item Cserjan, Lukas  Barriers to Socio-Ecological Transition in the Face of Structural Global Inequalities Young Economists Conference 2025 2025-10-02
	\item Scharnreitner, Franz  Barriers to Socio-Ecological Transition in the Face of Structural Global Inequalities Young Economists Conference 2025 2025-10-02
	\item Hornykewycz, Anna  Barriers to Socio-Ecological Transition in the Face of Structural Global Inequalities Young Economists Conference 2025 2025-10-02
	\item Eder, Julia Theresa  Producing the mobility transition: Opportunities and challenges for the Austrian railway supply industry from a global production network perspective EAEPE Conference 2025 2025-09-26
	\item Gräbner-Radkowitsch, Claudius  Ecological externalization in the EU? EAEPE Conference 2025 2025-09-25
	\item Gräbner-Radkowitsch, Claudius  A Post-Colonial Challenge for Institutionalist Economics -- A Pragmatist Response EAEPE Conference 2025 2025-09-25
	\item Theine, Hendrik  Beyond Text-As-Data: Integrating Structural Topic Modelling and Critical Discourse Analysis for Contextualized Transitions Research European Association for Evolutionary Political Economy 2025-09-25
	\item Verita, Carlotta  Beyond Text-As-Data: Integrating Structural Topic Modelling and Critical Discourse Analysis for Contextualized Transitions Research European Association for Evolutionary Political Economy 2025-09-25
	\item   Giving Back Wealth: Analysis of the Communication and Media Reception of the \glqq Good Council for Redistribution\grqq{} European Association for Evolutionary Political Economy 2025-09-25
	\item Verita, Carlotta  Giving Back Wealth: Analysis of the Communication and Media Reception of the \glqq Good Council for Redistribution\grqq{} European Association for Evolutionary Political Economy 2025-09-25
	\item Theine, Hendrik  Giving Back Wealth: Analysis of the Communication and Media Reception of the \glqq Good Council for Redistribution\grqq{} European Association for Evolutionary Political Economy 2025-09-25
	\item Terhorst, Carlotta  The hydrogen fairytale -- discursive coalitions in EU renewable energy policy Annual Conference of the European Association of Evolutionary Political Economy 2025-09-25
	\item Verita, Carlotta  The hydrogen fairytale -- discursive coalitions in EU renewable energy policy Annual Conference of the European Association of Evolutionary Political Economy 2025-09-25
	\item Porak, Laura  The hydrogen fairytale -- discursive coalitions in EU renewable energy policy Annual Conference of the European Association of Evolutionary Political Economy 2025-09-25
	\item Pühringer, Stephan  Gendered Competitive Practices in Economics: A Multi-Layer Model of Women's Underepresentation EAEPE Conference 2025 2025-09-25
	\item Hager, Theresa  Gendered Competitive Practices in Economics: A Multi-Layer Model of Women's Underepresentation EAEPE Conference 2025 2025-09-25
	\item Pühringer, Stephan  The plurality of economics behind socio-ecological transformations EAEPE Conference 2025 2025-09-25
	\item Bäuerle, Lukas  The plurality of economics behind socio-ecological transformations EAEPE Conference 2025 2025-09-25
	\item   Media Narratives on wealth-based taxation in German speaking countries EAEPE Conference 2025 2025-09-24
	\item   Media Narratives on wealth-based taxation in German speaking countries EAEPE Conference 2025 2025-09-24
	\item   Media Narratives on wealth-based taxation in German speaking countries EAEPE Conference 2025 2025-09-24
	\item Theine, Hendrik  Media Narratives on wealth-based taxation in German speaking countries EAEPE Conference 2025 2025-09-24
	\item Stäudelmayr, Alexander  Media Narratives on wealth-based taxation in German speaking countries EAEPE Conference 2025 2025-09-24
	\item Aistleitner, Matthias  Barriers to Socio-Ecological Transition in the Face of Structural Global Inequalities EAEPE Conference 2025 2025-09-24
	\item Hager, Theresa  Barriers to Socio-Ecological Transition in the Face of Structural Global Inequalities EAEPE Conference 2025 2025-09-24
	\item Cserjan, Lukas  Barriers to Socio-Ecological Transition in the Face of Structural Global Inequalities EAEPE Conference 2025 2025-09-24
	\item Scharnreitner, Franz  Barriers to Socio-Ecological Transition in the Face of Structural Global Inequalities EAEPE Conference 2025 2025-09-24
	\item Hornykewycz, Anna  Barriers to Socio-Ecological Transition in the Face of Structural Global Inequalities EAEPE Conference 2025 2025-09-24
	\item   The social and epistemic structure of research on Socio-ecological Transformation. Challenging the economic mainstream? EAEPE Conference 2025 2025-09-24
	\item   The social and epistemic structure of research on Socio-ecological Transformation. Challenging the economic mainstream? EAEPE Conference 2025 2025-09-24
	\item Aistleitner, Matthias  The social and epistemic structure of research on Socio-ecological Transformation. Challenging the economic mainstream? EAEPE Conference 2025 2025-09-24
	\item Theine, Hendrik  Mediated Discourses, Digital Capitalism, and the Struggle for Socio-Ecological Futures Pluralist Perspectives on Digital Technologies and Societal Transformation 2025-09-23
	\item Eder, Julia Theresa  Kommentar \glqq Industriepolitik für die Mobilitätswende" Jahrestagung des Fachbereich "Finanzwissenschaft und Infrastrukturplanung\grqq{} (IFIP) des Institut für Raumplanung der TU Wien 2025-09-11
	\item Kapeller, Jakob  It’s the economy stupid – Was ist eigentlich die Wirtschaft?  2025-09-01
	\item Kapeller, Jakob  Pluralism in Economics  2025-09-01
	\item Stäudelmayr, Alexander  Media Capture and Authoritarian Media Policy: A Scoping Review of Right-Wing Strategies in the 21st Century Communication and Capital(ism): ESA RN18 Mid-Term Conference 2025-08-30
	\item Verita, Carlotta  Media Capture and Authoritarian Media Policy: A Scoping Review of Right-Wing Strategies in the 21st Century Communication and Capital(ism): ESA RN18 Mid-Term Conference 2025-08-30
	\item Theine, Hendrik  Media Capture and Authoritarian Media Policy: A Scoping Review of Right-Wing Strategies in the 21st Century Communication and Capital(ism): ESA RN18 Mid-Term Conference 2025-08-30
	\item   Giving Back Wealth: Analysis of the Communication and Media Reception of the "Good Council for Redistribution Communication and Capital(ism): ESA RN18 Mid-Term Conference 2025-08-29
	\item Theine, Hendrik  Giving Back Wealth: Analysis of the Communication and Media Reception of the "Good Council for Redistribution Communication and Capital(ism): ESA RN18 Mid-Term Conference 2025-08-29
	\item Verita, Carlotta  Giving Back Wealth: Analysis of the Communication and Media Reception of the "Good Council for Redistribution Communication and Capital(ism): ESA RN18 Mid-Term Conference 2025-08-29
	\item   \glqq Disturbing Narratives in Economics: The Role of the Media regarding Notions on Wealth (Re-) Distribution\grqq{} Communication and Capital(ism): ESA RN18 Mid-Term Conference 2025-08-27
	\item   \glqq Disturbing Narratives in Economics: The Role of the Media regarding Notions on Wealth (Re-) Distribution\grqq{} Communication and Capital(ism): ESA RN18 Mid-Term Conference 2025-08-27
	\item Theine, Hendrik  \glqq Disturbing Narratives in Economics: The Role of the Media regarding Notions on Wealth (Re-) Distribution\grqq{} Communication and Capital(ism): ESA RN18 Mid-Term Conference 2025-08-27
	\item   \glqq Disturbing Narratives in Economics: The Role of the Media regarding Notions on Wealth (Re-) Distribution\grqq{} Communication and Capital(ism): ESA RN18 Mid-Term Conference 2025-08-27
	\item Stäudelmayr, Alexander  \glqq Disturbing Narratives in Economics: The Role of the Media regarding Notions on Wealth (Re-) Distribution\grqq{} Communication and Capital(ism): ESA RN18 Mid-Term Conference 2025-08-27
	\item Eder, Julia Theresa  Umkämpfte Mobilitätswende Marxistische Studienwoche 2025 2025-08-12
	\item Hornykewycz, Anna  Gestalte das perfekte Wirtschaftssystem JKU Science Holidays 2025-08-07
	\item Terhorst, Carlotta  Gestalte das perfekte Wirtschaftssystem JKU Science Holidays 2025-08-07
	\item Hornykewycz, Anna  Gestalte das perfekte Wirtschaftssystem JKU Science Holidays 2025-07-30
	\item Terhorst, Carlotta  Gestalte das perfekte Wirtschaftssystem JKU Science Holidays 2025-07-30
	\item Pühringer, Stephan  Wie soll Reichtum verteilt werden, auf welcher Insel willst du leben? Science Holidays 2025 2025-07-23
	\item Terhorst, Carlotta  Gestalte das perfekte Wirtschaftssystem JKU Science Holidays 2025-07-17
	\item Hornykewycz, Anna  Gestalte das perfekte Wirtschaftssystem JKU Science Holidays 2025-07-17
	\item Altreiter, Carina  Political Economy of Knowledge Production in Capitalist Academia: Challenges for Socio-Ecological Transformation SASE'S 37th Annual Conference 2025-07-10
	\item Pühringer, Stephan  Political Economy of Knowledge Production in Capitalist Academia: Challenges for Socio-Ecological Transformation SASE'S 37th Annual Conference 2025-07-10
	\item Hager, Theresa  Political Economy of Knowledge Production in Capitalist Academia: Challenges for Socio-Ecological Transformation SASE'S 37th Annual Conference 2025-07-10
	\item Pühringer, Stephan  Competing to Belong: Gendered Competitive Practices and the Persistent Underrepresentation of Women in Economics SASE'S 37th Annual Conference 2025-07-09
	\item Hager, Theresa  Competing to Belong: Gendered Competitive Practices and the Persistent Underrepresentation of Women in Economics SASE'S 37th Annual Conference 2025-07-09
	\item Pühringer, Stephan  The polysemy of sustainabilities. Insights from the Social Studies of Economics Discourse Net Conference: Discourse and the imaginaries of past, present and future societies: media and representations of (inter)national (dis)orders 2025-07-08
	\item Bäuerle, Lukas  The polysemy of sustainabilities. Insights from the Social Studies of Economics Discourse Net Conference: Discourse and the imaginaries of past, present and future societies: media and representations of (inter)national (dis)orders 2025-07-08
	\item Terhorst, Carlotta  Ungleichheiten in der grünen Transformation: Die vernachlässigte Rolle von kollektiven Infrastrukturen und Institutionen Transformationssoziologie konkret 2025-07-01
	\item Kapeller, Jakob  Ökonomische Polarisierung in Europa  2025-07-01
	\item Theine, Hendrik  Beyond Text-As-Data: Integrating Structural Topic Modelling and Critical Discourse Analysis for Contextualised Transitions Research International Sustainability Transitions (IST) 2025-06-25
	\item Verita, Carlotta  Beyond Text-As-Data: Integrating Structural Topic Modelling and Critical Discourse Analysis for Contextualised Transitions Research International Sustainability Transitions (IST) 2025-06-25
	\item Porak, Laura  Discursive Coalitions in EU Renewable Energy Politics: Building the Hydrogen Fairytale IST Conference 2025 2025-06-25
	\item Verita, Carlotta  Discursive Coalitions in EU Renewable Energy Politics: Building the Hydrogen Fairytale IST Conference 2025 2025-06-25
	\item Terhorst, Carlotta  Discursive Coalitions in EU Renewable Energy Politics: Building the Hydrogen Fairytale IST Conference 2025 2025-06-25
	\item Eder, Julia Theresa  Mobilitätswende produzieren -- Beschäftigungspotenziale, Arbeitsbedingungen und Arbeitskräftebedarf der österreichischen Bahnindustrie  2025-06-24
	\item Theine, Hendrik  Giving back wealth: Analysis of the communication and media reception of the \glqq Good Council for Redistribution\grqq{} Union for Democratic Communication Conference 2025-06-20
	\item Gräbner-Radkowitsch, Claudius  Kommentar zum Buch \glqq Kapitalismus am Limit\grqq{} Kapitalismus am Limit 2025-06-18
	\item Bäuerle, Lukas  Forschung zu (pluraler) ökonomischer Bildung: Überblick und Beispiel Strategy Platform pluraler Hochschulstandorte in Deutschland 2025-06-17
	\item Eder, Julia Theresa  Mobilitätswende produzieren -- Chancen und Herausforderungen für die österreichische Bahnindustrie im europäischen und globalen Kontext  2025-06-12
	\item Pühringer, Stephan  Das Recht auf gute und nachhaltige Mobilität Akademie für den Sozialen und Ökologischen Umbau 2025-06-11
	\item Eder, Julia Theresa  Tracing the EU’s quest for green hydrogen in Chile Kongress des Consejo Europeo de Investigación en Ciencias Sociales de América Latina (CEISAL) 2025 2025-06-03
	\item Eder, Julia Theresa  Industriepolitik als Ausweg aus der Krise?  2025-05-27
	\item Gräbner-Radkowitsch, Claudius  Liegt die Wahrheit irgendwo dazwischen? Eine plurale Perspektive auf globale Ungleichheit  2025-05-22
	\item Pühringer, Stephan  Die Politische Ökonomie der Vermögenskonzentration in Österreich  2025-05-21
	\item Eder, Julia Theresa  Das globale Produktionsnetzwerk der österreichischen Bahnindustrie  2025-05-12
	\item Bäuerle, Lukas  Zukunftsfähiges Wirtschaften. Welche innovativen Wege weisen Wissenschaft und Praxis?  2025-05-08
	\item Porak, Laura  Das globale Produktionsnetzwerk der österreichischen Bahnindustrie  2025-05-06
	\item Eder, Julia Theresa  Das globale Produktionsnetzwerk der österreichischen Bahnindustrie  2025-05-06
	\item Hornykewycz, Anna  Gastvortrag Propädeutikum Doktorat Sozial- und Wirtschaftswissenschaften  2025-04-29
	\item Hornykewycz, Anna  Gestalte das perfekte Wirtschaftssystem Stiftung für Wirtschaftsbildung 2025-04-23
	\item Sommer, Marlene  Herausforderungen für eine Wirtschaft im 21. Jahrhundert Schüler*innen Workshop an der JKU 2025-04-22
	\item Hager, Theresa  Herausforderungen für eine Wirtschaft im 21. Jahrhundert Schüler*innen Workshop an der JKU 2025-04-22
	\item Hager, Theresa  Podiumsdiskussion zum Thema \glqq Perspektiven auf eine klimasoziale Stadt" Podiumsdiskussion zum Thema "Perspektiven auf eine klimasoziale Stadt\grqq{} 2025-04-10
	\item Hornykewycz, Anna  Konzernmacht in globalen Warenketten  2025-04-09
	\item Pühringer, Stephan  Präsentation der Endergebnisse der Studie „Mobilitätswende produzieren“  2025-04-07
	\item Porak, Laura  Präsentation der Endergebnisse der Studie „Mobilitätswende produzieren“  2025-04-07
	\item Hornykewycz, Anna  Präsentation der Endergebnisse der Studie „Mobilitätswende produzieren“  2025-04-07
	\item Eder, Julia Theresa  Präsentation der Endergebnisse der Studie „Mobilitätswende produzieren“  2025-04-07
	\item Cserjan, Lukas  Präsentation der Endergebnisse der Studie „Mobilitätswende produzieren“  2025-04-07
	\item Terhorst, Carlotta  Die Bedeutung der Freiheit: Eine philosophische Perspektive auf die Grenzen des Wachstums\&nbsp; Who is in charge here?Konzeptionen von Natur, Kultur \& Agency angesichts der ökologischen Krise 2025-04-04
	\item Eder, Julia Theresa  Industriepolitik im Zeitalter von globalen Warenketten“ im Rahmen der VU „Globale Warenketten  2025-04-02
	\item Pühringer, Stephan  Netzwerke des Überreichtums in Österreich und ihre Implikationen auf Einladung des Netzwerke der Initiativen für Philanthropie in Österreich 2025-03-19
	\item Altreiter, Carina  Political Economy of Knowledge Production in Capitalist Academia: Challenges for Socio-Ecological Transformation STS Hub Conference 2025 2025-03-11
	\item Pühringer, Stephan  Political Economy of Knowledge Production in Capitalist Academia: Challenges for Socio-Ecological Transformation STS Hub Conference 2025 2025-03-11
	\item Hager, Theresa  Political Economy of Knowledge Production in Capitalist Academia: Challenges for Socio-Ecological Transformation STS Hub Conference 2025 2025-03-11
	\item Kapeller, Jakob  Konzernmacht in globalen Güterketten  2025-03-01
	\item Cserjan, Lukas  Präsentation der Zwischenergebnisse der Studie „Mobilitätswende produzieren“  2025-01-24
	\item Eder, Julia Theresa  Präsentation der Zwischenergebnisse der Studie „Mobilitätswende produzieren“  2025-01-24
	\item Pühringer, Stephan  Präsentation der Zwischenergebnisse der Studie „Mobilitätswende produzieren“  2025-01-16
	\item Porak, Laura  Präsentation der Zwischenergebnisse der Studie „Mobilitätswende produzieren“  2025-01-16
	\item Hornykewycz, Anna  Präsentation der Zwischenergebnisse der Studie „Mobilitätswende produzieren“  2025-01-16
	\item Eder, Julia Theresa  Präsentation der Zwischenergebnisse der Studie „Mobilitätswende produzieren“  2025-01-16
	\item Cserjan, Lukas  Präsentation der Zwischenergebnisse der Studie „Mobilitätswende produzieren“  2025-01-16
	\item Gräbner-Radkowitsch, Claudius  An Institutionalist Approach to Socio-Ecological Transformations: From Goals and Strategies to the Key Role of Relatedness Annual Allied Social Science Associations Meeting 2025-01-03
	\item   Media Concentration Across Wireless, Online Advertising, and Online Search​. Findings from the US Team 75th International Communication Association 2025-01-01
	\item   Media Concentration Across Wireless, Online Advertising, and Online Search​. Findings from the US Team 75th International Communication Association 2025-01-01
	\item   Media Concentration Across Wireless, Online Advertising, and Online Search​. Findings from the US Team 75th International Communication Association 2025-01-01
	\item Theine, Hendrik  Media Concentration Across Wireless, Online Advertising, and Online Search​. Findings from the US Team 75th International Communication Association 2025-01-01
	\item Theine, Hendrik  Discourse of Justice: Giving back wealth -- Analysis of the communication and media reception of the good council for redistribution 75th International Communication Association 2025-01-01
	\item Porak, Laura  Delegating Decarbonization: How InvestEU empowers financial capital at the expense of a democratic governance of EU industrial policy  2025-01-01
	\item   Delegating Decarbonization: How InvestEU empowers financial capital at the expense of a democratic governance of EU industrial policy  2025-01-01
	\item Terhorst, Carlotta  Inequalities in green transitions: The neglected role of collective infrastructures and institutions ICAE Research Seminar 2024-12-17
	\item Pühringer, Stephan  Die Nobelpreise 2024: Wirtschaft unbekannt/unknown 2024-12-05
	\item Rath, Johanna  Institutional Evolutionary Economics: Introduction LV VU Exploring Economics (Universität Innsbruck) 2024-12-03
	\item Kapeller, Jakob  Philosophical perspectives on moral erosion in markets  2024-12-01
	\item Sommer, Marlene  Gestalte das perfekte Wirtschaftssystem unbekannt/unknown 2024-11-29
	\item Hornykewycz, Anna  Gestalte das perfekte Wirtschaftssystem unbekannt/unknown 2024-11-29
	\item Pühringer, Stephan  Pressekonferenz zur Studie: \glqq Netzwerke der Superreichen in Österreich\grqq{} unbekannt/unknown 2024-11-26
	\item Porak, Laura  Degrowth als neue Wirtschaftsordnung? unbekannt/unknown 2024-11-21
	\item Rath, Johanna  Unravelling the social-epistemological costs of short-term contracts in Austrian research funding schemes EAEPE 2024 2024-11-21
	\item Eder, Julia Theresa  Input bei Konferenz \glqq Monster verstehen" in Schiene III zu "Wer verfolgt welche Transformationsstrategien in Deutschland und Europa? Und was wäre eine progressive Strategie für den wirtschaftlichen Umbau?\grqq{} unbekannt/unknown 2024-11-16
	\item Pühringer, Stephan  Sozial-ökologische Transformation braucht eine andere Wirtschaft unbekannt/unknown 2024-11-12
	\item Hager, Theresa  Forschen mit Ablaufdatum unbekannt/unknown 2024-11-07
	\item Kapeller, Jakob  Ökonomische Polarisierung in Europa unbekannt/unknown 2024-11-06
	\item Eder, Julia Theresa  Input bei Research Seminar des ICAE: Identifying and overcoming the challenges of the Austrian mobility unbekannt/unknown 2024-11-05
	\item Cserjan, Lukas  Mobility Transition in Austria’s Railway Sector:Economic Preconditions and Impact Forum for Macroeconomics and Macroeconomic Policies 2024-10-24
	\item   Wealth Inequality and Media Narratives: Exploring Redistribution Policy Debates on Taxation in German-Speaking Countries Momentum Kongress: Alternativen 2024-10-18
	\item Porak, Laura  Kritische Utopien als Methode und Praxis: Der Forschungsstandort Österreich unbekannt/unknown 2024-10-18
	\item Hager, Theresa  Kritische Utopien als Methode und Praxis: Der Forschungsstandort Österreich unbekannt/unknown 2024-10-18
	\item   Kritische Utopien als Methode und Praxis: Der Forschungsstandort Österreich unbekannt/unknown 2024-10-18
	\item Kapeller, Jakob  Woher kommt das Geld? unbekannt/unknown 2024-10-15
	\item Theine, Hendrik  The climate crisis portrayal in Austrian media: A Cross-Paradigm approach combining structural topic modelling and critical discourse analysis unbekannt/unknown 2024-10-15
	\item Porak, Laura  Openness only thrives where fairness survives: The geo-economic turn in EU competitiveness discourse unbekannt/unknown 2024-10-10
	\item Eder, Julia Theresa  Vortrag bei Lernwerkstatt der Gewerkschaftsschule Oberösterreich zur Einführung in wissenschaftliches Schreiben \glqq Von der Idee zum Schreiben\grqq{} unbekannt/unknown 2024-10-05
	\item   Make big tech pay their fair share? A comparative analysis of news-based policies and the underlying economic imaginaries that support them International Conference on Historical-Materialist Policy Analysis 2024-10-03
	\item Theine, Hendrik  Make big tech pay their fair share? A comparative analysis of news-based policies and the underlying economic imaginaries that support them International Conference on Historical-Materialist Policy Analysis 2024-10-03
	\item Kapeller, Jakob  Pluralism in Economics unbekannt/unknown 2024-10-01
	\item Bäuerle, Lukas  Aufgeheizte Debatten. Narrative, Technologien und imaginierte Zukünfte der deutschen Wärmewende 7. Jahrestagung der Gesellschaft für sozioökonomische Bildung und Wissenschaft (GSÖBW) 2024-09-26
	\item Pühringer, Stephan  Gendered Competitive Practices in Economics. A Multi-Layer Model of Women’s Underrepresentation unbekannt/unknown 2024-09-26
	\item Hager, Theresa  Gendered Competitive Practices in Economics. A Multi-Layer Model of Women’s Underrepresentation unbekannt/unknown 2024-09-26
	\item Eder, Julia Theresa  The green transformation in Europe: who pays the price? Workshop der Young Scholar Initiative im Rahmen der Young Economists Conference der Arbeiterkammer Wien 2024-09-25
	\item Kapeller, Jakob  Pluralism in economics and the split between heterodoxy and orthodoxy SummerSchool „Wealth Inequality Research in Africa“ 2024-09-17
	\item Kapeller, Jakob  Orthodox vs. Heterodox understandings of wealth dynamics SummerSchool „Wealth Inequality Research in Africa“ 2024-09-16
	\item Porak, Laura  Economic Preconditions and Impacts of the Mobility Transition in Austria's Railway Sector unbekannt/unknown 2024-09-06
	\item Cserjan, Lukas  Economic Preconditions and Impacts of the Mobility Transition in Austria's Railway Sector unbekannt/unknown 2024-09-06
	\item Hornykewycz, Anna  Economic Preconditions and Impacts of the Mobility Transition in Austria's Railway Sector unbekannt/unknown 2024-09-06
	\item Eder, Julia Theresa  Economic Preconditions and Impacts of the Mobility Transition in Austria's Railway Sector unbekannt/unknown 2024-09-06
	\item Pühringer, Stephan  Economic Preconditions and Impacts of the Mobility Transition in Austria's Railway Sector unbekannt/unknown 2024-09-06
	\item Porak, Laura  Openness only thrives where fairness survives: The geo-economic turn in EU competitiveness discourse EAEPE 2024-09-05
	\item   The social and epistemic structure of research on Socio-ecological Transformation. Challenging the economic mainstream? EAEPE 2024-09-04
	\item   The social and epistemic structure of research on Socio-ecological Transformation. Challenging the economic mainstream? EAEPE 2024-09-04
	\item Aistleitner, Matthias  The social and epistemic structure of research on Socio-ecological Transformation. Challenging the economic mainstream? EAEPE 2024-09-04
	\item Eder, Julia Theresa  Identifying and overcoming the challenges of the Austrian mobility transition Konferenz der European Association for Evolutionary Political Economy 2024-09-04
	\item   The social and epistemic structure of research on Socio-ecological Transformation. Challenging the economic mainstream? unbekannt/unknown 2024-09-04
	\item Pühringer, Stephan  The social and epistemic structure of research on Socio-ecological Transformation. Challenging the economic mainstream? unbekannt/unknown 2024-09-04
	\item Aistleitner, Matthias  The social and epistemic structure of research on Socio-ecological Transformation. Challenging the economic mainstream? unbekannt/unknown 2024-09-04
	\item Porak, Laura  Identifying and overcoming challenges for the Austrian mobility transition unbekannt/unknown 2024-09-04
	\item Hornykewycz, Anna  Identifying and overcoming challenges for the Austrian mobility transition unbekannt/unknown 2024-09-04
	\item Cserjan, Lukas  Identifying and overcoming challenges for the Austrian mobility transition unbekannt/unknown 2024-09-04
	\item Pühringer, Stephan  Identifying and overcoming challenges for the Austrian mobility transition unbekannt/unknown 2024-09-04
	\item Eder, Julia Theresa  Identifying and overcoming challenges for the Austrian mobility transition unbekannt/unknown 2024-09-04
	\item   Gender Norms and Network Structure: A Model of the Intra-Household Division of Labor 36th Annual EAEPE Conference 2024-09-04
	\item Hager, Theresa  Gender Norms and Network Structure: A Model of the Intra-Household Division of Labor 36th Annual EAEPE Conference 2024-09-04
	\item   Gender Norms and Network Structure: A Model of the Intra-Household Division of Labor 36th Annual EAEPE Conference 2024-09-04
	\item Pühringer, Stephan  Superreichen-Netzwerke in Österreich unbekannt/unknown 2024-08-28
	\item Kapeller, Jakob  The climate crisis from a pluralist perspective pluralumn Conference 2024-08-06
	\item Hornykewycz, Anna  JKU Science Holidays: Gestalte das perfekte Wirtschaftssystem unbekannt/unknown 2024-07-24
	\item Hornykewycz, Anna  JKU Science Holidays: Gestalte das perfekte Wirtschaftssystem unbekannt/unknown 2024-07-23
	\item Porak, Laura  Woher kommt Schokolade und wer macht eigentlich mein Handy? unbekannt/unknown 2024-07-22
	\item Hornykewycz, Anna  JKU Science Holidays: Gestalte das perfekte Wirtschaftssystem unbekannt/unknown 2024-07-16
	\item   The social and epistemic structure of research on Socio-ecological Transformation. Challenging the economic mainstream? 36th Annual SASE Conference 2024-06-29
	\item   The social and epistemic structure of research on Socio-ecological Transformation. Challenging the economic mainstream? 36th Annual SASE Conference 2024-06-29
	\item Aistleitner, Matthias  The social and epistemic structure of research on Socio-ecological Transformation. Challenging the economic mainstream? 36th Annual SASE Conference 2024-06-29
	\item   Unraveling Explicit and Implicit Rules of Competition: An Interdisciplinary Framework for Analysis 36th Annual SASE Conference 2024-06-29
	\item Hager, Theresa  Unraveling Explicit and Implicit Rules of Competition: An Interdisciplinary Framework for Analysis 36th Annual SASE Conference 2024-06-29
	\item Porak, Laura  Unraveling Explicit and Implicit Rules of Competition: An Interdisciplinary Framework for Analysis 36th Annual SASE Conference 2024-06-29
	\item   The social and epistemic structure of research on Socio-ecological Transformation. Challenging the economic mainstream? unbekannt/unknown 2024-06-28
	\item Aistleitner, Matthias  The social and epistemic structure of research on Socio-ecological Transformation. Challenging the economic mainstream? unbekannt/unknown 2024-06-28
	\item Pühringer, Stephan  The social and epistemic structure of research on Socio-ecological Transformation. Challenging the economic mainstream? unbekannt/unknown 2024-06-28
	\item Porak, Laura  Die grüne Transformation des EU-Verkehrssektors AK EU-Länderreferent*innentagung 2024-06-20
	\item Cserjan, Lukas  Wirtschaftliche Herausforderungen im 21. Jahrhundert unbekannt/unknown 2024-06-18
	\item Porak, Laura  Wirtschaftliche Herausforderungen im 21. Jahrhundert unbekannt/unknown 2024-06-18
	\item Kapeller, Jakob  Ökonomische Polarisierung in Europa unbekannt/unknown 2024-06-11
	\item Pühringer, Stephan  Wie viel Wettbewerb wollen wir (uns leisten)? Die „Kosten“ kompetitiver Drittmittelvergabe an österreichischen Unis unbekannt/unknown 2024-05-29
	\item Porak, Laura  Studienpräsentation \glqq Mobilitätswende produzieren\grqq{}. unbekannt/unknown 2024-05-27
	\item Eder, Julia  Studienpräsentation \glqq Mobilitätswende produzieren\grqq{}. unbekannt/unknown 2024-05-27
	\item Hornykewycz, Anna  Studienpräsentation \glqq Mobilitätswende produzieren\grqq{}. unbekannt/unknown 2024-05-27
	\item Cserjan, Lukas  Studienpräsentation \glqq Mobilitätswende produzieren\grqq{}. unbekannt/unknown 2024-05-27
	\item Pühringer, Stephan  Studienpräsentation \glqq Mobilitätswende produzieren\grqq{}. unbekannt/unknown 2024-05-27
	\item Porak, Laura  The European green transition and the role of industrial policy unbekannt/unknown 2024-05-21
	\item Pühringer, Stephan  Die Macht des ökonomischen Denkens: Narrative, Akteur*innen und Netzwerke unbekannt/unknown 2024-05-15
	\item Cserjan, Lukas  Die Macht des ökonomischen Denkens: Narrative, Akteur*innen und Netzwerke unbekannt/unknown 2024-05-15
	\item Hager, Theresa  Die Macht des ökonomischen Denkens: Narrative, Akteur*innen und Netzwerke unbekannt/unknown 2024-05-15
	\item Porak, Laura  Die Macht des ökonomischen Denkens: Narrative, Akteur*innen und Netzwerke unbekannt/unknown 2024-05-15
	\item Pühringer, Stephan  Netzwerke von Überreichtum in Österreich unbekannt/unknown 2024-04-11
	\item Pühringer, Stephan  Promoters of academic competition? The role of academic social networks and platforms in the competitization of science unbekannt/unknown 2024-04-04
	\item Hager, Theresa  Podiumsdiskussion zum Thema \glqq Mehr als Geld: Was braucht gute Wissenschaft?\grqq{} unbekannt/unknown 2024-03-14
	\item Aistleitner, Matthias  Wirtschaftsmodelle ohne Klimakrise unbekannt/unknown 2024-03-06
	\item Pühringer, Stephan  Wirtschaftsmodelle ohne Klimakrise unbekannt/unknown 2024-03-06
	\item Hornykewycz, Anna  Junge Forschung \#9 mit Anna Hornykewycz unbekannt/unknown 2024-03-01
	\item Pühringer, Stephan  Welche Universität braucht gute Wissenschaft? unbekannt/unknown 2024-02-26
	\item Hirte, Katrin  Konversionsstrategien in der Landwirtschaft: Agrarische sozial-ökologische Transformation und die Landwirte unbekannt/unknown 2024-02-26
	\item Altreiter, Carina  How capitalist academia needs to change to assist structured socio-ecological transformation unbekannt/unknown 2024-02-23
	\item Pühringer, Stephan  How capitalist academia needs to change to assist structured socio-ecological transformation unbekannt/unknown 2024-02-23
	\item Rath, Johanna  Herausforderungen für eine Wirtschaft im 21. Jahrhundert Schüler*innen Workshop an der JKU 2024-02-06
	\item Cserjan, Lukas  Herausforderungen für eine Wirtschaft im 21. Jahrhundert Schüler*innen Workshop an der JKU 2024-02-06
	\item Kapeller, Jakob  Pluralism in Economics unbekannt/unknown 2024-02-01
	\item Gräbner-Radkowitsch, Claudius  Agentenbasiertes Modellieren und Komplexitätsökonomik unbekannt/unknown 2024-01-24
	\item Gräbner-Radkowitsch, Claudius  Pluralismus und Komplexität: Ein Gegenentwurf zur neoklassischen Gleichgewichtsökonomik? unbekannt/unknown 2024-01-19
	\item Rath, Johanna  Short-term Employment and Social Stratification in Austrian Academia ICAE Research Seminar 2023/24 2024-01-16
	\item Hornykewycz, Anna  Herausforderungen für eine Wirtschaft im 21. Jahrhundert Schüler*innen Workshop an der JKU 2023-12-19
	\item Rath, Johanna  Herausforderungen für eine Wirtschaft im 21. Jahrhundert Schüler*innen Workshop an der JKU 2023-12-19
	\item Pühringer, Stephan  Welche Wirtschaft(swissenschaft) braucht Sozialökologische Transformation? unbekannt/unknown 2023-12-14
	\item Pühringer, Stephan  Sozialökologische Transformation braucht andere Wirtschaft unbekannt/unknown 2023-12-14
	\item Kapeller, Jakob  Pluralismus in der Ökonomik unbekannt/unknown 2023-12-13
	\item Hirte, Katrin  Bodendaten und Bodenpolitik im Agrarbereich: Bodendaten (Reichsbodenschätzung), Bodennutzungsdaten (INVEKOS; BMEL und Testbetriebsnetz) sowie Bodenbelastung (BMVU) unbekannt/unknown 2023-12-13
	\item Kapeller, Jakob  Pluralism in Economics unbekannt/unknown 2023-12-01
	\item Pühringer, Stephan  Wissensproduktion und Publikationskulturen in der “Wettbewerbswissenschaft” Konferenz Wissenschaftliche Publikationskulturen im Zeitalter von Open Access 2023-11-30
	\item Pühringer, Stephan  Unternehmenslobbys drängen immer stärker in die Schulen. Expert*innen diskutierten über eine zukunftsfähige Wirtschaftsbildung unbekannt/unknown 2023-11-24
	\item Kapeller, Jakob  Kurt Rothschild Preisverleihung 2022 unbekannt/unknown 2023-11-23
	\item Aistleitner, Matthias  A world full of nails: Über die eigenwillige (Nicht-)Thematisierung Sozial-Ökologischer Transformation in den Wirtschaftswissenschaften unbekannt/unknown 2023-11-21
	\item Rath, Johanna  Kapitalismus und Demokratie im Wandel: Marx, Polanyi und Schumpeter im Dialog Entwicklungspolitische Hochschulwochen Linz 2023 2023-11-20
	\item Aistleitner, Matthias  Beschäftigungspolitische Implikationen einer sozial-ökologischen Transformation am Beispiel der Fahrzeuglieferindustrie in OÖ unbekannt/unknown 2023-11-20
	\item Kapeller, Jakob  Modeling power laws in economics? unbekannt/unknown 2023-11-15
	\item Pühringer, Stephan  Konsequenzen der Wettbewerbsorientierung von und an Universitäten unbekannt/unknown 2023-11-14
	\item Pühringer, Stephan  Verteilung von Reichtum, Finanz- und Wirtschaftsbildung sowie neoliberale Denkweisen FM4-Sendung \glqq Auf Laut\grqq{} 2023-11-08
	\item Porak, Laura  Implicit and explicit rules of competition unbekannt/unknown 2023-11-08
	\item Kapeller, Jakob  Why are there so many power laws in economics? AK Young Economists Conference 2024 2023-10-06
	\item Kapeller, Jakob  Heating up Houses instead of the Climate A transparent and realistic pathway for transforming the German housing sector unbekannt/unknown 2023-10-06
	\item Hornykewycz, Anna  Heating up Houses instead of the Climate A transparent and realistic pathway for transforming the German housing sector unbekannt/unknown 2023-10-06
	\item   ‘The charm of emission trading’: Ideas of German economists on economic policy in times of crises Annual Conference of the Political Economy Section of the German Association for Political Sciences 2023-09-25
	\item Porak, Laura  ‘The charm of emission trading’: Ideas of German economists on economic policy in times of crises Annual Conference of the Political Economy Section of the German Association for Political Sciences 2023-09-25
	\item Pühringer, Stephan  Konsequenzen der Wettbewerbsorientierung von und an Universitäten unbekannt/unknown 2023-09-23
	\item Rath, Johanna  Employment Relationships in the Age of Digitalisation: Analysing Policy Measures in an Agent-Based Framework EAEPE 2023 2023-09-14
	\item Hager, Theresa  Competing for sustainability? An institutionalist analysis of the new development model of the European Union unbekannt/unknown 2023-09-14
	\item Gräbner-Radkowitsch, Claudius  Competing for sustainability? An institutionalist analysis of the new development model of the European Union unbekannt/unknown 2023-09-14
	\item Hornykewycz, Anna  Competing for sustainability? An institutionalist analysis of the new development model of the European Union unbekannt/unknown 2023-09-14
	\item Kapeller, Jakob  Heating up houses instead of the climate: A transparent and realistic pathway for transforming the German housing sector unbekannt/unknown 2023-09-13
	\item Aistleitner, Matthias  What money can buy. Networks of the superriches in Austria unbekannt/unknown 2023-09-13
	\item Pühringer, Stephan  What money can buy. Networks of the superriches in Austria unbekannt/unknown 2023-09-13
	\item   ‘The charm of emission trading’: Ideas of German economists on economic policy in times of crises EAEPE Annual Conference 2023-09-12
	\item Porak, Laura  ‘The charm of emission trading’: Ideas of German economists on economic policy in times of crises EAEPE Annual Conference 2023-09-12
	\item Kapeller, Jakob  Pluralismus in der Ökonomik unbekannt/unknown 2023-09-09
	\item Kapeller, Jakob  Can a European wealth tax close the green investment gap? unbekannt/unknown 2023-09-08
	\item Ötsch, Walter  Wie funktioniert die Ökonomie als Wissenschaft? Radiosendung 2023-08-21
	\item Pühringer, Stephan  Wie funktioniert die Ökonomie als Wissenschaft? Radiosendung 2023-08-21
	\item Pühringer, Stephan  Netzwerke der Superreichen in Österreich unbekannt/unknown 2023-08-01
	\item Aistleitner, Matthias  Netzwerke der Superreichen in Österreich unbekannt/unknown 2023-08-01
	\item Pühringer, Stephan  Denkanstöße liefern -- Interview mit START-Preisträger Stephan Pühringer unbekannt/unknown 2023-07-28
	\item Pühringer, Stephan  Gendered Competitive Practices in Economics. A Multi-Layer Model of Women’s Underrepresentation 35th Annual SASE Conference, Rio de Janeiro 2023-07-11
	\item Hager, Theresa  Gendered Competitive Practices in Economics. A Multi-Layer Model of Women’s Underrepresentation 35th Annual SASE Conference, Rio de Janeiro 2023-07-11
	\item Hirte, Katrin  Agrarische sozio-ökonomische Transformation und Transformationswissen unbekannt/unknown 2023-07-04
	\item Gräbner-Radkowitsch, Claudius  Competing for Sustainability? An institutionalist analysis of the new development model of the EU unbekannt/unknown 2023-06-27
	\item Hager, Theresa  Competing for Sustainability? An institutionalist analysis of the new development model of the EU unbekannt/unknown 2023-06-27
	\item Hornykewycz, Anna  Competing for Sustainability? An institutionalist analysis of the new development model of the EU unbekannt/unknown 2023-06-27
	\item Rath, Johanna  Ranking the best of Research? Values and Valuation in Science Workshop „Value and Valuation“ 2023-06-23
	\item Hager, Theresa  Gendered Competitive Practices in Economics. A Multi-Layer Model of Women’s Underrepresentation Value and Valuation. Challenges in Political Economy Analysis. 2023-06-23
	\item Pühringer, Stephan  Gendered Competitive Practices in Economics. A Multi-Layer Model of Women’s Underrepresentation Value and Valuation. Challenges in Political Economy Analysis. 2023-06-23
	\item Hirte, Katrin  Zur aktuellen agrarischen Transformationsforschung – Profile, Zugänge, Kernauffassungen unbekannt/unknown 2023-06-21
	\item Rath, Johanna  Herausforderungen für eine Wirtschaft im 21. Jahrhundert Schüler*innen Workshop an der JKU 2023-06-20
	\item Hornykewycz, Anna  Herausforderungen für eine Wirtschaft im 21. Jahrhundert Schüler*innen Workshop an der JKU 2023-06-20
	\item Pühringer, Stephan  Sustainable Socio-Economic Transition and Economic Reasoning unbekannt/unknown 2023-06-16
	\item Porak, Laura  Wettbewerbsfähige Nachhaltigkeit: Eine Historisch-Materialistische Analyse der Europäischen grünen Transformation unbekannt/unknown 2023-05-09
	\item Pühringer, Stephan  Gendered Competitive Practices in Economics A Multi-Layer Model of Women’s Underrepresentation 8. Österreichischer Workshop Feministischer Ökonom*innen (FemÖk) 2023-05-05
	\item Hager, Theresa  Gendered Competitive Practices in Economics A Multi-Layer Model of Women’s Underrepresentation 8. Österreichischer Workshop Feministischer Ökonom*innen (FemÖk) 2023-05-05
	\item Kapeller, Jakob  Wirtschaft anders denken? – Der Beitrag einer pluralen Ökonomie unbekannt/unknown 2023-05-04
	\item Hirte, Katrin  Eliten und Demokratie — zur Etablierung der funktionalistischen Elite- und Demokratieauffassung nach 1945 und die Rolle von C. J. Friedrich unbekannt/unknown 2023-05-02
	\item Pühringer, Stephan  Exzellent prekär unbekannt/unknown 2023-04-22
	\item Hager, Theresa  NUWiss vor Ort. Die lokalen Unterbauinitiativen: Unterbau Linz unbekannt/unknown 2023-04-22
	\item Hirte, Katrin  Elitetheorien und Demokratieverständnis. Auffassungen und Vorgänge vor und nach 1945 unbekannt/unknown 2023-04-18
	\item Hornykewycz, Anna  Herausforderungen für eine Wirtschaft im 21. Jahrhundert Schüler*innen Workshop an der JKU 2023-04-17
	\item Hager, Theresa  Herausforderungen für eine Wirtschaft im 21. Jahrhundert Schüler*innen Workshop an der JKU 2023-04-17
	\item Rath, Johanna  Herausforderungen für eine Wirtschaft im 21. Jahrhundert Schüler*innen Workshop an der JKU 2023-03-30
	\item Theine, Hendrik  Catering for the interests of the richest or staunch fighters for social justice? A long-term perspective on media discourses of economic inequality in the US press unbekannt/unknown 2023-03-28
	\item   Backlash or Progress? -- Exploring the long-time effect of Covid-policies on the division of care work unbekannt/unknown 2023-03-21
	\item Pühringer, Stephan  Wettbewerb in der Wissenschaft unbekannt/unknown 2023-03-16
	\item Pühringer, Stephan  Wissenschaft und Wettbewerb unbekannt/unknown 2023-03-16
	\item Pühringer, Stephan  What economic education is missing: the real world unbekannt/unknown 2023-03-07
	\item Pühringer, Stephan  Power and Influence of Economists: Contributions to the Social Studies of Economics unbekannt/unknown 2023-02-09
	\item   Power and Influence of Economists: Contributions to the Social Studies of Economics unbekannt/unknown 2023-02-09
	\item   Power and Influence of Economists: Contributions to the Social Studies of Economics unbekannt/unknown 2023-02-09
	\item Kapeller, Jakob  Rethinking the economy – The contribution of pluralist economics unbekannt/unknown 2023-02-06
	\item Aistleitner, Matthias  Forschung über Wissenschaft unbekannt/unknown 2023-02-06
	\item Kapeller, Jakob  Wirtschaft anders denken! Was ist plurale Ökonomik? unbekannt/unknown 2023-02-05
	\item Pühringer, Stephan  Competitization and Quantification in academic Economics: Origins, Evolution and Implications unbekannt/unknown 2023-01-31
	\item Gräbner-Radkowitsch, Claudius  Competitization and Quantification in academic Economics: Origins, Evolution and Implications unbekannt/unknown 2023-01-31
	\item Gräbner-Radkowitsch, Claudius  Agentenbasierte Modelle \& Komplexitätsökonomik unbekannt/unknown 2023-01-26
	\item Ferschli, Benjamin  Underdemand or undersupply of labour? Automation and employment in the Austrian automotive sector. A value chain perspective unbekannt/unknown 2023-01-24
	\item Gräbner-Radkowitsch, Claudius  Agentenbasierte Modellierung und Komplexitätsökonomik unbekannt/unknown 2023-01-23
	\item Hirte, Katrin  Geschichte der Wirtschaftspolitik unbekannt/unknown 2023-01-19
	\item   Supply, Demand, and Preferences: Determinants of Meat Consumption in Spain: 1964 – 2019 unbekannt/unknown 2023-01-17
	\item Pühringer, Stephan  Konsequenzen der Wettbewerbsorientierung von und an Universitäten. Eine Perspektive der interdisziplinären Wettbewerbsforschung unbekannt/unknown 2023-01-12
	\item Gräbner-Radkowitsch, Claudius  Degrowth and the Global South? How institutionalism can complement a timely discourse on ecologically sustainable development in an unequal world ASSA, New Orleans 2023-01-11
	\item Gräbner-Radkowitsch, Claudius  Competing for sustainability? An institutionalist analysis of the new development model of the European Union ICAPE 2023 Conference 2023-01-05
	\item Hornykewycz, Anna  Competing for sustainability? An institutionalist analysis of the new development model of the European Union ICAPE 2023 Conference 2023-01-05
	\item Hager, Theresa  Competing for sustainability? An institutionalist analysis of the new development model of the European Union ICAPE 2023 Conference 2023-01-05
	\item Kapeller, Jakob  Wechselseitige Kritik ist nur möglich mit nachvollziehbarer Forschung unbekannt/unknown 2022-12-29
	\item Pühringer, Stephan  Prekäre Wissenschaft: Warum Unis reformiert werden müssen unbekannt/unknown 2022-12-21
	\item Hager, Theresa  Bane or Boon: On Technological Competitiveness and Joining the European Union Open Research Seminar ICAE 2022-12-20
	\item   How public economists in Germany are thinking about the economy, economic policy, and economics unbekannt/unknown 2022-12-13
	\item Porak, Laura  How public economists in Germany are thinking about the economy, economic policy, and economics unbekannt/unknown 2022-12-13
	\item Rath, Johanna  Battle over Indicators or a battle of ideas? unbekannt/unknown 2022-12-06
	\item Pühringer, Stephan  Battle over Indicators or a battle of ideas? unbekannt/unknown 2022-12-06
	\item Porak, Laura  Battle over Indicators or a battle of ideas? unbekannt/unknown 2022-12-06
	\item Hager, Theresa  Battle over Indicators or a battle of ideas? unbekannt/unknown 2022-12-06
	\item Aistleitner, Matthias  Battle over Indicators or a battle of ideas? unbekannt/unknown 2022-12-06
	\item Hirte, Katrin  Die deutsche universitäre Agrarsoziologie als Wissenschaftsdisziplin unbekannt/unknown 2022-12-06
	\item   Competition Universalism und einige Zusammenhänge zwischen Wettbewerb und Raum unbekannt/unknown 2022-11-29
	\item Pühringer, Stephan  Competition Universalism und einige Zusammenhänge zwischen Wettbewerb und Raum unbekannt/unknown 2022-11-29
	\item   Don’t fear the reaper — Tax policy frames in German print media unbekannt/unknown 2022-11-29
	\item Hornykewycz, Anna  Arbeitsbeziehungen im Wandel. Der Effekt von Digitalisierung auf die Ausgestaltung von Arbeitsverträgen und Kooperation Technikfolgenabschätzung aus Arbeitnehmer:innenperspektive 2022-11-25
	\item Rath, Johanna  Arbeitsbeziehungen im Wandel. Der Effekt von Digitalisierung auf die Ausgestaltung von Arbeitsverträgen und Kooperation Technikfolgenabschätzung aus Arbeitnehmer:innenperspektive 2022-11-25
	\item Pühringer, Stephan  Konsequenzen der Wettbewerbsorientierung von und an Universitäten unbekannt/unknown 2022-11-24
	\item   The implementation of policies for excellence in French higher education unbekannt/unknown 2022-11-22
	\item Gräbner-Radkowitsch, Claudius  Komplexitätsökonomik unbekannt/unknown 2022-11-17
	\item Pühringer, Stephan  Konsequenzen der Wettbewerbsorientierung von und an Universitäten unbekannt/unknown 2022-11-16
	\item Schnitzer, Hendrik  Criticizing Economics for its Lack of Realism? unbekannt/unknown 2022-11-15
	\item Rath, Johanna  The Translation of Social Relations into Computer-Based Models: New Possibilities and Familiar Limitations Gastvortrag MA \glqq digital society", LV "Interdisziplinäres Projekt: Digitalisierung interdisziplinär\grqq{} (LVA-Nr. 232477 2022W) 2022-11-14
	\item Hornykewycz, Anna  The Translation of Social Relations into Computer-Based Models: New Possibilities and Familiar Limitations Gastvortrag MA \glqq digital society", LV "Interdisziplinäres Projekt: Digitalisierung interdisziplinär\grqq{} (LVA-Nr. 232477 2022W) 2022-11-14
	\item Pühringer, Stephan  Verzerrte Expertise. Zur (Frei)handelsdebatte in den Wirtschaftswissenschaften Entwicklungstagung 2022: Ungleiche Möglichkeiten 2022-11-12
	\item Hager, Theresa  Verzerrte Expertise. Zur (Frei)handelsdebatte in den Wirtschaftswissenschaften Entwicklungstagung 2022: Ungleiche Möglichkeiten 2022-11-12
	\item Hornykewycz, Anna  Employment Relationships in the Age of Digitalization unbekannt/unknown 2022-11-08
	\item Gräbner-Radkowitsch, Claudius  Kompexitätsökonomik unbekannt/unknown 2022-11-03
	\item Kapeller, Jakob  Pluralismus in der Ökonomie (und die Klimafrage) unbekannt/unknown 2022-11-02
	\item Hirte, Katrin  Verschweigen als gewolltes Nichtwissen in der Wissenschaft: Strategien in Umbruchzeiten unbekannt/unknown 2022-10-26
	\item Pühringer, Stephan  Netzwerke der Superreichen in Österreich unbekannt/unknown 2022-10-25
	\item Aistleitner, Matthias  Netzwerke der Superreichen in Österreich unbekannt/unknown 2022-10-25
	\item Kapeller, Jakob  Pluralismus in der Ökonomie: Philosophische Perspektiven unbekannt/unknown 2022-10-25
	\item Gräbner-Radkowitsch, Claudius  Was ist plurale Ökonomik? unbekannt/unknown 2022-10-19
	\item   European Cohesion Policy in Times of Austerity: A Regional Analysis of Germany and Portugal unbekannt/unknown 2022-10-18
	\item Pühringer, Stephan  Neoliberale Think Tanks und ihre Verbindungen zu Reichen-Netzwerken 4. Österreichische Reichtumskonferenz 2022-10-17
	\item Hornykewycz, Anna  Neoliberale Think Tanks und ihre Verbindungen zu Reichen-Netzwerken 4. Österreichische Reichtumskonferenz 2022-10-17
	\item Hager, Theresa  Neoliberale Think Tanks und ihre Verbindungen zu Reichen-Netzwerken 4. Österreichische Reichtumskonferenz 2022-10-17
	\item Aistleitner, Matthias  Ökonomie: Enge Verflechtung verzerrt den Wettbewerb der Ideen unbekannt/unknown 2022-10-14
	\item Kapeller, Jakob  Socio-ecological transition for the sustainable development of well-being unbekannt/unknown 2022-10-13
	\item Pühringer, Stephan  Networks of the super-rich in Austria Young Economists Conference 2022 2022-10-10
	\item Griesebner, Teresa  Networks of the super-rich in Austria Young Economists Conference 2022 2022-10-10
	\item Aistleitner, Matthias  Networks of the super-rich in Austria Young Economists Conference 2022 2022-10-10
	\item Gräbner-Radkowitsch, Claudius  Bane or Boon: On Technological Competitiveness and Joining the European Union Young Economists Conference 2022 2022-10-09
	\item Hager, Theresa  Bane or Boon: On Technological Competitiveness and Joining the European Union Young Economists Conference 2022 2022-10-09
	\item Rath, Johanna  The political economy of academic publishing: On the commodification of a public good Young Economist Conference 2022 2022-10-07
	\item Kapeller, Jakob  Wirtschaft anders denken! Was ist plurale Ökonomik? unbekannt/unknown 2022-09-30
	\item Gräbner-Radkowitsch, Claudius  Applied economic methodology: oxymoron or ideal? unbekannt/unknown 2022-09-29
	\item Gräbner-Radkowitsch, Claudius  The Political Economy of Measuring Competitiveness Mission Competition. Transdisziplinäre Perspektiven auf Wettbewerbsgesellschaften. 2022-09-28
	\item Hager, Theresa  The Political Economy of Measuring Competitiveness Mission Competition. Transdisziplinäre Perspektiven auf Wettbewerbsgesellschaften. 2022-09-28
	\item Pühringer, Stephan  Wettbewerb in akademischen, politischen und öffentlichen Diskursen unbekannt/unknown 2022-09-28
	\item Rath, Johanna  Wettbewerb und Wettbewerbsfähigkeit im öffentlichen, politischen und akademischen Diskurs unbekannt/unknown 2022-09-28
	\item Porak, Laura  Wettbewerb und Wettbewerbsfähigkeit im öffentlichen, politischen und akademischen Diskurs unbekannt/unknown 2022-09-28
	\item Pühringer, Stephan  Wettbewerb und Wettbewerbsfähigkeit im öffentlichen, politischen und akademischen Diskurs unbekannt/unknown 2022-09-28
	\item Aistleitner, Matthias  Wettbewerb und Wettbewerbsfähigkeit im öffentlichen, politischen und akademischen Diskurs unbekannt/unknown 2022-09-28
	\item Pühringer, Stephan  Wettbewerb in der Wissenschaft: Ambivalenzen und (un-)in-tendierte Folgen unbekannt/unknown 2022-09-28
	\item Gräbner-Radkowitsch, Claudius  Curse or Blessing: On Technological Competitiveness and Joining the European Union 34th Annual EAEPE Conference, Neapel 2022-09-09
	\item Hager, Theresa  Curse or Blessing: On Technological Competitiveness and Joining the European Union 34th Annual EAEPE Conference, Neapel 2022-09-09
	\item Rath, Johanna  EU Policy-making and the construction of economic indicators: the case of Competitiveness in the European Semester 34th Annual EAEPE Conference 2022 2022-09-08
	\item Hornykewycz, Anna  Modelling the Demand for Heterogeneous Consumption Products. A Macroeconomic Stock-Flow Consistent Agent-Based Approach 34th Annual EAEPE Conference, Neapel 2022-09-08
	\item Pühringer, Stephan  Networks of the super-rich in Austria 34th Annual EAEPE Conference 2022-09-08
	\item Griesebner, Teresa  Networks of the super-rich in Austria 34th Annual EAEPE Conference 2022-09-08
	\item Aistleitner, Matthias  Networks of the super-rich in Austria 34th Annual EAEPE Conference 2022-09-08
	\item Rath, Johanna  EU Policy-making and the construction of economic indicators: the case of Competitiveness in the European Semester 34th Annual EAEPE Conference, Neapel 2022-09-08
	\item Pühringer, Stephan  Competitive Performativity of Academic Social Networks. The Subjectification of Competition on Researchgate, Twitter and Google Scholar 34th Annual EAEPE Conference, Neapel 2022-09-07
	\item Gräbner-Radkowitsch, Claudius  The heterogeneous effects of digitalization on macroeconomic developments -- a comparative perspective 34th Annual EAEPE Conference, Neapel 2022-09-06
	\item Pühringer, Stephan  Competitive Performativity of Academic Social Networks. The Subjectification of Competition on Researchgate, Twitter and Google Scholar Workshop „Rankings and the structure of Economics”, JKU Linz 2022-07-21
	\item Pühringer, Stephan  Reichennetzwerke in Österreich unbekannt/unknown 2022-07-14
	\item Gräbner-Radkowitsch, Claudius  The Toll it Takes: On Technological Competitiveness and Joining the European Union 34th Annual SASE Conference, Amsterdam 2022-07-11
	\item Hager, Theresa  The Toll it Takes: On Technological Competitiveness and Joining the European Union 34th Annual SASE Conference, Amsterdam 2022-07-11
	\item Aistleitner, Matthias  Development and Interdisciplinarity: re-examining the \glqq economics silo\grqq{} 34th Annual SASE Conference, Amsterdam 2022-07-11
	\item Gräbner-Radkowitsch, Claudius  Technological capabilities, globalisation and economic growth 34th Annual SASE Conference, Amsterdam 2022-07-11
	\item Heimberger, Philipp  Technological capabilities, globalisation and economic growth 34th Annual SASE Conference, Amsterdam 2022-07-11
	\item Kapeller, Jakob  Technological capabilities, globalisation and economic growth 34th Annual SASE Conference, Amsterdam 2022-07-11
	\item Rath, Johanna  Shaping Employment Relationships in the Age of Digitalisation: Analysing Policy Measures in an Agent-Based Framework 34th Annual SASE Conference, Amsterdam 2022-07-11
	\item Hornykewycz, Anna  Shaping Employment Relationships in the Age of Digitalisation: Analysing Policy Measures in an Agent-Based Framework 34th Annual SASE Conference, Amsterdam 2022-07-11
	\item Kapeller, Jakob  Can the European Green Deal Contribute to a Socio-Ecological Transformation? 34th Annual SASE Conference, Amsterdam 2022-07-10
	\item   Can the European Green Deal Contribute to a Socio-Ecological Transformation? 34th Annual SASE Conference, Amsterdam 2022-07-10
	\item Pühringer, Stephan  Winning urban competition with a social agenda. The competition imaginary in Viennese urban development plans 34th Annual SASE Conference, Amsterdam 2022-07-10
	\item   Opinion Dynamics under Conflicting Interests unbekannt/unknown 2022-06-30
	\item Hager, Theresa  Herausforderungen für eine Wirtschaft im 21. Jahrhundert Schüler*innen Workshop an der JKU 2022-06-29
	\item Hornykewycz, Anna  Herausforderungen für eine Wirtschaft im 21. Jahrhundert Schüler*innen Workshop an der JKU 2022-06-29
	\item Gräbner-Radkowitsch, Claudius  MAN Steyr und der internationale Wettbewerb unbekannt/unknown 2022-06-26
	\item Kapeller, Jakob  MAN Steyr und der internationale Wettbewerb unbekannt/unknown 2022-06-26
	\item Pühringer, Stephan  Konsequenzen der Wettbewerbsorientierung von und an Universitäten unbekannt/unknown 2022-06-25
	\item   Dedicated Innovation Systems for the Transformation towards Sustainability unbekannt/unknown 2022-06-23
	\item Hirte, Katrin  Performativity: Vom Entstehen von agrarischen Verarbeitungsstrukturen am Beispiel Schlachthöfe unbekannt/unknown 2022-06-15
	\item Hirte, Katrin  Agrarpolitik und Agrarökonomie unbekannt/unknown 2022-06-14
	\item   The Performativity of Competitiveness in Academia unbekannt/unknown 2022-06-09
	\item Pühringer, Stephan  The Performativity of Competitiveness in Academia unbekannt/unknown 2022-06-09
	\item Kapeller, Jakob  Pluralismus in der Ökonomie: Philosophische Perspektiven unbekannt/unknown 2022-06-09
	\item Kapeller, Jakob  Interview zu: Ist die Preisexplosion begründet unbekannt/unknown 2022-06-08
	\item   EU-Vermögenssteuer für ein grünes und gerechtes Europa Diskurs.Das Wissenschaftlernetz 2022-06-02
	\item Kapeller, Jakob  EU-Vermögenssteuer für ein grünes und gerechtes Europa Diskurs.Das Wissenschaftlernetz 2022-06-02
	\item Porak, Laura  Re-Inventing Competitiveness — tracing the development of competitiveness as European governmental problem unbekannt/unknown 2022-06-02
	\item Gräbner-Radkowitsch, Claudius  Agent Based Modelling unbekannt/unknown 2022-05-27
	\item Kapeller, Jakob  Das Menschenbild moderner Ökonomie unbekannt/unknown 2022-05-25
	\item Pühringer, Stephan  The political economy of academic publishing: On the commodification of a public good unbekannt/unknown 2022-05-13
	\item Rath, Johanna  The political economy of academic publishing: On the commodification of a public good unbekannt/unknown 2022-05-13
	\item Aistleitner, Matthias  Development and Interdisciplinarity: a comment on Mitra et al. (2020) unbekannt/unknown 2022-05-12
	\item Theine, Hendrik  Klimakrise, klimasoziale Politik und die Rolle der Superreichen unbekannt/unknown 2022-05-05
	\item Kapeller, Jakob  Konzernmacht in globalen Güterketten unbekannt/unknown 2022-05-03
	\item Gräbner-Radkowitsch, Claudius  Liegt die Wahrheit irgendwo dazwischen? Eine plurale Perspektive auf globale Ungleichheit unbekannt/unknown 2022-05-02
	\item Pühringer, Stephan  Universitäten und Forschung im Wettbewerb unbekannt/unknown 2022-04-29
	\item Hornykewycz, Anna  Introducing heterogeneous consumer products into a single-country, SFC macro-ABM framework unbekannt/unknown 2022-04-28
	\item Bäuerle, Lukas  Wirtschaft als transformativer Prozess. Grundlagen und Implikationen für die Wirtschaftswissenschaften unbekannt/unknown 2022-04-21
	\item Hirte, Katrin  Euphemismen in der Ökonomik: Das Beispiel „Arbeitgeber“ – Herkunft und Etablierung eines Begriffs Open Research Seminar, ICAE 2022-04-07
	\item Kapeller, Jakob  Dilemmata marktliberaler Globalisierung Globalisierungs- und Wachstumsgrenzen 2022-03-09
	\item Pühringer, Stephan  Machtstrukturen in der Ökonomie und der performative Einfluss auf Politik und Gesellschaft unbekannt/unknown 2022-03-01
	\item   Konfliktlinien in den Wirtschaftswissenschaften und ihre Bedeutung für ‚Bildung und nachhaltige Entwicklung‘ (BNE) unbekannt/unknown 2022-02-24
	\item Kapeller, Jakob  Pluralism in economics (\& the role of heterodoxy) unbekannt/unknown 2022-02-07
	\item   Think-Tanks: Neue Akteure im politischen Diskurs unbekannt/unknown 2022-02-02
	\item Pühringer, Stephan  Think-Tanks: Neue Akteure im politischen Diskurs unbekannt/unknown 2022-02-02
	\item Pühringer, Stephan  Konsequenzen der Wettbewerbsorientierung von und an Universitäten: Eine Perspektive der interdisziplinären Wettbewerbsforschung Tagung des Zentralausschusses Universitätslehrerinnen und Universitätslehrer beim Bundesministerium für Bildung, Wissenschaft und Forschung 2022-01-28
	\item Eder, Julia  Mit neuen Eigentumsformen durch die sozial-ökologische Transformation? Die europäische Industrie zukunftsfit machen unbekannt/unknown 2022-01-27
	\item   Modern Central Banks: Navigating between 'the stars' and a state-led growth model unbekannt/unknown 2022-01-13
	\item   Competition in Transitional Processes: Polanyi \& Schumpeter Pluralumni Netzwerk 2022-01-10
	\item Hager, Theresa  Competition in Transitional Processes: Polanyi \& Schumpeter Pluralumni Netzwerk 2022-01-10
	\item Rath, Johanna  Competition in Transitional Processes: Polanyi \& Schumpeter Pluralumni Netzwerk 2022-01-10
	\item Hirte, Katrin  Geschichte des Neoliberalismus unbekannt/unknown 2021-12-28
	\item Hager, Theresa  Lobbyismus und gesamtwirtschaftliche Entwicklung – ein Literaturüberblick zu Mancur Olsons Theorie Open Research Seminar ICAE 2021-12-09
	\item Gräbner-Radkowitsch, Claudius  Komplexitätsökonomik – eine Einführung unbekannt/unknown 2021-12-03
	\item Rath, Johanna  Shaping Sustainable Employment Relationships in the Age of Digitalisation: Analysing Policy Measures in an Agent-Based Framework unbekannt/unknown 2021-12-02
	\item Hornykewycz, Anna  Shaping Sustainable Employment Relationships in the Age of Digitalisation: Analysing Policy Measures in an Agent-Based Framework unbekannt/unknown 2021-12-02
	\item Pühringer, Stephan  Sozialer Schein, aber wirtschaftsliberale Positionen unbekannt/unknown 2021-12-02
	\item Pühringer, Stephan  Handelsnarrative in der ökonomischen Debatte Entwicklungspolitische Hochschulwochen 2021-11-30
	\item Gräbner-Radkowitsch, Claudius  Freihandelsdebatten in der Ökonomie unbekannt/unknown 2021-11-30
	\item Pühringer, Stephan  Freihandelsdebatten in der Ökonomie unbekannt/unknown 2021-11-30
	\item Heimberger, Philipp  The political importance of technical details in the European Semester unbekannt/unknown 2021-11-29
	\item Heimberger, Philipp  Reforming the EU’s fiscal rules: Output gap estimates and fiscal policy in Belgium and beyond unbekannt/unknown 2021-11-26
	\item Heimberger, Philipp  Fiscal policy and the Covid-19 crisis unbekannt/unknown 2021-11-24
	\item Aistleitner, Matthias  The industrial policy research discourse (2000-2020) – a topic modelling approach Young Economists Conference 2021: Industrial policy for sustainable work and a green transformation 2021-11-22
	\item Kapeller, Jakob  Der Soziologe Robert K. Merton: Alles misslingt nach Plan unbekannt/unknown 2021-11-19
	\item Schütz, Bernhard  The Emergence of Debt and Secular Stagnation in an unequal Society: A stock-flow consistent agent-based approach unbekannt/unknown 2021-11-18
	\item Hornykewycz, Anna  The Emergence of Debt and Secular Stagnation in an unequal Society: A stock-flow consistent agent-based approach unbekannt/unknown 2021-11-18
	\item Gräbner-Radkowitsch, Claudius  The Emergence of Debt and Secular Stagnation in an unequal Society: A stock-flow consistent agent-based approach unbekannt/unknown 2021-11-18
	\item Kapeller, Jakob  The micro-macro link in heterodox economics unbekannt/unknown 2021-11-17
	\item Gräbner-Radkowitsch, Claudius  Komplexitätsökonomik – Alternative zur Gleichgewichtsökonomik? unbekannt/unknown 2021-11-17
	\item Porak, Laura  Governing the Ungovernable. Recontextualizations of Competition in European Policy Discourse Young Economists Conference 2021 2021-11-07
	\item Rath, Johanna  Shaping Sustainable Employment Relationships in the Age of Digitalisation: Analysing Policy Measures in an Agent-Based Framework Young Economists Conference 2021 2021-11-06
	\item Hornykewycz, Anna  Shaping Sustainable Employment Relationships in the Age of Digitalisation: Analysing Policy Measures in an Agent-Based Framework Young Economists Conference 2021 2021-11-06
	\item Gräbner-Radkowitsch, Claudius  (Mis)Measuring Competitiveness: The Quantification of a Malleable Concept in the European Semester Young Economists Conference 2021 2021-11-06
	\item Hager, Theresa  (Mis)Measuring Competitiveness: The Quantification of a Malleable Concept in the European Semester Young Economists Conference 2021 2021-11-06
	\item Porak, Laura  Kulturelle Wachstumskritik unbekannt/unknown 2021-11-02
	\item Heimberger, Philipp  Do corporate tax cuts boost growth? unbekannt/unknown 2021-10-30
	\item Hornykewycz, Anna  The Emergence of Debt and Secular Stagnation in an unequal Society: A stock-flow consistent agent-based Approach 25th FMM Conference 2021 2021-10-29
	\item Gräbner-Radkowitsch, Claudius  The Emergence of Debt and Secular Stagnation in an unequal Society: A stock-flow consistent agent-based Approach 25th FMM Conference 2021 2021-10-29
	\item Schütz, Bernhard  The Emergence of Debt and Secular Stagnation in an unequal Society: A stock-flow consistent agent-based Approach 25th FMM Conference 2021 2021-10-29
	\item Schütz, Bernhard  The evolution of debtor-creditor relationships within a monetary union:
 Trade imbalances, excess reserves and economic policy 25th FMM Conference 2021-10-29
	\item Landesmann, Michael  The evolution of debtor-creditor relationships within a monetary union:
 Trade imbalances, excess reserves and economic policy 25th FMM Conference 2021-10-29
	\item Kapeller, Jakob  The evolution of debtor-creditor relationships within a monetary union:
 Trade imbalances, excess reserves and economic policy 25th FMM Conference 2021-10-29
	\item Heimberger, Philipp  The evolution of debtor-creditor relationships within a monetary union:
 Trade imbalances, excess reserves and economic policy 25th FMM Conference 2021-10-29
	\item Gräbner-Radkowitsch, Claudius  The evolution of debtor-creditor relationships within a monetary union:
 Trade imbalances, excess reserves and economic policy 25th FMM Conference 2021-10-29
	\item Kapeller, Jakob  What is heterodox economics? unbekannt/unknown 2021-10-27
	\item Pühringer, Stephan  Trade narratives of economists: A CDA approach Workshop: Discourse Analysis in Economics 2021-10-21
	\item Hornykewycz, Anna  „Hinter jeder erfolgreichen Frau steht ein Mann, der ihr den Rücken stärkt.“ Momentum-Kongress 2021-10-16
	\item Porak, Laura  „Hinter jeder erfolgreichen Frau steht ein Mann, der ihr den Rücken stärkt.“ Momentum-Kongress 2021-10-16
	\item Rath, Johanna  „Hinter jeder erfolgreichen Frau steht ein Mann, der ihr den Rücken stärkt.“ Momentum-Kongress 2021-10-16
	\item   „Hinter jeder erfolgreichen Frau steht ein Mann, der ihr den Rücken stärkt.“ Momentum-Kongress 2021-10-16
	\item Hager, Theresa  „Hinter jeder erfolgreichen Frau steht ein Mann, der ihr den Rücken stärkt.“ Momentum-Kongress 2021-10-16
	\item Kapeller, Jakob  (How) can heterodox economics inform the study of institutions? Workshop \glqq Institutions\grqq{} 2021-10-16
	\item Hornykewycz, Anna  Shaping Sustainable Employment Relationships in the Age of Digitalisation: Analysing Policy Measures in an Agent-Based Framework Momentum2021: Arbeit | Momentum Kongress 2021-10-16
	\item Rath, Johanna  Shaping Sustainable Employment Relationships in the Age of Digitalisation: Analysing Policy Measures in an Agent-Based Framework Momentum2021: Arbeit | Momentum Kongress 2021-10-16
	\item Beyer, Karl  \glqq Für die „Leistungsträger“ und „uns Österreicher“: Eine Mediendiskursanalyse zu Sozialreformen der ÖVP/FPÖ-Regierung 2017-2019 in Österreich\grqq{} Momentum: Arbeit 2021-10-16
	\item Pühringer, Stephan  \glqq Für die „Leistungsträger“ und „uns Österreicher“: Eine Mediendiskursanalyse zu Sozialreformen der ÖVP/FPÖ-Regierung 2017-2019 in Österreich\grqq{} Momentum: Arbeit 2021-10-16
	\item Porak, Laura  Warum müssen wir noch immer arbeiten? unbekannt/unknown 2021-10-16
	\item Hornykewycz, Anna  The Emergence of Debt and Secular Stagnation in an unequal Society: A stock-flow consistent agent-based Approach EAEPE Annual Conference 2021-10-15
	\item Schütz, Bernhard  The Emergence of Debt and Secular Stagnation in an unequal Society: A stock-flow consistent agent-based Approach EAEPE Annual Conference 2021-10-15
	\item Gräbner-Radkowitsch, Claudius  The Emergence of Debt and Secular Stagnation in an unequal Society: A stock-flow consistent agent-based Approach EAEPE Annual Conference 2021-10-15
	\item Heimberger, Philipp  Budgetpolitik im Kontext der Covid-19-Krise unbekannt/unknown 2021-10-12
	\item Heimberger, Philipp  Budgetpolitik im Kontext der Covid-19-Krise unbekannt/unknown 2021-10-12
	\item Kapeller, Jakob  Economic Polarization in Europe Six Decades of the Past, Six Decades of the Future 2021-10-09
	\item   Competition in Transitional Processes: Polanyi and Schumpeter 24th Annual Conference of the European Society of the History of Economic Thought 2021-10-09
	\item Hager, Theresa  Competition in Transitional Processes: Polanyi and Schumpeter 24th Annual Conference of the European Society of the History of Economic Thought 2021-10-09
	\item Rath, Johanna  Competition in Transitional Processes: Polanyi and Schumpeter 24th Annual Conference of the European Society of the History of Economic Thought 2021-10-09
	\item Kapeller, Jakob  Structural Change in times of increasing openness Six Decades of the Past, Six Decades of the Future 2021-10-08
	\item Kapeller, Jakob  Pluralism in economics unbekannt/unknown 2021-10-07
	\item Aistleitner, Matthias  Power structures in economics and their impact on trade policies: An SSE approach unbekannt/unknown 2021-10-06
	\item Pühringer, Stephan  Power structures in economics and their impact on trade policies: An SSE approach unbekannt/unknown 2021-10-06
	\item Heimberger, Philipp  How to reform the EU’s fiscal rules? “Fiscal 2.0: A roadmap for the future?” 2021-09-30
	\item Kapeller, Jakob  A European Wealth Tax? Context, methods, revenues Conference “Fiscal Matters”, European Parliament 2021-09-28
	\item Heimberger, Philipp  Italy myths and the future of Europe unbekannt/unknown 2021-09-23
	\item Pühringer, Stephan  Neoliberale Denkmuster als Basis des degressiven Arbeitslosengeldes unbekannt/unknown 2021-09-18
	\item Kapeller, Jakob  A European Wealth Tax unbekannt/unknown 2021-09-07
	\item Gräbner-Radkowitsch, Claudius  The evolution of debtor-creditor relationships within a monetary union:
 Trade imbalances, excess reserves and economic policy EAEPE Annual Conference 2021-09-04
	\item Landesmann, Michael  The evolution of debtor-creditor relationships within a monetary union:
 Trade imbalances, excess reserves and economic policy EAEPE Annual Conference 2021-09-04
	\item Kapeller, Jakob  The evolution of debtor-creditor relationships within a monetary union:
 Trade imbalances, excess reserves and economic policy EAEPE Annual Conference 2021-09-04
	\item Heimberger, Philipp  The evolution of debtor-creditor relationships within a monetary union:
 Trade imbalances, excess reserves and economic policy EAEPE Annual Conference 2021-09-04
	\item   The Authors of Economics Journals Revisited: Evidence from a Large-Scale Replication of Hodgson \& Rothman (1999) 33rd Annual EAEPE Conference 2021-09-03
	\item Kapeller, Jakob  The Authors of Economics Journals Revisited: Evidence from a Large-Scale Replication of Hodgson \& Rothman (1999) 33rd Annual EAEPE Conference 2021-09-03
	\item Aistleitner, Matthias  The Authors of Economics Journals Revisited: Evidence from a Large-Scale Replication of Hodgson \& Rothman (1999) 33rd Annual EAEPE Conference 2021-09-03
	\item Rath, Johanna  The Political Economy of Academic Publishing: On the Commodification of a Public Good unbekannt/unknown 2021-09-03
	\item Aistleitner, Matthias  Top economists as free-trade cheerleaders in the public 33rd Annual EAEPE Conference 2021-09-03
	\item Pühringer, Stephan  Top economists as free-trade cheerleaders in the public 33rd Annual EAEPE Conference 2021-09-03
	\item Schütz, Bernhard  The Emergence of Debt and Secular Stagnation in an unequal Society: A stock-flow consistent agent-based Approach EAEPE Annual Conference 2021-09-03
	\item Hornykewycz, Anna  The Emergence of Debt and Secular Stagnation in an unequal Society: A stock-flow consistent agent-based Approach EAEPE Annual Conference 2021-09-03
	\item Gräbner-Radkowitsch, Claudius  The Emergence of Debt and Secular Stagnation in an unequal Society: A stock-flow consistent agent-based Approach EAEPE Annual Conference 2021-09-03
	\item Heimberger, Philipp  Still learning to catch-up? Technological capabilities, globalisation and economic growth 33 rd Annual EAEPE Conference 2021-09-03
	\item Porak, Laura  Governing the Ungovernable. Recontextualizations of Competition in European Policy Discourse 33rd Annual EAEPE Conference 2021-09-03
	\item Gräbner-Radkowitsch, Claudius  (Mis)Measuring Competitiveness: The Quantification of a Malleable Concept in the European Semester 33rd Annual EAEPE Conference 2021-09-03
	\item Hager, Theresa  (Mis)Measuring Competitiveness: The Quantification of a Malleable Concept in the European Semester 33rd Annual EAEPE Conference 2021-09-03
	\item Kapeller, Jakob  Tracing the invisible rich Conference of the European Economic Association 2021-08-25
	\item Kapeller, Jakob  Tracing the invisible rich: The rank correction approach to inferring tails in survey data Conference of the European Economic Association 2021-08-25
	\item Heimberger, Philipp  How should we deal with the budgetary costs due to the Covid-19 crisis? Austrian trade union federation 2021-08-25
	\item Heimberger, Philipp  Why the fiscal rules should be reformed: Output gap estimates and fiscal policy in Italy Event on the EU’s fiscal rules followed by a panel discussion with representatives of different parties from the Draghi government, Rome 2021-07-12
	\item   Competition in Transitional Processes: Polanyi \& Schumpeter Conference of the International Schumpeter Society 2021-07-09
	\item Rath, Johanna  Competition in Transitional Processes: Polanyi \& Schumpeter Conference of the International Schumpeter Society 2021-07-09
	\item Hager, Theresa  Competition in Transitional Processes: Polanyi \& Schumpeter Conference of the International Schumpeter Society 2021-07-09
	\item Kapeller, Jakob  Heterodox Economics and Economic Criticism unbekannt/unknown 2021-07-08
	\item Hager, Theresa  Wettbewerb in und rund um die Europäische Union unbekannt/unknown 2021-07-06
	\item Gräbner-Radkowitsch, Claudius  Wettbewerb in und rund um die Europäische Union unbekannt/unknown 2021-07-06
	\item Porak, Laura  Wettbewerb in und rund um die Europäische Union unbekannt/unknown 2021-07-06
	\item   Comments on “Can the state be a good investor?” Polish Economic Institute, Warsaw 2021-07-06
	\item Heimberger, Philipp  Comments on “Can the state be a good investor?” Polish Economic Institute, Warsaw 2021-07-06
	\item Kapeller, Jakob  Author Meets Critics: “Intangible Flow Theory in Economics: Human Participation in Economic and Societal Production” 33rd Annual SASE Meeting 2021-07-05
	\item Kapeller, Jakob  EU taxation capabilities and the way forward towards institutional progress in Europe unbekannt/unknown 2021-07-05
	\item   Theorizing Competition: An Interdisciplinary Framework 33rd Annual SASE Meeting 2021-07-04
	\item   Theorizing Competition: An Interdisciplinary Framework 33rd Annual SASE Meeting 2021-07-04
	\item Altreiter, Carina  Theorizing Competition: An Interdisciplinary Framework 33rd Annual SASE Meeting 2021-07-04
	\item Pühringer, Stephan  Theorizing Competition: An Interdisciplinary Framework 33rd Annual SASE Meeting 2021-07-04
	\item Gräbner-Radkowitsch, Claudius  Theorizing Competition: An Interdisciplinary Framework 33rd Annual SASE Meeting 2021-07-04
	\item Porak, Laura  Governing the Ungovernable. Recontextualizations of Competition in European Policy Discourse 33rd Annual SASE Meeting 2021-07-04
	\item Pühringer, Stephan  The political economy of academic publishing: On the commodification of a public good Annual Conference of the Society for the Advancement of Socio-Economics (SASE) 2021-07-03
	\item Rath, Johanna  The political economy of academic publishing: On the commodification of a public good Annual Conference of the Society for the Advancement of Socio-Economics (SASE) 2021-07-03
	\item Aistleitner, Matthias  The Authors of Economics Journals Revisited: Evidence from a Large-Scale Replication of Hodgson \& Rothman (1999) SASE 2021 Conference: After Covid? Critical Conjunctures and Contingent Pathways of Contemporary Capitalism 2021-07-02
	\item Kapeller, Jakob  The Authors of Economics Journals Revisited: Evidence from a Large-Scale Replication of Hodgson \& Rothman (1999) SASE 2021 Conference: After Covid? Critical Conjunctures and Contingent Pathways of Contemporary Capitalism 2021-07-02
	\item   The Authors of Economics Journals Revisited: Evidence from a Large-Scale Replication of Hodgson \& Rothman (1999) SASE 2021 Conference: After Covid? Critical Conjunctures and Contingent Pathways of Contemporary Capitalism 2021-07-02
	\item Heimberger, Philipp  Italy and Europe will rise or fall together Participation in a panel discussion International Weekly Seminar, Le Grand Continent 2021-06-24
	\item Heimberger, Philipp  Italy -- Europe’s basket case? Forum New Economy 2021-06-15
	\item Porak, Laura  Social Limits to Growth unbekannt/unknown 2021-06-10
	\item Strunk, Birte  Warum braucht es Pluralismus in den Wirtschaftswissenschaften? unbekannt/unknown 2021-06-09
	\item Gräbner-Radkowitsch, Claudius  Warum braucht es Pluralismus in den Wirtschaftswissenschaften? unbekannt/unknown 2021-06-09
	\item   Competition in Transitional Processes: Polanyi and Schumpeter Duke University CHOPE Summer Institute 2021 2021-06-04
	\item Hager, Theresa  Competition in Transitional Processes: Polanyi and Schumpeter Duke University CHOPE Summer Institute 2021 2021-06-04
	\item Rath, Johanna  Competition in Transitional Processes: Polanyi and Schumpeter Duke University CHOPE Summer Institute 2021 2021-06-04
	\item Hirte, Katrin  Oligopole Strukturen in der Schlachthofindustrie und ihre Herkunft Research Seminar, ICAE 2021-05-27
	\item Porak, Laura  Governing the Ungovernable. Recontextualizations of European Policy Discourse Economic Policy Conference 2021-05-27
	\item Kapeller, Jakob  Conversations for the Future of Europe unbekannt/unknown 2021-05-19
	\item Heimberger, Philipp  Fiskalpolitik und europäischer Integrationsprozess unbekannt/unknown 2021-05-11
	\item Pühringer, Stephan  Der Wert, zitiert zu werden unbekannt/unknown 2021-05-04
	\item Hirte, Katrin  Plurale ökonomische Ideengeschichte Initiative „Mehr ökonomische Vielfalt erreichen“ 2021-04-29
	\item Hirte, Katrin  Plurale volkswirtschaftliche Ideengeschichte unbekannt/unknown 2021-04-29
	\item Aistleitner, Matthias  Wissenschaftstheoretische Ansätze in den WiWi und ihre Auswirkungen auf gesellschaftliche Prozesse (Teil 2) Wissensproduktion: Unis als Wissensfabriken und ihre Rolle in der Gesellschaft 2021-04-28
	\item Kapeller, Jakob  Pandemic pushes polarization: Macroeconomic divergence in the Eurozone unbekannt/unknown 2021-04-23
	\item Aistleitner, Matthias  Wissenschaftstheoretische Ansätze in den WiWi und ihre Auswirkungen auf gesellschaftliche Prozesse (Teil 1) Wissensproduktion: Unis als Wissensfabriken und ihre Rolle in der Gesellschaft 2021-04-20
	\item Porak, Laura  Governing the Ungovernable. Recontextualizations of European Policy Discourse First Doctoral Conference on the Social and Political Constitution of the Economy 2021-03-26
	\item Pühringer, Stephan  German neoliberalism in crisis? Varieties of Neoliberalization 2021-03-18
	\item Kapeller, Jakob  Kosten und Nutzen des „billigen“ Geldes unbekannt/unknown 2021-03-09
	\item Rath, Johanna  Wie wird über Wettbewerb gesprochen? unbekannt/unknown 2021-03-02
	\item Heimberger, Philipp  Corona-Krise und EU-Wiederaufbauplan: Ein ‚fiskalpolitischer Durchbruch‘ für Europa? unbekannt/unknown 2021-02-20
	\item Heimberger, Philipp  Does economic globalisation promote economic growth? A meta analysis Presentation 13th FIW-Research Conference ‘International Economics 2021-02-19
	\item Heimberger, Philipp  Outputlücken-Schätzungen und Fiskalpolitik im Kontext der Covid19-Krise unbekannt/unknown 2021-02-18
	\item Pühringer, Stephan  Private Verlage profitieren von Forschungsgeldern unbekannt/unknown 2021-02-01
	\item Kapeller, Jakob  Past and future of pluralism in economics unbekannt/unknown 2021-01-29
	\item Schütz, Bernhard  Nach „Koste es was es wolle“: Eine neue Ära in der Ökonomie? unbekannt/unknown 2021-01-27
	\item Pühringer, Stephan  Wettbewerb infrage -- neue Podcastreihe unbekannt/unknown 2021-01-05
	\item Pühringer, Stephan  Wettbewerb als Ordnungsprinzip unbekannt/unknown 2021-01-05
	\item Gräbner-Radkowitsch, Claudius  Komplexitätsökonomik -- ein Teaser unbekannt/unknown 2021-01-05
	\item Gräbner-Radkowitsch, Claudius  Country capabilities, product complexity and finance in the EU Rebuilding Macroeconomics Globalization Hub Conference 2021-01-05
	\item Kapeller, Jakob  Finanzialisierung und globale Ungleichheit unbekannt/unknown 2020-12-18
	\item   Competition in Transformational Processes: Polanyi \& Schumpeter Open Research Seminar ICAE 2020-12-17
	\item Hager, Theresa  Competition in Transformational Processes: Polanyi \& Schumpeter Open Research Seminar ICAE 2020-12-17
	\item Rath, Johanna  Competition in Transformational Processes: Polanyi \& Schumpeter Open Research Seminar ICAE 2020-12-17
	\item Gräbner-Radkowitsch, Claudius  Pluralismus in der Ökonomik -- Wissenschaftstheoretische Hintergründe unbekannt/unknown 2020-12-15
	\item   Evolutionary political economic policies in a multisector, multi-regional agent-based model of a global value chain, resource extraction and climate change Open Research Seminar ICAE 2020-12-10
	\item Heimberger, Philipp  Corporate Tax Competition: A meta-analysis Open Research Seminar ICAE 2020-11-26
	\item Kapeller, Jakob  Past and future of pluralism in economics unbekannt/unknown 2020-11-23
	\item Griesebner, Teresa  Global Child Chain -- Indische Leihmutterschaft im Kontext globaler Güterkettenforschung Open Research Seminar ICAE 2020-11-19
	\item Strohmaier, Rita  The Role of Public Procurement for Driving Corporate Social Innovation in Global IT Supply Chains Open Research Seminar ICAE 2020-11-12
	\item Kapeller, Jakob  Which are the most inspiring economic questions of our time? unbekannt/unknown 2020-11-11
	\item Schütz, Bernhard  Minsky-Veblen Cycles: A stock-flow consistent, agent-based approach Open Research Seminar ICAE 2020-11-05
	\item Hornykewycz, Anna  Minsky-Veblen Cycles: A stock-flow consistent, agent-based approach Open Research Seminar ICAE 2020-11-05
	\item   Narrative Economics: Introduction to Literary and Media Representations of the Economy Open Research Seminar ICAE 2020-10-29
	\item Pühringer, Stephan  Theorizing Competition: An interdisciplinary approach to the genesis of a contested concept Open Research Seminar ICAE 2020-10-22
	\item Gräbner-Radkowitsch, Claudius  Theorizing Competition: An interdisciplinary approach to the genesis of a contested concept Open Research Seminar ICAE 2020-10-22
	\item Porak, Laura  Allbetroffenheit in der Pandemie? Ein soziologischer Blick auf das Erleben der Auswirkungen der Corona-Krise Momentum 2020 2020-10-17
	\item Porak, Laura  Wohin steuert die Europäische Union? Ein Klärungsversuch der strategischen Ausrichtung der EU seit Lissabon Momentum 2020 2020-10-16
	\item   Competition in Transformational Processes: Polanyi \& Schumpeter Momentum-Kongress 2020-10-16
	\item Hager, Theresa  Competition in Transformational Processes: Polanyi \& Schumpeter Momentum-Kongress 2020-10-16
	\item Rath, Johanna  Competition in Transformational Processes: Polanyi \& Schumpeter Momentum-Kongress 2020-10-16
	\item   Presente y futuro del desarrollo económico desde una perspectiva Latinoamericana unbekannt/unknown 2020-10-02
	\item   Presente y futuro del desarrollo económico desde una perspectiva Latinoamericana unbekannt/unknown 2020-10-02
	\item Kapeller, Jakob  Presente y futuro del desarrollo económico desde una perspectiva Latinoamericana unbekannt/unknown 2020-10-02
	\item Pühringer, Stephan  Theorizing Competition: An interdisciplinary approach to the genesis of a contested concept EAEPE Online Conference 2020-09-04
	\item   Theorizing Competition: An interdisciplinary approach to the genesis of a contested concept EAEPE Online Conference 2020-09-04
	\item   Competition in Transformational Processes: Polanyi \& Schumpeter EAEPE 2020-09-03
	\item Hager, Theresa  Competition in Transformational Processes: Polanyi \& Schumpeter EAEPE 2020-09-03
	\item Rath, Johanna  Competition in Transformational Processes: Polanyi \& Schumpeter EAEPE 2020-09-03
	\item Rath, Johanna  Talking about Competition. Discursive shifts in the economic imaginary of competition in public debates EAEPE 2020 2020-09-03
	\item Pühringer, Stephan  Talking about Competition. Discursive shifts in the economic imaginary of competition in public debates EAEPE 2020 2020-09-03
	\item Aistleitner, Matthias  Exploring the trade (policy) narratives in economic elite discourse EAEPE annual conference 2020 2020-09-03
	\item Pühringer, Stephan  Exploring the trade (policy) narratives in economic elite discourse EAEPE annual conference 2020 2020-09-03
	\item   Inequality in a zero growth economy Open Research Seminar ICAE 2020-07-23
	\item Rath, Johanna  Talking about Competition: Discursive shifts in the economic imaginary of competition in public debates SASE 2020 2020-07-22
	\item Pühringer, Stephan  Talking about Competition: Discursive shifts in the economic imaginary of competition in public debates SASE 2020 2020-07-22
	\item Pühringer, Stephan  Theorizing Competition: An interdisciplinary approach to the genesis of a contested concept SASE Annual Conference 2020-07-18
	\item Pühringer, Stephan  Talking about Competition: Discursive shifts in the economic imaginary of competition Open Research Seminar ICAE 2020-07-16
	\item Rath, Johanna  Talking about Competition: Discursive shifts in the economic imaginary of competition Open Research Seminar ICAE 2020-07-16
	\item Porak, Laura  Die strategische Ausrichtung der Europäischen Union Open Research Seminar ICAE 2020-07-09
	\item Pühringer, Stephan  Monopolies in Science Publishing: A black hole for public spending SPACE-Kickoff 2020-06-30
	\item Gräbner-Radkowitsch, Claudius  The Emergence of Core-Periphery Structures in the European Trade Network Open Research Seminar ICAE 2020-06-18
	\item Rath, Johanna  Monopolies in Science Publishing: A black hole for public spending? Open Research Seminar ICAE 2020-05-14
	\item Pühringer, Stephan  Monopolies in Science Publishing: A black hole for public spending? Open Research Seminar ICAE 2020-05-14
	\item Hirte, Katrin  Zettelkasten ist nicht gleich Zettelkasten: Eine Kontrastierung der Arbeiten von Mark Lombardi und Niklas Luhmann Open Research Seminar ICAE 2020-04-23
	\item Hirte, Katrin  Zettelkasten ist nicht gleich Zettelkasten. Zum ontologischen Problem in der Netzwerkforschung mit Fokus auf Mark Lombardi und Niklas Luhmann Tagung: Warum Netzwerkforschung 2020-03-24
	\item Kapeller, Jakob  Was ist heterodoxe Ökonomie? Selbst-Organisierte Lehrveranstaltung: \glqq Alternative Wirtschaftstheorien -- Heterodoxe Ökonomie\grqq{} 2020-03-19
	\item Kapeller, Jakob  Refeudalisierung als Gefahr für die Demokratie Armutskonferenz 2020-03-10
	\item Pühringer, Stephan  Wo lassen denken? Die unsichtbare Macht von Think Tanks unbekannt/unknown 2020-02-20
	\item Hornykewycz, Anna  Theory and empirics of capability accumulation: Implications for macroeconomic modelling Open Research Seminar ICAE 2020-02-13
	\item Gräbner-Radkowitsch, Claudius  Theory and empirics of capability accumulation: Implications for macroeconomic modelling Open Research Seminar ICAE 2020-02-13
	\item Aistleitner, Matthias  Theory and empirics of capability accumulation: Implications for macroeconomic modelling Open Research Seminar ICAE 2020-02-13
	\item Pühringer, Stephan  Diskursanalyse zu Schwarz-Blauen Arbeitsmarkt- und Sozialpolitikreformen Open Research Seminar ICAE 2020-01-23
	\item Schütz, Bernhard  The conflict over income in a capitalist society: A stock-flow consistent approach Open Research Seminar ICAE 2020-01-09
	\item Kapeller, Jakob  City Science Talk: Solidarität unbekannt/unknown 2020-01-08
	\item Kapeller, Jakob  Solidarität – Wandel eines starken Begriffs unbekannt/unknown 2020-01-08
	\item Pühringer, Stephan  Neoliberalismus und „Rechtspopulismus“ Analogien und Widersprüche Tagung Neoliberalismus und Rechtspopulismus: Zusammenhänge, Strategien und Umgangsformen 2019-12-19
	\item Heimberger, Philipp  Does economic globalization affect income inequality? A meta-analysis unbekannt/unknown 2019-12-05
	\item Kapeller, Jakob  Wirtschaftliche Polarisierung in Europa Aktuelle Gegensätze \& ökonomische Betrachtungen 2019-12-05
	\item Eder, Julia Theresa  Prestons Community Wealth Building-Ansatz – Eine lokale Entwick-lungsstrategie (nicht nur) für Orte, die ‚nichts zählen‘ Open Research Seminar ICAE 2019-12-05
	\item Kapeller, Jakob  Der Feind im Inneren – Das alte Dilemma des Liberalismus 18. IT- \& Beratertag der Wirtschaftskammer Österreich 2019-12-04
	\item Kapeller, Jakob  Gesellschaftliche Verantwortung der Wissenschaft Fachkonferenz Wissenschaft in Verantwortung 2019-11-21
	\item Heimberger, Philipp  Does employment protection affect unemployment? A meta-study Open Research Seminar ICAE 2019-11-21
	\item Aistleitner, Matthias  Towards exploring the genesis of competition in economic thought Open Research Seminar ICAE 2019-11-14
	\item Gräbner-Radkowitsch, Claudius  Getting the best of both worlds: potentials for triangulating agent-based and equilibrium-based analysis Agent-based economics 2019-10-30
	\item Hirte, Katrin  Funktionale Konstrukte von Bevölkerung im Kontext der „Agrar- versus Industriestaatsdebatte unbekannt/unknown 2019-10-28
	\item Gräbner-Radkowitsch, Claudius  Is the Eurozone disintegrating? Macroeconomic divergence, structural polarization, trade and fragility FMM Conference 2019 2019-10-26
	\item Schütz, Bernhard  Is the Eurozone disintegrating? Macroeconomic divergence, structural polarization, trade and fragility FMM Conference 2019 2019-10-26
	\item Kapeller, Jakob  Is the Eurozone disintegrating? Macroeconomic divergence, structural polarization, trade and fragility FMM Conference 2019 2019-10-26
	\item Heimberger, Philipp  Is the Eurozone disintegrating? Macroeconomic divergence, structural polarization, trade and fragility FMM Conference 2019 2019-10-26
	\item Kapeller, Jakob  The conflict over income in a capitalist society: A stock-flow consistent approach 23rd Conference of the Forum for Macroeconomics and Macroeconomic Policies (FMM) 2019-10-25
	\item   The conflict over income in a capitalist society: A stock-flow consistent approach 23rd Conference of the Forum for Macroeconomics and Macroeconomic Policies (FMM) 2019-10-25
	\item   The conflict over income in a capitalist society: A stock-flow consistent approach 23rd Conference of the Forum for Macroeconomics and Macroeconomic Policies (FMM) 2019-10-25
	\item Schütz, Bernhard  The conflict over income in a capitalist society: A stock-flow consistent approach 23rd Conference of the Forum for Macroeconomics and Macroeconomic Policies (FMM) 2019-10-25
	\item Kapeller, Jakob  Vermögen wächst auch mit Vermögenssteuer weiter unbekannt/unknown 2019-10-23
	\item Rath, Johanna  Widersprüche im technologischen Fortschritt: am Beispiel der Blockchain Technologie Momentum19: Widerspruch 2019-10-12
	\item Kapeller, Jakob  Economic Polarization in Europe Momentum 2019: Widerspruch 2019-10-12
	\item Kapeller, Jakob  Auftragsvergabe, Leistungsqualität und Kostenintensität im Schienenpersonenverkehr Momentum 2019 2019-10-12
	\item Aistleitner, Matthias  Auftragsvergabe, Leistungsqualität und Kostenintensität im Schienenpersonenverkehr Momentum 2019 2019-10-12
	\item Grimm, Christian  Auftragsvergabe, Leistungsqualität und Kostenintensität im Schienenpersonenverkehr Momentum 2019 2019-10-12
	\item Schütz, Bernhard  Ein pluralistisches Paradigma? Anwendungsbeispiel Mindestlohn Momentum19: Widerspruch 2019-10-11
	\item Porak, Laura  Der Wert des Widerspruchs für die demokratische Praxis Momentum 2019 2019-10-11
	\item Heimberger, Philipp  'Output gap nonsense': Reforming the EU's fiscal rules ETUC Policy Committee 2019-10-10
	\item Aistleitner, Matthias  (Towards) exploring the genesis of competition in economic thought Young Economists Conference 2019 2019-10-02
	\item Heimberger, Philipp  Does economic globalization affect income inequality? A meta-analysis Young Economists Conference 2019 2019-10-01
	\item Hirte, Katrin  Relationalität in der Soziologie und „Poppers Fluch“ unbekannt/unknown 2019-09-28
	\item Hirte, Katrin  \glqq Alte Bekannte im neuen Gewand\grqq{} – Zum Phänomen kontextualer Neuinterpretation von Wissen unbekannt/unknown 2019-09-27
	\item Hirte, Katrin  Too big to fail…? Subventionierte Strukturplan-Politik in Deutschland zur Forcierung von Massentierhaltung unbekannt/unknown 2019-09-26
	\item Kapeller, Jakob  Structural change in times of increasing openness: Assessing path dependency in European Economic integration Jahrestagung des Vereins für Socialpolitik 2019-09-23
	\item Gräbner-Radkowitsch, Claudius  Structural change in times of increasing openness: Assessing path dependency in European Economic integration Jahrestagung des Vereins für Socialpolitik 2019-09-23
	\item Gräbner-Radkowitsch, Claudius  Country capabilities, product complexity, and finance in the EU: An AB-SFC multi country model for policy analysis How Can Interdisciplinary Research Enhance the Policy Relevance of Macroeconomics? 2019-09-19
	\item Gräbner-Radkowitsch, Claudius  Unrealistic models and how to identify them: on accounts of model realisticness Annual Conference of the European Association for Evolutionary Political Economy 2019-09-14
	\item Kapeller, Jakob  Convergence and Polarization Dynamics in Europa and globally unbekannt/unknown 2019-09-14
	\item Schütz, Bernhard  Creating a Pluralist Paradigm: An Application to the Minimum Wage Debate 31st EAEPE Annual Conference 2019-09-14
	\item Pühringer, Stephan  Divided We Stand? Professional Consensus and Political Conflict in Academic Economics EAEPE Annual Conference 2019-09-13
	\item Hornykewycz, Anna  Models of Capability Accumulation European Association for Evolutionary Political Economy 2019-09-13
	\item Heimberger, Philipp  Is the Eurozone disintegrating? Macroeconomic divergence, structural polarization EAEPE Conference 2019 2019-09-13
	\item Kapeller, Jakob  Economics \& Philosophy Pre-Conference at European Association for Evolutionary Economy 2019-09-12
	\item Gräbner-Radkowitsch, Claudius  Pluralism, complexity and the effective triangulation of methods Summer School on Complexity Economics, Behavioural Economics, and Data Science 2019-09-02
	\item Kapeller, Jakob  Interview zu pluraler Ökonomik, Teil II unbekannt/unknown 2019-08-11
	\item Kapeller, Jakob  Interview zu pluraler Ökonomik, Teil I unbekannt/unknown 2019-07-28
	\item Pühringer, Stephan  Wie funktioniert Wettbewerb? unbekannt/unknown 2019-07-12
	\item Gräbner-Radkowitsch, Claudius  Kommentar zu Komplexität und Evolution in der Ökonomik Jahrestagung Ausschuss für Evolutorische Ökonomik (VfS) 2019-07-05
	\item Pühringer, Stephan  Divided We Stand? Professional Consensus and Political Conflict in Academic Economics Annual Conference Society for the Advancement of Socio-Economics (SASE) 2019-06-29
	\item Heimberger, Philipp  The power of economic models: The case of the EU's fiscal regulation framework SASE 2019 2019-06-28
	\item Heimberger, Philipp  Fiscal policies in international perspective Applied economics seminar at the Joint Vienna Institute 2019-06-04
	\item Kapeller, Jakob  Der Feind im Inneren: Das alte Dilemma des Liberalismus Top Management Symposium \glqq Open Society and its Enemies\grqq{} 2019-05-31
	\item Hirte, Katrin  Ökonomisches Denken selbst bewertet – Muster, Mechanismen und Dynamiken in der ökonomischen Dogmenhistorie Tagung des DFG-Netzwerkes \glqq Soziologie des ökonomischen Denkens\grqq{} 2019-05-16
	\item Ötsch, Walter  Market and society in Neo-liberalism and right-wing populism: A “reversed embeddedness”? International Karl Polanyi Conference 2019-05-04
	\item Pühringer, Stephan  Market and society in Neo-liberalism and right-wing populism: A “reversed embeddedness”? International Karl Polanyi Conference 2019-05-04
	\item Pühringer, Stephan  Machtstrukturen Neoliberaler Think Tanks in Österreich unbekannt/unknown 2019-04-30
	\item Kapeller, Jakob  Development and globalization: a pluralist perspective 2nd Vienna Conference on Pluralism in Economics 2019-04-16
	\item Heimberger, Philipp  Structural change in times of increasing openness: Assessing path dependency in European economic integration 2nd Vienna Conference on Pluralism in Economics 2019-04-16
	\item Pühringer, Stephan  Exploring the trade narrative in the ‘top 5’ journals in economics 2. Vienna Conference on Pluralism in Economics 2019-04-16
	\item Aistleitner, Matthias  Exploring the trade narrative in the ‘top 5’ journals in economics 2. Vienna Conference on Pluralism in Economics 2019-04-16
	\item Pühringer, Stephan  Why Economics is still Political Economy. Economists as enlightened experts and/or political proclaimers 2. Vienna Conference on Pluralism in Economics 2019-04-15
	\item Kapeller, Jakob  Philosophy and economics Still Rethinking? The Need for Pluralism in Economics 2019-03-31
	\item Kapeller, Jakob  The future of pluralism in economics Still Rethinking? The Need for Pluralism in Economics 2019-03-31
	\item Hirte, Katrin  Netzwerkanalysen zur Entwicklung der professoralen Agrarpolitiker und Agrarökonomen Deutschlands 1933 bis 2000 unbekannt/unknown 2019-03-29
	\item Rath, Johanna  Solving the Hold-up-Problem: Mechanism Design vs Social Norms Open Research Seminar ICAE 2019-02-27
	\item Grimm, Christian  Paradigmen und Politik. Bestandsaufnahme der deutschen Ökonomik im Vergleich zu den USA unbekannt/unknown 2019-02-22
	\item Strunk, Birte  Harvesting the Benefits of Different Perspectives: Theoretical and Practical Reflections on Pluralistic Education in Economics Grenzen überwinden, Pluralismus wagen. Perspektiven sozio*ökonomischer Hochschullehre 2019-02-22
	\item Gräbner-Radkowitsch, Claudius  Harvesting the Benefits of Different Perspectives: Theoretical and Practical Reflections on Pluralistic Education in Economics Grenzen überwinden, Pluralismus wagen. Perspektiven sozio*ökonomischer Hochschullehre 2019-02-22
	\item Kapeller, Jakob  Pluralist and interdisciplinary research in academic economics GSÖBW-Meeting: \glqq crossing borders, embracing pluralism\grqq{} 2019-02-21
	\item Pühringer, Stephan  Warum gibt es die Staatsschuldenkrise? SchülerInnenworkshop 2019-02-11
	\item Schütz, Bernhard  Warum gibt es die Staatsschuldenkrise? SchülerInnenworkshop 2019-02-11
	\item Schütz, Bernhard  Menschen- und Umweltrechte in globalen Beschaffungsketten Workshop Global denken, global handeln: Über Globalisierung 2019-01-18
	\item Heimberger, Philipp  10 Fragen an Philipp Heimberger oder: Auf den Spuren der NAIRU unbekannt/unknown 2018-12-20
	\item Pühringer, Stephan  Polit-ökonomische Machtstrukturen unter „öffentlichen ÖkonomInnen“ Momentum 2018: Klasse 2018-12-19
	\item Pühringer, Stephan  Freiheitliche Flügelkämpfe? (Historische) Konfliktlinien in der FPÖ Kurswechsel-Heftpräsentation 2018-12-18
	\item Heimberger, Philipp  Das ABC der Finanzwelt -- G wie Great Depression B unbekannt/unknown 2018-12-12
	\item Hirte, Katrin  Intended and unintended non-knowledge – a neglected area in the debate on economic pluralism Forms of Power in Economics: New perspectives for the Social Studies of Economics between networks, discourses and fields 2018-12-07
	\item Beyer, Karl  Why Economics is Still Political Economy: Economists as enlightened experts and/or political proclaimers Forms of Power in Economics 2018-12-06
	\item Pühringer, Stephan  Why Economics is Still Political Economy: Economists as enlightened experts and/or political proclaimers Forms of Power in Economics 2018-12-06
	\item Hirte, Katrin  Entitäten und Relationen – Die relationale Forschungsperspektive Tagung: Das Paradigma der Relationalität, Deutsche Gesellschaft für Netzwerkforschung 2018-12-04
	\item Pühringer, Stephan  Divided we stand? Die politischen Orientierungen ‚öffentlicher ÖkonomInnen‘ in den USA 10 Jahre nach der Weltfinanzkrise 2018-11-16
	\item Schütz, Bernhard  Country capabilities, product complexity, and finance in the EU: An AB-SFC multi country model for policy analysis unbekannt/unknown 2018-11-15
	\item Gräbner-Radkowitsch, Claudius  Country capabilities, product complexity, and finance in the EU: An AB-SFC multi country model for policy analysis unbekannt/unknown 2018-11-15
	\item Kapeller, Jakob  The focus of academic economics 22. FMM-Konferenz 2018-10-28
	\item Heimberger, Philipp  Strukturelle Polarisierung: Warum Europa trotz Aufschwungs ökonomisch auseinanderdriftet Momentum 2018 2018-10-22
	\item Hirte, Katrin  Andersdenkende als Vordenker im wirtschaftswissenschaftlichen Feld Tagung des DFG-Netzwerkes \glqq Soziologie des ökonomischen Denkens\grqq{} 2018-10-19
	\item Pühringer, Stephan  Machtstrukturen Neoliberaler Think Tanks unbekannt/unknown 2018-10-15
	\item Heimberger, Philipp  Do labor market rigidities increase unemployment? Evidence for 23 OECD countries over 1985-2013 Young Economists Conference 2018 2018-10-10
	\item Kapeller, Jakob  Structural change in time of increasing openness: assessing path dependency in European economic integration Annual Meeting of the German Economic Association 2018-09-24
	\item Heimberger, Philipp  Structural change in times of increasing openness: Assessing path dependency in European economic integration EAEPE Conference 2018 2018-09-08
	\item Kapeller, Jakob  Structural change in times of increasing openness: Assessing path dependency in European economic integration EAEPE Conference 2018 2018-09-08
	\item Gräbner-Radkowitsch, Claudius  Structural change in times of increasing openness: Assessing path dependency in European economic integration EAEPE Conference 2018 2018-09-08
	\item Aistleitner, Matthias  The Focus of Academic Economics -- Before and After the Crisis EAEPE 2018: Evolutionary foundations at a crossroad: Assessments, outcomes and implications for policymakers 2018-09-07
	\item Heimberger, Philipp  The Kaldor paradox revisited: New evidence in times of increasing openness“ Annual Meeting of the European Association for Evolutionary Economy 2018-09-07
	\item Gräbner-Radkowitsch, Claudius  The Kaldor paradox revisited: New evidence in times of increasing openness“ Annual Meeting of the European Association for Evolutionary Economy 2018-09-07
	\item Kapeller, Jakob  The Kaldor paradox revisited: New evidence in times of increasing openness“ Annual Meeting of the European Association for Evolutionary Economy 2018-09-07
	\item Pühringer, Stephan  What economics education is missing: the real world unbekannt/unknown 2018-09-06
	\item Kapeller, Jakob  Widersprüche in der Ökonomie unbekannt/unknown 2018-08-24
	\item   Zivilisierte Märkte Pressegespräch 2018-07-18
	\item Kapeller, Jakob  Zivilisierte Märkte Pressegespräch 2018-07-18
	\item Beyer, Karl  Silvio Gesell und die Freiwirtschaftslehre unbekannt/unknown 2018-07-18
	\item Gräbner-Radkowitsch, Claudius  Die Bedeutung von Vertrauen und sozialer Kontrolle für die Funktion informeller Wert-Transfer Systems Tagung des Evolutorischen Ausschusses des Vereins für Socialpolitik 2018-07-13
	\item Schütz, Bernhard  Employment and the minimum wage: A pluralist approach 20th Anniversary Conference of the Association for Heterodox Economics 2018-07-13
	\item Kapeller, Jakob  Die Spaltung als Modell: Europas Zerfall als Krise der Wirtschaftstheorie Ringvorlesung 10 Years After the Crash 2018-07-04
	\item Hirte, Katrin  Performativität und Ökonomik unbekannt/unknown 2018-06-21
	\item Hirte, Katrin  Performativität und Ökonomik – Worum sorgen sich Ökonom*innen? Vorlesungsreihe „Plurale Ökonomik“, Universität Jena 2018-06-21
	\item Hirte, Katrin  Vergessen – Verkennen – Vermeiden: Das Problem Andersdenkende in den Wissenschaften unbekannt/unknown 2018-06-20
	\item Gräbner-Radkowitsch, Claudius  Theory development through agent-based modelling Lessons from economics Theory Development Through Agent-Based Modeling 2018-06-14
	\item Kapeller, Jakob  Verteilungsgerechtigkeit im Familienrecht: Eine philosophische Perspektive 31. FamilienrichterInnen -- Tagung 2018-06-07
	\item Heimberger, Philipp  Fiscal policies in international perspective unbekannt/unknown 2018-06-05
	\item Kapeller, Jakob  Ökonomische Effekte der Verkehrsreform des Landes Tirol unbekannt/unknown 2018-06-04
	\item Schütz, Bernhard  Ökonomische Effekte der Verkehrsreform des Landes Tirol unbekannt/unknown 2018-06-04
	\item Heimberger, Philipp  Modellierung von Fiskalpolitik: Theorie und Empirie anhand von aktuellen Debatten SOLV XII – Debatten der Ökonomie 2018-05-09
	\item Pühringer, Stephan  „Netzwerke des Marktes“ in ihrem Einfluss auf Gesellschaft Reshaping Economics 2018-04-28
	\item Ötsch, Walter  „Netzwerke des Marktes“ in ihrem Einfluss auf Gesellschaft Reshaping Economics 2018-04-28
	\item Pühringer, Stephan  VWL-Studium als alltägliche Lebensrealität Reshaping Economics 2018-04-27
	\item Bäuerle, Lukas  VWL-Studium als alltägliche Lebensrealität Reshaping Economics 2018-04-27
	\item Schütz, Bernhard  Geld -- Eine post-keynesianische Perspektive unbekannt/unknown 2018-04-18
	\item Kapeller, Jakob  Ökonomische Diskurse: Historische und philosophische Perspektiven Einführung zu SOLVXII -- Debatten in der Ökonomie 2018-04-11
	\item Pühringer, Stephan  Verteilungsgerechtigkeit und ökonomische Ungleichheit unbekannt/unknown 2018-04-10
	\item Pühringer, Stephan  Wie denken zukünftige Ökonom\_innen? Diskussionsveranstaltung zur Zukunft der deutschen Ökonomik 2018-03-20
	\item   Politik und Paradigmen in der Ökonomie in Deutschland und den USA FGW Vernetzungstreffen 2018-03-16
	\item Pühringer, Stephan  Politik und Paradigmen in der Ökonomie in Deutschland und den USA FGW Vernetzungstreffen 2018-03-16
	\item Hirte, Katrin  Ernährungswende gemeinsam gestalten unbekannt/unknown 2018-03-06
	\item Pühringer, Stephan  10 Jahre Krise: Neoliberale Denkmuster, wirtschaftspolitische Debatten und der Einfluss von Think Tanks Ökonomie: Was ist eigentlich neoliberal? 2018-02-09
	\item Hirte, Katrin  Zur Transformation der ostdeutschen Agrarstrukturen 1990/1991 sowie zu ihrer neuerlichen Transformation unbekannt/unknown 2018-01-31
	\item Heimberger, Philipp  Österreichs Staatsausgabenstrukturen im europäischen Vergleich Budget Jour Fixe WIFO 2018-01-10
	\item   To trust or to control -- Informal value transfer systems and computational analysis in institutional economics Allied Social Sciences Associations Meeting 2018-01-07
	\item Gräbner-Radkowitsch, Claudius  To trust or to control -- Informal value transfer systems and computational analysis in institutional economics Allied Social Sciences Associations Meeting 2018-01-07
	\item Pühringer, Stephan  The Anti-democratic Logic of Right-wing Populism and Neoliberal Market-fundamentalism ASSA Annual Conference 2018-01-07
	\item Kapeller, Jakob  Government Policies and Financial Crises: Mitigation, Postponement or Prevention? ASSA Annual Meeting 2018 2018-01-05
	\item Schütz, Bernhard  Government Policies and Financial Crises: Mitigation, Postponement or Prevention? ASSA Annual Meeting 2018 2018-01-05
	\item Aistleitner, Matthias  Citation Patterns in Economics and Beyond: Assessing the Peculiarities of Economics from Two Scientometric Perspectives Momentum 2017: Vielfalt 2017-12-20
	\item Kapeller, Jakob  Wissenschaftstheorie und Ökonomie Ringvorlesung \glqq Frontiers of Economics – Die Wirtschaftswissenschaft zwischen Krise und Aufbruch\grqq{} 2017-12-20
	\item Pühringer, Stephan  Economists' Narratives and Metaphors of the Financial Crisis unbekannt/unknown 2017-11-30
	\item Gräbner-Radkowitsch, Claudius  Institutionenökonomik, Komplexitätsökonomik und Pluralismus Ringvorlesung \glqq Denkschulen und aktuelle Kontroversen der Ökonomik\grqq{} 2017-11-28
	\item Schütz, Bernhard  Die vielen Facetten des Reichtums in Österreich unbekannt/unknown 2017-11-23
	\item Schütz, Bernhard  Ab wann ist jemand reich? unbekannt/unknown 2017-11-23
	\item Heimberger, Philipp  Ungleichheit in den USA: Ökonomische und gesellschaftliche Auswirkungen unbekannt/unknown 2017-11-22
	\item Heimberger, Philipp  The dynamic effects of fiscal consolidation episodes on income inequality: Evidence for 17 OECD countries over 1978-2013 FMM Conference 2017 2017-11-10
	\item Hirte, Katrin  Wirtschaftswissenschaften und sozial-ökologische Transformation unbekannt/unknown 2017-11-06
	\item Kapeller, Jakob  Reichtum in Österreich unbekannt/unknown 2017-10-30
	\item Pühringer, Stephan  Ordoliberalismus als politisches Gestaltungswissen: ein deutscher Sonderweg Bedingungen und politische Funktionen exakter Wissenschaften 2017-10-27
	\item Kapeller, Jakob  Citation metrics and the development of economics 2017 INET Plenary Conference 2017-10-24
	\item   On defining institutions and how to study them Annual Conference of the European Association for Evolutionary Political Economy (EAEPE) 2017-10-21
	\item Gräbner-Radkowitsch, Claudius  On defining institutions and how to study them Annual Conference of the European Association for Evolutionary Political Economy (EAEPE) 2017-10-21
	\item Grimm, Christian  Paradigmatische Homogenität? Aktueller Stand und Zukunftsperspektiven der Ökonomik in Deutschland und den USA Momentum 2017: Vielfalt 2017-10-20
	\item Beyer, Karl  The political consequences of paradigmatic monism in economics. Evidences from a comparative analysis of German and US economics Annual Conference European Ass. for Evol. Pol. Economy (EAEPE) 2017-10-20
	\item Pühringer, Stephan  The political consequences of paradigmatic monism in economics. Evidences from a comparative analysis of German and US economics Annual Conference European Ass. for Evol. Pol. Economy (EAEPE) 2017-10-20
	\item   Sources of cooperation, performance, and stability in informal value transfer systems Annual Conference of the European Association for Evolutionary Political Economy (EAEPE) 2017-10-19
	\item Gräbner-Radkowitsch, Claudius  Sources of cooperation, performance, and stability in informal value transfer systems Annual Conference of the European Association for Evolutionary Political Economy (EAEPE) 2017-10-19
	\item Ötsch, Walter  Aus der moralischen Entrüstung gehen unbekannt/unknown 2017-10-16
	\item Aistleitner, Matthias  In cars we trust. Automobile manufacturing as part of an integrative European industrial policy. Young Economists Conference 2017 (‘European integration at a crossroads’) 2017-10-13
	\item Gräbner-Radkowitsch, Claudius  On the many ways a model can be unrealistic – and still useful What to make of highly unrealistic models? 2017-10-12
	\item Heimberger, Philipp  The dynamic effects of fiscal consolidation episodes on income inequality: Evidence for 17 OECD countries over 1978-2013 European Integration at a Crossroads 2017-10-12
	\item Pühringer, Stephan  Ordoliberale Netzwerke und ihre Wirkmächtigkeit der(wirtschafts-)politische Einfluss des Ordoliberalismus 2017-10-10
	\item   Beyond Equilibrium: Revisting Two-Sided Markets from an Agent-Based Perspective Conference on Complex Systems 2017-09-29
	\item Gräbner-Radkowitsch, Claudius  Beyond Equilibrium: Revisting Two-Sided Markets from an Agent-Based Perspective Conference on Complex Systems 2017-09-29
	\item   The dance of Godzilla and the earthquake: on the sectoral and structural foundations of macroeconomic dynamics Conference on Complex Systems 2017-09-20
	\item Gräbner-Radkowitsch, Claudius  The dance of Godzilla and the earthquake: on the sectoral and structural foundations of macroeconomic dynamics Conference on Complex Systems 2017-09-20
	\item   Towards a computational understanding of institutions – A review and a new proposition Jahrestagung des World Interdisciplinary Network for Institutional Research (WINIR) 2017-09-15
	\item Gräbner-Radkowitsch, Claudius  Towards a computational understanding of institutions – A review and a new proposition Jahrestagung des World Interdisciplinary Network for Institutional Research (WINIR) 2017-09-15
	\item Kapeller, Jakob  Citation metrics and the development of economics The Fragmentation of Economics and the New Role of the History of Economic Thought 2017-09-15
	\item Kapeller, Jakob  Agenten, Auktionen, Ausschreibungen: Von Nobelpreisen und Vergabeverfahren Bahnenquete 2017-09-13
	\item Kapeller, Jakob  The Political Economy of Bank Regulatory Evasion Summer School 2017-09-06
	\item Kapeller, Jakob  Globalization and Free Trade Forum Alpbach 2017 2017-08-29
	\item Heimberger, Philipp  The dynamic effects of fiscal consolidation episodes on income inequality: Evidence for 17 OECD countries over 1978-2013 ECINEQ Conference 2017 2017-07-19
	\item Pühringer, Stephan  Selling the “economic truth” to the public. Herbert Giersch and Hans-Werner Sinn as “public intellectuals" Language and Economics 2017-07-17
	\item Pühringer, Stephan  Profil der deutschsprachigen Volkswirtschaftslehre im internationalen Vergleich Heimann-Colloquiums an der Universität Hamburg 2017-06-29
	\item Griesser, Markus  MigrantInnen als Zielgruppe der österreichischen Gewerkschaftsbewegung Präsentation von Heft 2/2017 der Österreichischen Zeitschrift für Soziologie 2017-06-14
	\item Gräbner-Radkowitsch, Claudius  Komplexitätsökonomik Ringvorlesung \glqq Ökonomische Denkschulen und Grundlagen der Wissenschaftstheorie\grqq{} 2017-06-12
	\item Gräbner-Radkowitsch, Claudius  The complexity approach to economic development as a new rationale for industrial policy? EAEPE Symposium on Development Economics 2017-06-11
	\item Schütz, Bernhard  Wie wird gerechter Handel möglich? Öffentlicher ExpertInnenworkshop der Allianz gerechter Handel 2017-06-09
	\item Hirte, Katrin  Mendelejews Traum – Faktuale und fiktionale Komponenten einer Wissenschaftsgeschichte unbekannt/unknown 2017-05-26
	\item Pühringer, Stephan  Bilder von ÖkonomInnen zur Finanzkrise 2008 Macht der Bilder, Macht der Sprache 2017-05-25
	\item Gräbner-Radkowitsch, Claudius  Verification and validation of agent-based models: Epistemological importance and theoretical challenges Agent-based modeling in Ecological Economics -- From toy model to verified tool of analysis 2017-05-19
	\item Heimberger, Philipp  Die Eurokrise: Wettbewerbsdenken vs. koordinierte Wirtschaftspolitik unbekannt/unknown 2017-05-19
	\item Schütz, Bernhard  EU-Krise: Neue Chancen für Europa? unbekannt/unknown 2017-05-11
	\item Kapeller, Jakob  Wissenschaftstheorie und Ökonomie unbekannt/unknown 2017-05-08
	\item Kapeller, Jakob  Verteilungstendenzen im globalen Kapitalismus. Eine plurale Perspektive. unbekannt/unknown 2017-05-04
	\item Kapeller, Jakob  Economics and Philosophy: Models and Pluralism unbekannt/unknown 2017-04-27
	\item Kapeller, Jakob  Paradigm lost? Herausforderungen der Ökonomik nach der Krise unbekannt/unknown 2017-04-26
	\item Pühringer, Stephan  Neoliberale Think Tanks als Katalysatoren gesellschaftlicher Reformdiskurse. ESPAnet Austria: 1. Forschungskonferenz Sozialpolitik 2017-04-21
	\item Stelzer-Orthofer, Christine  Neoliberale Think Tanks als Katalysatoren gesellschaftlicher Reformdiskurse. ESPAnet Austria: 1. Forschungskonferenz Sozialpolitik 2017-04-21
	\item Griesser, Markus  Arbeitsmarktpolitik als Gesellschaftspolitik ESPAnet Austria: 1. Forschungskonferenz Sozialpolitik 2017-04-20
	\item   Moralität, Wettbewerb und internationaler Handel: Eine europäische Perspektive AK LänderreferentInnentagung EU und Internationales 2017-04-04
	\item Kapeller, Jakob  Moralität, Wettbewerb und internationaler Handel: Eine europäische Perspektive AK LänderreferentInnentagung EU und Internationales 2017-04-04
	\item Schütz, Bernhard  Moralität, Wettbewerb und internationaler Handel: Eine europäische Perspektive AK LänderreferentInnentagung EU und Internationales 2017-04-04
	\item Heimberger, Philipp  Fiscal consolidation and disintegration tendencies in Europe: Is there a link? „Europe at a crossroads": Member seminar of the Vienna Institute for International Economic Studies 2017-03-30
	\item Pühringer, Stephan  Zur zentralen Rolle ordoliberaler Netzwerke in wirtschaftspolitischen Reformprozessen in Deutschland Gründungskonferenz der Gesellschaft für Soziökomischen Bildung und Wissenschaft (GSÖBW) 2017-03-17
	\item Ötsch, Walter  Zur zentralen Rolle ordoliberaler Netzwerke in wirtschaftspolitischen Reformprozessen in Deutschland Gründungskonferenz der Gesellschaft für Soziökomischen Bildung und Wissenschaft (GSÖBW) 2017-03-17
	\item Griesser, Markus  Wohlstandsorientierte Politik: Möglichkeiten für eine gesellschaftliche Verankerung? unbekannt/unknown 2017-03-16
	\item Kapeller, Jakob  Pluralist and heterodox economics: Conceptual clarification and exemplary applications COST-Workshops 2017-03-15
	\item Heimberger, Philipp  Did fiscal consolidation cause the double-dip recession in the euro area? Keynes-Tagung „Keynes, Geld und Finanzen“ 2017-02-20
	\item Griesser, Markus  Ansätze einer Verankerung wohlstandsorientierter Politik in Österreich Kongress „Gutes Leben für alle“ 2017-02-11
	\item Hirte, Katrin  The two transitions from algebraic to analytical thinking in economics in the 18th and 19th century Tagung des DFG-Netzwerkes \glqq Soziologie des ökonomischen Denkens\grqq{} 2017-02-09
	\item Pühringer, Stephan  Diskursive und politische Wirkmächtigkeit ökonomischen Denkens unbekannt/unknown 2017-01-17
	\item Griesser, Markus  Uneven waves of commodification, decommodification, and recommodification. Karl Polanyi and the Analysis of Welfare State Transformation A Great Transformation? Global Perspectives on Contemporary Capitalisms 2017-01-13
	\item Heimberger, Philipp  Enforcing economic liberalism in European fiscal policy-making: On the role of the European Commission’s poential output model A Great Transformation? 2017-01-11
	\item Aistleitner, Matthias  Prospects of a sustainable automotive industry A Great Transformation? Global Perspectives on Contemporary Capitalisms 2017-01-11
	\item Gräbner-Radkowitsch, Claudius  The Nature of Institutions: A Computational Perspective Allied Social Sciences Associations (ASSA) Annual Meeting 2017-01-07
	\item Gräbner-Radkowitsch, Claudius  The Complementary Relationship Between Evolutionary-Institutional and Complexity Economics Allied Social Sciences Associations (ASSA) Annual Meeting 2017-01-07
	\item Ötsch, Walter  Populismus, Demagogie und die „Wut von unten“ unbekannt/unknown 2016-12-21
	\item Kapeller, Jakob  Was ist heterodoxe Ökonomie? unbekannt/unknown 2016-12-15
	\item Gräbner-Radkowitsch, Claudius  The complexity challenge: Systemist thinking, computational modeling, and pluralism in economics Verleihung des WIWA Nachwuchspreises für Plurale Ökonomik 2016-12-12
	\item   Studienpräsentation \glqq Verankerung wohlstandsorientierter Politik" Workshop "Verankerung wohlstandsorientierter Politik\grqq{} 2016-12-12
	\item Griesser, Markus  Studienpräsentation \glqq Verankerung wohlstandsorientierter Politik" Workshop "Verankerung wohlstandsorientierter Politik\grqq{} 2016-12-12
	\item Kapeller, Jakob  Citation metrics and the development of economics Governing Economics: Institutional Changes, New Frontiers and the State of Pluralism 2016-12-07
	\item Hirte, Katrin  Ökonomen-Netzwerke – Zu historischen und aktuellen Netzwerken der deutschen VWL-Professorinnen und Professoren unbekannt/unknown 2016-12-05
	\item Gräbner-Radkowitsch, Claudius  Eine Einführung in die Institutionenökonomik(en) Ringvorlesung \glqq Denkschulen und aktuelle Kontroversen der Ökonomik\grqq{} 2016-11-21
	\item Kapeller, Jakob  Selbstverwirklichung vs. Selbstoptimierung: Das ambivalente Erbe der Aufklärung 5. Internationale Hartheim-Konferenz 2016-11-18
	\item Kapeller, Jakob  Die Spaltung als Modell: Europas Zerfall als Krise der Wirtschaftstheorie unbekannt/unknown 2016-11-10
	\item Grimm, Christian  Pluralismus in der Ökonomik? Zur Frage von Machtverhältnissen und Homogenität in der deutschsprachigen Ökonomik. unbekannt/unknown 2016-11-07
	\item Grimm, Christian  Profil der deutschsprachigen Volkswirtschaftslehre. Paradigmatische Ausrichtung und politische Wirkmächtigkeit deutschsprachiger Ökonom\_innen FGW Jahrestagung 2016-11-05
	\item Heimberger, Philipp  The performativity of potential output: Pro-cyclicality and path dependency in coordinating European fiscal policies The 28th annual EAEPE conference 2016-11-05
	\item Kapeller, Jakob  The performativity of potential output: Pro-cyclicality and path dependency in coordinating European fiscal policies The 28th annual EAEPE conference 2016-11-05
	\item Griesser, Markus  Workshop \glqq Armut und soziale Ausgrenzung\grqq{} Tagung „Helfe sich wer kann! Unser Sozialsystem im Umbruch“ 2016-11-05
	\item Pühringer, Stephan  The “Performative Footprint” of economists political and societal impact of post-war German economists Annual Conference \glqq European Association for Evolutionary and Political Economy\grqq{} 2016-11-04
	\item Gräbner-Radkowitsch, Claudius  The effect of selection and reputation mechanisms on the emergence of cooperation, Annual Conference of the European Association for Evolutionary Political Economy (EAEPE) 2016-11-04
	\item Kapeller, Jakob  Epistemology of Economics PreConference-session at the annual conference of the European Association for Evolutionary Political Economy (EAEPE) 2016-11-03
	\item Pühringer, Stephan  Pluralismus in der Ökonomie? Wissenschaftstheoretische Grundlagen und wissenschaftspolitische Konsequenzen Ringvorlesung \glqq Rethinking Economics\grqq{} 2016-10-26
	\item Kapeller, Jakob  The performativity of potential output: Pro-cyclicality and path dependency in coordinating European fiscal policies FMM conference -- Towards Pluralism in Macroeconomics? 2016-10-21
	\item Heimberger, Philipp  The performativity of potential output: Pro-cyclicality and path dependency in coordinating European fiscal policies FMM conference -- Towards Pluralism in Macroeconomics? 2016-10-21
	\item Aistleitner, Matthias  Perspektiven für eine nachhaltige Automobilindustrie Momentum 2016: Macht 2016-10-15
	\item Kapeller, Jakob  Spezialisierung, Stratifikation und internationale Wirtschaft Momentum16: Macht 2016-10-15
	\item Grimm, Christian  Performativität, Machtverhältnisse und Ökonomik. Zur Frage der Homogenität in der deutschsprachigen Ökonomik Momentum 2016: Macht 2016-10-14
	\item Heimberger, Philipp  Die Macht ökonomischer Modelle am Beispiel des \glqq Potential-Output\grqq{}-Modells der Europäischen Kommission Momentum -- Macht 2016-10-14
	\item   Strategien einer permanenten neoliberalen Diskurshegemonie? Momentum Kongress 2016: Macht 2016-10-14
	\item Pühringer, Stephan  Strategien einer permanenten neoliberalen Diskurshegemonie? Momentum Kongress 2016: Macht 2016-10-14
	\item Kapeller, Jakob  Der Kapitalismus und die schöne neue Welt: Eine ambivalente Beziehung Jahrestagung der Sozialplattform OÖ 2016-10-11
	\item Pühringer, Stephan  Wie denken zukünftige Ökonom\_innen? unbekannt/unknown 2016-09-27
	\item   Netzwerke, Paradigmen, Attitüden -- Der deutsche Sonderweg im Fokus FGW-Vernetzung und Projektpräsentation 2016-09-27
	\item Pühringer, Stephan  Netzwerke, Paradigmen, Attitüden -- Der deutsche Sonderweg im Fokus FGW-Vernetzung und Projektpräsentation 2016-09-27
	\item Pühringer, Stephan  Netzwerke, Paradigmen, Attitüden -- Der deutsche Sonderweg im Fokus FGW-Vernetzung und Projektpräsentation 2016-09-27
	\item Hirte, Katrin  Zeitlichkeit und Tauschfähigkeit bei Rosa Luxemburg und Joseph Alois Schumpeter unbekannt/unknown 2016-09-24
	\item Kapeller, Jakob  The performativity of potential output: Pro-cyclicality and path-dependency in coordinating European fiscal policies Conference on Complex Systems (CCS) 2016-09-20
	\item Pühringer, Stephan  Podiumsdiskussion: Was kommt nach der Krise im Anschluss an die Filmpremiere \glqq Zero Crash\grqq{} 2016-09-19
	\item Gräbner-Radkowitsch, Claudius  Computational Game Theory and the Micro Foundations of Institutions Institutions -- Emergence or Design? 2016-09-08
	\item Pühringer, Stephan  Workshop Verteilung und Gerechtigkeit: Ökonomische und politphilosophische Perspektiven Herbstakademie der Cusanus Hochschule 2016-09-07
	\item Kapeller, Jakob  The future of economic teaching Paneldiskussion im Rahmen der Jahrestagung des Vereins für Sozialpolitik (VfS) 2016-09-05
	\item Kapeller, Jakob  Globalisierung und Konzernmacht – Zur Moral des Profits im 21. Jahrhundert unbekannt/unknown 2016-09-05
	\item Kapeller, Jakob  Wissenschaftstheorie und Ökonomie 2. IMK-Workshop \glqq Plurale Ökonomik\grqq{} 2016-08-13
	\item Pühringer, Stephan  The role of democracy in the history of liberal (economic) thought Summerschool Alternative Economic and Monetary Systems 2016-08-01
	\item Ötsch, Walter  Kann man soziale Verantwortung lehren? unbekannt/unknown 2016-07-28
	\item Pühringer, Stephan  Wirkungen ökonomischer Lehre Treffen AG Ökonomie und Studierende 2016-07-27
	\item Griesser, Markus  Discursive Shifts in the Transition from an “Active” to an “Activating” Labour Market Policy (LMP) Interpretive Policy Analysis Conference 2016-07-07
	\item Pühringer, Stephan  Discursive Shifts in the Transition from an “Active” to an “Activating” Labour Market Policy (LMP) Interpretive Policy Analysis Conference 2016-07-07
	\item Pühringer, Stephan  Zur zentralen Rolle ordoliberaler Netzwerke in wirtschaftspolitischen Reformprozessen in Deutschland Jahrestagung des Ausschusses für Evolutorische Ökonomik im Verein für Socialpolitik an der Leibniz Universität Hannover 2016-07-02
	\item Ötsch, Walter  Zur zentralen Rolle ordoliberaler Netzwerke in wirtschaftspolitischen Reformprozessen in Deutschland Jahrestagung des Ausschusses für Evolutorische Ökonomik im Verein für Socialpolitik an der Leibniz Universität Hannover 2016-07-02
	\item Pühringer, Stephan  Diskursive und politische Wirkmächtigkeit ökonomischen Denkens Ringvorlesung „Theorie, Modell, Wirklichkeit“ an der Universität Wien 2016-06-21
	\item Pühringer, Stephan  Still the Queens of Social Sciences. (Post-)Crisis power balances of “public economists” in Germany Crisis and the Social Sciences: New Challenges and Perspectives 2016-06-11
	\item Hubmann, Georg  Workshop: Think Tanks und Macht ICAE-Sommerakademie „Globalisierung und Konzernmacht“ 2016-06-03
	\item Pühringer, Stephan  Workshop: Think Tanks und Macht ICAE-Sommerakademie „Globalisierung und Konzernmacht“ 2016-06-03
	\item Pühringer, Stephan  Ökonomisches Denken in der Krise? Wirtschaft, menschengerecht gedacht? 2016-05-23
	\item Gräbner-Radkowitsch, Claudius  Computations, Mechanisms, and Socio-Ecological Systems: A meta-theoretical appraisal of ABM Agent-based modeling in Ecological Economics -- A useful tool or just a fancy gadget? 2016-05-20
	\item Bräutigam, Lars  Das europäische Schattenbankensystem – Typologisierung und Bewertung regulatorischer Initiativen auf europäischer Ebene unbekannt/unknown 2016-05-12
	\item Beyer, Karl  Das europäische Schattenbankensystem – Typologisierung und Bewertung regulatorischer Initiativen auf europäischer Ebene unbekannt/unknown 2016-05-12
	\item Pühringer, Stephan  Die Denkfabrik der Millionäre unbekannt/unknown 2016-05-12
	\item Gräbner-Radkowitsch, Claudius  Netzwerke und Komplexität in der Ökonomik Ringvorlesung \glqq Alternative Mikroökonomie\grqq{} 2016-05-10
	\item Kapeller, Jakob  Spezifische Aspekte ökonomischen Modelldenkens unbekannt/unknown 2016-04-26
	\item Kapeller, Jakob  Großer Betrug am kleinen Bürger: Warum zahlen wir noch Steuern? unbekannt/unknown 2016-04-08
	\item Kapeller, Jakob  Panama Papers unbekannt/unknown 2016-04-07
	\item Kapeller, Jakob  The social philosophy of globalized markets unbekannt/unknown 2016-04-07
	\item Pühringer, Stephan  Krisen als Krankheiten und Katastrophen: Zur diskursiven Wirkmächtigkeit ökonomischen Denkens Lectures and Debates zur \glqq Sprache des Geldes\grqq{} 2016-04-06
	\item Pühringer, Stephan  Think Tanks, Ökonomen und Politik unbekannt/unknown 2016-03-31
	\item Pühringer, Stephan  Think Tank networks of German Neoliberalism More Roads from Mont Pèlerin. Neoliberalism Studies 2016-03-20
	\item Pühringer, Stephan  Vortrag und Workshop zur Finanzkrise und deren wirtschaftspolitischen Folgen unbekannt/unknown 2016-03-01
	\item Kapeller, Jakob  Wie liberal ist der Neoliberalismus? unbekannt/unknown 2016-02-23
	\item Hirte, Katrin  Entstehen und Vergehen der deutschen Heterodoxen ab den 1970ern Tagung des DFG-Netzwerkes \glqq Soziologie des ökonomischen Denkens\grqq{}, Universität Frankfurt am Main 2016-02-11
	\item Gräbner-Radkowitsch, Claudius  Komplexitätsökonomik Ringvorlesung \glqq Denkschulen und aktuelle Kontroversen der Ökonomik\grqq{} 2016-02-09
	\item Kapeller, Jakob  Geschichte und Praxis der Austeritätspolitik unbekannt/unknown 2016-02-05
	\item Kapeller, Jakob  Humanistische Handelspolitik und inklusives Wachstum unbekannt/unknown 2016-01-26
	\item Pühringer, Stephan  Workshop zu: Philosophien der Verteilungsgerechtigkeit unbekannt/unknown 2016-01-07
	\item Grimm, Christian  Wirtschaftspolitische Ausrichtungen in österreichischen Parteiprogrammen 7. Wintertagung des ICAE: Ökonomie! Welche Ökonomie? Zu Stand und Status der Wirtschaftswissenschaft 2015-12-05
	\item Pühringer, Stephan  Krise und ökonomische Politikberatung 7. Wintertagung des ICAE: Ökonomie! Welche Ökonomie? Zu Stand und Status der Wirtschaftswissenschaft 2015-12-05
	\item Aistleitner, Matthias  Der Einfluss der Wissenschaftsstatistik auf den ökonomischen Diskurs 7. Wintertagung des ICAE: Ökonomie! Welche Ökonomie? Zu Stand und Status der Wirtschaftswissenschaft 2015-12-05
	\item Heimberger, Philipp  Austeritätspolitik in Zeiten der Eurokrise: Wachstumseffekte der fiskalischen Konsolidierungsmaßnahmen 2011-2013 Ökonomie! Welche Ökonomie? Zu Stand und Status der Wirtschaftswissenschaft 2015-12-05
	\item Hirte, Katrin  „Landnahme“. Zur Ökonomie Rosa Luxemburgs und dem Defizit-Problem in der Ökonomie. unbekannt/unknown 2015-12-05
	\item Kapeller, Jakob  Ökonomische Forschung und die Finanzkrise 7. Wintertagung des ICAE: Ökonomie! Welche Ökonomie? Zu Stand und Status der Wirtschaftswissenschaft 2015-12-05
	\item Beyer, Karl  Die Rolle der Politik in der Entwicklung des Schattenbankensektors Ökonomie! Welche Ökonomie? Zu Stand und Status der Wirtschaftswissenschaft 2015-12-04
	\item Bräutigam, Lars  Die Rolle der Politik in der Entwicklung des Schattenbankensektors Ökonomie! Welche Ökonomie? Zu Stand und Status der Wirtschaftswissenschaft 2015-12-04
	\item Hirte, Katrin  Der Einfluss europäischer Agrarpolitikmaßnahmen auf die Arbeit im Agrarsektor unbekannt/unknown 2015-12-01
	\item Schütz, Bernhard  Government policies and financial crises: On the mitigation, postponement and prevention of crisis within a Minsky-Veblen cycle MitarbeiterInnenseminar des Insituts für VWL der JKU 2015-11-28
	\item   Government policies and financial crises: On the mitigation, postponement and prevention of crisis within a Minsky-Veblen cycle MitarbeiterInnenseminar des Insituts für VWL der JKU 2015-11-28
	\item Landesmann, Michael  Government policies and financial crises: On the mitigation, postponement and prevention of crisis within a Minsky-Veblen cycle MitarbeiterInnenseminar des Insituts für VWL der JKU 2015-11-28
	\item Kapeller, Jakob  Government policies and financial crises: On the mitigation, postponement and prevention of crisis within a Minsky-Veblen cycle MitarbeiterInnenseminar des Insituts für VWL der JKU 2015-11-28
	\item Griesser, Markus  Funktionen, Policies, Diskurse. Ein Beitrag zur Erklärung (sozial-)staatlicher Transformation Tag der Politikwissenschaft 2015 2015-11-28
	\item Pühringer, Stephan  Welche Ökonomie wollen wir? unbekannt/unknown 2015-11-27
	\item Hirte, Katrin  Exklusionen in der Ökonomie? Die Situation der Heterodoxen in Deutschland und ihre Ursachen unbekannt/unknown 2015-11-27
	\item Pühringer, Stephan  Workshop zu: Philosophien der Verteilungsgerechtigkeit unbekannt/unknown 2015-11-17
	\item Kapeller, Jakob  Pluralismus in der Ökonomie: Eine wissenschaftstheoretische Perspektive unbekannt/unknown 2015-11-04
	\item Kapeller, Jakob  Verteilungstendenzen im Kapitalismus -- Nationale und globale Perspektiven Präsentation Kurswechsel 2/15: Vermögensungleichheit, Kapitalismus und Demokratie 2015-11-03
	\item Aistleitner, Matthias  Die Macht der Wissenschaftsstatistik und die Entwicklung der Ökonomie Momentum 15: Kritik 2015-10-23
	\item Heimberger, Philipp  Did Fiscal Consolidation Cause the Double Dip Recession in the Euro Area? The Spectre of Stagnation? Europe in the World Economy 2015-10-23
	\item Hirte, Katrin  Politik und ihre Ad-hoc- Gremien in Krisenzeiten: Public Management zwischen Demokratie und Wirkungsorientierung? unbekannt/unknown 2015-10-23
	\item Pühringer, Stephan  Märkte als Richter – Angela Merkels Sprechen über die Krise unbekannt/unknown 2015-10-19
	\item Pühringer, Stephan  \glqq Märkte als Richter\grqq{} -- Stephan Pühringer über Merkels Sprechen über die Krise unbekannt/unknown 2015-10-19
	\item Aistleitner, Matthias  Die Dominanz des ökonomischen Mainstreams aus szientometrischer Perspektive: Befunde zum \glqq Matthäus-Effekt\grqq{} in den Wirtschaftswissenschaften 5. Studentischer Soziologiekongress 2015-10-02
	\item Gräbner-Radkowitsch, Claudius  Eine algorithmische Perspektive auf die Evolution von Institutionen Buchenbachworkshop 2015-10-01
	\item Hirte, Katrin  Agrarsoziologie ohne Aufgaben? Zum vergangenen und zukünftigem Aufgaben-verständnis in der deutschsprachigen Agrarsoziologie unbekannt/unknown 2015-10-01
	\item   It’s not only about inequality: the role of unmet aspirations and speculative capital for explaining social protest Annual Conference of the European Association for Evolutionary Political Economy (EAEPE) 2015-09-18
	\item Gräbner-Radkowitsch, Claudius  It’s not only about inequality: the role of unmet aspirations and speculative capital for explaining social protest Annual Conference of the European Association for Evolutionary Political Economy (EAEPE) 2015-09-18
	\item Gräbner-Radkowitsch, Claudius  The Nature of Institutions: A Computational Perspective Allied Social Sciences Associations (ASSA) Annual Meeting 2015-09-18
	\item Hirte, Katrin  Zum performativen Fußabdruck der ÖkonomInnen 1955-1992 2te pluralistische Ergänzungsveranstaltung des Netzwerkes Real World Economics 2015-09-07
	\item   Zum performativen Fußabdruck der ÖkonomInnen 1955-1992 2te pluralistische Ergänzungsveranstaltung des Netzwerkes Real World Economics 2015-09-07
	\item Kapeller, Jakob  Some implications of traditional philosophy for current economics unbekannt/unknown 2015-09-03
	\item Griesser, Markus  Economics return to a dismal science? A CPE approach to the role of economic thought in German labour market reforms from the 1960s to the 2000s Inaugural Conference of Cultural Political Economy 2015-09-02
	\item Pühringer, Stephan  Economics return to a dismal science? A CPE approach to the role of economic thought in German labour market reforms from the 1960s to the 2000s Inaugural Conference of Cultural Political Economy 2015-09-02
	\item Pühringer, Stephan  Economics return to a dismal science? The changing role of economic thought in German labour market reforms from the 1960s to the 2000s Inaugural Conference on Cultural Political Economy 2015-09-02
	\item Griesser, Markus  Economics return to a dismal science? The changing role of economic thought in German labour market reforms from the 1960s to the 2000s Inaugural Conference on Cultural Political Economy 2015-09-02
	\item Kapeller, Jakob  Pluralism in Economics unbekannt/unknown 2015-09-02
	\item   Beyond Equilibrium: Revisting Two-Sided Markets from an Agent-Based Perspective Annual Conference of the European Association for Evolutionary Political Economy (EAEPE) 2015-08-19
	\item Gräbner-Radkowitsch, Claudius  Beyond Equilibrium: Revisting Two-Sided Markets from an Agent-Based Perspective Annual Conference of the European Association for Evolutionary Political Economy (EAEPE) 2015-08-19
	\item Hirte, Katrin  Ökonom*innen und Politik. Analyse zur politischen Einflußnahme deutschsprachiger Ökonom*innen und Ökonomie. unbekannt/unknown 2015-07-16
	\item Pühringer, Stephan  The narrow concept of “economic normality” in economists public crisis discourse 10th Conference of Interpretive Policy Analysis. 2015-07-08
	\item Hirte, Katrin  Zur Finanzkrise und ihren forschungsseitigen Herausforderungen unbekannt/unknown 2015-07-01
	\item Hirte, Katrin  Europäische Agrarpolitik unbekannt/unknown 2015-06-22
	\item Kapeller, Jakob  Piketty verstehen 6. Sommerakademie des ICAE 2015-06-20
	\item Pühringer, Stephan  Philosophien der Verteilungsgerechtigkeit 6. Sommerakademie des ICAE: Kapitalismus und Gerechtigkeit -- Die Rolle der Ungleichheit im 21. Jahrhundert 2015-06-20
	\item Kapeller, Jakob  Pluralismus in der Ökonomie: Eine wissenschaftstheoretische Perspektive unbekannt/unknown 2015-06-09
	\item Kapeller, Jakob  Pluralismus in der Ökonomie: Eine wissenschaftstheoretische Perspektive unbekannt/unknown 2015-06-01
	\item Griesser, Markus  MigrantInnen als Zielgruppe? Studienpräsentation zu solidarischen Beratungs- und Unterstützungsangeboten von ArbeitnehmerInnenorganisationen in Österreich 2015-05-20
	\item Kapeller, Jakob  Die Rolle pluraler Ökonomik in der ökonomischen Bildung unbekannt/unknown 2015-05-05
	\item Hirte, Katrin  Denkschulenentwicklungen in der deutschen VWL nach 1945 Kolloquium Dogmenhistorie 2015-04-24
	\item Kapeller, Jakob  The crisis as a game changer? Der Einfluss der Krise auf die Entwicklung ökonomischer Forschung [The crisis and the development of economic research] Conference on the OECD's \glqq New Approaches to Economic Challenges\grqq{}-Program 2015-04-22
	\item Hirte, Katrin  Moderation zu: Wirtschaftspolitische Herausforderungen einer starken Vermögenskonzentration Vortragsreihe \glqq Eigentum\grqq{} 2015-04-22
	\item Kapeller, Jakob  Two perspectives on citation metrics in economics Conference of the Institute for New Economic Thinking (INET) "Egalite, Liberte, Fragilite 2015-04-12
	\item Pühringer, Stephan  Markets as „ultimate judges“ of economic policies. Angela Merkel´s interpretation of the economic crisis and her conclusions for European crisis policies 1st Vienna Conference on Pluralism in Economics 2015-04-11
	\item Hirte, Katrin  Zum Entstehen und „Vergehen“ der deutschen universitären Agrarsoziologie 79. Tagung AG Ländliche Sozialforschung Österreich 2015-03-20
	\item Kapeller, Jakob  Krise der Wirtschaftswissenschaften -- Chancen für eine plurale Ökonomik? unbekannt/unknown 2015-02-23
	\item Kapeller, Jakob  Pluralism in Economics: Perspectives from Philosophy of Science unbekannt/unknown 2015-02-05
	\item Gräbner-Radkowitsch, Claudius  Pluralismus, Komplexität und Mikroökonomik unbekannt/unknown 2015-01-13
	\item Kapeller, Jakob  Beyond Foundations: Systemism in Economic Thinking ASSA Converence 2015 2015-01-03
	\item Gräbner-Radkowitsch, Claudius  Agent-Based Computational Models: A Useful Heuristic for Institutional Pattern Modeling? Allied Social Sciences Associations (ASSA) Annual Meeting 2015-01-03
	\item Kapeller, Jakob  Ranglisten und andere Evaluationskriterien – die Macht der Wissenschaftsstatistik unbekannt/unknown 2014-12-18
	\item Bräutigam, Lars  Markt, Marktakteure und die Rolle des Staates im Wirtschaftsleben Markt! Welcher Markt? Der interdisziplinäre Diskurs um Märkte und Marktwirtschaft 2014-12-12
	\item Pühringer, Stephan  „Der Markt als Richter“. Marktdisziplin und Austeritätspolitik Markt! Welcher Markt? Der interdisziplinäre Diskurs um Märkte und Marktwirtschaft 2014-12-12
	\item Hirte, Katrin  Märkte und die Anerkennung von Arbeit – schlechte und ungleiche Bezahlung als nur eine geschlechterspezifische Frage? Ein erkenntnistheoretischer Zugang zur Frage der schlechten Bezahlung bestimmter Tätigkeiten Markt! Welcher Markt? Der interdisziplinäre Diskurs um Märkte und Marktwirtschaft 2014-12-12
	\item Ötsch, Walter  “Der” Markt: Genese und Wirkungsweise eines vieldeutigen Begriffs Markt! Welcher Markt? Der interdisziplinäre Diskurs um Märkte und Marktwirtschaft 2014-12-12
	\item Ötsch, Walter  Moderation einer Diskussion zur Sozialen Markwirtschaft Markt! Welcher Markt? Der interdisziplinäre Diskurs um Märkte und Marktwirtschaft 2014-12-11
	\item Ötsch, Walter  The Neoliberal Utopia as Negation of All Other Utopias Unterwegs zu einer neuen Zivilisation geteilter Genügsamkeit. Perspektiven utopischen Denkens 25 Jahre nach dem Tod Ignacio Ellacurías 2014-12-05
	\item Hirte, Katrin  ÖkonomInnen und Ökonomie – Theorie, Erhebungen, Zugänge Workshop „Zur Pluralität der wissenschaftlichen Lehre in Deutschland“ 2014-12-01
	\item Griesser, Markus  Gewerkschaftliche Erneuerung durch Bewegungen der Migration? Workshop „(Kritische) Gewerkschaftsforschung in Österreich“ 2014-11-29
	\item Hirte, Katrin  Internationaler Freihandel auf Basis komparativer Kostenvorteile? -- zu Theorie und Praxis eines Versprechens Workshop Wirtschaftstheorie und Politik: Freihandel 2014-11-24
	\item Ötsch, Walter  Keynotespeaker zum Thema Mythos Markt – Die Wichtigkeit von Narrativen und von diskursiven Netzwerken unbekannt/unknown 2014-11-21
	\item Ötsch, Walter  Podiumsdiskussion über das Programm von \glqq Agenda Austria\grqq{} unbekannt/unknown 2014-11-19
	\item Bräutigam, Lars  Financial Markets: A Critical Approach to their Functions and Impact on Economy. Unemployment and Austerity in Mediterranean Countries 2014-11-19
	\item Hirte, Katrin  Kritische Betrachtung der Entwicklung der EU-Agrarpolitik unbekannt/unknown 2014-11-12
	\item Kapeller, Jakob  Die Rückkehr des Rentiers. Ansichten zu Thomas Pikettys ‘Capital in the 21st century’ unbekannt/unknown 2014-11-12
	\item Hirte, Katrin  Europäische Agrarpolitik. Institutionen, Innovationen, Interpretationen. unbekannt/unknown 2014-11-12
	\item Gräbner-Radkowitsch, Claudius  About Implicit Assumptions in Formal Modelling – The Case of DSGE and ABM in Economics Annual Conference of the European Association for Evolutionary Political Economy (EAEPE) 2014-11-07
	\item Kapeller, Jakob  Wirtschaftspolitik, Verteilungsgerechtigkeit und Demokratie Jahreshauptversammlung der Arbeiterkammer Burgenland 2014-11-07
	\item Kapeller, Jakob  Pluralismus in der Ökonomie: Eine wissenschaftstheoretische Perspektive unbekannt/unknown 2014-11-05
	\item Kapeller, Jakob  Relevanz, Realität, Modelle. Was braucht die Ökonomie? unbekannt/unknown 2014-10-29
	\item Kapeller, Jakob  Pluralismus in der Ökonomie: Eine wissenschaftstheoretische Perspektive unbekannt/unknown 2014-10-23
	\item Ötsch, Walter  Wie Bilder und Metaphern Wirtschaft beeinflussen: mit Beispielen vom 18. Jahrhundert bis zum Neoliberalismus unbekannt/unknown 2014-10-23
	\item Kapeller, Jakob  Pluralismus in der Ökonomie: Eine wissenschaftstheoretische Perspektive unbekannt/unknown 2014-10-21
	\item Hirte, Katrin  ÖkonomInnen in der Finanzkrise -- Wie performativ ist Wissenschaft unbekannt/unknown 2014-10-19
	\item Beyer, Karl  Emanzipation bei Marx und seine Kritik an Proudhon. Momentum 14 -- Ökonomik und Emanzipation 2014-10-18
	\item Beyer, Karl  Emanzipation bei Marx und seine Kritik an Proudhon und dessen ideengeschichtlichen Nachfahren unbekannt/unknown 2014-10-18
	\item Pühringer, Stephan  Crisis resistance of inequality The Future of Capitalism Development, Un(der)employment and Inequality 2014-09-24
	\item Bräutigam, Lars  Crisis resistance of inequality The Future of Capitalism Development, Un(der)employment and Inequality 2014-09-24
	\item   From free to civilised markets. First steps towards Eutopia WINIR conference \glqq Institutions that change the world\grqq{} 2014-09-12
	\item Schütz, Bernhard  From free to civilised markets. First steps towards Eutopia WINIR conference \glqq Institutions that change the world\grqq{} 2014-09-12
	\item Kapeller, Jakob  From free to civilised markets. First steps towards Eutopia WINIR conference \glqq Institutions that change the world\grqq{} 2014-09-12
	\item Kapeller, Jakob  Die Verteilung von Einkommen und Vermögen in Europa und International unbekannt/unknown 2014-09-10
	\item Kapeller, Jakob  Inequality before and after the Crisis European Forum Alpbach 2014-08-27
	\item Kapeller, Jakob  Wie können alternative ökonomische Forschungsansätze an Universitäten gefördert werden?\glqq  IMK-Workshop "Pluralismus in der Ökonomie\grqq{} 2014-08-10
	\item   Workshop: Einführung in die Komplexitätsökonomik Pluralismus in der Ökonomik 2014-08-09
	\item   Workshop: Einführung in die Komplexitätsökonomik Pluralismus in der Ökonomik 2014-08-09
	\item Gräbner-Radkowitsch, Claudius  Workshop: Einführung in die Komplexitätsökonomik Pluralismus in der Ökonomik 2014-08-09
	\item Kapeller, Jakob  Wissenschaftstheorie und Ökonomie unbekannt/unknown 2014-08-08
	\item Hirte, Katrin  Zur Performativität der Ökonomik und aktuellen ÖkonomInnen-Netzwerken in Deutschland unbekannt/unknown 2014-07-17
	\item Ötsch, Walter  Ökonomische Theorie und öffentliche Meinung unbekannt/unknown 2014-07-09
	\item Pühringer, Stephan  Podiumsdiskussion zum Film Too big to Tell von Johanna Tschautscher unbekannt/unknown 2014-07-09
	\item Pühringer, Stephan  Vom „Über-die-Verhältnisse-leben“ in der Krise Wirtschaftspolitik in der EU – das Scheitern des neoklassischen Paradigmas 2014-06-02
	\item Ötsch, Walter  Vom mittelalterlichen Dienst an Gott zum neoklassischen Produktionsfaktor. Ein Überblick über die Kulturgeschichte von Arbeit und Zeit unbekannt/unknown 2014-05-07
	\item Ötsch, Walter  Die unsichtbare Hand des Marktes: Wahrheit oder Glaubenslehre unbekannt/unknown 2014-05-06
	\item Ötsch, Walter  Mechanismen der Finanzmärkte, Globalisierung und Wirtschaftskrise unbekannt/unknown 2014-04-17
	\item Hirte, Katrin  Europäische Agrarpolitik – eine kritische Sicht unbekannt/unknown 2014-04-14
	\item Ötsch, Walter  Der Papst als Kritiker des Wirtschaftssystems unbekannt/unknown 2014-03-29
	\item Schütz, Bernhard  From Free to Civilized Markets: First Steps towards Eutopia Progressive Economy Forum 2014-03-06
	\item Kapeller, Jakob  From Free to Civilized Markets: First Steps towards Eutopia Progressive Economy Forum 2014-03-06
	\item Nordmann, Jürgen  Die Funktionslogik von Think-Tanks in neoliberalen Machtapparaten unbekannt/unknown 2014-02-04
	\item Ötsch, Walter  Demokratie und Finanzkapitalismus unbekannt/unknown 2014-02-04
	\item Hirte, Katrin  Zur Performativität der Ökonomik als Wissenschaft in Verantwortung Wissen! Welches Wissen? Wahrheit, Theorie und Glauben in der ökonomischen Theorie 2013-12-14
	\item Pühringer, Stephan  Zur Performativität der Ökonomik als Wissenschaft in Verantwortung Wissen! Welches Wissen? Wahrheit, Theorie und Glauben in der ökonomischen Theorie 2013-12-14
	\item Beyer, Karl  Diskussionsbeitrag zum Vortrag Modellierungskulturen in der Ökonomik: Vom Disziplinierungsinstrument zum Treiber von Theoriepluralismus? ICAE-Wintertagung: Wissen! Welches Wissen 2013-12-13
	\item Plaimer, Wolfgang  Möglichkeiten zur Bürgerbeteiligung in Vorchdorf unbekannt/unknown 2013-12-09
	\item Hirte, Katrin  ÖkonomInnen in der Finanzkrise -- Diskurse, Netzwerke, Initiativen Workshop der AK Linz 2013-12-05
	\item Pühringer, Stephan  ÖkonomInnen in der Finanzkrise -- Diskurse, Netzwerke, Initiativen Workshop der AK Linz 2013-12-05
	\item Plaimer, Wolfgang  Möglichkeiten vom Bürgerbeteiligungsmodellen in Mehrheits- und Minderheitsgemeinden unbekannt/unknown 2013-11-28
	\item Pühringer, Stephan  Ökonomie und ÖkonomInnen -- Interdependenz von Wirtschaftstheorie und ihrer Praxis Ringvorlesung Ökonomik zwischen Modell und Wirklichkeit 2013-11-26
	\item Gräbner-Radkowitsch, Claudius  Formal Foundations of Agent-Based Models and Simulations Annual Conference of the European Association for Evolutionary Political Economy (EAEPE) 2013-11-08
	\item Ötsch, Walter  Was ist eigentlich liberal? unbekannt/unknown 2013-11-06
	\item Pühringer, Stephan  Was ist eigentlich liberal? unbekannt/unknown 2013-11-06
	\item Ötsch, Walter  Ökonomik als soziale Physik – Von der Moralwissenschaft zum Naturalismus unbekannt/unknown 2013-10-29
	\item Ötsch, Walter  Populismus und Demagogie. Inhalte und Muster eines gefährlichen Denkens, mit Beispielen bis zur Tea Party und Frank Stronach Alltagsrassismus 2013-10-24
	\item Ötsch, Walter  Eigene Bilder von Identität und Zugehörigkeit erfahren und verändern Alltagsrassismus 2013-10-24
	\item Plaimer, Wolfgang  Der Fiskalpakt und seine Implementation in Österreich Momentum 2013: Fortschritt 2013-10-18
	\item Pühringer, Stephan  Der Fiskalpakt und seine Implementation in Österreich Momentum 2013: Fortschritt 2013-10-18
	\item Plaimer, Wolfgang  Bürgerbeteiligungsmodelle unbekannt/unknown 2013-10-07
	\item Gräbner-Radkowitsch, Claudius  Simulationen in der Evolutorischen Ökonomik Buchenbachworkshop 2013-10-03
	\item Pühringer, Stephan  Aus den Vorhöfen der Macht in die Medien hin zur eigenen Partei. Formen der der Einflussnahme von ÖkonomInnen auf Politik und Wirtschaft im Zuge der Finanz- und Wirtschaftskrisenpolitik. unbekannt/unknown 2013-09-26
	\item Beyer, Karl  Vortrag CDOs -- A Critical Phenomenon of the Financial System in the Crisis The Economy in Crisis and the Crisis in Economics 2013-09-10
	\item Bräutigam, Lars  Vortrag CDOs -- A Critical Phenomenon of the Financial System in the Crisis The Economy in Crisis and the Crisis in Economics 2013-09-10
	\item Pühringer, Stephan  Tsunami, Earthquake, Fever. Economists’ framing of the financial crisis in the public economic discourse The Economy in Crisis and the Crisis of Economics 2013-09-10
	\item Plaimer, Wolfgang  Bürgerbeteiligungsmodelle unbekannt/unknown 2013-08-29
	\item Plaimer, Wolfgang  Möglichkeiten zur Bürgerbeteiligung in Steyr unbekannt/unknown 2013-08-28
	\item Kapeller, Jakob  Wie reich ist Österreich? Studienautor Kapeller live im Studio unbekannt/unknown 2013-08-05
	\item Beyer, Karl  Workshop zum Schattenbankensystem ATTAC Sommer-Akademie „Reset Finance! Wege zu einem gesellschaftlich kontrollierten Finanz- und Bankensystem“, 2013-07-20
	\item Ötsch, Walter  Die Finanzkrise 2007-2009 als Krise von Schattenbanken. Eine institutionelle Analyse Jahrestagung des Ausschusses Evolutorische Ökonomik im Verein für Socialpolitik 2013-07-06
	\item Plaimer, Wolfgang  Möglichkeiten vom Bürgerbeteiligungsmodellen in Mehrheits- und Minderheitsgemeinden unbekannt/unknown 2013-07-01
	\item Pühringer, Stephan  The implementation of the European Fiscal Compact in Austria as a post-democratic phenomenon 1st World Keynes Conference 2013-06-27
	\item Nordmann, Jürgen  Allegorien des Lesens. Paul de Man im Kontext unbekannt/unknown 2013-06-21
	\item Plaimer, Wolfgang  Bürgerbeteiligungsmodelle unbekannt/unknown 2013-06-12
	\item Plaimer, Wolfgang  Bürgerbeteiligungsmodelle unbekannt/unknown 2013-06-10
	\item Ötsch, Walter  Finanzkapitalismus: was ist das? Strukturen eines Machtsystems 9. Internationales Open Space „Geld oder Leben – Symposium für eine lebensfreudige Finanzwirtschaft“ 2013-05-18
	\item Nordmann, Jürgen  Was ist Demokratie? unbekannt/unknown 2013-04-24
	\item Plaimer, Wolfgang  Bürgerbeteiligungsmodelle Bezirksparteiausschuss der SPÖ 2013-04-15
	\item Ötsch, Walter  Finanzkrise und Staatschuldenkrise Bezirkskonferenz des SPÖ-Pensionistenverbandes 2013-03-27
	\item Plaimer, Wolfgang  Die Implementierung des Fiskalpakts als postdemokratisches Phänomen Dominanz der Wirtschaft, Wiedererwachen der Politik, internationale Politische Ökonomie? 2013-03-22
	\item Pühringer, Stephan  Die Implementierung des Fiskalpakts als postdemokratisches Phänomen Dominanz der Wirtschaft, Wiedererwachen der Politik, internationale Politische Ökonomie? 2013-03-22
	\item Ötsch, Walter  Finanzkrise und Staatschuldenkrise Bezirkskonferenz des SPÖ-Pensionistenverbandes 2013-03-20
	\item Hirte, Katrin  Europäische Agrarpolitik Ernährungssouveränität 2013-03-12
	\item Ötsch, Walter  Einführung in das Schattenbankensystem Workshop Schattenbanken 2013-03-01
	\item Ötsch, Walter  Bankenrettungen unbekannt/unknown 2013-02-10
	\item Nordmann, Jürgen  Podiumsdiskussion: Zur Lage der Demokratie Demokratiewerkstatt 2013-01-24
	\item Hirte, Katrin  Das Problem der Anerkennung von Arbeit Workshop Feministische ÖkonomInnen 2013-01-18
	\item Plaimer, Wolfgang  Mehr Beteiligung durch direkte Demokratie? unbekannt/unknown 2012-12-03
	\item Ötsch, Walter  The Political Economy of Offshore Jurisdictions: Its Neoliberal Background Die Politische Ökonomie von Regulierungsoasen 2012-11-30
	\item Pühringer, Stephan  Diskussionsbeitrag zu: Ötsch, Silke: Our banking secrecy is a strong castle Die Politische Ökonomie von Regulierungsoasen 2012-11-30
	\item Ötsch, Walter  Eröffnung der 4. Jahrestagung Die Politische Ökonomie von Regulierungsoasen Die Politische Ökonomie von Regulierungsoasen 2012-11-29
	\item Hirte, Katrin  Alte und neue Erklärungsansätze zum Aufstieg und Fall von Eliten Lunch Lectures 2012-11-14
	\item Ötsch, Walter  Wieviel Sturm verträgt der Euro-Rettungsschirm? unbekannt/unknown 2012-11-12
	\item Plaimer, Wolfgang  Postdemokratie in Österreich Momentum Kongress 2012 im Panel “Recht, Freiheit, Demokratie” 2012-09-28
	\item Pühringer, Stephan  How liberalism lost its concept of democracy. Momentum Kongress 2012 im Panel “Recht, Freiheit, Demokratie”. 2012-09-28
	\item Ötsch, Walter  Lenken im Hamsterrad? Politisches Handeln in Zeiten von Finanz-Spekulationen, Schuldenkrisen und allmächtigen Konzernen Werkstatt-Gespräche Bludenz 2012-09-25
	\item Ötsch, Walter  Der Verlust der Kategorien von Moral und der Gesellschaft in der Geschichte der Nationalökonomie Ökonomisch-philosophischen Herbstakademie 2012: Die Ökonomien des Gemeinsamen. Neue Orte ökonomischer Bildung 2012-09-18
	\item Ötsch, Walter  Wider die Marktgläubigkeit. Ökonomie neu denken lernen (Publikumsdiskussion) Ökonomisch-philosophischen Herbstakademie 2012: Die Ökonomien des Gemeinsamen. Neue Orte ökonomischer Bildung 2012-09-18
	\item Ötsch, Walter  Neoliberalismus, ökonomische Theorien und Propaganda Ökonomisch-philosophischen Herbstakademie 2012: Die Ökonomien des Gemeinsamen. Neue Orte ökonomischer Bildung 2012-09-18
	\item Ötsch, Walter  Hintergründe der Staatsschuldenkrise – wie denken die handelnden Akteure? unbekannt/unknown 2012-09-17
	\item Ötsch, Walter  Diskussion zum Film Inside Job Dokumentarfilmreihe Die Politik in der Krise 2012-09-12
	\item Hirte, Katrin  Das Konzept einer Gesamtanalyse der Wirtschaft 1te pluralistische Ergänzungsveranstaltung zur Jahrestagung des Vereins für Socialpolitik 2012-09-11
	\item Hirte, Katrin  Das Institut für die Gesamtanalyse der Wirtschaft 1. pluralistischen Ergänzungsveranstaltung zur Jahrestagung 2012-09-11
	\item Ötsch, Walter  Theoriegeschichte als (kritische) Kulturgeschichte 1. pluralistischen Ergänzungsveranstaltung zur Jahrestagung 2012-09-10
	\item Ötsch, Walter  Finanzmärkte und Postdemokratie unbekannt/unknown 2012-08-27
	\item Nordmann, Jürgen  Grenzen der Krisendebatte. Zum Verhältnis von Sach- und Grundsatzdiskussionen in den Printmedien und in den Sozialwissenschaften Sprachliche Konstruktionen sozial- und wirtschaftspolitischer \glqq Krisen\grqq{} in der BRD 2012-07-16
	\item Nordmann, Jürgen  Think-Tanks und die Institutionen der EU. Eine kritische Analyse aus demokratietheoretischer Sicht IPA-Konferenz 2012-07-08
	\item Pühringer, Stephan  Economists and Economics -- Discourse Profiles of Exonomists in the Financial Crisis Joint Conference of AHE, IIPPE and FAPE 2012-07-07
	\item Hirte, Katrin  Economists and Economics -- Discourse Profiles of Exonomists in the Financial Crisis Joint Conference of AHE, IIPPE and FAPE 2012-07-07
	\item Ötsch, Walter  Geschichte und Ausprägungen des Neoliberalismus unbekannt/unknown 2012-06-18
	\item Nordmann, Jürgen  Theorien der liberalen Gesellschaft unbekannt/unknown 2012-06-15
	\item Ötsch, Walter  Folgt nach der Finanz- die Demokratiekrise? unbekannt/unknown 2012-06-15
	\item Ötsch, Walter  Ausgeliefert? Aufgaben und Chancen der Politik angesichts der 'Macht der Märkte' und einer ausgehöhlten Demokratie unbekannt/unknown 2012-05-31
	\item Nordmann, Jürgen  Moderation zu Werner Rügemer: Ratingagenturen unbekannt/unknown 2012-05-11
	\item Ötsch, Walter  Hintergründe zur Staatsschuldenkrise: wie denken die handelnden Akteure? unbekannt/unknown 2012-05-09
	\item Ötsch, Walter  Muss Forschung ökonomisch sein? CityScienceTalk der Langen Nacht der Forschung unbekannt/unknown 2012-04-27
	\item Ötsch, Walter  Ratingagenturen in der neoliberalen Wirtschaft Länder- und Regionenrating 2012-04-26
	\item Ötsch, Walter  Krise in Europa Frühjahrstagung 2012: Katastrophen 2012-04-12
	\item Nordmann, Jürgen  Moderation zum Vortrag \glqq Wem gehört die Welt? Zur Wiederentdeckung der Gemeingüter\grqq{} von Silke Helfrich unbekannt/unknown 2012-03-21
	\item Ötsch, Walter  Staatsschulden: Wer ist schuld, wer zahlt? unbekannt/unknown 2012-03-19
	\item Ötsch, Walter  Ausgeliefert? Aufgaben und Chancen der Politik angesichts der \glqq Macht der Märkte\grqq{} und einer ausgehöhlten Demokratie unbekannt/unknown 2012-03-15
	\item Nordmann, Jürgen  Moderation zum Vortrag "West-End. Das Scheitern der Moderne als kapitalistisches Patriarchat und die Logik der Alternativen von Univ.Prof. Claudia von Werlhof unbekannt/unknown 2012-03-14
	\item Hirte, Katrin  Moderation zum Vortrag \glqq Geld regiert die Welt! Wer regiert das Geld?\grqq{} von Prof. Dr. Margit Kennedy unbekannt/unknown 2012-03-07
	\item Nordmann, Jürgen  Kommentar zum Vortrag \glqq Rebooting Is Not An Option: Toward Equitable Social and Economic Development\grqq{} von Stephanie Seguino unbekannt/unknown 2012-02-29
	\item Hirte, Katrin  Kommentar zum Vortrag \glqq Rebooting Is Not An Option: Toward Equitable Social and Economic Development\grqq{} von Stephanie Seguino unbekannt/unknown 2012-02-29
	\item Hirte, Katrin  Moderation zu: „’Rebooting’ is not an option: Toward equitable social and economic development” von Stephanie Seguino (University of Vermont/USA), (via skype), Wissensturm Linz. „Ökonomia“ -- Wirtschaft aus feministischer Sicht 2012-02-29
	\item Ötsch, Walter  Politik und Wirtschaft, insbesondere zum Neoliberalismus unbekannt/unknown 2011-12-16
	\item Ötsch, Walter  Podiumsdiskussion aus Anlass des 90. Geburtstages von Prof. Kazimierz Laski unbekannt/unknown 2011-12-06
	\item Plaimer, Wolfgang  Postdemokratie in Österreich? Am Beispiel von politischen Entscheidungsprozessen Demokratie! Welche Demokratie? 2011-12-02
	\item Nordmann, Jürgen  Die neoliberale Oligarchie. Zum aktuellen Verhältnis von Besitz und Macht in der Demokratie Demokratie! Welche Demokratie? 2011-12-02
	\item Ötsch, Walter  Retten wir €uropa! Die Frage ist nur, wie? unbekannt/unknown 2011-11-28
	\item Nordmann, Jürgen  Der Gesellschaftsvertrag von Locke bis Rawls unbekannt/unknown 2011-11-25
	\item Ötsch, Walter  Ist der Euro am Ende? unbekannt/unknown 2011-11-22
	\item Hirte, Katrin  Performativity -- ein neues Konzept zur Analyse des Einflusses von Ökonomen? unbekannt/unknown 2011-11-17
	\item Nordmann, Jürgen  Krisen und Alternativen des neoliberalen Modells unbekannt/unknown 2011-11-15
	\item Ötsch, Walter  Wer ist schuld an der Staatsschuld? unbekannt/unknown 2011-11-07
	\item Hirte, Katrin  Gleichheit und Vielfalt als normative Konzeptionen? -- zu den philosophischen Implikationen bei Simone de Beauvoir Momentum 4, Track 5 2011-10-29
	\item Nordmann, Jürgen  Veränderung von Machtverhältnissen in politischen Entscheidungsprozessen Momentum 4, Track 5: Was ist Gleichheit? 2011-10-28
	\item Plaimer, Wolfgang  Veränderung von Machtverhältnissen in politischen Entscheidungsprozessen Momentum 4, Track 5: Was ist Gleichheit? 2011-10-28
	\item Pühringer, Stephan  Gleichheit versus Vielfalt. Ein konstruierter Widerspruch? Momentum 4, Track 5: Was ist Gleichheit? 2011-10-27
	\item Nordmann, Jürgen  Liberalismus und Neoliberalismus unbekannt/unknown 2011-10-15
	\item Ötsch, Walter  Gesellschaft bei Marx und im Neoliberalismus: die Gesellschaft \glqq des Marktes" Arbeitskreis Politische Ökonomie zum Thema "Karl Marx 2011\grqq{} 2011-10-14
	\item Hirte, Katrin  Arbeit bei Marx Arbeitskreis Politische Ökonomie zum Thema \glqq Karl Marx 2011\grqq{} 2011-10-14
	\item Hirte, Katrin  Das Performativity-Konzept und seine soziologietheoretische Fundierung unbekannt/unknown 2011-10-05
	\item Hirte, Katrin  Crowdsourcing -- temporäre virtuelle Gemeinschaften und ihr Regelbezug. Der Fall Guttenplag. 41. Jahrestagung der Gesellschaft für Informatik 2011-10-04
	\item Hirte, Katrin  Netzwerke im Internet -- eine neue kritische Öffentlichkeit? Das Beispiel Guttenberg Dreiländerkongress der Deutschen Gesellschaft für Soziologie, der Österreichischen Gesellschaft für Soziologie und der Schweizerischen Gesellschaft für Soziologie 2011-10-01
	\item Hirte, Katrin  Konkurrierende Vergangenheit -- Geschichte und Gegengeschichte zur Vergangenheit der Nachkriegsgeneration der deutschen Agrarpolitikprofessoren Dreiländerkongress der Deutschen Gesellschaft für Soziologie, der Österreichischen Gesellschaft für Soziologie und der Schweizerischen Gesellschaft für Soziologie 2011-09-29
	\item Nordmann, Jürgen  Keine Alternative. Neoliberale Positionen in den Printmedien nach dem Crash Dreiländerkongress der Deutschen Gesellschaft für Soziologie, der Österreichischen Gesellschaft für Soziologie und der Schweizerischen Gesellschaft für Soziologie 2011-09-29
	\item Hirte, Katrin  Performativity -- ein theoretischer Ansatz zur Wirkungsanalyse der Ökonomie auf die Gesellschaft? Tagung: Krise des Kapitalismus und die Zukunft der ökonomischen Wissenschaft 2011-09-28
	\item Ötsch, Walter  Reasons behind the actual cris of Euro World Festival der International Union of Socialist Youth 2011-07-30
	\item Hirte, Katrin  Ernährungssouveränität unbekannt/unknown 2011-06-09
	\item Ötsch, Walter  Mechanismen der Finanzmärkte, Globalisierung und Wirtschaftskrise Management und Leadership für Frauen 2011-05-28
	\item Nordmann, Jürgen  Thomas Kuhns wissenschaftliche Revolution. Grenzen der Paradigmenlehre unbekannt/unknown 2011-05-27
	\item Hirte, Katrin  Geographische Bezüge in professoralen Netzwerken -- universitäre Lieferbeziehungen in der deutschen Agrarpolitik und Agrarökonomie unbekannt/unknown 2011-05-27
	\item Nordmann, Jürgen  Die Österreichische Schule im Neoliberalismus unbekannt/unknown 2011-05-23
	\item Nordmann, Jürgen  Foucault. Eine Einführung in sein Denken unbekannt/unknown 2011-05-18
	\item Hirte, Katrin  Die professorale Scientific Community der deutschen Agrarökonomie und Agrarpolitik vor und nach 1945 Prosopographische Arbeiten. Inhalte -- Methoden -- Erfahrungen -- Desiderata 2011-05-05
	\item Ötsch, Walter  Akteur und Markt, Subjekt und Ordnung in der ökonomischen Theorie unbekannt/unknown 2011-02-24
	\item Ötsch, Walter  Kapitalismus und Moral Max-Delbrück-Forum 2011-02-23
	\item Ötsch, Walter  Wirtschaftskrise, Krise der Mainstream-Ökonomik -- Chance für die Sozialökonomie? Eduard-Heimann-Colloquiumsreihe: Nach der Krise -- Folgen für Wirtschaft, Wissenschaft und Politik 2011-02-09
	\item Nordmann, Jürgen  Schreibreflexion und Subjekt im Spätwerk von Roland Barthes unbekannt/unknown 2011-01-21
	\item Nordmann, Jürgen  Gibt es eine neoliberale Gesellschaft? Theoretische Überlegungen zur Analyse eines ambivalenten Phänomens unbekannt/unknown 2010-12-03
	\item Ötsch, Walter  Der Markt-Begriff im Neoliberalismus unbekannt/unknown 2010-11-30
	\item Hirte, Katrin  Ego-Netzwerke 1933 bis nach 1945 im Zeitverlauf -- Brüche? Kontinuitäten? Typica? unbekannt/unknown 2010-11-13
	\item   Solidarität im Kapitalismus 3. Momentum Kongress in Hallstatt 2010-10-22
	\item Pühringer, Stephan  Solidarität im Kapitalismus 3. Momentum Kongress in Hallstatt 2010-10-22
	\item Ötsch, Walter  Der Begriff \glqq der Markt" im Diskurs um die Wirtschaft -- Die Tiefenbedeutung von "Markt\grqq{}. Ein Schlüssel zum Verständnis der neoliberal-marktradikalen Gesellschaft unbekannt/unknown 2010-10-22
	\item Hirte, Katrin  Das neoklassische Freihandelsmodell -- Fundament für Entwicklungszusammenarbeit oder Zementierung globaler Ungleichheiten? 3. Momentum Kongress, Hallstadt 2010-10-22
	\item Nordmann, Jürgen  Trash, Skandale und Ratschläge statt Aufklärung und politische Bildung. Über das Zusammenspiel von kommerzialisierten Medien und gemachter Meinung in der neoliberalen Gesellschaft 3. Momentum Kongress, Hallstadt 2010-10-22
	\item Hirte, Katrin  Vielschichtigkeit in der Realität -- Stringenz in der Analyse? Das Problem der Datenbewältigung bei der Erstellung von Ego-Netzwerken. Methodenworkshop Netzwerkforschung in den Geistes- und Sozialwissenschaften 2010-09-18
	\item Nordmann, Jürgen  Die Evolution ökonomischen Wissens und des Wissens über den Kapitalismus. Performativity als Analyseinstrument: das Beispiel der Fabian Society, der Mont Pelerin Society und der Chicagoer Schule Jahrestagung des Ausschusses Evolutorische Ökonomik 2010-07-03
	\item Hirte, Katrin  Die Evolution ökonomischen Wissens und des Wissens über den Kapitalismus. Performativity als Analyseinstrument: das Beispiel der Fabian Society, der Mont Pelerin Society und der Chicagoer Schule Jahrestagung des Ausschusses Evolutorische Ökonomik 2010-07-03
	\item Ötsch, Walter  Die Evolution ökonomischen Wissens und des Wissens über den Kapitalismus. Performativity als Analyseinstrument: das Beispiel der Fabian Society, der Mont Pelerin Society und der Chicagoer Schule Jahrestagung des Ausschusses Evolutorische Ökonomik 2010-07-03
	\item Hirte, Katrin  Historie der professoralen Agrarökonomen Deutschlands 1933 – 1955 unbekannt/unknown 2010-06-24
	\item Hirte, Katrin  Netzwerkanalyse und Zeitverläufe Forschungskolloquiums 2010-06-24
	\item Ötsch, Walter  Wirtschaftskrise, Sparpakete und politische Optionen Klausur der Landtagklubs der SPÖ 2010-06-22
	\item   Aktionsforschung – ein Weg zum Design institutioneller Neuerungen unbekannt/unknown 2010-06-10
	\item Hirte, Katrin  Macht- und Elitetheorien – alte und neue Ansätze Sommerakademie 2010 2010-06-05
	\item Hirte, Katrin  Moderation zu: Neue Machtsysteme in der Postdemokratie Sommerakademie 2010 2010-06-04
	\item Nordmann, Jürgen  Moderation zu: Neue Machtsysteme in der Postdemokratie Sommerakademie 2010 2010-06-04
	\item Ötsch, Walter  Moderation zu: Neue Machtsysteme in der Postdemokratie Sommerakademie 2010 2010-06-04
	\item Hirte, Katrin  Netzwerkanalytische Betrachtungen zu historischen Verläufen -- Vorteile und neue Erkenntnisse? (am Beispiel der Historie der professoralen Agrarökonomen Deutschlands 1933 bis 1955) Workshop Historische Netzwerkforschung 2010-05-30
	\item Ötsch, Walter  Wie pluralistisch ist die Volkswirtschaftslehre an der JKU wirklich? Podiumsdiskussion Initiative kritischer Studierender 2010-05-18
	\item Hirte, Katrin  Performativität -- ein tragfähiger Ansatz, um den Einfluss von ökonomischen Theorien auf reale Wirtschaftsabläufe zu analysieren? Forschungskolloquiums 2010-04-29
	\item Nordmann, Jürgen  Globale Tendenzen der Weltwirtschaft im 20. Jahrhundert: von der Krise der 70er Jahre zur Großen Krise ab 2008 Ringvorlesung: Global History -- Society and Governance 2010-04-27
	\item Ötsch, Walter  Globale Tendenzen der Weltwirtschaft im 20. Jahrhundert: von der Krise der 70er Jahre zur Großen Krise ab 2008 Ringvorlesung: Global History -- Society and Governance 2010-04-27
	\item Ötsch, Walter  Mythos Markt. Marktradikale Propaganda und ökonomische Theorie Bündnis für Eine Welt/ÖIE mit ATTAC-Kärnten 2010-04-22
	\item Ötsch, Walter  Wie Wirtschaft Welt bewegt. Die großen ökonomischen Modelle auf dem Prüfstand Podiumsdiskussion zum Buch von Hans Bürger und K.W. Rothschild 2010-04-12
	\item Ötsch, Walter  Was ist Grün Alternative gefragt?! Politische Theorie und Geschichte der Grünen 2010-03-13
	\item Ötsch, Walter  Arbeitsmarkt der Zukunft, Zukunft der Sozialwirtschaft Sozialwirtschaft. Wandel-Zukunft-Chancen 2010-02-25
	\item Ötsch, Walter  Keynesianismus und Neoliberalismus. Zur Geschichte des Wirtschaftssystems und Finanzmarktkrise und Lösungsvorschläge. Zur aktuellen Situation des Kapitalismus Aktuelle Fragen der Wirtschaftspolitik 2010-02-23
	\item Ötsch, Walter  Globale Wirtschaft -- Globale Krise Lehrerfortbildung 2010-02-22
	\item Ötsch, Walter  Nach der Wirtschaftskrise: was haben wir gelernt? Podiumsdiskussion 2010-02-18
	\item Ötsch, Walter  Dem Volk auf's Maul g'schaut? -- Politische Sprache in der aktuellen Asyldebatte unbekannt/unknown 2010-01-28
	\item Hirte, Katrin  Preistheorien und Preisgestaltung – eine systematisierende Hinterfragung (Beispiel Milchmarkt und Milchmarktregelungen). Milchtagung 2009 2009-03-03
	\item Hirte, Katrin  Institutionalisierungsprozesse im Ökologischen Landbau in den Neuen Bundesländern. Die GÄA e. V. und Biopark. 10. Wissenschaftstagung Ökolandbau 2009-02-11
\end{enumerate}
\subsection*{Mitgliedschaft/Funktion}
\begin{enumerate}
	\item Eder, Julia Theresa  Mitglied Vorstand Mattersburger Kreis für Entwicklungspolitik (Externe Organisation)  2025-01-14
	\item Hornykewycz, Anna  Society for the Advancement of Socio-Economics (SASE) (Externe Organisation)  2025-01-01
	\item Rath, Johanna  YSI Philosophy of Economics Working Group Coordinator (Externe Organisation)  2024-11-21
	\item Pühringer, Stephan  Organization: Research Area Economic Sociology (Externe Organisation)  2024-09-03
	\item Gräbner-Radkowitsch, Claudius  European Association for Evolutionary Political Economy (EAEPE) (Externe Organisation)  2024-03-01
	\item Kapeller, Jakob  Member of the Equal Opportunities Commission for Germany (Externe Organisation)  2024-02-01
	\item Theine, Hendrik  Kontext Institut (Externe Organisation)  2024-01-01
	\item Hornykewycz, Anna  Member of Pre-Conference Organizing Team, European Association for Evolutionary Political Economy (Externe Organisation)  2023-01-01
	\item Kapeller, Jakob  Gleichstellungskommission zur Verfassung des Gleichstellungsberichtes der Bundesregierung (Externe Organisation)  2023-01-01
	\item Hager, Theresa  Member of Social Media Team of European Association for Evolutionary Political Economy (EAEPE) (Externe Organisation)  2022-12-01
	\item Hager, Theresa  Women and Gender Forum Commitee of the Society for the Advancement of Socio-Economics (Externe Organisation)  2022-11-01
	\item Hornykewycz, Anna  SASE (Society for the Advancement of Socio-Economics) (Externe Organisation)  2022-06-01
	\item Hager, Theresa  SASE (Society for the Advancement of Socio-Economics) (Externe Organisation)  2022-03-01
	\item Hager, Theresa  European Association for Evolutionary Political Economy (Externe Organisation)  2022-01-01
	\item Kapeller, Jakob  Inclusive Economics Prize Jury (Externe Organisation)  2022-01-01
	\item Gräbner-Radkowitsch, Claudius  Journal of Institutional Economics (Externe Organisation)  2021-10-01
	\item Aistleitner, Matthias  Österreichische Gesellschaft für Soziologie (Externe Organisation)  2021-01-01
	\item Pühringer, Stephan  Österreichische Gesellschaft für Soziologie (Externe Organisation)  2021-01-01
	\item Kapeller, Jakob  Jury Vorsitz Kurt-Rothschild-Preis (Externe Organisation)  2020-01-01
	\item Pühringer, Stephan  Research Area Coordinator im Bereich \glqq Economic Sociology\grqq{} der European Association for Evolutionary Political Economy (EAEPE) (Externe Organisation)  2019-09-13
	\item Hornykewycz, Anna  European Association for Evolutionary Political Economy (Externe Organisation)  2019-06-01
	\item Aistleitner, Matthias  American Economic Association (Externe Organisation)  2019-01-01
	\item Gräbner-Radkowitsch, Claudius  Gesellschaft für sozio*ökonomische Bildung \& Wissenschaft (Externe Organisation)  2019-01-01
	\item Gräbner-Radkowitsch, Claudius  Verein für Socialpolitik (Externe Organisation)  2019-01-01
	\item Kapeller, Jakob  SASE (Society for the Advancement of Socio-Economics) (Externe Organisation)  2019-01-01
	\item Hafele, Jakob  Structural change and technological upgrading in times of globalization (Veranstaltung) Structural change and technological upgrading in times of globalization 2018-11-08
	\item Gräbner-Radkowitsch, Claudius  Structural change and technological upgrading in times of globalization (Veranstaltung) Structural change and technological upgrading in times of globalization 2018-11-08
	\item Schütz, Bernhard  Young economists conference der Arbeiterkammer Wien zum Thema \glqq Welfare State under Attack" (Veranstaltung) Young economists conference der Arbeiterkammer Wien zum Thema "Welfare State under Attack\grqq{} 2018-10-08
	\item Pühringer, Stephan  Young economists conference der Arbeiterkammer Wien zum Thema \glqq Welfare State under Attack" (Veranstaltung) Young economists conference der Arbeiterkammer Wien zum Thema "Welfare State under Attack\grqq{} 2018-10-08
	\item Aistleitner, Matthias  SASE (Society for the Advancement of Socio-Economics) (Externe Organisation)  2018-01-01
	\item Gräbner-Radkowitsch, Claudius  Netzwerk Ökonomische Bildung und Beratung (Externe Organisation)  2018-01-01
	\item Pühringer, Stephan  SASE (Society for the Advancement of Socio-Economics) (Externe Organisation)  2018-01-01
	\item Kapeller, Jakob  Gesellschaft für sozio*ökonomische Bildung \& Wissenschaft (Externe Organisation)  2018-01-01
	\item Pühringer, Stephan  Association for Social Economics (Externe Organisation)  2018-01-01
	\item Pühringer, Stephan  Arbeitskreis Politische Ökonomie (Externe Organisation)  2018-01-01
	\item Aistleitner, Matthias  European Association for Evolutionary Political Economy (Externe Organisation)  2018-01-01
	\item Schütz, Bernhard  Young Economists Conference der AK Wien: European Integration at a Crossroads (Veranstaltung) Young Economists Conference der AK Wien: European Integration at a Crossroads 2017-10-12
	\item   Institutions, Industry Structure, Evolution -- Complexity Approaches to Economics (Veranstaltung) Institutions, Industry Structure, Evolution -- Complexity Approaches to Economics 2017-09-21
	\item Gräbner-Radkowitsch, Claudius  Institutions, Industry Structure, Evolution -- Complexity Approaches to Economics (Veranstaltung) Institutions, Industry Structure, Evolution -- Complexity Approaches to Economics 2017-09-21
	\item Gräbner-Radkowitsch, Claudius  World Interdisciplinary Network for Institutional Research (Externe Organisation)  2017-01-01
	\item Pühringer, Stephan  Gesellschaft für sozio*ökonomische Bildung \& Wissenschaft (Externe Organisation)  2017-01-01
	\item Kapeller, Jakob  World Economics Association (Externe Organisation)  2017-01-01
	\item Pühringer, Stephan  European Association for Evolutionary Political Economy (Externe Organisation)  2017-01-01
	\item Hirte, Katrin  Deutsche Gesellschaft für Netzwerkforschung (Externe Organisation)  2016-12-05
	\item Schütz, Bernhard  Young Economists Conference der AK Wien: (UN-)EMPLOYMENT IN TIMES OF TECHNICAL PROGRESS AND ECONOMIC CRISIS (Externe Organisation)  2016-10-04
	\item Pühringer, Stephan  Young Economists Conference der AK Wien: (UN-)EMPLOYMENT IN TIMES OF TECHNICAL PROGRESS AND ECONOMIC CRISIS (Externe Organisation)  2016-10-04
	\item Gräbner-Radkowitsch, Claudius  Complex Systems Society (Externe Organisation)  2016-01-01
	\item Heimberger, Philipp  European Association for Evolutionary Political Economy (Externe Organisation)  2016-01-01
	\item Gräbner-Radkowitsch, Claudius  Association for Evolutionary Economics (Externe Organisation)  2015-01-01
	\item Heimberger, Philipp  Forum for Macroeconomics and Macroeconomic Policies (Externe Organisation)  2015-01-01
	\item Kapeller, Jakob  Association for Heterodox Economics (Externe Organisation)  2015-01-01
	\item Kapeller, Jakob  Association for Evolutionary Economics (Externe Organisation)  2015-01-01
	\item Kapeller, Jakob  American Economic Association (Externe Organisation)  2014-01-01
	\item Pühringer, Stephan  American Economic Association (Externe Organisation)  2014-01-01
	\item Gräbner-Radkowitsch, Claudius  European Association for Evolutionary Political Economy (Externe Organisation)  2014-01-01
	\item Gräbner-Radkowitsch, Claudius  American Economic Association (Externe Organisation)  2014-01-01
	\item Kapeller, Jakob  European Association for Evolutionary Political Economy (Externe Organisation)  2013-01-01
	\item Gräbner-Radkowitsch, Claudius  Netzwerk Plurale Ökonomik (Externe Organisation)  2010-01-01
	\item Hirte, Katrin  Historical Network Research (Externe Organisation)  2009-11-01
	\item Kapeller, Jakob  Momentum Congress Hallstatt (Externe Organisation)  2008-01-01
\end{enumerate}
\subsection*{Gutachtertätigkeiten}
\begin{enumerate}
	\item Pühringer, Stephan  Routledge, Frontiers of Political Economy  2023-02-01
	\item Hirte, Katrin  Evangelisches Studienwerk Villigst  2022-12-01
	\item Gräbner-Radkowitsch, Claudius  DFG-Einzelmaßnahme  2022-11-01
	\item Gräbner-Radkowitsch, Claudius  Laudes Foundation  2022-10-01
	\item Pühringer, Stephan  Routledge Frontiers of Political Economy Book Series  2022-07-01
	\item Gräbner-Radkowitsch, Claudius  Handbook of Complexity Economics  2022-07-01
	\item Gräbner-Radkowitsch, Claudius  Marietta Blau Stipendium  2022-05-01
	\item Kapeller, Jakob  Israel Science Foundation (ISF)  2022-03-01
	\item Pühringer, Stephan  Joan Robinson Prize EAEPE  2021-06-01
	\item Pühringer, Stephan  Sozioökonomische Bildung und Wissenschaft  2019-06-05
	\item Pühringer, Stephan  Themenband GSÖBW  2019-06-01
	\item Pühringer, Stephan  Routledge Political Economy Book Series  2019-02-01
	\item Hirte, Katrin  Perspektiven einer pluralen Ökonomik  2017-08-01
	\item Kapeller, Jakob  Perspektiven einer pluralen Ökonomik  2017-04-01
	\item Hirte, Katrin  Perspektiven einer pluralen Ökonomik  2017-04-01
\end{enumerate}
