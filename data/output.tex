\begin{itemize}
\item The political economy of academic publishing: On the commodification of a public good: Pühringer S., Rath J., Griesebner T. , in PLoS One, Vol. 16, Nr. 6, 2021
\item Stability , fairness and random walks in the bargaining problem: Kapeller J., Steinerberger S. , in Physica A: Statistical Mechanics and its Applications, Vol. 488, Elsevier, Seite(n) 60-71, 2017
\item Biased trade narratives and its impact on development studies: a multi-level mixed-method approach: Aistleitner M., Pühringer S. , in European Journal of Development Research, Vol. 35, Seite(n) 1322-1346, 2023
\item Can a European wealth tax close the green investment gap?: Kapeller J., Wildauer R., Leitch S. , in Ecological Economics, Nr. 209, 2023
\item Competing for Sustainability? An Institutionalist Analysis of the New Development Model of the European Union: Gräbner-Radkowitsch C., Hager T., Hornykewycz A. , in Journal of Economic Issues, Vol. 57, Nr. 2, Seite(n) 676-683, 2023
\item Degrowth and the Global South: The Twin Problem of Global Dependencies: Gräbner-Radkowitsch C., Strunk B. , in Ecological Economics, Vol. 213, 2023
\item Degrowth and the Global South? How Institutionalism can Complement a Timely Discourse on Ecologically Sustainable Development in an Unequal World: Gräbner-Radkowitsch C., Strunk B. , in Journal of Economic Issues, Vol. 57, Nr. 2, Seite(n) 476-483, 2023
\item Political sovereignty in tension with global capitalist accumulation: the case of the European socio-economic strategy: Porak L. , in Critical Policy Studies, 2023
\item The authors of economics journals revisited: evidence from a large-scale replication of Hodgson and Rothman (1999): Aistleitner M., Kapeller J., Kronberger D. , in Journal of Institutional Economics, Vol. 19, Nr. 1, Seite(n) 86–101, 2023
\item Winning city competition with a social agenda. The competition imaginary in Viennese urban development plans: Altreiter C., Azevedo S., Porak L., Pühringer S., Wolfmayr G. , in Urban Research & Practice, Seite(n) 10.1080/17535069.2022.2161834, 2023
\item Capability accumulation and product innovation: an agent-based perspective: Gräbner C., Hornykewycz A. , in Journal of Evolutionary Economics, Vol. 32, Seite(n) 87-121, 2022
\item Critical junctures of hope: how to bridge the gap between the necessary and the feasible?: Kapeller J., Huwe V. , in GAIA - Ecological Perspectives for Science and Society, Vol. 31, Nr. 1, Seite(n) 10-13, 2022
\item Die Netzwerkanalyse und der Umgang mit ihren Forschungsergebnissen: Hirte K., Ötsch W., Pühringer S. , in Berliner Journal für Soziologie, Nr. 32, Seite(n) 153-163, 2022
\item Divided We Stand? Professional Consensus and Political Conflict in Academic Economics: Pühringer S., Beyer K. , in Journal of Economic Issues, forthcoming, 2022
\item Paradigms and Policies: The state of economics in the German-speaking countries: Kapeller J., Pühringer S., Grimm C. , in Review of International Political Economy, Vol. 29, Nr. 4, Seite(n) 1183-1210, 2022
\item Polanyi and Schumpeter: Transitional processes via societal spheres: Hager T., Heck I., Rath J. , in The European Journal for the History of Economic Thought, Vol. 29, Nr. 6, Seite(n) 1089–1110, 2022
\item The evolution of debtor-creditor relationships within a monetary union: Trade imbalances, excess reserves and economic policy: Gräbner-Radkowitsch C., Heimberger P., Kapeller J., Landesmann M., Schütz B. , in Structural Change and Economic Dynamics, Vol. 62, Seite(n) 262-289, 2022
\item Tracing the invisible rich: a new approach to modelling Pareto tails in survey data: Kapeller J., Wildauer R. , in Labour Economics, Vol. 75, Nr. 102145, 2022
\item Corporate tax competition: A meta-analysis: Heimberger P. , in European Journal of Political Economy, Vol. 69, Nr. 1020002, 2021
\item Creating a pluralist paradigm: An application to the minimum wage debate: Schütz B. , in Journal of Economic Issues, Vol. 55, Nr. 1, Seite(n) 103-124, 2021
\item Does economic globalisation promote economic growth?: Heimberger P. , in The World Economy, forthcoming, doi.org/10.1111/twec.13235, 2021
\item Does economic globalization affect government spending? A meta-analysis: Heimberger P. , in Public Choice, Vol. 187, Seite(n) 349-374, 2021
\item Does employment protection affect unemployment? A meta-analysis: Heimberger P. , in Oxford Economic Papers, Vol. 73, Nr. 3, Seite(n) 982-1007, 2021
\item The Trade (Policy) Discourse in Top Economic Journals: Aistleitner M., Pühringer S. , in New Political Economy, Vol. 26, Nr. 5, Seite(n) 748-764, 2021
\item The collapse of cooperation: The endogeneity of institutional break-up and its asymmetry with emergence: Cordes C., Elsner W., Gräbner C., Heinrich T., Henkel J., Schwardt H., Schwesinger G., Su T. , in Journal of Evolutionary Economics, Vol. 31, Nr. 4, Seite(n) 1291-1315, 2021
\item Theory and Empirics of Capability Accumulation: Implications for Macroeconomic Modelling: Aistleitner M., Gräbner C., Hornykewycz A. , in Research Policy, Vol. 50, Nr. 6, e-no. 104258, 2021
\item Trust and Social Control. The Sources of Stability in Informal Value Transfer Systems: Gräbner C., Elsner W., Lascaux A. , in Computational Economics, Nr. 58, Seite(n) 1077-1102, 2021
\item Understanding economic openness: A review of existing measures: Gräbner C., Heimberger P., Kapeller J., Springholz F. , in Review of World Economics, Vol. 157, Nr. 1, Seite(n) 87-120, 2021
\item Does economic globalisation affect income inequality? A meta-analysis: Heimberger P. , in The World Economy, Seite(n) 1-23, 2020
\item From the ʻplanning euphoriaʼ to the ʻbitter economic truthʼ: The Transmission of economic ideas into German Labour Market Policies in the 1960s and 2000s: Pühringer S., Griesser M. , in Critical Discourse Studies, Vol. 17, Nr. 5, Seite(n) 476-493, 2020
\item Is the Eurozone disintegrating? Macroeconomic divergence, structural polarization, trade and fragility: Gräbner C., Heimberger P., Kapeller J., Schütz B. , in Cambridge Journal of Economics, Vol. 44, Nr. 3, Seite(n) 647-669, 2020
\item Pluralism in economics: its critiques and their lessons: Gräbner C., Strunk B. , in Journal of Economic Methodology, Vol. 27, Nr. 4, Seite(n) 311-329, 2020
\item Structural change in times of increasing openness: assessing path dependency in European economic integration: Gräbner C., Heimberger P., Kapeller J., Schütz B. , in Journal of Evolutionary Economics, Vol. 30, Nr. 5, Seite(n) 1467-1495, 2020
\item The dynamic effects of fiscal consolidation episodes on income inequality: Evidence for 17 OECD countries over 1978-2013: Heimberger P. , in Empirica, Vol. 47, Seite(n) 53-81, 2020
\item The power of economic models: The case of the EU's fiscal regulation framework: Heimberger P., Huber J., Kapeller J. , in Socio-Economic Review, Vol. 18, Nr. 2, Seite(n) 337-366, 2020
\item Citation Patterns in Economics and Beyond: Assessing the Peculiarities of Economics from Two Scientometric Perspectives: Aistleitner M., Kapeller J., Steinerberger S. , in Science in Context, Vol. 32, Nr. 4, Seite(n) 361-380, 2019
\item Getting the Best of Both Worlds? Developing Complementary Equation-Based and Agent-Based Models: Gräbner C., Bale C., Furtado B., Alvarez-Pereira B., Gentile J., Henderson H., Lipari F. , in Computational Economics, Vol. 53, Nr. 2, Seite(n) 763-782, 2019
\item Introduction: change and persistence in contemporary economics: Kapeller J., Meyer D. , in Science in Context, Vol. 32, Nr. 4, Seite(n) 357-360, 2019
\item Government policies and financial crises: mitigation, postponement or prevention?: Landesmann M., Kapeller J., Mohr F., Schütz B. , in Cambridge Journal of Economics, Vol. 42, Nr. 2, Seite(n) 309-330, 2018
\item How to Relate Models to Reality? An Epistemological Framework for the Validation and Verification of Computational Models: Gräbner C. , in Journal of Artificial Societies and Social Simulation, Vol. 21, Nr. 3, 2018
\item Open strategy-making with crowds and communities: Comparing Wikimedia and Creative Commons: Kapeller J., Dobusch L. , in Long Range Planning, Vol. 51, Nr. 4, Seite(n) 561-579, 2018
\item The Power of Scientometrics and the Development of Economics: Aistleitner M., Kapeller J., Steinerberger S. , in Journal of Economic Issues, Vol. 52, Nr. 3, Routledge, Seite(n) 816-834, 2018
\item To trust or to control: Informal value transfer systems and computational analysis in institutional economics: Gräbner C., Elsner W., Lascaux A. , in Journal of Economic Issues, Vol. 52, Nr. 2, Seite(n) 559-569, 2018
\item Did Fiscal Consolidation Cause the Double-Dip Recession in the Euro Area?: Heimberger P. , in Review of Keynesian Economics, Vol. 5, Nr. 3, Seite(n) 439-458, 2017
\item Images and imaginaries of unemployed people. Discursive shifts in the transition from active to activating labour market policies in Germany: Griesser M. , in Critical Social Policy, Seite(n) [online first], 2017
\item Right-wing populism and market-fundamentalism. Two mutually reinforcing threats to democracy in the 21st century: Ötsch W., Pühringer S. , in Journal of Language and Politics, Vol. 16, Nr. 4, Seite(n) 497-509, 2017
\item The Complementary Relationship Between Institutional and Complexity Economics: The Example of Deep Mechanismic Explanations: Gräbner C. , in Journal of Economic Issues, Vol. 51, Nr. 2, Seite(n) 392-400, 2017
\item The NAIRU determinants: what’s structural about unemployment in Europe?: Heimberger P., Kapeller J., Schütz B. , in Journal of Policy Modeling, Vol. 39, Nr. 5, Seite(n) 883-908, 2017
\item The performativity of potential output: Pro-cyclicality and path dependency in coordinating European fiscal policies: Heimberger P., Kapeller J. , in Review of International Political Economy, Vol. 24, Nr. 5, Seite(n) 904-928, 2017
\item Agent-based computational models - a formal heuristic for institutionalist pattern modelling?: Gräbner C. , in Journal of Institutional Economics, Vol. 12, Nr. 1, Seite(n) 241-261, 2016
\item Correcting for the Missing Rich: An Application to Wealth Survey Data: Eckerstorfer P., Halak J., Kapeller J., Schütz B., Springholz F., Wildauer R. , in Review of Income and Wealth, Vol. 62, Nr. 4, Seite(n) 605-627, 2016
\item Emergent Phenomena in Scientific Publishing: A Simulation Exercise: Kapeller J., Steinerberger S. , in Research Policy, Vol. 45, Nr. 10, Seite(n) 1945–1952, 2016
\item Errata in 'The Political Economy of the Kuznets Curve': Álvarez Pereira B., Henderson H., Lipari F., Furtado B., Bale C., Gräbner C., Gentile J. , in Review of Development Economics, Vol. 20, Nr. 4, Seite(n) 817-819, 2016
\item Conspicuous Consumption, Inequality and Debt: The Nature of Consumption-driven Profit-led Regimes: Kapeller J., Schütz B. , in Metroeconomica, Vol. 66, Nr. 1, Seite(n) 51-70, 2015
\item New Perspectives on  Institutionalist Pattern Modeling: Systemism, Complexity and  Agent-Based modeling: Gräbner C., Kapeller J. , in Journal of Economic Issues, Vol. 49, Nr. 2, Seite(n) 433-440, 2015
\item The financial crisis as a heart attack: Discourse profiles of economists in the financial crisis: Pühringer S., Hirte K. , in Journal of Language and Politics, Vol. 14, Nr. 4, Seite(n) 599-626, 2015
\item Modeling the Evolution of Preferences: An Answer to Schubert and Cordes: Kapeller J., Steinerberger S. , in Journal of Institutional Economics, Vol. 10, Nr. 2, Seite(n) 337-347, 2014
\item Exploring Pluralist Economics: The Case of the Minsky-Veblen Cycles: Kapeller J., Schütz B. , in Journal of Economic Issues, Vol. 47, Nr. 2, Seite(n) 515-524, 2013
\item Model-Platonism in Economics: On a classical epistemological critique: Kapeller J. , in Journal of Institutional Economics, Vol. 9, Nr. 2, Seite(n) 199-221, 2013
\item Practicing Pluralism: A Rejoinder to W. Robert Brazelton: Dobusch L., Kapeller J. , in Journal of Economic Issues, Vol. 47, Nr. 4, Seite(n) 1035-1037, 2013
\item The grounds of solidarity: From liberty to loyalty: Kapeller J., Wolkenstein F. , in European Journal of Social Theory, Vol. 16, Nr. 4, Seite(n) 476-491, 2013
\item The impossibility of rational consumer choice - A problem and its solution: Kapeller J., Schütz B., Steinerberger S. , in Journal of Evolutionary Economics, Vol. 23, Nr. 1, Seite(n) 39-60, 2013
\item Heterodox United vs. Mainstream City? Sketching a framework for interested pluralism in economics: Dobusch L., Kapeller J. , in Journal of Economic Issues, Vol. 46, Nr. 4, Seite(n) 1035-1057, 2012
\item Citation Metrics: Serious drawbacks, perverse incentives and strategic options for heterodox economics: Kapeller J. , in American Journal of Economics and Sociology, Vol. 69, Nr. 5, Seite(n) 1376-1408, 2010
\item Some critical notes on citation metrics and heterodox economics: Kapeller J. , in Review of Radical Political Economics, Vol. 42, Nr. 3, Seite(n) 330-337, 2010
\item Why is Economics not an Evolutionary Science? New Answers to Veblen's old Question.: Dobusch L., Kapeller J. , in Journal of Economic Issues, Vol. 43, Nr. 4, Seite(n) 867-898, 2009
\item Rechtspopulismus: Ein Gesellschaftsbild mit eskalierender Wirkung: Ötsch W. , in Salzburger Theologische Zeitschrift, Vol. 21, Nr. 1, Seite(n) 7-22, 2018
\item Imaginative Grundlagen bei Adam Smith. Aspekte von Bildlichkeit und ihrem Verlust in der Geschichte der Ökonomik: Ötsch W. , in Allgemeine Zeitschrift für Philosophie, Vol. 41, Nr. 3, Seite(n) 315-340, 2016
\item Differential impacts of electricity access on educational outcomes: Evidence from Uganda: Buyinza F., Kapeller J., Senono V., Anber M. , in Electricity Journal, Vol. 37, Nr. 1, Seite(n) 1-10, 2024
\item Economization: The (re-)organization of knowledge and ignorance according to ‘the market’: Steffestun T., Ötsch W. , in ephemera, Vol. 23, Nr. 1, Seite(n) 133-159, 2023
\item Rezension zu Markus Marterbauer/Martin Schürz: Angst und Angstmacherei: Hubmann G., Kapeller J. , in WISO – Wirtschafts- und Sozialpolitische Zeitschrift des ISW, Vol. 46, Nr. 1, Seite(n) 102-109, 2023
\item The Social Field of Elite Trade Economists: A Quantitative Social Studies of Economics Perspective: Aistleitner M., Pühringer S. , in Oeconomia, Vol. 13, Nr. 2, Seite(n) 475-515, 2023
\item Wettbewerbsfähige Nachhaltigkeit: eine Historisch-Materialistische Analyse der Ideen, Institutionen und Machtverhältnisse in der europäischen grünen Transformation: Porak L. , in Momentum Quarterly, Vol. 12, Nr. 1, Seite(n) 65-83, 2023
\item Die Budgetsemielastizität und ihre Auswirkungen auf Verschuldungsspielräume im Rahmen der Schuldenbremse: Heimberger P., Schütz B. , in Wirtschaftsdienst, Vol. 102, Nr. 11, Seite(n) 834-837, 2022
\item Notizen zum ökonomischen Element in der politischen Doktrinbildung: Kapeller J., Hubmann G. , in Zeitschrift für Wirtschafts- und Unternehmensethik, Vol. 23, Seite(n) 34-37, 2022
\item Trade Models in the European Union: Gräbner-Radkowitsch C., Tamesberger D., Heimberger P., Kapelari T., Kapeller J. , in Economic Annals, Vol. 67, Nr. 235, Seite(n) 7-36, 2022
\item A Fitting Pareto tails to wealth survey data: A practitioners’ guide: Kapeller J., Wildauer R. , in Journal of Income Distribution, 2021
\item Intangible Flow Theory: A New Way for Conceptualizing Embeddedness?: Kapeller J. , in Accounting, Economics and Law, Vol. 14, Nr. 1, Seite(n) 159-164, 2021
\item Introduction to the symposium: The Complexity of Institutions: Theory and Computational Models: Gräbner C., Heinrich T. , in Forum for Social Economics, Vol. 50, Nr. 2, Seite(n) 153-156, 2021
\item Monopolies in Science Publishing. A Black Hole for Public Spending?: Pühringer S., Rath J. , in Journal of Management Information and Decision Sciences, Vol. 24, Nr. 6, Seite(n) 1-5, 2021
\item Polarisierung oder Konvergenz? Zur ökonomischen Zukunft des vereinten Europas: Kapeller J. , in WISO direkt (Analysen und Konzepte zur Wirtschafts- und Sozialpolitik), 2021
\item Standortwettbewerb und Deindustrialisierung: Das Beispiel MAN als Lehrbuchfall: Kapeller J., Gräbner C. , in WISO - Wirtschafts- und sozialpolitische Zeitschrift, Vol. 44, Nr. 4, Seite(n) 34-52, 2021
\item Strategien für einen Wandel der ökonomischen Lehre: Porak L., Schröter G. , in Forschungsjournal Soziale Bewegungen, Vol. 34, Nr. 4, Seite(n) 718-729, 2021
\item Unternehmenskonzentrationen in der Fleischbranche und die performative Rolle der Agrarökonomik: Hirte K. , in ÖZS - Österreichische Zeitschrift für Soziologie, Vol. 46, Nr. 2, Seite(n) 187-206, 2021
\item Vom empiristischen Humanismus zum partizipativen Sozialismus – Review von Thomas Piketty ‚Kapital und Ideologie‘: Kapeller J., Rehm M. , in Soziologische Revue, Vol. 44, Nr. 1, Seite(n) 25-33, 2021
\item What is structural about unemployment in OECD countries?: Heimberger P. , in Review of Social Economy, Vol. 79, Nr. 2, Seite(n) 380-412, 2021
\item Zur Pluralität in der ökonomischen Politikberatung in Deutschland. Eine empirische Untersuchung: Pühringer S. , in Leviathan - Berliner Zeitschrift für Sozialwissenschaft, Vol. 49, Seite(n) 243-265, 2021
\item Pandemic pushes polarisation: the Corona crisis and macroeconomic divergence in the Eurozone: Gräbner C., Heimberger P., Kapeller J. , in Journal of Industrial and Business Economics, Vol. 47, Nr. 3, Seite(n) 425-438, 2020
\item Vermögenskonzentration in Österreich: Ein Update auf Basis des HFCS 2017: Kapeller J., Wildauer R., Heck I. , in Wirtschaft und Gesellschaft, Nr. 206, Seite(n) 1-38, 2020
\item Arbeitsmarktinstitutionen, Kapitalakkumulation und Arbeitslosigkeit in OECD-Ländern, Wirtschaft und Gesellschaft: Heimberger P. , in Wirtschaft und Gesellschaft, Vol. 45, Nr. 1, 2019
\item Beyond equilibrium: revisiting two-sided markets from an agent-based modelling perspective: Heinrich T., Gräbner C. , in International Journal of Computational Economics and Econometrics, Vol. 9, Nr. 3, Seite(n) 153-180, 2019
\item Deutungsrahmen der aktiven Arbeitsmarktpolitik: ein deutsch-österreichischer Vergleich von diskursiven Frames aus Anlass von 50 Jahren Arbeits(markt)förderungsgesetz: Griesser M. , in Momentum Quarterly, Vol. 8, Nr. 3, Seite(n) 166-182, 2019
\item Ideas have Consequences: Eine vergleichende Analyse zur transformativen Rolle von Ideen.: Grimm C. , in Momentum Quarterly, Vol. 8, Nr. 4, Seite(n) 183-247, 2019
\item The heterogeneous relationship between income and inequality: a panel co-integration approach: Flechtner S., Gräbner C. , in Economics Bulletin, Vol. 39, Nr. 4, Seite(n) 2540–2549, 2019
\item What economics education is missing: the real world: Pühringer S., Bäuerle L. , in International Journal of Social Economics, Vol. 46, Nr. 8, Seite(n) 977-991, 2019
\item ’Erkühne Dich, weise zu sein!‘ Grundlegung einer Gemeinsinn-Ökonomie: Ötsch W., Graupe S., Loske R. , in GWP - Gesellschaft, Wirtschaft, Politik, Vol. 68, Nr. 2, Seite(n) 243-250, 2019
\item Formal Approaches to Socio-economic Analysis - Past and Perspectives: Gräbner C. , in Forum for Social Economics, Vol. 47, Nr. 1, Seite(n) 32-63, 2018
\item Freiheitliche Flügelkämpfe? (Historische) Konfliktlinien in der FPÖ: Beyer K., Pühringer S. , in BEIGEWUM, in Kurswechsel, Nr. 3, Seite(n) 19-27, 2018
\item Neoliberalism and Right-wing Populism: conceptual analogies: Pühringer S., Ötsch W. , in Forum for Social Economics, Vol. 47, Nr. 2, Seite(n) 192-203, 2018
\item The dynamics of and on networks: an introduction: Gräbner C., Heinrich T., Kudic M., Vermeulen B. , in International Journal of Computational Economics and Econometrics, Vol. 8, Nr. 3/4, Seite(n) 229-241, 2018
\item Wachstum? Wohlstand und Lebensqualität!: Griesser M., Brand U. , in Momentum Quarterly, Vol. 7, Nr. 2, Seite(n) 53-72, 2018
\item Wirtschaftspolitische Positionen österreichischer Parteien im historischen Verlauf. Die Ausgestaltung österreichischer Parteiprogrammatiken hinsichtlich neoliberalen Gedankenguts: Grimm C. , in Momentum Quarterly, Vol. 7, Nr. 3, Seite(n) 136-154, 2018
\item Bestände und Konzentration privater Vermögen in Österreich: Kapeller J., Ferschli B., Schütz B., Wildauer R. , in Arbeiterkammer Wien, in Wirtschaft und Gesellschaft, Vol. 43, Nr. 4, Seite(n) 499-534, 2017
\item The Complexity of Economies and Pluralism in Economics: Gräbner C. , in Journal of Contextual Economics, Vol. 137, Nr. 3, Seite(n) 193-225, 2017
\item Von Onlineplattformen und mittelalterlichen Märkten - Gleichgewichtsmodelle und agentenbasierte Modellierung zweiseitiger Märkte: Gräbner C., Heinrich T. , in TATuP - Zeitschrift für Technikfolgenabschätzung in Theorie und Praxis, Vol. 26, Nr. 3, Seite(n) 23-29, 2017
\item Von der sozialen Neuzusammensetzung zur gewerkschaftlichen Erneuerung? MigrantInnen als Zielgruppe der österreichischen Gewerkschaftsbewegung: Griesser M., Sauer B. , in ÖZS - Österreichische Zeitschrift für Soziologie, Seite(n) 147-166, 2017
\item Was denken (zukünftige) ÖkonomInnen?: Einblicke in die politische und gesellschaftliche Wirkmächtigkeit ökonomischen Denkens.: Pühringer S., Bäuerle L., Engarntner T. , in GWP - Gesellschaft, Wirtschaft, Politik, Vol. 66, Nr. 4, Seite(n) 547-556, 2017
\item From free to civilized trade: a European perspective: Kapeller J., Schütz B., Tamesberger D. , in Review of Social Economy, Vol. 74, Nr. 3, Seite(n) 320-328, 2016
\item Populismus und Demagogie. Mit Beispielen von Jörg Haider, Heinz–Christian Strache und Frank Stronach: Ötsch W. , in Foreign Theoretical Trends, Nr. 10, Seite(n) 39-46, 2016
\item Relationships are Constructed from Generalized Unconscious Social Images Kept in Steady Locations in Mental Space: Derks L., Ötsch W., Walker W. , in Journal of Experiential Psychotherapy, Vol. 19, Nr. 1, Seite(n) 3-16, 2016
\item Wie krank ist unser Wirtschaftssystem? Krisen als Krankheiten im ökonomischen Diskurs: Pühringer S., Egger J. , in Kuckuck. Notizen zur Alltagskultur, Vol. 31, Nr. 1, Seite(n) 32-37, 2016
\item Der Staat als Wissensapparat. Konzeptionelle Überlegungen zu einer nicht-funktionalistischen Funktionsanalyse des Sozialstaats: Griesser M. , in Zeitschrift für Sozialreform, Vol. 61, Seite(n) 103-124, 2015
\item Markets as “ultimate judges” of economic policies - Angela Merkel´s discourse profile during the economic crisis and the European crisis policies.: Pühringer S. , in On the Horizon, Vol. 23, Nr. 3, Seite(n) 246-259, 2015
\item “Harte” Sanktionen für “budgetpolitische Sünder”. Kritische Diskursanalyse der Debatte zum Fiskalpakt in meinungsbildenden österreichischen Qualitätsmedien.: Pühringer S. , in Momentum Quarterly, Vol. 4, Nr. 1, Seite(n) 23-41, 2015
\item Fortschrittsidee und Politische Vision [Progress and Politics]: Kapeller J., Hubmann G. , in Momentum Quarterly, Vol. 3, Nr. 4, Seite(n) 235-245, 2014
\item ÖkonomInnen und Ökonomie in der Krise? Eine diskurs- und netzwerkanalytische Sicht.: Hirte K., Pühringer S. , in WISO - Wirtschafts- und sozialpolitische Zeitschrift, Vol. 1, Seite(n) 159-178, 2014
\item Breaking New Paths: Theory and Method in Path Dependence Research: Dobusch L., Kapeller J. , in Schmalenbach Business Review, Vol. 65, Nr. 2, Seite(n) 288-311, 2013
\item How Formalism shapes Perception: An Experiment on Mathematics as a Language: Kapeller J., Steinerberger S. , in International Journal of Pluralism and Economics Education, Vol. 4, Nr. 2, Seite(n) 138-156, 2013
\item The Deep Meening of ‘Market’: Understanding Neoliberal-Market-Radical Reasoning: Ötsch W. , in Human Geography, Vol. 6, Nr. 2, Seite(n) 11-25, 2013
\item A Guide to Paradigmatic Self-Marginalization - Lessons for Post-Keynesian Economists: Dobusch L., Kapeller J. , in Review of Political Economy, Seite(n) 469-487, 2012
\item Solidarisch Handeln: Konzeptionen, Ursachen und Implikationen: Kapeller J., Hubmann G. , in Momentum Quarterly, Vol. 1, Nr. 3, Seite(n) 139-152, 2012
\item Würdigungs-Netzwerke, gewolltes Nichtwissen und Geschichtsschreibung.: Hirte K. , in Österreichische Zeitschrift für Geschichtswissenschaften, Vol. 23, Nr. 1, Studienverlag Innsbruck, Seite(n) 155-185, 2012
\item Wirtschaft, Wissenschaft und Politik: Die sozialwissenschaftliche Bedingtheit linker Reformpolitik: Dobusch L., Kapeller J. , in Prokla, Vol. 41, Nr. 3, Seite(n) 389-404, 2011
\item Ökonomische Ausrichtung und Netzwerke - das Beispiel des Sachverständigenrates.: Hirte K., Ötsch W. , in Prokla, Nr. 3, Seite(n) 423-447, 2011
\item Perpetuing the failure: Economic Education and the Current Crisis: Ötsch W., Kapeller J. , in Journal of Social Science Education, Vol. 9, Nr. 2, Seite(n) 16-25, 2010
\item Protektionismus. Die Grenzen der Staatsintervention in den 1930er Jahren: Nordmann J. , in European Journal of Economics and Economic Policies: Intervention, Vol. 7, Nr. 1, Seite(n) 42-49, 2010
\item The internal consistency of perfect competition: Kapeller J., Pühringer S. , in Journal of Philosophical Economics, Vol. 3, Nr. 2, Seite(n) 134-152, 2010
\item Diskutieren und Zitieren: Zur paradigmatischen Konstellation aktueller ökonomischer Theorie: Dobusch L., Kapeller J. , in European Journal of Economics and Economic Policies: Intervention, Vol. 6, Nr. 2, Seite(n) 145-152, 2009
\item Neokonservativer Markt-Radikalismus. Das Fallbeispiel des Iraks: Ötsch W., Kapeller J. , in Internationale Politik und Gesellschaft, Nr. 2, Seite(n) 40-55, 2009
\item Politische Paradigmata und neoliberale Einflüsse am Beispiel von vier sozialdemokratischen Parteien in Europa.: Kapeller J., Huber J. , in ÖZP - Österreichische Zeitschrift für Politikwissenschaft, Nr. 2, Seite(n) 163-192, 2009
\item Dekarbonisierung des Gebäudesektors als Teil einer sozial-ökologischen Transformation. Ein Gestaltungsvorschlag: Kapeller J., Hornykewycz A., Weber J., Cserjan L. , in ifso expertise, Vol. 25, Seite(n) 1-23, 2024
\item Wert und Werte in der Ökonomik: Ötsch W. , in Agora42 (Philosophisches Wirtschaftsmagazin), Nr. 02/2023, Seite(n) 9-13, 2023
\item Networks of the super-rich in Austria. Evidence from an explorative case study: Pühringer S., Aistleitner M., Griesebner T. , in Materialien zu Wirtschaft und Gesellschaft, Nr. 238, 2022
\item "Koste es, was es wolle". Eine neue Ära der Ökonomie?: Schütz B. , in economy, 2021
\item A European wealth tax for a fair and green recovery: Kapeller J., Leitch S., Wildauer R. , in Renner Institut & Foundation for European Progressive Studies (FEPS), in Policy Study, 2021
\item Is a € 10 Trillion European climate investment initiative fiscally sustainable?: Kapeller J., Leitch S., Wildauer R. , in Renner Institut & Foundation for European Progressive Studies (FEPS), in Policy Study, 2021
\item Keynes, die Outputlücke und Probleme mit den Fiskalregeln: Heimberger P. , in Blog der Keynes-Gesellschaft, 2021
\item Pluralism in economics – its critiques and their lessons: Gräbner C., Strunk B. , in Developing Economics, 2021
\item Soziale Rhetorik, neoliberale Praxis: Pühringer S., Beyer K., Kronberger D. , in Beuteler Extradienst, 2021
\item Ökonomische Polarisierung in Europa: Kapeller J. , in Zeitschrift Bürger und Staat, Vol. 71, Nr. 4, Seite(n) 246-251, 2021
\item Die moderne Lehrbuchwissenschaft als Zombiewissenschaft: Porak L., Neuffer S. , in Agora42 (Philosophisches Wirtschaftsmagazin), 2020
\item Structural polarisation and path dependent development models in the EU: Heimberger P. , in Blog Developing Economics, 2020
\item ‘Output gap nonsense’ and the EU’s fiscal rules: A response to the European Commission’s economists: Heimberger P., Kapeller J. , in Blog Institute for New Economic Thinking, 2020
\item Das Gemeine an der Gemeinwohldebatte: Hirte K., Poppinga O. , in Wege für eine bäuerliche Zukunft – Zeitschrift der ÖBV/ Via Campesina Austria, Vol. 42, Nr. 3 (358), Seite(n) 7-9, 2019
\item Was ist eine Krise? Ein Rückblick auf die Wirtschafts- und Finanzkrisen 2008 und 2010: Ötsch W., Pühringer S. , in Blickpunkt WISO, 2019
\item Die deutschsprachige Volkswirtschaftslehre: Pühringer S., Grimm C. , in beigewum.at, 2018
\item Es wächst das Bewusstsein einer multiplen Krise: Ötsch W. , in Agora42 (Philosophisches Wirtschaftsmagazin), 2018
\item The Top Journals Club in Economics: Kapeller J. , in Institute for New Economic Thinking (INET), Commentaries, 2018
\item Was ist eine Krise? Wie ökonomische Theorien Wahrnehmung formen: Ötsch W., Pühringer S. , in Kurswechsel, Nr. 4/2018, Seite(n) 7-17, 2018
\item Wie viel bringt die Vermögenssteuer? Neue Aufkommensschätzungen für Österreich.: Kapeller J., Ferschli B., Schütz B., Wildauer R. , in ISW Institut für Sozial- und Wirtschaftswissenschaften, in Wirtschafts- und Sozialwissenschaftliche Zeitschrift, Vol. 40, Nr. 1, Seite(n) 146-160, 2018
\item Bestände und Konzentration privater Vermögen in Österreich: Kapeller J., Ferschli B., Schütz B., Wildauer R. , in Abteilung Wirtschaftswissenschaft der AK Wien, in Materialen zu Wirtschaft und Gesellschaft, Nr. 167, 2017
\item Agrarische Regelungspolitik und die drei agrarpolitischen „Syndrome“: Hirte K. , in Via Campesina Austria, in Wege für eine bäuerliche Zukunft – Zeitschrift der ÖBV/ Via Campesina Austria, Vol. 39, Nr. 3 (343), Seite(n) 10-11, 2016
\item Das "strukturelle Defizit" in der österreichischen Budgetpolitik: Berechnungsprobleme, Revisionen und wirtschaftspolitische Relevanz: Heimberger P. , in Wirtschaft und Gesellschaft, Vol. 42, Nr. 3, Seite(n) 451-464, 2016
\item Die Widersprüche des Mister Perfect: Berechnung – Beherrschung – Perfektion: Ötsch W. , in Agora42 (Philosophisches Wirtschaftsmagazin), Nr. 01/2017, Seite(n) 18-23, 2016
\item How economic policy drives European (dis)integration: Heimberger P., Kapeller J. , in Institute for New Economic Thinking (INET), 2016
\item Internationaler Freihandel: Theoretische Ausgangspunkte und empirische Folgen: Kapeller J. , in Wirtschafts- und Sozialwissenschaftliche Zeitschrift, Vol. 39, Nr. 1, Seite(n) 99-122, 2016
\item Neoliberale Think Tanks als (neue) Akteure in österreichischen gesellschafts- und sozialpolitischen Diskursen. Das Beispiel des Hayek Institut und der Agenda Austria: Pühringer S., Stelzer-Orthofer C. , in SWS-Rundschau - Sozialwissenschaftliche Studiengesellschaft Rundschau, Vol. 56, Nr. 1, Seite(n) 75-96, 2016
\item Replik zur Replik: Von Vorwürfen der Unwissenschaftlichkeit: Pühringer S., Stelzer-Orthofer C. , in SWS-Rundschau - Sozialwissenschaftliche Studiengesellschaft Rundschau, Vol. 3, Seite(n) 447-449, 2016
\item Wahrheit und Ökonomie: Grimm C., Kapeller J. , in Kurswechsel, Nr. 1, Seite(n) 18-29, 2016
\item Warum die Volkswirtschaften der Eurozone den USA und Großbritannien seit der Finanzkrise hinterherhinken: Zur Rolle von Unterschieden in der Geld– und Fiskalpolitik: Heimberger P. , in Vienna Institute for International Economic (wiiw), in Studies Research Report, Nr. 5, Wien, 2016
\item 'Strukturreformen' und Lohnkürzungen in Griechenland: Erwartungen, Ergebnisse und Folgen: Heimberger P. , in ISW, in WISO - Wirtschafts- und sozialpolitische Zeitschrift, Vol. 38, Nr. 3, Seite(n) 104-121, 2015
\item Allgemeine Modelltheorie und ökonomische Modelle: Kapeller J. , in EWE - Erwägen, Wissen, Ethik, Vol. 26, Nr. 3, Seite(n) 387-389, 2015
\item Bezeichnende Konstellation. Zum Eröffnungstag „Agrarpolitik“ in Österreich auf dem Podium: REWE, RWA Raiffeisen und Landwirtschaftskammer: Hirte K. , in Via Campesina Austria, in Wege für eine bäuerliche Zukunft – Zeitschrift der ÖBV/ Via Campesina Austria, Vol. 38, Nr. 2 (337), Seite(n) 15, 2015
\item Das Ökonomie-Monopol an den Agrarfakultäten: Hirte K. , in Wege für eine bäuerliche Zukunft – Zeitschrift der ÖBV/ Via Campesina Austria, Vol. 38, Nr. 3 (338), Seite(n) 22-23, 2015
\item Die griechische Schuldendebatte und das Mantra von den "notwendigen Strukturreformen": Heimberger P. , in WISO direkt, Nr. 05, 2015
\item Griechenland: Das Scheitern der europäischen Krisenpolitik: Heimberger P. , in Arbeiterkammer Wien, in EU-Infobrief, Nr. 3, Seite(n) 11-15, 2015
\item Verteilung und Gerechtigkeit: Philosophische  Perspektiven: Aistleitner M., Fölker M., Kapeller J., Mohr F., Pühringer S. , in Wirtschaft und Gesellschaft, Vol. 41, Nr. 1, Seite(n) 71-106, 2015
\item Verteilungstendenzen im Kapitalismus - Nationale und globale Perspektiven: Kapeller J., Schütz B. , in Kurswechsel, Nr. 2/2015, Seite(n) 54-68, 2015
\item Zwischen Konjunkturpuffer und Tauschobjekt. Gewerkschaftliche Perspektiven auf Migration im Österreich der Zweiten Republik: Griesser M., Sauer B. , in Kurswechsel, Nr. Heft 4, Seite(n) 58-66, 2015
\item Die Vermögensverteilung in Österreich und das Aufkommenspotenzial einer Vermögenssteuer: Eckerstorfer P., Halak J., Kapeller J., Schütz B., Springholz F., Wildauer R. , in Wirtschaft und Gesellschaft, Vol. 40, Nr. 1, Seite(n) 63-81, 2014
\item Landwirtschaft, Ideologien und "...ismen": Hirte K. , in Via Campesina Austria, in Wege für eine bäuerliche Zukunft – Zeitschrift der ÖBV/ Via Campesina Austria, Vol. 37, Nr. 2 (332), Seite(n) 14-15, 2014
\item Making Morality Matter: Civilized Markets and European Values.: Kapeller J., Schütz B., Tamesberger D. , in Journal for a Progressive Economy, 2014
\item Deutsche Agrarpolitikprofessoren vor und nach 1945: Hirte K. , in Via Campesina Austria, in Wege für eine bäuerliche Zukunft – Zeitschrift der ÖBV/ Via Campesina Austria, Vol. 36, Nr. 2 (327), Seite(n) 18-20, 2013
\item Deutsche Europapolitik vor und nach 1945: Hirte K. , in Wege für eine bäuerliche Zukunft – Zeitschrift der ÖBV/ Via Campesina Austria, Vol. 36, Nr. 3 (328), Seite(n) 22-23, 2013
\item Die Regulation der Routine: Über die regulatorischen Spielräume zur Etablierung nachhaltigen Konsums: Kapeller J., Schütz B., Tamesberger D. , in Wirtschaft und Gesellschaft, Vol. 39, Nr. 2, Seite(n) 207-231, 2013
\item Diskutieren statt Ignorieren: Eckpfeiler für interessierten Pluralismus in der Ökonomie: Dobusch L., Kapeller J. , in Der öffentliche Sektor - The Public Sector, Vol. 39, Nr. 3, 2013
\item Reichtumsverteilung in Österreich: Eckerstorfer P., Halak J., Kapeller J., Schütz B., Springholz F., Wildauer R. , in WISO - Wirtschafts- und sozialpolitische Zeitschrift, Vol. 36, Nr. 4, 2013
\item „Arbeitsmarktferne“ Personen – wer sind die? Zu veränderten Exklusionsdynamiken in neokapitalistischen Gesellschaften: Pühringer S. , in SWS-Rundschau - Sozialwissenschaftliche Studiengesellschaft Rundschau, Vol. 53, Nr. 4, Seite(n) 361-381, 2013
\item Die intellektuelle Geschichte des Neoliberalismus im Spiegel des alten Liberalismus: Nordmann J. , in ksoe-Dossier der katholischen Erwachsenenbildung Österreich, 2012
\item Konsum demokratisch gestalten: Spielräume zur Etablierung nachhaltigen Konsums.: Kapeller J., Schütz B., Tamesberger D. , in WISO - Wirtschafts- und sozialpolitische Zeitschrift, Seite(n) 167-183, 2012
\item Markt. Sichtweisen auf die Wirtschaft: Ötsch W. , in Praxis Politik, Nr. 2/2011, Seite(n) 4-8, 2011
\item Die Rolle der Agrarpolitik und Agrarökonomie in agrarpolitischen Diskursverläufen: Hirte K. , in TRANS Internet-Zeitschrift für Kulturwissenschaften , Vol. 17, Nr. 2, 2010
\item Neoklassische Sozialdemokratie und Sozialdemokratie am Beispiel des Hamburger Programms der SPD: Kapeller J., Huber J. , in TRANS Internet-Zeitschrift für Kulturwissenschaften , Vol. 17, Nr. 2, 2010
\item Markt als soziale Struktur - Zum Diskursszenario zur "Märkte-Störung" durch den Milchstreik: Hirte K. , in arbeitsergebnisse, Nr. 62, University Press, Seite(n) 14-26, 2009
\item Editorial: Freie Fahrt für reiche Burschen? Schwarz-Blau ist zurück!: Griesser M., Hofmann J. , in Kurswechsel, Nr. 3, 2018
\item Wettbewerbsfähige Nachhaltigkeit – Die Lösung unserer Probleme?: Porak L. : Momentum Kongress Paper, Seite(n) 1-14, 2022
\item Für die „Leistungsträger“ und „uns Österreicher“: Eine Mediendiskursanalyse zu Sozialreformen der ÖVP/FPÖ-Regierung 2017-2019 in Österreich: Griesser M., Beyer K., Pühringer S. : Momentum Kongress Paper, 2021
\item Shaping sustainable employment relationships in the age of Digitalisation: analysing policy measures in an agent-based framework: Hornykewycz A., Rath J. : Momentum Kongress Paper, 2021
\item The emergence of debt and secular stagnation in an unequal society: A stock-flow consistent agent-based approach: Gräbner C., Hornykewycz A., Schütz B. : Momentum Kongress Paper, 2021
\item Warum müssen wir (noch immer) arbeiten? Eine hegemonietheoretische Analyse der Bedeutung und des Wertes von Lohnarbeit für den modernen Staat: Porak L. : Momentum Kongress Paper, 2021
\item „Hinter jeder erfolgreichen Frau steht ein Mann, der ihr den Rücken stärkt.“: Hager T., Hornykewycz A., Jonjic M., Porak L., Rath J. : Momentum Kongress Paper, 2021
\item Allbetroffenheit in der Pandemie? Ein soziologischer Blick auf das Erleben der Auswirkungen der Corona-Krise: Vogel L., Jühlke R., Porak L., Quinz H. : Momentum Kongress Paper, Seite(n) 1-19, 2020
\item Competition in Transformational Processes: Polanyi & Schumpeter: Hager T., Heck I., Rath J. , in Momentum: Momentum Kongress Paper, Seite(n) 1-25, 2020
\item Wohin steuert die Europäische Union? Ein Klärungsversuch der strategischen Ausrichtung der EU seit Lissabon: Porak L. : Momentum Kongress Paper, Seite(n) 1-22, 2020
\item Auftragsvergabe, Leistungsqualität und Kostenintensität im Schienenpersonenverkehr: Aistleitner M., Grimm C., Kapeller J. : Momentum Kongress Paper, Seite(n) 1-56, 2019
\item Creating a pluralist paradigm: An application to the minimum wage debate: Schütz B. : Momentum Kongress Paper, Seite(n) 1-41, 2019
\item Hans Albert und die Kritik am Modell-Platonismus in den Wirtschaftswissenschaften: Kapeller J., Ferschli B. , in Franco, Giuseppe: Handbuch Karl Popper, Springer Fachbuch, Heidelberg, Seite(n) 733-749, 2019
\item Desintegration in Europa? Makroökonomische Divergenz und strukturelle Polarisierung: Gräbner C., Heimberger P., Kapeller J., Schütz B. : Momentum Kongress Paper, Serie Momentum quarterly, Seite(n) 1-32, 2018
\item Die Positionierung von Studierenden an öffentlichen Hochschulen: Porak L. : Momentum Kongress Paper, Seite(n) 1-18, 2018
\item Politische und gesellschaftliche Wirkmächtigkeit von ÖkonomInnen-Netzwerken: Pühringer S. : Momentum Kongress Paper, Seite(n) 1-10, 2018
\item Work or Die? How Wage dependency determines the production process: Hager T., Rath J., Wimmler L. : Momentum Kongress Paper, Seite(n) 1-37, 2018
\item Zur Verteilung und Klassenstruktur der Österreichischen Vermögen: Ferschli B., Kapeller J., Wildauer R. : Momentum Kongress Paper, Seite(n) 1-55, 2018
\item Citation Patterns in Economics and Beyond: Assessing the Peculiarities of Economics from Two Scientometric Perspectives: Aistleitner M., Kapeller J., Steinerberger S. : Momentum Kongress Paper, Seite(n) 1-22, 2017
\item Paradigmatische Homogenität? Aktueller Status und Zukunftsperspektiven der Ökonomik in Deutschland und den USA: Grimm C. : Momentum Kongress Paper, Seite(n) 1-16, 2017
\item Zur Performativität ökonomischen Wissens und aktuellen ÖkonomInnen-Netzwerken in Deutschland: Hirte K., Pühringer S. , in Maeße, Jens; Pahl, Hanno; Sparsam, Jan: Die Innenwelt der Ökonomie. Wissen, Macht und Performativität in der Wirtschaftswissenschaft, Springer VS Verlag, Wiesbaden, Seite(n) 363-390, 2017
\item Agenda Austria: Diskursstrategien einer neoliberalen Reformagenda: Pühringer S. : Momentum Kongress Paper, Seite(n) 1-21, 2016
\item Die Macht ökonomischer Modelle am Beispiel des »Potential Output«-Modells der Europäischen Kommission: Heimberger P. : Momentum Kongress Paper, Seite(n) 1-9, 2016
\item Perspektiven für eine nachhaltige Automobilindustrie: Aistleitner M. : Momentum Kongress Paper, Seite(n) 1-27, 2016
\item Postdemokratie, Machtverhältnisse und Ökonomie: Grimm C. : Momentum Kongress Paper, Seite(n) 1-22, 2016
\item Spezialisierung, Stratifikation und internationale Wirtschaft: Verteilung, Arbeitsteilung und Klassenlagen aus globaler Perspektive: Kapeller J., Heimberger P. : Momentum Kongress Paper, Seite(n) 1-20, 2016
\item Still the queens of social sciences? Economists as “public intellectuals” in/after the crisis.: Pühringer S. , in International Conference in Contemporary Social Sciences (Conference Proceedings): Crisis and the social sciences: New challenges and perspectives, Seite(n) 507-528, 2016
\item Die Macht der Wissenschaftsstatistik und die Entwicklung der Ökonomie: Aistleitner M., Fölker M., Kapeller J. : Momentum Kongress Paper, Seite(n) 1-21, 2015
\item Kontinuitäten neoliberaler Wirtschaftspolitik. Die Austeritätsdebatte als Spiegelbild diskursiver Machtverwerfungen innerhalb der Ökonomik: Pühringer S. , in Marterbauer, Markus/Mesch, Michael/Rehm, Miriam/Reiterlechner, Christine: Das Scheitern des neoklassischen Paradigmas – Wirtschaftspolitik in der EU, ÖGB Verlag, Wien, Seite(n) 159-174, 2015
\item Marktmetaphoriken in Krisennarrativen von Angela Merkel.: Pühringer S. , in Ötsch, Walter/Hirte, Katrin/Pühringer, Stephan/Bräutigam, Lars: Markt! Welcher Markt? Der interdisziplinäre Diskurs um Märkte und Marktwirtschaft., Metropolis, Marburg, Seite(n) 229-252, 2015
\item Märkte und die Anerkennung von Arbeit. Zum Zusammenhang schlecht bezahlter Arbeiten und der Struktur der Arbeitsergebnisse: Hirte K. , in Ötsch, Walter; Hirte, Katrin; Pühringer, Stephan; Bräutigam, Lars: Markt! Welcher Markt?, Serie Kritische Studien zu Markt und Gesellschaft, Metropolis, Marburg, Seite(n) 281-322, 2015
\item Politik und ihre Ad-hoc- Gremien in Krisenzeiten: Hirte K. , in Momentum-Kongress: Momentum Kongress Paper, Seite(n) 1-22, 2015
\item Verteilungstendenzen im Kapitalismus: Kapeller J., Schütz B. : Momentum Kongress Paper, Seite(n) 1-21, 2015
\item Emanzipation bei Marx und seine Kritik an Proudhon und dessen ideengeschichtlichen Nachfahren: Beyer K. : Momentum Kongress Paper, Seite(n) 1-18, 2014
\item From Free to Civilized Markets: Kapeller J., Schütz B., Tamesberger D. : Momentum Kongress Paper, Seite(n) 1-29, 2014
\item How to Hide Secrecy Jurisdictions: Ötsch W. , in Ötsch, Walter O.; Grözinger, Gerd; Beyer, Karl M.; Bräutigam, Lars: The Political Economy of Offshore Jurisdictions, : Metropolis Verlag, : Metropolis Verlagrburg, Seite(n) 61-75, 2014
\item Performative Wissenschaft: Ökonomiekritik, Ökonomietheorien und die Verantwortung von ÖkonomInnen.: Hirte K., Pühringer S. , in Hirte Katrin, Thieme Sebastian, Ötsch Walter Otto: Wissen! Welches Wissen? Zu Wahrheit, Theorien und Glauben sowie ökonomischen Theorien, Metropolis Verlag, Marburg, Seite(n) 267-302, 2014
\item Der Fiskalpakt und seine Implementation in Österreich: Plaimer W., Pühringer S. : Momentum Kongress Paper, Seite(n) 1-20, 2013
\item Grenzen aktueller Krisendebatten. Über Konstruktionen der öffentlichen Meinung und das Verhältnis von Sach- und Grundsatzdiskussionen in (neo)liberalen Demokratien: Nordmann J. , in Wengeler Martin; Ziem, Alexander: Sprachliche Konstruktionen von Krisen : interdisziplinäre Perspektiven auf ein fortwährend aktuelles Phänomen, Hempen, Bremen, Seite(n) 53-66, 2013
\item Marx, Keynes und die Idee des gesellschaftlichen Fortschritts: Suche nach neuen politischen Visionen: Schütz B. : Momentum Kongress Paper, Seite(n) 1-21, 2013
\item „Persilschein“ – Netzwerke: Für Bruchlosigkeit in Umbruchzeiten.: Hirte K. , in Schönhuth, Michael; Gamper, Markus; Kronenwett, Michael; Stark, Martin: Visuelle Netzwerkforschung. Qualitative, quantitative und partizipative Zugänge., Transcript Verlag, Seite(n) 331-353, 2013
\item Bridges to Past Polls: Die oberösterreichische Erfahrung mit Vorwahlen als demokratisches Instrument: Huber J., Kaindlstorfer L., Kapeller J. : Momentum Kongress Paper, 2012
\item Die Regulation der Routine: Kapeller J., Schütz B., Tamesberger D. : Momentum Kongress Paper, Seite(n) 1-31, 2012
\item How Liberalism lost its concept of democracy: Pühringer S., Kapeller J. : Momentum Kongress Paper, Seite(n) 1-22, 2012
\item Postdemokratie in Österreich?: Plaimer W. , in Momentum-Kongress: Momentum Kongress Paper, Seite(n) 1-17, 2012
\item Aktionsforschung - ein Weg zum Design institutioneller Neuerungen zur regionalen Anpassung an den Klimawandel: Knierim A., Hirte K. : Anpassung an den Klimawandel - regional umsetzen!, oecom Verlag, München, Seite(n) 156-174, 2011
\item Braucht die aktuelle Gesellschaft einen Gesellschaftsvertrag? Der politische Neoliberalismus im Spiegel von John Locke und John Rawls: Nordmann J. : Gesellschaft! Welche Gesellschaft?, Metropolis-Verlag, Marburg, Seite(n) 33-60, 2011
\item Crowdsourcing und Regelbezüge - der Fall GuttenPlag.: Hirte K. , in Heiss, Hans-Ulrich: INFORMATIK 2011 - Informatik schafft Communities. 41. Jahrestagung der Gesellschaft für Informatik , 4.-7.10.2011, Berlin, Serie Lecture Notes in Informatics (LNI), Vol. P-192, Springer, 2011
\item Der Markt und die großen Ratingagenturen: Ötsch W. : Momentum Kongress Paper, Serie Momentum quarterly, Seite(n) 1-11, 2011
\item Gleichheit und Vielfalt – normative Konzeptionen? Die philosophischen Implikationen zum Problem Anerkennung bei Simone de Beauvoir und Hannah Arendt: Hirte K. : Momentum Kongress Paper, Seite(n) 1-15, 2011
\item Gleichheit versus Vielfalt. Ein konstruierter Widerspruch?: Pühringer S. : Momentum Kongress Paper, Seite(n) 1-23, 2011
\item Veränderung von Machtverhältnissen in politischen Entscheidungsprozessen: Plaimer W., Nordmann J. , in Momentum-Kongress: Momentum Kongress Paper, Seite(n) 1-21, 2011
\item Das neoklassische Freihandelsmodell: Hirte K. : Momentum Kongress Paper, Seite(n) 1-16, 2010
\item Die Tiefenbedeutung von ‘Markt’: Ötsch W. : Momentum Kongress Paper, Seite(n) 1-27, 2010
\item Performativity of Economics - ein tragfähiger Ansatz zur Analyse der Rolle der Ökonomen in der Ökonomie?: Hirte K. , in Walter Ötsch, Katrin Hirte, Jürgen Nordmann: Krise. Welche Krise?, Metropolis Verlag, Marburg, Seite(n) 49-75, 2010
\item Solidarisch Handeln: Konzeptionen, Ursachen und Implikationen: Hubmann G., Kapeller J. : Momentum Kongress Paper, Seite(n) 1-30, 2010
\item Solidarität im Kapitalismus: Pühringer M., Pühringer S. : Momentum Kongress Paper, Seite(n) 1-32, 2010
\item Trash, Skandale und Ratschläge statt Aufklärung und politische Bildung. Über das Zusammenspiel von kommerzialisierten Medien und gemachter Meinung in der neoliberalen Gesellschaft: Nordmann J. : Momentum Kongress Paper, Seite(n) 1-8, 2010
\item Frei Handeln. Überlegungen zur Überwindung des neoliberalen Freiheitsbegriffs: Pühringer S., Wolfmayer G. : Momentum Kongress Paper, Seite(n) 1-21, 2009
\item Krisenbilder von ÖkonomInnen in der Presse: Pühringer S., Egger J. , in Walter Ötsch; Silja Graupe: Macht der Bilder, Macht der Sprache, Seite(n) 75-86, 2018
\item Ökonomische Expertise und polit-ökonomische Machtstrukturen: Pühringer S., Liedl B. , in AK Kärnten: Welt aus den Fugen. Wie der Neoliberalismus unser Leben verändert., ÖGB-Verlag, Wien, Seite(n) 41-56, 2018
\item Agrarpolitik und Arbeit – der Einfluss europäischer Agrarpolitikmaßnahmen auf die Arbeit im Agrarsektor: Hirte K., Kuschel S. , in Ahlert, Maximilian; Fiederer, Franca; Varelmann, Katharina; Kuschel, Sarah; Ewers, Sylvia; Politor, Merlin; Stamp, Katharina: Frohes Schaffen!? Arbeit in der Landwirtschaft, Universtity Press, Kassel, Seite(n) 9-15, 2016
\item Die Politische Ökonomie „des“ Marktes. Eine Zusammenfassung zur Wirkungsgeschichte von Friedrich A. Hayek: Ötsch W. , in Kapeller, Jakob; Pühringer, Stephan; Hirte, Katrin; Ötsch, Walter O.: Ökonomie! Welche Ökonomie? Stand und Status der Wirtschaftswissenschaften, Metropolis Verlag, Marburg, Seite(n) 19-50, 2016
\item Die „Landnahme“-These von Rosa Luxemburg – empirisch beobachtbar, aber theoretisch falsifiziert?: Hirte K. , in Kapeller, Jakob; Pühringer, Stephan; Hirte, Katrin; Ötsch, Walter: Ökonomie! Welche Ökonomie?, Metropolis Verlag, Marburg, Seite(n) 273-313, 2016
\item Vorwort im Tagungsband Ökonomie! Welche Ökonomie?: Hirte K., Kapeller J., Pühringer S., Ötsch W. : Ökonomie! Welche Ökonomie?, Metropolis Verlag, Marburg, Seite(n) 2-10, 2016
\item Markt und Markttheorie. Vorwort und Überblick: Ötsch W. , in Ötsch, Walter O.; Hirte, Katrin; Pühringer, Stephan; Bräutigam, Lars: Markt! Welcher Markt? Der interdisziplinäre Diskurs um Märkte und Marktwirtschaft, Metropolis Verlag, Marburg, Seite(n) 7-24, 2015
\item Offshore Aspects of Shadow Banking. With Considerations on the Recent Financial Crisis: Beyer K., Bräutigam L. , in Ötsch, Walter O.; Grözinger, Gert; Beyer, Karl M.: The Political Economy of Offshore Jurisdictions, Metropolis Verlag, Marburg, 2014
\item The Political Economy of Offshore Jurisdictions. An Introduction: Ötsch W., Schmidt M. , in Ötsch, Walter O.; Grözinger, Gerd; Beyer, Karl M.; Bräutigam, Lars: The Political Economy of Offshore Jurisdictions, Metropolis Verlag, Marburg, Seite(n) 7-23, 2014
\item Vorwort im Sammelband "Wissen! Welches Wissen?": Hirte K., Ötsch W. , in Hirte Katrin, Thieme Sebastian, Ötsch Walter Otto: Wissen! Welches Wissen? Zu Wahrheit, Theorien und Glauben sowie ökonomischen Theorien, Metropolis Verlag, Marburg, Seite(n) 7-16, 2014
\item Economists and Economics. Discourse profiles of economists in the financial crisis: Pühringer S., Hirte K. , in Association française d'économie politique: Joint conference of AHE, IIPPE, FAPE. Kongress Political economy and the outlook for capitalism 05.-07.07.2012, Paris, Seite(n) 1-16, 2012
\item Vorwort: Hirte K., Ötsch W. , in Ötsch, Walter O.; Hirte, Katrin; Nordmann, Jürgen: Gesellschaft! Welche Gesellschaft?, Metropolis, Marburg, Seite(n) 7-15, 2011
\item Was sind ökonomische Modelle?: Kapeller J. , in Volker Gadenne und Reinhard Neck: Philosophie und Wirtschaftswissenschaft, Mohr Siebeck, Seite(n) 29-50, 2011
\item Institutionalisierung zivilgesellschaftlicher Partizipation - Zwischen Ignoranz, Integration und Invasion: Dobusch L., Kapeller J. , in Blaha, Barbara; Weidenholzer, Josef: Freiheit - Beiträge für eine demokratische Gesellschaft, Wilhelm Braumüller Universitäts-Verlagsbuchhandlung, Wien, Seite(n) 201-217, 2010
\item Was ist eine Krise?: Nordmann J. , in Ötsch, Walter O.; Hirte, Katrin; Nordmann, Jürgen: Krise! Welche Krise?, Metropolis, Marburg, Seite(n) 7-20, 2010
\item Ökonomisierung an den Hochschulen.: Hirte K. , in Johanna Besier, Hannah Fritsch, Arne Rost, Sven Schmidt, Hannes Schulz, Maggie Selle, Katharina Wenzel: Agrarpolitik in der Lehre?, ABL Bauernblatt Verlags-GmbH, Seite(n) 17-25, 2010
\item Lobbying and Macroeconomic Development: Hager T. , in Mause, Karsten; Polk, Andreas: The Political Economy of Lobbying, Serie Studies in Public Choice, Vol. 43, Springer, Cham, Seite(n) 77-99, 2024
\item Performativität: Geplante Landwirtschaftsstrukturen – das Beispiel Böckenhoff-Plan: Hirte K. , in Gruber, Holle; Henkel, Anna; Scheel, Laura: Land: Digitalisierung, Agrarwandel, Energiewende – soziologische Perspektiven zu ländlichen Räumen im Umbruch, Serie 10 Minuten Soziologie, trancript, Bielefeld, Seite(n) 87-100, 2024
\item Dilemmata marktliberaler Globalisierung: Kapeller J., Hubmann G. , in Sturn, Richard; Klüh, Ulrich: Wachstums- und Globalisierungsgrenzen, Serie Jahrbuch Normative und institutionelle Grundfragen der Ökonomik, Vol. 20, Metropolis Verlag, Marburg, 2023
\item Economics and Power: Benz P., Maesse J., Pühringer S., Rossier T. , in Macknight, Viski; Medvecky, Fabien: Making Economics Public, Routledge, London, Seite(n) 18-25, 2023
\item Elements of an evolutionary approach to comparative economic studies: Gräbner-Radkowitsch C. , in Casagrande, Sara; Dallago, Bruno: The Routledge Handbook of Comparative Economic Systems, Routledge, London, forthcoming, Seite(n) 81-102, 2023
\item Il potere e l'economics: Benz P., Maesse J., Pühringer S., Rossier T. , in Nicoletta, Gerardo C.; di Carlo, Michele S.; Ventrone, Oreste: Economisti e Società. Nuove sociologie dell'expertise economica, Liguori Editore, Napoli, Seite(n) 17-24, 2023
\item Islands in the Privately Dominated Sea of Capitalist Media: Porak L., Schamberger K. , in Güney, Selma; Hille, Lina; Pfeiffer, Juliane; Porak, Laura; Theine, Hendrik: Eigentum, Medien, Öffentlichkeit, Westend Verlag, Frankfurt am Main, Seite(n) 443-449, 2023
\item L´expertise parziale dell´economics: il caso della ricerca (delle politiche) sul commercio: Aistleitner M., Pühringer S. , in Nicoletta, Gerardo C.; di Carlo, Michele S.; Ventrone, Oreste: Economisti e Società. Nuove sociologie dell'expertise economica, Liguori Editore, Napoli, Seite(n) 25-40, 2023
\item Soziale und ökologische Probleme müssen zusammen betrachtet werden: Pühringer S. , in Renner Institut: Ein aktiver Staat, der die Menschen stärkt und schützt, Seite(n) 24-27, 2023
\item The role of power in the Social Studies of Economics: an introduction: Maesse J., Pühringer S., Rossier T., Benz P. , in Jens Maesse, Stephan Pühringer, Thierry Rossier,  Pierre Benz: Power and Influence of Economists: Contributions to the Social Studies of Economics., Routledge, London, 2022
\item Who are the economists Germany listens to? What it needs to have academic, public or political impact: Pühringer S., Beyer K. , in Maesse, Jens; Pühringer, Stephan; Rossier, Thierry; Benz, Pierre: Power and Influence of Economists: Contributions to the Social Studies of Economics., Routledge, London, Seite(n) 147-169, 2022
\item Why Fostering Socio-economic Convergence in the EU Is Necessary for Successful Climate Change Mitigation: Gräbner-Radkowitsch C., Hafele J. , in Heinrich Böll Foundation, ZOE-Institute for Future-Fit Economies and Finanzwende Recherche: Making the great turnaround work: Economic policy for a green and just transition, Seite(n) 104-114, 2022
\item Wissenschaftstheoretische Grundlagen: Koch L., Ötsch W., Graupe S. , in Lehmann-Waffenschmidt, Marco; Peneder, Michael: Evolutorische Ökonomik. Konzepte, Wegbereiter und Anwendungsfelder, Metropolis Verlag, Marburg, Seite(n) 349-359, 2022
\item Agent of Sustainable Change - Der unternehmerische Staat und sozial-ökologische Transformation: Strohmaier R. , in Klüh, Ulrich; Sturn, Richard: Der Staat in der großen Transformation. Jahrbuch Normative und institutionelle Grundfragen der Ökonomik, Serie Jahrbuch Normative und institutionelle Grundfragen der Ökonomik, Metropolis, Weimar, Seite(n) 169-192, 2021
\item Konzernmacht in globalen Güterketten: Gräbner C., Kapeller J. , in Karin Fischer, Christian Reiner und Cornelia Staritz: Globale Güterketten und ungleiche Entwicklung. Arbeit, Kapital, Natur und Konsum, Mandelbaum, Seite(n) 195-217, 2021
\item Narration und Imagination. Die Rolle von imaginierten Bildern in der Geschichte der Wirtschaftstheorie: Ötsch W. , in Künzel, Christine; Priddat, Birger: Fiktion und Narration in der Ökonomie. Interdisziplinäre Perspektiven auf den Umgang mit ungewisser Zukunft, Metropolis, Marburg, Seite(n) 241-267, 2021
\item Ordoliberalismus: Ötsch W., Pühringer S. , in Michael G. Festl: Handbuch Liberalismus, J.B. Metzler, Stuttgart, Seite(n) 372-378, 2021
\item Populismus: Ötsch W., Wodak R. , in Ferstl, Michael G.: Handbuch Liberalismus, J.B. Metzler, Stuttgart, Seite(n) 535-541, 2021
\item Das doppelte Reflektionsproblem: Hirte K. , in Hochmann, Lars: Economists4future, Murmann Verlag, Hamburg, Seite(n) 43-58, 2020
\item Die Auswirkung von Mindestlöhnen auf die Arbeitslosigkeit: Ein Paradigmenvergleich: Schütz B. , in Stephan Pühringer, Silja Graupe, Katrin Hirte, Jakob Kapeller, Stephan Panther: Jenseits der Konventionen: Alternatives Denken zu Wirtschaft, Gesellschaft und Politik, Metropolis, Marburg, Seite(n) 157-173, 2020
\item Friedman’s Instrumentalismus und das Problem von Kopernikus: Hirte K. , in Pühringer, Stephan; Graupe, Silja; Hirte, Katrin; Kapeller, Jakob; Panther, Stephan: Jenseits der Konventionen, Metropolis Verlag, Marburg, Seite(n) 97-122, 2020
\item Miteinander und voneinander lernen. Vielfalt in der ökonomischen Lehre: Porak L. , in Hochmann, Lars: Economists4future, Murmann Verlag, Hamburg, Seite(n) 127-142, 2020
\item Paradigmen und Politik. Der derzeitige Stand der Ökonomie: Kapeller J., Pühringer S. , in Jakob Kapeller, Stephan Pühringer, Silja Graupe, Kathrin Hirte, Stephan Panther: Jenseits der Konventionen: Alternatives Denken zu Wirtschaft, Gesellschaft und Politik, Metropolis, Marburg, Seite(n) 221-252, 2020
\item Plurale Ökonomik - Eine kurze Einführung: Piétron D., Porak L., Thieme S. , in Thielscher, Christian: Wirtschaftswissenschaften verstehen, Springer Gabler, Wiesbaden, Seite(n) 189-205, 2020
\item Think Tank Networks of German Neoliberalism. Power Structures in Economics and Economic Policies in Post-War Germany: Pühringer S. , in Mirowski, Philip; Plehwe, Dieter; Slobodian, Quinn: Nines Lives of Neoliberalism, Verso Books, New York, Seite(n) 283-306, 2020
\item Das dritte gossensche Gesetz: Hirte K. , in Hochmann, Lars; Graupe, Silja; Korbun, Thomas; Panther, Stephan; Schneidewind, Uwe: Möglichkeits¬wissen¬schaften. Ökonomie mit Möglichkeitssinn, Metropolis Verlag, Marburg, Seite(n) 133-176, 2019
\item Humankapital: Kapeller J. , in von Braunmühl, Claudia; Gerstenberger, Heide; Ptak, Ralf; Wichterich, Christa: ABC der globalen Unordnung. Von »Anthropozän« bis »Zivilgesellschaft«, VSA-Verlag, Hamburg, Seite(n) 120-121, 2019
\item Marktfundamentalismus als Kollektivgedanke. Mises und die Ordoliberalen: Ötsch W., Pühringer S. , in Richard Sturn, Nenad Pantelic: Dem Markt vertrauen? Beiträge zur Tiefenstruktur neoliberaler Regulierung., Metropolis, Marburg, Seite(n) 185-210, 2019
\item Pluralism in Economics: Epistemological Rationales and Pedagogical Implementation: Kapeller J. , in Decker, Samuel; Elsner, Wolfram; Flechtner, Svenja: Advancing Pluralism in Teaching Economics, Routledge, London, Seite(n) 55-77, 2019
\item The “eternal character” of austerity measures in European crisis policies. Evidences from the Fiscal Compact discourse in Austria.: Pühringer S. , in Power, Kate; Ali, Tanweer; Lebduskova, Eva: Discourse Analysis and Austerity: Critical Studies from Economics and Linguistics, Routledge, London, Seite(n) 134-158, 2019
\item Wissen und Nichtwissen angesichts ‚des Marktes‘. Das Konzept von Hayek: Ötsch W. , in Graupe, Silja; Ötsch, Walter O.; Rommel, Florian: Spielräume des Denkens, Metropolis, Marburg, Seite(n) 311-339, 2019
\item Bilder des Rechtspopulismus: Ötsch W. , in Ötsch, Walter O.; Graupe, Silja: Macht der Bilder, Macht der Sprache, Angelika Lenz Verlag, Isenburg, Seite(n) 113-128, 2018
\item Heterodoxie in der Ökonomik: Hirte K., Thieme S. , in Schetsche, Michael; Schmied-Knittel, Ina: Heterodoxie. Konzepte, Traditionen, Figuren der Abweichung, Halem Verlag, Köln, 2018
\item Zeitlichkeit und Tauschfähigkeit bei Rosa Luxemburg und Joseph A. Schumpeter: Hirte K. , in Bies, Michael; Giacovelli, Sebastian; Langenohl, Andreas: Ästhetische Eigenzeiten von Tausch und Gabe, Wehrhahn Verlag, Hannover, 2018
\item Dealing adequately with the political element in formal modelling: Gräbner C. , in Katsikides, Savas and Hanappi, Hardy and Scholz-Wäckerle, Manuel: Theory and Method of Evolutionary Political Economy, Routledge, New York, Seite(n) 236-254, 2017
\item Die Rolle des Gleichgewichtskonzepts in der mikroökonomischen Ausbildung: Gräbner C. , in Till van Treek, Janina Urban: Wirtschaft neu denken, iRIGHTS media, Berlin, Seite(n) 60-73, 2017
\item Internationale Tendenzen und Potentiale der Vermögensbesteuerung: Kapeller J., Schütz B., Springholz F. , in Dimmel, Nikolaus; Hofmann, Julia; Schenk, Martin; Schürz, Martin: Handbuch Reichtum – Neue Erkenntnisse aus der Ungleichheitsforschung, Studienverlag, Wien, Seite(n) 477-492, 2017
\item The Micro-Macro Link in Heterodox Economics: Gräbner C., Kapeller J. , in Tae-Hee Jo, Lynne Chester Carlo and D'Ippoliti: The Routledge Handbook of Heterodox Economics, Routledge, Seite(n) 145-159, 2017
\item The success story of ordoliberalism as guiding principle of German economic policy: Pühringer S. , in Hien, Josef; Joerges, Christian: Ordoliberalism. Law and the rule of economics, Hart Publishing, Oxford, Portland, Seite(n) 134-158, 2017
\item Zur Performativität in den Wirtschaftswissenschaften. Kernaussagen, Anwendungspotentiale und Grenzen eines Konzepts: Hirte K. , in Pfriem, Reinhard; Schneidewind, Uwe: Transformative Wirtschaftswissenschaft im Kontext nachhaltiger Entwicklung, Metropolis Verlag, Marburg, 2017
\item sezonieri.at: Kollektive Handlungsfähigkeit von ErntearbeiterInnen in Österreich: Griesser M. , in Schmidjell, Cornelia; Sedmak, Clemens; Koch, Andreas; Kapferer, Elisabeth; Gaisbauer, Helmut P.; Bogner, Stefan; Wimmer, Bernd: Lesebuch Soziale Ausgrenzung III, Mandelbaum Verlag, Wien, Seite(n) 89-92, 2017
\item Die neoliberale Utopie als Ende aller Utopien: Ötsch W. , in Pittl, Sebastian; Prüller-Jagenteufel; Gunter: Unterwegs zu einer neuen ‚Zivilisation geteilter Genügsamkeit‘. Perspektiven utopischen Denkens 25 Jahre nach dem Tod Ignacio Ellacurías, Vandenhoeck & Ruprecht uni press, Wien, Seite(n) 105-119, 2016
\item Economic Complexity and Trade-Offs in Policy Decisions: Schwardt H., Gräbner C., Heinrich T., Cordes C., Schwesinger G. , in Gräbner, Claudius; Heinrich, Torsten; Schwardt, Henning: Policy Implications of Evolutionary and Institutional Economics, Routledge, London, New York, Seite(n) 3-19, 2016
\item Evolutionary Political Economy and the Complexity of Economic Policy: Kapeller J., Scholz-Wäckerle M. , in Gräbner,  Claudius / Heinrich, Torsten / Schwardt, Henning: Policy  Implications of Recent Advances in Evolutionary and Institutional  Economics, Routledge, London, Seite(n) 99-122, 2016
\item Geld und Raum. Anmerkungen zum Homogenisierungsprogramm der beginnenden Neuzeit: Ötsch W. , in Brodbeck, Karl-Heinz; Graupe, Silja: Geld! Welches Geld? Geld als Denkform, Metropolis Verlag, Marburg, Seite(n) 71-101, 2016
\item Netzwerke im Internet – eine neue kritische Öffentlichkeit? Das Beispiel Guttenberg: Hirte K. , in Imhof, Kurt; Welz, Frank; Fleck, Christian; Vobruba, Georg: Neuer Strukturwandel der Öffentlichkeit. Verhandlungen des dritten gemeinsamen Kongresses der Deutschen, Österreichischen und Schweizerischen Gesellschaft für Soziologie, Springer, Wiesbaden, Seite(n) 1-16, 2016
\item Verteilungstendenzen im Kapitalismus. Globale Perspektiven: Kapeller J., Schütz B. , in Bundesarbeitskammer: Die Verteilungsfrage. Von Reichtum, Krisen und Ablenkungsmanövern, ÖGB-Verlag, Wien, Seite(n) 49-54, 2016
\item Beyond Foundations: Systemism in Economic  Thinking: Kapeller J. , in Jo, Tae-Hee / Todorovka, Zdravka: Advancing the Frontiers of Heterodox Economics: Essays in Honor of Frederic S. Lee, Routledge, London, Seite(n) 115-134, 2015
\item Demokratie in Liberalismus und Neoliberalismus: Kapeller J., Pühringer S. , in Seckauer, Hansjörg/Stelzer-Orthofer, Christine/Kepplinger, Brigitte: Das Vorgefundene und das Mögliche. Beiträge zur Gesellschafts- und Sozialpolitik zwischen Ökonomie und Moral, Mandelbaum, Wien, Seite(n) 111-127, 2015
\item Moralität, Wettbewerb und internationaler Handel: Eine europäische Perspektive: Kapeller J., Schütz B., Tamesberger D. , in Seckauer, Hansjörg; Stelzer-Orthofer, Christine; Kepplinger, Brigitte: Das Vorgefundene und das Mögliche. Beiträge zur Gesellschafts- und Sozialpolitik zwischen Ökonomie und Moral. Festschrift für Josef Weidenholzer, Mandelbaum Verlag, Wien, Seite(n) 213-227, 2015
\item Ökonomie und Moral. Eine kurze Theoriegeschichte: Ötsch W. , in Seckauer, Hansjörg; Stelzer-Orthofer, Christine; Kepplinger, Brigitte: Das Vorgefundene und das Mögliche. Beiträge zur Gesellschafts- und Sozialpolitik zwischen Ökonomie und Moral, Mandelbaum Verlag, Wien, Seite(n) 100-110, 2015
\item Agrargiganten im Osten. Zur neuerlichen Transformation der transformierten deutschen Agrarstrukturen.: Hirte K. , in Brähler, Elmar; Wagner, Wolf: 25 Jahre Mauerfall – kein Ende mit der Wende?, Psychosozial-Verlag, Gießen, Seite(n) 277-­290, 2014
\item Führt Pluralismus in der ökonomischen Theorie zu mehr Wahrheit?: Kapeller J., Grimm C., Springholz F. , in Hirte, Katrin, Thieme, Sebastian, Ötsch, Walter: Wissen! Welches Wissen?, Metropolis, Marburg, Seite(n) 147-163, 2014
\item Subventionierung von Lohnkosten als Mittel zur Armutsvermeidung: Stelzer-Orthofer C., Pühringer S. , in Dimmel, N./Schenk, M./Stelzer-Orthofer, C.: Handbuch Armut in Österreich, Studienverlag, Innbruck, Seite(n) 817-831, 2014
\item Die Macht der Ratingagenturen: Governance in der Ideologie 'des Marktes': Ötsch W. , in Brodbeck, Karl-Heinz: Alternative Länder-Ratings, Schriftenreihe der Finance & Ethics Academy, Band 5, Shaker Verlag, Aachen, Seite(n) 58-98, 2013
\item Marktradikalität. Der Diskurs von „dem Markt“: Ötsch W. , in Günther, Lea-Simone; Hertlein, Saskia;  Klüsener, Vea und Raasch, Markus: Radikalität. Religiöse, politische und künstlerische Radikalismen in Geschichte und Gegenwart.  Band 2: Frühe Neuzeit und Moderne, Königshausen & Neumann, Würzburg, Seite(n) 254-279, 2013
\item Die ersten Professoren für Agrarpolitik und Agrarökonomie ab 1945 an den westdeutschen Universitäten und ihre Vergangenheit.: Hirte K. , in Buchsteiner Martin, Strahl Antje: Thünen-Jahrbuch, Nr. 7, Seite(n) 87-114, 2012
\item Regulatorische Unsicherheit und Private Standardisierung: Koordination durch Ambiguität: Dobusch L., Kapeller J. : Steuerung durch Regeln, Serie Managementforschung, Vol. 22, Seite(n) 43-81, 2012
\item Lobbyismus und gesamtwirtschaftliche Entwicklung: Hager T. , in Andreas Polk, Karsten Mause: Handbuch Lobbyismus, Springer VS, Wiesbaden, Seite(n) 817–842, 2023
\item Vorwort. Jenseits der Konventionen: Pühringer S., Graupe S., Hirte K., Kapeller J., Panther S. , in Pühringer, Stephan; Graupe, Silja; Hirte, Katrin; Kapeller, Jakob; Panther, Stephan: Jenseits der Konventionen. Eine Festschrift für Walter Ötsch, Metropolis, Marburg, Seite(n) 9-16, 2020
\item Wissen, Selbstwissen und Nichtwissen der marktfundamentalen Ökonomie: Ötsch W. , in Ötsch, Walter O.; Steffestun, Theresa: Wissen und Nichtwissen der ökonomisierten Gesellschaft, Metropolis Verlag, Marburg, Seite(n) 85-131, 2020
\item Spielräume des Denkens. Zur Einführung: Graupe S., Ötsch W., Rommel F. , in Graupe, Silja; Ötsch, Walter O.; Rommel, Florian: Spielräume des Denkens, Metropolis, Marburg, Seite(n) 9-32, 2019
\item Ökonomie als Lachnummer: Ötsch W. , in Krauß, Dietrich: Die Rache des Mainstreams an sich selbst. 5 Jahre »Die Anstalt«, Westend Verlag, Frankfurt am Main, Seite(n) 260-271, 2019
\item Einführung: Die Bedeutung von Bildern für das Denken: Ötsch W., Graupe S. , in Ötsch, Walter O.; Graupe, Silja: Macht der Bilder, Macht der Sprache, Angelika Lenz Verlag, Isenburg, Seite(n) 9-19, 2018
\item Ein philosophischer Blick auf die Grundlagen internationaler Ökonomie.: Kapeller J. , in Till van Treeck, Janina Urban: Wirtschaft neu denken – blinde Flecken der Lehrbuchökonomie, Seite(n) 108-116, 2016
\item Introduction: Gräbner C., Heinrich T., Schwardt H. , in Gräbner, Claudius; Heinrich, Torsten; Schwardt, Henning: Policy Implications of Recent Advances in Evolutionary and Institutional Economics, Routledge, London, New York, Seite(n) xxi-xxx, 2016
\item Schönheit und Macht. Drei Beispiele aus der Kulturgeschichte: Ötsch W. , in Ridler, Gerda: Mythos Schönheit. Facetten des Schönen in Natur, Kunst und Gesellschaft, Hatje Cantz Verlag, Stuttgart, Seite(n) 259-263, 2015
\item Wirtschaftspolitik, Verteilungsgerechtigkeit und Demokratie: Kapeller J. , in AK Burgenland: Gerechtigkeit muss sein, Seite(n) 150-165, 2015
\item Mythen über Reichtum und Macht: Demokratie ist nicht käuflich: Pühringer S. , in Beigewum/ATTAC/Armutskonferenz: Mythen des Reichtums, VSA Verlag, Hamburg, Seite(n) 149-158, 2014
\item Populismus und Demagogie. Mit Beispielen von Jörg Haider, Hans-­‐Christian Strache und Frank Stronach sowie der Tea Party”.: Ötsch W. , in Gressl, Martin, Klemenjak, Martin, Klepp. Cornelia, Pichler, Heinz, Rottermann, Doris und Scherling, Josefine: Populismus und Rassismus im Vormarsch?, Schriftenreihe „Arbeit und Bildung“, Klagenfurt, Seite(n) 12-26, 2014
\item A guide to paradigmatic Self-marginalization - Lessons for Post-Keynesian Economists: Kapeller J., Dobusch L. , in Lavoie M. und Lee F.S.: In Defense of Post-Keynesian and Heterodox Economics., Routledge, London, Seite(n) 62-86, 2012
\item Der freie Markt: Ötsch W. , in Czejkowska, Agnieszka: Imagine Economy. Neoliberale Metaphern im wirtschaftlichen Diskurs, Erhard Löcker Verlag, Wien, Seite(n) 39-45, 2012
\item Krise des Euroraums: Ötsch W. , in Starke, Herbert; Horvath, Patrick; Weinzierl, Rupert: Die Vision Zentraleuropa im 21. Jahrhundert, Arbeitsgemeinschaft für wissenschaftliche Wirtschaftspolitik, Wien, Seite(n) 68-71, 2012
\item Politische Ökonomie und Gesellschaft. Eine theoriegeschichtliche Skizze: Ötsch W. , in Grözinger, Gerd; Reich, Utz-Peter: Entfremdung – Ausbeutung – Revolte, Metropolis Verlag, Marburg, Seite(n) 145-165, 2012
\item Postdemokratie in Österreich?: Plaimer W. , in Nordmann, Jürgen; Hirte, Katrin; Ötsch, Walter O.: Demokratie! Welche Demokratie? Postdemokratie kritisch hinterfragt, Metropolis Verlag, Marburg, Seite(n) 159-174, 2012
\item Staatsschuldenkrise und ökonomisches Denken – im Euroraum und in Zentraleuropa: Ötsch W. , in Horvath, Patrick, Skarke, Herbert  und Weinzierl, Rupert: Die “Vision Zentraleuropa” im 21. Jahrhundert. Festschrift zum 90. Geburtstag von Heinz Kienzl, Arbeitsgemeinschaft für wissenschaftliche Wirtschaftspolitik (WIWIPOL), Wien, Seite(n) 68-71, 2012
\item Vorwort in "Demokratie! Welche Demokratie? Postdemokratie kritisch hinterfragt": Nordmann J. : Demokratie! Welche Demokratie? Postdemokratie kritisch hinterfragt, Seite(n) 7-14, 2012
\item Kurt Rothschild als politischer Ökonom: Ötsch W. , in Bartel, Rainer; Bürger, Hans; Klug, Friedrich: In Memoriam Univ. Prof. Kurt W. Rothschild, Serie Schriftenreihe des Instituts für Kommunalwissenschaft (IKW), Nr. Band 122, Institut für Kommunalwirtschaften, Wien, Seite(n) 73-84, 2011
\item Das Bewusstsein des Homo Oeconomicus: Ötsch W. , in Bauer, Renate: Bewusstsein und Ich, Angelika Lenz Verlag, Neu-Isenberg, Seite(n) 105–117, 2010
\item Diskursverläufe in der universitären Agrarpolitik als neoliberales Hegemonialprojekt – Struktur, Ursache und Wirkungen: Hirte K. , in Ötsch, Walter; Thomasberger, Claus: Der neoliberale Marktdiskurs, Metropolis, Marburg, Seite(n) 187-212, 2009
\item Wirtschaft(lich) studieren. Erlebniswelten von Studierenden der Volkswirtschaftslehre: Bäuerle L., Ötsch W., Pühringer S. , Springer, Wiesbaden, 2020
\item Die deutsche Agrarpolitik und Agrarökonomik. Entstehung und Wandel zweier ambivalenter Disziplinen: Hirte K. , Springer, Wiesbaden, 2019
\item Mythos Markt. Mythos Neoklassik. Das Elend des Marktfundamentalismus: Ötsch W. , Serie Kritische Studien zu Markt und Gesellschaft, Vol. 11, Metropolis, Marburg, 2018
\item Netzwerke des Marktes : Ordoliberalismus als Politische Ökonomie: Ötsch W., Pühringer S., Hirte K. , Springer VS, Wiesbaden, 2018
\item ÖkonomInnen in der Finanzkrise. Diskurse. Netzwerke. Initiativen.: Hirte K. , Metropolis Verlag, Marburg, 2013
\item Frei handeln? Liberales und neoliberales Freiheitskonzept und ihre Auswirkungen auf die Verteilung von Macht und Eigentum: Pühringer S. , Peter Lang, Frankfurt am Main, 2011
\item Modell-Platonismus in der Ökonomie: Zur Aktualität einer klassischen epistemologischen Kritik: Kapeller J. , Peter Lang, Frankfurt/Main, 2011
\item Mythos Markt. Marktradikale Propaganda und ökonomische Theorie: Ötsch W. , Metropolis Verlag, Marburg, 2009
\item Das europäische Schattenbankensystem – Typologisierung und die Bewertung regulatorischer Initiativen auf europäischer Ebene: Beyer K., Bräutigam L. , Serie Materialien zu Wirtschaft und Gesellschaft. Working Paper-Reihe der AK Wien, Nr. 154, Arbeiterkammer Wien, 2016
\item Pluralism and Economics Education: Theine H., Porak L. , Serie International Journal of Pluralism and Economics Education, Vol. 13 (1), 2022
\item Der neoliberale Markt–Diskurs. Ursprünge, Geschichte, Wirkungen.: Ötsch W., Thomasberger C. , Metropolis Verlag, Marburg, 2009
\item Ökonomie! Welche Ökonomie? Stand und Status der Wirtschaftswissenschaften: Kapeller J., Pühringer S., Hirte K., Ötsch W. , Metropolis, Marburg, 2016
\item Markt! Welcher Markt? Der interdisziplinäre Diskurs um Märkte und Marktwirtschaft: Ötsch W., Hirte K., Pühringer S., Bräutigam L. , Metropolis, Marburg, 2015
\item Wissen! Welches Wissen? Zu Wahrheit, Theorien und Glauben sowie ökonomischen Theorien: Hirte K., Thieme S., Ötsch W. , Metropolis Verlag, Marburg, 2014
\item Demokratie! Welche Demokratie? Postdemokratie kritisch hinterfragt: Nordmann J., Hirte K., Ötsch W. , Metropolis Verlag, Marburg, 2012
\item Gesellschaft! Welche Gesellschaft? Nachdenken über eine sich wandelnde Gesellschaft: Ötsch W., Hirte K., Nordmann J. , Metropolis-Verlag, Marburg, 2011
\item Solidarität - Beiträge für eine gerechte Gesellschaft: Blaha B., Kapeller J., Weidenholzer J. , Braumüller, Wien, 2011
\item Krise! Welche Krise? Zur Problematik aktueller Krisendebatten: Ötsch W., Hirte K., Nordmann J. , Metropolis, Marburg, 2010
\item Das Imaginative der Politischen Ökonomie: Ötsch W., Hilt A. , Serie Kritische Studien zu Markt und Gesellschaft, Vol. 15, Metropolis Verlag, Marburg, 2024
\item Wissen und Nichtwissen der ökonomisierten Gesellschaft. Aufgaben einer neuen Politischen Ökonomie: Ötsch W., Steffestun T. , Metropolis Verlag, Marburg, 2021
\item Jenseits der Konventionen. Alternatives Denken zu Wirtschaft, Gesellschaft und Politik. Festschrift für Walter Ötsch: Pühringer S., Graupe S., Hirte K., Kapeller J., Panther S. , Metropolis, Marburg, 2020
\item Macht der Bilder, Macht der Sprache: Graupe S., Ötsch W. , Angelika Lenz Verlag, Isenburg, 2018
\item Policy Implications of Recent Advances in Evolutionary and Institutional Economics: Gräbner C., Heinrich T., Schwardt H. , Routledge, London, New York, 2016
\item The Political Economy of Offshore Jurisdictions: Ötsch W., Grözinger G., Beyer K., Bräutigam L. , Metropolis Verlag, Marburg, 2014
\item Change and persistence in contemporary economics: Kapeller J., Meyer D. , Serie Special Issue Science in Context, Vol. 32, 2019
\item The dynamics of and on networks: Gräbner C., Heinrich T., Kudic M., Vermeulen B. , in wiiw Opinion Piece, Serie International Journal of Computational Economics and Econometrics, Vol. 8, 2018
\item Power and Influence of Economists: Contributions to the Social Studies of Economic: Maesse J., Pühringer S., Rossier T., Benz P. , Routledge, London, 2022
\item Spielräume des Denkens: Graupe S., Ötsch W., Rommel F. , Metropolis, Marburg, 2019
\item Ökonomische Effekte der Verkehrsreform des Landes Tirol: Kapeller J., Böck M., Schütz B., Zens G. , Johannes Kepler Universität, Linz, 2018
\item "Ohne Effizienz geht es nicht". Ergebnisse einer qualitativ-empirischen Erhebung unter Studierenden der Volkswirtschaftslehre: Bäuerle L., Pühringer S., Ötsch W. , Serie FGW-Studien, Nr. 13, FGW, Düsseldorf, 2019
\item How to boost the European Green Deal’s scale and ambition: Kapeller J., Wildauer R., Leitch S. , Serie FEPS Policy Paper, 2020
\item Wirtschaftliche Polarisierung in Europa: Ursachen und Handlungsoptionen: Kapeller J., Gräbner C., Heimberger P. , Friedrich-Ebert Stiftung, Bonn, 2019
\item Netzwerke, Paradigmen, Attitüden. Der deutsche Sonderweg im Fokus. Paradigmatische Ausrichtung und politische Orientierung von deutschen und US-amerikanischen Ökonomi_nnen im Vergleich: Beyer K., Grimm C., Kapeller J., Pühringer S. , Nr. 7, FGW, Düsseldorf, 2018
\item Zum Profil der deutschsprachigen Volkwirtschaftslehre. Paradigmatische Ausrichtung und politische Orientierung deutschsprachiger Ökonom_innen: Kapeller J., Pühringer S., Grimm C. , Nr. 2, FGW-Studien, Düsseldorf, 2017
\item ÖkonomInnen und Politik – Analyse zur politischen Einflussnahme deutschsprachiger ÖkonomInnen: Griesser M., Hirte K., Pühringer S. , Forschungsbericht: Förderer: Jubiläumsfonds, Universität Linz, 2015
\item ÖkonomInnen und Ökonomie. Eine wissenschaftssoziologische Entwicklungsanalyse zum Verhältnis von ÖkonomInnen und Ökonomie im deutschsprachigen Raum ab 1945: Heise A., Hirte K., Ötsch W., Pühringer S., Reichl A., Sander H., Thieme S. , Hans-Böckler-Stiftung, Düsseldorf, 2015
\item MigrantInnen als Zielgruppe. Solidarische Beratungs- und Unterstützungsangebote von ArbeitnehmerInnenorganisationen in Österreich: Griesser M., Sauer B. , Serie Studie - Abschlussbericht, Universität, Institut für Politikwissenschaften, Wien, 2014
\item Vermögen in Österreich: Eckerstorfer P., Halak H., Kapeller J., Schütz B., Springholz F., Wildauer R. , Serie Materialien zu Wirtschaft und Gesellschaft, Nr. 126, AK Wien, WIen, 2014
\item Bestände und Verteilung der Vermögen in Österreich: Eckerstorfer P., Halak H., Kapeller J., Schütz B., Springholz F., Wildauer R. , Serie Materialien zu Wirtschaft und Gesellschaft, Vol. 122, Abteilung Wirtschaftswissenschaft und Statstik der Kammer für Arbeiter und Angestellte, Wien, 2013
\item Soziale Rhetorik, neoliberale Praxis: Eine Analyse der Wirtschafts- und Sozialpolitik der AfD: Pühringer S., Beyer K., Kronberger D. , Serie OBS Arbeitspapier, Nr. 52, Otto Brenner Stiftung, 2021
\item ÖkonomInnen in der Finanzkrise. Analyse zur Positionierung deutschsprachiger Ökonomen im Kontext ihrer strukturellen Verankerung: Hirte K., Pühringer S. , 2012
\item Agrarpolitik und Agrarökonomie. Zur Ambivalenz zweier wissenschaftlicher Disziplinen: Hirte K. , Universität Jena, 2017
\item Economic Change and Change in Economics: Kapeller J. , Universität Linz, 2014
\item Fiscal multipliers, unemployment and debt: Heimberger P. , Wirtschaftsuniversität Wien, 2018
\item The strange non-crisis of economics. Economic crisis and the crisis policies in economic and political discourses.: Pühringer S. , Universität Linz, 2015
\item Modell-Platonismus in der Ökonomie. Zur Aktualität einer klassisch-epistemologischen Kritik: Kapeller J. , Universität Linz, 2011
\item Wirtschaftspolitische Ausrichtung österreichischer Parteien im historischen Verlauf. Die Ausgestaltung österreichischer Parteiprogrammatiken unter dem Einfluss neoliberalen Gedankenguts: Grimm C. , 2015
\item Illusionen eines Zirkulationskünstlers? Pierre-Joseph Proudhon auf dem ökonomiekritischen Prüfstand: Beyer K. , Universität Wien, 2012
\item Delayed by outsourcing? Zur Stabilität des Kapitalismus im 21. Jahrhundert (Doppelrezension): Kapeller J. , in Hrsg. v. Hollstein, Betina / Schimank, Uwe / Struck, Olaf / Weiß, Anja, in Soziologische Revue, Vol. 40, Nr. 4, DeGruyter, Seite(n) 547–555, 2017
\item Antonino Palumbo and Alan Scott (2018): Remaking Market Society: A Critique of Social Theory and Political Economy in Neoliberal Times: Beyer K. , in ÖZS - Österreichische Zeitschrift für Soziologie, Vol. 44, Nr. 2, Seite(n) 249-252, 2019
\item Dialog nicht erwünscht: Graupe S., Ötsch W. , in Forschung & Lehre, Vol. 23, Nr. 11, Seite(n) 1000, 2016
\item Rezension von „Martina Benz: Zwischen Migration und Arbeit. Worker Centers und die Organisierung prekär und informell Beschäftigter in den USA“: Griesser M. , in Bund demokratischer Wissenschaftlerinnen und Wissenschaftler, in Forum Wissenschaft, BdWi-Verlag, 2015
\item Rezension von „Monika Burmester, Emma Dowling & Norbert Wohlfahrt (Hg.) (2017): Privates Kapital für soziale Dienste? Wirkungsorientiertes Investment und seine Folgen für die Soziale Arbeit“: Griesser M. , in Soziales Kapital, Seite(n) 264-267, 2017
\item Zunehmende Ungleichheit der Vermögensverteilung: Rezension von Michael Schneider, Mike Pottenger und John E. King „The Distribution of Wealth – Growing Inequality?“: Schütz B. , in Wirtschaft und Gesellschaft, Vol. 43, Nr. 3, Seite(n) 449-451, 2017
\item Die aktuelle Krise im wirtschaftshistorischen Vergleich mit der Großen Depression der 1930er-Jahre: Heimberger P. , in Wirtschaft und Gesellschaft, Vol. 42, Nr. 1, Seite(n) 161-173, 2016
\item Helikoptergeld zur Überwindung der Wachstumsprobleme in Europa?: Heimberger P. , in AK Wien, in Wirtschaft und Gesellschaft, Vol. 42, Nr. 4, Wien, Seite(n) 690-695, 2016
\item Minsky, die globle Finanzkrise und der nächste Finanz-Crash: Heimberger P. , in Wirtschaft und Gesellschaft, Vol. 42, Nr. 3, Seite(n) 515-520, 2016
\item Wirtschaftliche Stagnation als "neue Normalsituation"?: Heimberger P. , in Wirtschaft und Gesellschaft, Vol. 42, Nr. 2, Seite(n) 356-361, 2016
\item Eine fiskalpolitische Lösung für die Eurozone: Heimberger P. , in Kammer für Arbeiter und Angestellte Wien, in Wirtschaft und Gesellschaft, Vol. 41, Nr. 3, Lexis Nexis, Seite(n) 449-458, 2015
\item Raus aus dem Euro?: Heimberger P. , in Wirtschaft und Gesellschaft, Vol. 41, Nr. 4, Seite(n) 603-614, 2015
\item Die Rückkehr des Rentiers. Rezension zu: Piketty, Thomas (2014): Capital in the 21st century. Cambridge: Harvard University Press: Kapeller J. , in Wirtschaft und Gesellschaft, Vol. 40, Nr. 2, Seite(n) 329-346, 2014
\item Soziale Frage im Wandel. Probleme und Perspektiven des Sozialstaates und der Arbeitsgesellschaft: Pühringer S. , in Kontraste - Presse- und Informationsdienst für Sozialpolitik, Nr. 2/2012, Seite(n) 22-23, 2012
\item Öffentlicher Vernunftgebrauch - ein probantes Mittel zur Bekämpfung von Ungerechtigkeit?: Pühringer S. , in Studentisches Soziologiemagazin, 2012
\item Aktivierung und Mindestsicherung. Rezension zum gleichnamigen Buch von Christine Stelzer-Orthofer und Josef Weidenholzer: Pühringer S. , in WISO - Wirtschafts- und sozialpolitische Zeitschrift, Vol. 34, Nr. 02, Seite(n) 169-171, 2011
\item Exploring the trade (policy) narratives in economic elite discourse: Aistleitner M., Pühringer S. , Serie SSRN, 2020
\item Competitive Performativity of Academic Social Networks. The Subjectification of Competition on Researchgate, Twitter and Google Scholar.: Pühringer S., Wolfmayer G. , Serie SPACE Working Paper Series, Nr. 19, Seite(n) 1-25, 2023
\item Dilemmata marktliberaler Globalisierung – Globale Freiheit durch globalen Wettbewerb?: Kapeller J., Hubmann G. , Serie ICAE Working Paper Series, Nr. 151, Seite(n) 1-18, 2023
\item Endogenous Heterogeneous Gender Norms and the Distribution of Paid and Unpaid Work in an Intra-Household Bargaining Model: Hager T., Mellacher P., Rath M. , Serie ICAE Working Paper Series, Nr. 147, Seite(n) 1-32, 2023
\item Gendered Competitive Practices in Economics. A Multi-Layer Model of Women’s Underrepresentation: Hager T., Pühringer S. , Serie ICAE Working Paper Series, Vol. 148, Seite(n) 1-22, 2023
\item Organizers and promotors of academic competition? The role of (academic) social networks and platforms in the competitization of science: Pühringer S., Wolfmayer G. , Serie ICAE Working Paper Series, Nr. 152, Seite(n) 1-21, 2023
\item The charm of emission trading: Ideas of German public economists on economic policy in times of crises: Porak L., Reinke R. , Serie ICAE Working Paper Series, Nr. 145, 2023
\item The three faces of competitization: From marketization to a multiplicity of competition: Altreiter C., Gräbner-Radkowitsch C., Pühringer S., Rogojanu A., Wolfmayr G. , Serie SPACE Working Paper Series, Nr. 18, 2023
\item Wie viel Wettbewerb wollen wir (uns leisten)? Zur Verwettbewerblichung der Universitäten in Österreich und darüber hinaus: Pühringer S. , Serie ICAE Working Paper Series, Vol. 149, Seite(n) 1-19, 2023
\item Competing for sustainability? An institutionalist analysis of the new development model of the European Union: Gräbner-Radkowitsch C., Hager T., Hornykewycz A. , Serie SPACE Working Paper Series, Nr. 17, 2022
\item Degrowth and the Global South? How institutionalism can complement a timely discourse on ecologically sustainable development in an unequal world: Gräbner-Radkowitsch C., Strunk B. , Serie ICAE Working Paper Series, Nr. 144, 2022
\item Degrowth and the global South: the twin problem of global dependencies: Gräbner-Radkowitsch C., Strunk B. , Serie ICAE Working Paper Series, Nr. 142, 2022
\item Development and Interdisciplinarity: re-examining the “economic silo”: Aistleitner M. , Serie SPACE Working Paper Series, Nr. 14, 2022
\item Investment booms, diverging competitiveness and wage growth within a monetary union: An AB-SFC model: Schütz B. , Serie ICAE Working Paper Series, Nr. 138, 2022
\item Polanyi and Schumpeter: Transitional Processes via Societal Spheres: Hager T., Heck I., Rath J. , Serie SPACE Working Paper Series, Nr. 16, Seite(n) 1-25, 2022
\item Winning urban competition with a social agenda. The competition imaginary in Viennese urban development: Altreiter C., Azevedo S., Porak L., Pühringer S., Wolfmayr G. , Serie SPACE Working Paper Series, Nr. 13, 2022
\item (Mis)Measuring Competitiveness: The Quantification of a Malleable Concept in the European Semester: Gräbner C., Hager T. , Serie SPACE Working Paper Series, Nr. 10, 2021
\item (Mis)Measuring Competitiveness: The Quantification of a Malleable Concept in the European Semester: Gräbner C., Hager T. , Serie ICAE Working Paper Series, Nr. 130, 2021
\item Competition in Transitional Processes: Polanyi & Schumpeter: Hager T., Heck I., Rath J. , Serie ICAE Working Paper Series, Nr. 128, 2021
\item Do Corporate Tax Cuts Boost Economic Growth?: Heimberger P., Gechert S. , Serie wiiw Working Papers, Nr. 201, forthcoming, doi/full/10.1080/09692290.2021.1904269, 2021
\item Do Higher Public Debt Levels Reduce Economic Growth?: Heimberger P. , Serie wiiw Working Papers, Nr. 211, 2021
\item Exploring the trade (policy) narratives in economic elite discourse: Aistleitner M., Pühringer S. , Serie SPACE Working Paper Series, Nr. 1, Seite(n) 1-16, 2021
\item Governing the Ungovernable. Recontextualizations of Competition in European Policy Discourse: Porak L. , in ICAE Linz, Serie SPACE Working Paper Series, Nr. 8, 2021
\item Theorizing Competition - An interdisciplinary approach to the genesis of a contested concept: Altreiter C., Gräbner C., Pühringer S., Rogojanu A., Wolfmayr G. , Serie SPACE Working Paper Series, Nr. 3, 2021
\item Theorizing Competition. An Interdisciplinary Framework: Altreiter C., Gräbner C., Pühringer S., Rogojanu A., Wolfmayr G. , Serie SPACE Working Paper Series, Nr. 4, 2021
\item Capability Accumulation and Product Innovation: Agent-Based Perspective: Hornykewycz A., Gräbner-Radkowitsch C. , Serie Rebuilding Macroeconomics Working Paper Series, Nr. 9, Seite(n) 1-22, 2020
\item Das doppelte Relexionsproblem in der Ökonomik: Hirte K. , Serie ICAE Working Paper Series, Nr. 115, Seite(n) 1-19, 2020
\item Emergence of Core-Periphery Structures in the European Union: A Complexity Perspective: Gräbner-Radkowitsch C., Hafele J. , Serie Rebuilding Macroeconomics Working Paper Series, Nr. 17, Seite(n) 1-21, 2020
\item Friedman’s Instrumentalismus und das Problem von Kopernikus: Hirte K. , Serie ICAE Working Paper Series, Nr. 114, Seite(n) 1-25, 2020
\item Special Interest Groups & Growth: Hager T. , Serie ICAE Working Paper Series, Nr. 116, 2020
\item Substituting Trust by Technology: A Comparative Study: Rath J. , Serie ICAE Working Paper Series, Nr. 107, 2020
\item Talking about Competition? Discursive Shifts in the Economic Imaginary of Competition in Public Debates: Pühringer S., Porak L., Rath J. , Serie SPACE Working Paper Series, Nr. 6, 2020
\item Theory and empirics of capability accumulation: implications for macroeconomics modelling: Aistleitner M., Gräbner-Radkowitsch C., Hornykewycz A. , Serie Rebuilding Macroeconomics Working Paper Series, Nr. 6, Seite(n) 1-29, 2020
\item Zwischen Meritokratie und Wohlfahrtschauvinismus: Beyer K., Griesser M., Pühringer S. , Serie ICAE Working Paper Series, Nr. 109, 2020
\item A Comment on Fitting Pareto Tails to Complex Survey Data: Wildauer R., Kapeller J. , Serie ICAE Working Paper Series, Nr. 102, 2019
\item Confict as a closure: A Kaleckian model of growth and distribution under fnancialization: Raghavendra S., Piiroinen P. , Serie ICAE Working Paper Series, Nr. 96, 2019
\item Das dritte Gossensche Gesetz. Zur Überlieferungspraxis in der ökonomischen Dogmengeschichte: Hirte K. , Serie ICAE Working Paper Series, Nr. 93, Seite(n) 1-63, 2019
\item Defining institutions - A review and a synthesis: Gräbner C., Ghorbani A. , Serie ICAE Working Paper Series, Nr. 89, 2019
\item Die Wirkmacht der "Liebe zum Markt": Zum anhaltenden Einfluss ordoliberaler ÖkonomInnenNetzwerke in Politik und Gesellschaft.: Pühringer S., Ötsch W. , in Institut für Ökonomie, Cusanus Hochschule, Serie Working Paper Serie ök, Nr. 51, 2019
\item Divided we stand? Professional consensus and political conflict in academic economics: Beyer K., Pühringer S. , in Institut für Ökonomie, Cusanus Hochschule, Serie Working Paper Serie ök, Nr. 51, 2019
\item Does economic globalisation affect income inequality? A meta-analysis: Heimberger P. , Serie wiiw Working Papers, Nr. 165, 2019
\item Exploring the trade narrative in top economics journals: Aistleitner M., Pühringer S. , Serie ICAE Working Paper Series, Nr. 97, 2019
\item Export performance, price comeptitiveness and technology: Revisiting the Kaldor paradox: Gräbner C., Heimberger P., Kapeller J. , Serie ICAE Working Paper Series, Nr. 88, 2019
\item The Impact of Labour Market Institutions and Capital Accumulation on Unemployment: Evidence for the OECD, 1985-2013: Heimberger P. , Serie wiiw Working Papers, Nr. 164, 2019
\item The anti-democratic logic of right-wing Populism and neoliberal market-fundamentalism: Ötsch W., Pühringer S. , in Institut für Ökonomie, Cusanus Hochschule, Serie Working Paper Serie ök, Nr. 48, 2019
\item Unrealistic models and how to identify them: on accounts of model realisticness: Gräbner C. , Serie ICAE Working Paper Series, Nr. 90, 2019
\item Vorschlag für eine Jobgarantie für Langzeitarbeitslose in Österreich: Tamesberger D., Theurl S. , Serie ICAE Working Paper Series, Nr. 100, 2019
\item Wirtschaftliche Polarisierung in Europa: Ursachen und Handlungsoptionen: Kapeller J., Gräbner C., Heimberger P. , Serie ICAE Working Paper Series, Nr. 98, 2019
\item Arbeitsmarktpolitik als Gesellschaftspolitik: Griesser M. , Serie ICAE Working Paper Series, Nr. 78, 2018
\item Auftragsvergabe, Leistungsqualität und Kostenintensität im Schienenpersonenverkehr: Aistleitner M., Grimm C., Kapeller J. , Serie ICAE Working Paper Series, Nr. 86, 2018
\item Bilder in der Geschichte der Ökonomie: Das Beispiel der Metapher von der Wirtschaft als Maschine: Ötsch W. , in Institut für Ökonomie, Cusanus Hochschule, Bernkastel-Kues., Serie Working Paper Serie ök, Vol. 42, 2018
\item Der vergessene Lippmann: Politik, Propaganda und Markt: Ötsch W., Graupe S. , in Institut für Ökonomie, Cusanus Hochschule, Bernkastel-Kues, Serie Working Paper Serie ök, Nr. 39, 2018
\item Employment and the minimum wage: A pluralist approach: Schütz B. , Serie ICAE Working Paper Series, Nr. 81, 2018
\item Factors Affecting Acces to Formal Credit by Micro and Small Enterprises in Ugande: Buyinza F., Tibaingana A., Mutenyo J. , Serie ICAE Working Paper Series, Nr. 83, 2018
\item Folgen einer möglichen Abschaffung der Notstandshilfe in Oberösterreich: Foissner F. , Serie ICAE Working Paper Series, Nr. 87, 2018
\item Household Electrification and Education Outcomes: Panel Evidence from Uganda: Buyinza F., Kapeller J. , Serie ICAE Working Paper Series, Nr. 85, 2018
\item Marktfundamentalismus als Kollektivgedanke. Mises und die Ordoliberalen: Ötsch W., Pühringer S. , in Institut für Ökonomie, Cusanus Hochschule, Bernkastel-Kues, Serie Working Paper Serie ök, Nr. 41, 2018
\item The Focus of Academic Economics: Before and After the Crisis: Aigner E., Aistleitner M., Glötzl F., Kapeller J. , in Institute for New Economic Thinking (INET), Serie Institute for New Economic Thinking  Working Paper Series, 2018
\item The focus of academic economics: before and after the crisis: Aigner E., Aistleitner M., Glötzl F., Kapeller J. , Serie ICAE Working Paper Series, Nr. 75, 2018
\item What economics education is missing: The real world: Pühringer S., Bäuerle L. , in Institut für Ökonomie, Cusanus Hochschule, Serie Working Paper Serie ök, Nr. 37, 2018
\item Wissen und Nicht-Wissen angesichts "des Marktes": Das Konzept von Hayek: Ötsch W. , in Institut für Ökonomie, Cusanus Hochschule, Bernkastel-Kues, Serie Working Paper Serie ök, Vol. 43, 2018
\item Argumentationsstrategien einer neoliberalen Reformagenda: Zum Diskursprofil der Agenda Austria in medialen Debatten: Pühringer S., Liedl B. , in Institut für Ökonomie, Cusanus Hochschule, Bernkastel-Kues, Serie Working Paper Serie ök, Vol. 27, 2017
\item Bestände und Konzentration privater Vermögen in Österreich: Ferschli B., Kapeller J., Schütz B., Wildauer R. , Serie ICAE Working Paper Series, Nr. 72, 2017
\item Der deutsche Sonderweg im Fokus: Eine vergleichende Analyse der paradig¬matischen Struktur und der politischen Orientierung der deutschen und US-amerikanischen Ökonomie: Beyer K., Grimm C., Kapeller J., Pühringer S. , Serie ICAE Working Paper Series, Nr. 71, 2017
\item Emergent Phenomena in Scientific Publishing: A Simulation Exercise: Kapeller J., Steinerberger S. , Serie ICAE Working Paper Series, Nr. 66, 2017
\item From paradigms to policies: Economic models in the EU’s fiscal regulation framework: Huber J., Heimberger P., Kapeller J. , Serie ICAE Working Paper Series, Nr. 61, 2017
\item From the "planning euphoria" to the "bitter economic truth": The transmission of economic ideas into German labour market policies in the 1960s and 2000s: Pühringer S., Griesser M. , in Institut für Ökonomie, Cusanus Hochschule, Bernkastel-Kues, Serie Working Paper Serie ök, Nr. 30, 2017
\item Inheritances and the Accumulation of Wealth in the Eurozone: Humer S., Moser M., Schnetzer M. , Serie ICAE Working Paper Series, Nr. 73, 2017
\item Neoliberalism and Right-wing Populism: conceptual analogies: Pühringer S., Ötsch W. , in Institut für Ökonomie, Cusanus Hochschule, Bernkastel-Kues, Serie Working Paper Serie ök, Vol. 36, 2017
\item Pluralism in Economics: Epistemological Rationales and Pedagogical Implementation: Kapeller J. , Serie ICAE Working Paper Series, Nr. 68, 2017
\item Right-wing populism and market-fundamentalism: Ötsch W., Pühringer S. , Serie ICAE Working Paper Series, Vol. 59, 2017
\item The “eternal character” of austerity measures in European crisis policies. Evidences from the Fiscal Compact discourse in Austria.: Pühringer S. , in Institut für Ökonomie, Cusanus Hochschule, Bernkastel-Kues, Serie Working Paper Serie ök, Nr. 32, 2017
\item Think tank networks of German neoliberalism power structures in economics and economic policies in post-war Germany: Pühringer S. , in Institut für Ökonomie, Cusanus Hochschule, Bernkastel-Kues, Serie Working Paper Serie ök, Nr. 24, 2017
\item Zum Profil der deutschsprachigen Volkswirtschaftslehre: Grimm C., Kapeller J., Pühringer S. , Serie ICAE Working Paper Series, Nr. 70, 2017
\item A model-based measurement device in European fiscal policy-making: The ontology and epistemology of potential output: Heimberger P., Kapeller J. , Serie ICAE Working Paper Series, Nr. 55, 2016
\item Did Fiscal Consolidation Cause the Double Dip Recession in the Euro Area?: Heimberger P. , Serie wiiw Working Papers, Nr. 130, 2016
\item Die aktuelle Krise im wirtschaftshistorischen Vergleich mit der Großen Depression der 1930er-Jahre: Heimberger P. , Serie ICAE Working Paper Series, Nr. 42, 2016
\item Has economics returned to being the “dismal science”? The changing role of economic thought in German labour market reforms: Pühringer S., Griesser M. , Serie ICAE Working Paper Series, Vol. 49, 2016
\item Imaginierte Grundlagen bei Adam Smith: Ötsch W. , in Institut für Ökonomie, Cusanus Hochschule, Bernkastel-Kues, Serie Working Paper Serie ök, Nr. 19, 2016
\item Monocular Accounting and its Discontents: The case of the stability and growth pact: Dobusch L., Wandl S., Kapeller J. , Serie ICAE Working Paper Series, Nr. 43, 2016
\item Neoliberale Think Tanks als (neue) Akteure in österreichischen gesellschafts- und sozialpolitischen Diskursen: Pühringer S., Stelzer-Orthofer C. , Serie ICAE Working Paper Series, Nr. 44, 2016
\item Postdemokratie, Machtverhältnisse und Ökonomik: Grimm C. , Serie ICAE Working Paper Series, Nr. 54, 2016
\item Still the queen of the social sciences?: Pühringer S. , Serie ICAE Working Paper Series, Nr. 52, 2016
\item The Power of Scientometrics and the Development of Economics: Aistleitner M., Kapeller J., Steinerberger S. , Serie ICAE Working Paper Series, Nr. 46, 2016
\item The performativity of potential output: Heimberger P., Kapeller J. , Serie ICAE Working Paper Series, Nr. 50, 2016
\item The performativity of potential output: pro-cyclicality and path dependency in coordinating European fiscal policies: Heimberger P., Kapeller J. , in Institute for New Economic Thinking, Serie Institute for New Economic Thinking  Working Paper Series, Nr. 50, 2016
\item Think Tank Networks of German Neoliberalism: Pühringer S. , Serie ICAE Working Paper Series, Nr. 53, 2016
\item Wirtschaftliche Stagnation als "neue Normalsituation"?: Heimberger P. , Serie ICAE Working Paper Series, Nr. 48, 2016
\item Wirtschaftspolitische Positionen österreichischer Parteien im historischen Verlauf: Grimm C. , Serie ICAE Working Paper Series, Nr. 51, 2016
\item Did Fiscal Consolidation Cause the Double-Dip Recession in the Euro Area?: Heimberger P. , Serie ICAE Working Paper Series, Nr. 41, 2015
\item Emanzipation bei Marx und seine Kritik an Proudhon: Beyer K. , Serie ICAE Working Paper Series, Nr. 34, 2015
\item Markets as “ultimate judges” of economic policies - Angela Merkel´s discourse profile during the economic crisis and the European crisis policies.: Pühringer S. , Serie ICAE Working Paper Series, Nr. 31, 2015
\item Marktradikalismus als Politische Ökonomie. Wirtschaftswissenschaften und ihre Netzwerke in Deutschland ab 1945: Ötsch W., Pühringer S. , Serie ICAE Working Paper Series, Nr. 38, 2015
\item Shadow Banking and the Offshore Nexus - Some Considerations on the Systemic Linkages of Two Important Economic Phenomena: Beyer K., Bräutigam L. , Serie ICAE Working Paper Series, Nr. 40, 2015
\item Verteilung und Gerechtigkeit: Philosophische Perspektiven: Aistleitner M., Fölker M., Kapeller J., Mohr F., Pühringer S. , Serie ICAE Working Paper Series, Nr. 32, 2015
\item Wie wirken ÖkonomInnen und Ökonomik auf Politik und Gesellschaft? Darstellung des gesellschaftlichen und politischen Einflusspotenzials von ÖkonomInnen anhand eines „Performativen Fußabdrucks“: Pühringer S. , Serie ICAE Working Paper Series, Nr. 35, 2015
\item Ökonomie und Moral: Ötsch W. , Serie ICAE Working Paper Series, Nr. 39, 2015
\item From Free to Civilized Markets: First Steps towards Eutopia: Kapeller J., Schütz B., Tamesberger D. , Serie ICAE Working Paper Series, Nr. 28, 2014
\item Kontinuitäten neoliberaler Wirtschaftspolitik in der Krise: Pühringer S. , Serie ICAE Working Paper Series, Nr. 30, 2014
\item Relationships are Constructed from Generalized Unconscious Social Images Kept in Steady Locations in Mental Space: Ötsch W. , Serie ICAE Working Paper Series, Nr. 29, 2014
\item Warum war Keynes so erfolgreich? Eine Darstellung anhand der Methode von Bruno Latour: Ötsch W. , Serie ICAE Working Paper Series, Nr. 27, 2014
\item Aus den Vorhöfen der Macht in die Medien zur eigenen Partei. Formen der Einflussnahme von ÖkonomInnen auf Politik und Wirtschaft im Zuge der Finanz-­ und Wirtschaftskrisenpolitik.: Pühringer S. , Serie ICAE Working Paper Series, Nr. 20, 2013
\item CDOs – A Critical Phenomenon of the Financial System in Crisis.: Beyer K., Bräutigam L. , Serie ICAE Working Paper Series, Nr. 21, 2013
\item Das Team Stronach. Eine österreichische Tea Party?: Pühringer S., Ötsch W. , Serie ICAE Working Paper Series, Nr. 19, 2013
\item Das neoliberale Selbst. Zur Genese und Kritik neuer Subjektkonstruktionen: Nordmann J. , Serie ICAE Working Paper Series, Nr. 22, 2013
\item Der Fiskalpakt und seine Implementation in Österreich: Plaimer W., Pühringer S. , Serie ICAE Working Paper Series, Vol. 2, 2013
\item Die Finanzkrise 2007-2009 als Krise von Schattenbanken. Eine einführende institutionelle Analyse.: Beyer K., Ötsch W. , Serie ICAE Working Paper Series, Nr. 17, 2013
\item Die neoliberale Gesellschaft. Ein theoretischer Umriss.: Nordmann J. , Serie ICAE Working Paper Series, Nr. 24, 2013
\item Mainstream, Orthodoxie und Heterodoxie - Zur Klassifizierung der Wirtschaftswissenschaften.: Hirte K. , Serie ICAE Working Paper Series, Nr. 16, 2013
\item Populismus und Demagogie. Mit Beispielen von Jörg Haider, Heinz–Christian Strache und Frank Stronach: Ötsch W. , Serie ICAE Working Paper Series, Nr. 25, 2013
\item The financial crisis as a tsunami. Discourse profiles of economists in the financial crisis.: Pühringer S., Hirte K. , in ICAE Uni Linz, Serie ICAE Working Paper Series, Nr. 14, 2013
\item The implementation of the European Fiscal Compact in Austria as a post-democratic phenomenon.: Pühringer S. , Serie ICAE Working Paper Series, Nr. 15, 2013
\item Zauberwort „Entkopplung“ – Ein Rückblick auf den Diskurs zur Agrarreform 2003 sowie die dortigen Diskrepanzen zwischen Versprechen und Wirkprinzipien der so genannten „Entkopplung“.: Hirte K. , Serie ICAE Working Paper Series, Nr. 12, 2013
\item ÖkonomInnen in der Finanzkrise - Netzwerkanalytische Sicht auf die deutschsprachigen ÖkonomInnen in der Finanzkrise: Pühringer S., Hirte K. , Serie ICAE Working Paper Series, Nr. 18, Seite(n) 1-21, 2013
\item Democracy in liberalism and neoliberalism. The case of Popper and Hayek.: Kapeller J., Pühringer S. , Serie ICAE Working Paper Series, Nr. 10, 2012
\item Die Macht der Ratingagenturen: Ötsch W. , Serie ICAE Working Paper Series, Nr. 8, 2012
\item Erdbeben, Fieber und zarte Pflänzchen. Chronologischer Verlauf des Finanzkrisen‐Diskurses deutschsprachiger Ökonomen: Pühringer S., Hirte K. , in ICAE, JKU, Serie ICAE Working Paper Series, Nr. 9, Linz, 2012
\item Marktradikalität. Der Diskurs von "dem Markt": Ötsch W. , Serie ICAE Working Paper Series, Nr. 7, 2012
\item The Deep Meaning of “Market”. A Key to Understand the Neoliberal-­‐Market-­‐ Radical Society.: Ötsch W. , Serie ICAE Working Paper Series, Nr. 11, 2012
\item Demokratiedefizit im Gesetzgebungsprozess budgetrelevanter Gesetze auf Bundesebene in Österreich: Plaimer W. , Serie ICAE Working Paper Series, Nr. 4, 2011
\item Gleichheit versus Vielfalt. Ein konstruierter Widerspruch?: Pühringer S. , Serie ICAE Working Paper Series, Nr. 3, 2011
\item Politische Ökonomie und Gesellschaft. Eine theoriegeschichtliche Skizze: Ötsch W. , Serie ICAE Working Paper Series, Nr. 5, 2011
\item Solidarität im Kapitalismus. Zur Unmöglichkeit einer Forderung: Pühringer M., Pühringer S. , Serie ICAE Working Paper Series, Nr. 6, 2011
\item Die Evolution des ökonomischen Wissens und des Wissens über den Kapitalismus. Performativity als Analyseinstrument: das Beispiel der Fabian Society, der Mont Pèlerin Society und der Chicagoer Schule: Ötsch W., Hirte K., Nordmann J. , in ICAE, JKU, Serie ICAE Working Paper Series, Nr. 1, Linz, 2010
\item Die Finanz-­ und Wirtschaftskrise seit 2007: Ein Überblick: Ötsch W. , Serie ICAE Working Paper Series, Nr. 2, 2010
\item Quelle place pour le Sud global dans la décroissance?: Gräbner-Radkowitsch C., Strunk B. , in The Conversation, 2024
\item Degrowth and the Global South: remarks on the twin problem of structural interdependencies: Gräbner-Radkowitsch C., Strunk B. , in Developing economics blog, 2023
\item Degrowth und der globale Süden: Gräbner-Radkowitsch C., Strunk B. , in Blog Postwachstum, 2023
\item Die (selbstauferlegten) Grenzen der Wissenschaft: Pühringer S., Altreiter C. , in Makronom, 2023
\item Euphemistische Ökonomik: Hirte K. , in Blog Postwachstum, 2023
\item Prekäre Arbeit an Universitäten kann man nicht wegrechnen: Pühringer S., Partheymüller J. , in Der Standard, 2023
\item Arbeitsplatz Universität: Gefangen in der Teilzeit: Breth L., Part F., Pühringer S., Schlitz N., Sperner P., Völkl Y. , in Der Standard, 2022
\item Das wackelige Fundament der Schuldenbremse: Heimberger P., Schütz B. , in Makronom, 2022
\item Strukturkonzentrationen in der Schlachthofbranche und die Rolle von Ökonomen: Hirte K. , in Oxi - Wirtschaft anders denken, Nr. 01/22, Seite(n) 8, 2022
\item Verzerrter Wettbewerb in der Forschung: Aistleitner M. , in science.orf.at, 2022
\item Wechselseitige Kritik ist nur möglich mit nachvollziehbarer Forschung: Kapeller J. , in ZBW Leibniz Informationszentrum Wirtschaft, Seite(n) 1-2, 2022
\item Zur ökonomischen Bedeutung von Suffizienz: Gräbner-Radkowitsch C., Lage J., Wiese F. , in Makronom, 2022
\item 20 Jahre gute Absichten in der Wissenschaftspolitik: Altreiter C., Rogojanu A., Gräbner C., Pühringer S., Wolfmayr G. , in Die Presse, 2021
\item Arbeitnehmerrechte sind kein Jobkiller: Heimberger P. , in Handelsblatt, 2021
\item Beschäftigungsschutz erhöht die Arbeitslosigkeit nicht: Heimberger P. , in Ökonomenstimme, 2021
\item Der Nobelpreis für David Card hat die Mindestlohn-Befürworter in Deutschland gestärkt: Heimberger P. , in Handelsblatt, 2021
\item Die EU-Anleihen sind ein Zukunftsmodell für Europa: Heimberger P. , in Handelsblatt, 2021
\item Die Finanzglobalisierung verschärft die Einkommensungleichheit: Heimberger P. , in Handelsblatt, 2021
\item Die deutschen Inflationssorgen speisen sich aus einem verzerrten Geschichtsbild: Heimberger P. , in Handelsblatt, 2021
\item Draghi darf das Sparen nicht übertreiben: Heimberger P. , in Handelsblatt, 2021
\item Drei Gründe, warum Staatsschulden nicht zwingend problematisch sind: Heimberger P. , in Handelsblatt, 2021
\item Eine globale Mindeststeuer stoppt die Steuerflucht der Konzerne: Heimberger P. , in Handelsblatt, 2021
\item Erhöhen Unternehmenssteuersenkungen das Wirtschaftswachstum?: Heimberger P., Gechert S. , in Ökonomenstimme, 2021
\item Fakten-Mensch: Ökonom Philipp Heimberger kritisiert die aus seiner Sicht verzerrte Debatte über die EU: Heimberger P. , in Süddeutsche Zeitung, 2021
\item Höhere Staatsschulden = weniger Wachstum?: Heimberger P. , in Makronom, 2021
\item Sieben „überraschende“ Fakten zu Italien: Heimberger P., Kowall N. , in Makronom, 2021
\item Und ewig lockt der Wettbewerb: Porak L. , in Makronom, 2021
\item Wachstum durch Unternehmensteuersenkungen? Die FDP weckt übertriebene Hoffnungen: Heimberger P. , in Handelsblatt, 2021
\item Wie befristete Uni-Stellen Innovation verhindern: Altreiter C., Gräbner C., Pühringer S., Rogojanu A., Wolfmayr G. , in Der Standard, 2021
\item Wie stark der globale Steuerwettbewerb tatsächlich ist: Heimberger P. , in Makronom, 2021
\item Ökonomie und Ideologie: Hirte K. , in OXI - Wirtschaft anders denken, Seite(n) 16-17, 2021
\item Ökonomische Offenheit: Die Vermessung der Globalisierung?: Gräbner C., Heimberger P., Kapeller J. , in Ökonomenstimme, 2021
\item Agrarsubventionen als Preis der Marktwirtschaft: Hirte K. , in Wege für eine Bäuerliche Zukunft – Zeitschrift der ÖBV/ Via Campesina Austria, Vol. 43, Nr. 1 (361), Wien, Seite(n) 10-11, 2020
\item Beschäftigungsschutz ist kein Jobkiller: Heimberger P. , in Makronom, 2020
\item Der Outputlücken-Nonsense gefährdet Deutschlands Erholung von der Corona-Krise: Heimberger P., Truger A. , in Makronom, 2020
\item Die gefährliche Geschichtsverdrehung des Hans-Werner Sinn: Heimberger P. , in Handelsblatt, 2020
\item Einigung der Eurogruppe: Bestenfalls ein erster Schritt: Heimberger P. , in Makronom, 2020
\item Enttäuschender EU-Gipfel: Zeit für eine „Neue Südpolitik”: Heimberger P. , in Makronom, 2020
\item Gewohnter Umsatz wird auf sich warten lassen: Schütz B. , in Kronen Zeitung, Interview (von Elisabeth Rathenböck) mit Bernhard Schütz, 2020
\item Gleichwertige Lebensverhältnisse im Euroraum: Heimberger P., Krahé M., van't Klooster J., Ponattu D. , in Frankfurter Allgemeine Zeitung, 2020
\item Kein Weg in die „Schuldenunion“: Heimberger P. , in Der Standard, 2020
\item Kritik an der Pluralen Ökonomik – Was ist dran und warum ist das wichtig?: Gräbner C., Strunk B. , in Ökonomenstimme, 2020
\item Wahrheiten über Wertfreiheit – eine Replik auf den Gastkommentar ‚Werte, Wahrheit, Wirtschaftsforschung‘ von Harald Oberhofer: Kapeller J. , in Der Standard, 2020
\item Wie die EU-Kommission Deutschlands Budgetsituation schlechtrechnet: Heimberger P. , in Handelsblatt, 2020
\item Wie die ‚Herren der Modelle‘ versuchen, den Outputlücken-Nonsense zu rechtfertigen: Heimberger P., Kapeller J. , in Makronom, 2020
\item Wie ein Rechenfehler zu fatalen Konsequenzen für die Haushalte der EU-Staaten führen kann: Heimberger P. , in Handelsblatt, 2020
\item Zu niets doen voor Z-Europa is gevaarlijk (Nichts für Südeuropa zu tun ist gefährlich): Heimberger P. , in NRC Handelsblad, 2020
\item Ökonomische Offenheit: Die Vermessung der Globalisierung: Gräbner C., Heimberger P., Kapeller J. , in Makronom, 2020
\item „Um welches Ziel es in der Ökonomie geht, ist also bestimmbar“: Hirte K. , in Agora42 (Philosophisches Wirtschaftsmagazin), 2020
\item Auch Deutschland wird zum Opfer des "Outputlücken-Nonsens": Heimberger P., Kapeller J. , in Makronom, 2019
\item Den wirtschaftlichen Abschwung bekämpfen!: Heimberger P., Pekanov A. , in Der Standard, 2019
\item Eine Strategie gegen die ökonomische Polarisierung Europas: Heimberger P., Gräbner C., Kapeller J. , in Makronom, 2019
\item Männlich, mikroökonomisch, Mainstream? Eine Untersuchung der VWL-Lehrstühle in Deutschland: Pühringer S., Grimm C. , in Makronom, 2019
\item The danger of  'nonsense Output gaps': Heimberger P. , in Financial Times, 2019
\item Welche Rolle spielt der „Outputlücken-Nonsense“?: Heimberger P. , in Makronom, 2019
\item Wirtschaftlicher Abschwung: Relevanz und Gegenmaßnahmen: Heimberger P. , in Momentum Newsletter, 2019
\item Die Diskrepanz zwischen Reden und Handeln ist im ökonomischen Bereich besonders groß: Hirte K. , in Agora42 (Philosophisches Wirtschaftsmagazin), 2018
\item Hilf dir selbst, dann hilft dir Deutschland: Fortschritt bei der Reform der Eurozone: Heimberger P. , in Der Standard, 2018
\item Vier europäische Lehren aus den Turbulenzen in Italien: Heimberger P. , in Makronom, 2018
\item Warum Europa trotz Aufschwung ökonomisch weiter auseinander driftet: Heimberger P., Gräbner C., Kapeller J. , in Makronom, 2018
\item Was Märkte (nicht) mit Demokratie zu tun haben: Hirte K., Poppinga O. , in Wege für eine Bäuerliche Zukunft – Zeitschrift der ÖBV/ Via Campesina Austria, Vol. 41, Nr. 3 (353), Wien, Seite(n) 4-6, 2018
\item Die EU braucht einen wirtschaftspolitischen Kurswechsel: Heimberger P. , in Der Standard, 2017
\item Europa am Scheidepunkt: Österreichs wichtige Rolle: Heimberger P. , in Der Standard, 2017
\item Migration: It's still the economy, stupid!: Pühringer S. , in Der Standard, 2017
\item Zerfällt Europa? Trump als Entscheidungsbeschleuniger: Heimberger P. , in Der Standard, 2017
\item Austeritätspolitik in der Eurozone: Ein Schuss ins eigene Knie: Heimberger P. , in Makronom, 2016
\item Der Blick von oben und der Blick von unten: Ötsch W. , in Makroskop, 2016
\item Investitionen gegen die Dauerkrise im Euroland: Heimberger P. , in Der Standard, 2016
\item Markt-Glauben, Klima-Krise und Katastrophen-Leugnung, Teil 1-4: Ötsch W. , in Makroskop, 2016
\item Trumps Team: Politik von und für Vermögende: Heimberger P. , in Die Presse, 2016
\item Ökonomisches Denken, Rechtspopulismus und Rechtsextremismus: Ötsch W. , in Makroskop, 2016
\item Märkte als Richter: Zur Dominanz neoliberaler Krisennarrative: Pühringer S. , in Ksoe-Dossier (Ksoe Nachrichten der Katholischen Sozialakademie), Nr. 1, Seite(n) 1-3, 2015
\item Das Team Stronach: Die österreichische Tea Party: Ötsch W. , in Die Presse, 2013
\item Globale geo-ökonomische Unordnung: Europa braucht industriepolitische Antworten: Porak L. , 2024
\item Vermögensriesen und ein Heer von Zwergen: Kapeller J. , in Kammer für Arbeiter und Angestellte Oberösterreich, 2024
\item Endogenous Heterogeneous Gender Norms and the Distribution of Paid and Unpaid Work in an Intra-Household Bargaining Model: Hager T., Mellacher P., Rath M. , 2023
\item Im Netz der Einfluss-Reichen: Hager T., Pühringer S. , 2023
\item Man ist akademische Einzelunternehmer*in: Pühringer S. , 2023
\item Woran scheitert transformative Wissensproduktion?: Pühringer S., Altreiter C. , 2023
\item Evaluierung des Zusammenhangs von Produktionspotenzial und Budgetsemielastizität im Rahmen der deutschen Schuldenbremse: Heimberger P., Schütz B. , 2022
\item Klima, Markt und Zukunftsbilder: Ötsch W. , 2022
\item (Mis)Measuring Competitiveness:  The Quantification of a Malleable Concept in the European Semester: Gräbner C., Hager T. , 2021
\item A European wealth tax: Kapeller J., Leitch S., Wildauer R. , 2021
\item Beeld over Italiaanse economie klopt niet: Heimberger P. , 2021
\item Budgetkürzungen durch „Outputlücken-Nonsens“: Heimberger P. , 2021
\item Competition Universalism: Its Historical Origins and Timely Alternatives: Gräbner C., Pühringer S. , 2021
\item Corporate tax cuts do not boost growth: Heimberger P., Gechert S. , 2021
\item Draghi government: Seven ‘surprising’ facts about Italy: Heimberger P. , in Vienna Institute for International Economic (wiiw), 2021
\item EU bonds are a model for the future of Europe: Heimberger P. , in Vienna Institute for International Economic (wiiw), 2021
\item Eine europäische Vermögenssteuer für das Klima: Kapeller J., Wildauer R. , 2021
\item European fiscal rules: reform urgently needed: Heimberger P. , 2021
\item Financial globalisation has increased income inequality: Heimberger P. , 2021
\item Fiscal austerity and the rise of the Nazis: Heimberger P. , 2021
\item Il governo Draghi: sette fatti sorprendenti sull’Italia: Heimberger P., Kowall N. , 2021
\item Keynes, output gap nonsense and the EU’s fiscal rules: Heimberger P. , 2021
\item Keynes, the output gap and the EU’s fiscal rules: Heimberger P. , in Vienna Institute for International Economic (wiiw), 2021
\item So denken Ökonom*innen über Wettbewerb – eine Kritische Analyse des österreichischen Expert*innendiskurses: Porak L., Pühringer S., Rath J. , 2021
\item The push for a global minimum corporate tax rate: Heimberger P. , in Vienna Institute for International Economic (wiiw), 2021
\item Verschwenderisches, reformfaules Italien? Warum gängige Mythen falsch und gefährlich sind: Heimberger P. , in Marie Jahoda – Otto Bauer Institut, 2021
\item Vorwort: Ötsch W., Graupe S. , in Ötsch, Walter O.; Graupe, Silja, Fifty-fifty Verlag, Frankfurt am Main, 2021
\item Walter Lippmann: Die Illusion von Wahrheit oder die Erfindung der Fake News: Ötsch W., Graupe S. , Fivty-fivty Verlag, Edition Buchkomplizen, Frankfurt am Main, 2021
\item Budgetpolitik im Wirtschaftsabschwung: erhebliche Spielräume vorhanden: Heimberger P. , 2020
\item Budgetpolitik in der Corona-Krise: Reform der Budgetregeln erforderlich: Heimberger P. , 2020
\item EU-Wiederaufbaufonds als Kernstück europäischer Krisenbekämpfung: Progressiver Durchbruch oder Enttäuschung?: Heimberger P. , 2020
\item Hyperinflation and the Rise of the Nazis: Heimberger P. , 2020
\item Keeping the promise of eurozone convergence: Heimberger P., Krahé M., Ponattu D., van't Klooster J. , 2020
\item Polarisierung oder Konvergenz? Zur ökonomischen Zukunft des vereinten Europa: Kapeller J. , 2020
\item Seven ’surprising’ facts about the Italian economy: Heimberger P., Kowall N. , 2020
\item Türkis-blaue Arbeitsmarkt- und Sozialpolitik revisited: zwischen Meritokratie und Wohlfahrtschauvinismus: Beyer K., Griesser M., Pühringer S. , 2020
\item Wie die ökonomische Globalisierung die Einkommensungleichheit beeinflusst: Heimberger P. , 2020
\item Arbeitslosigkeit in Europa: Was man tun könnte: Heimberger P. , 2019
\item Dem wirtschaftlichen Abschwung entgegenwirken: Zur wichtigen Rolle der Fiskalpolitik: Heimberger P., Pekanov A. , 2019
\item Der Einfluss des Neoliberalismus auf österreichische Parteiprogramme: Grimm C. , 2019
\item Economic Polarisation in Europe: Causes and Policy Options: Gräbner C., Kapeller J., Heimberger P. , 2019
\item Holding Together what Belongs Together: A Strategy to Counteract Economic Polarisation in Europe: Kapeller J., Gräbner C., Heimberger P. , 2019
\item How much space for fiscal expansion? Germany falls victim to 'output gap nonsense’: Heimberger P. , in Vienna Institute for International Economic (wiiw), 2019
\item Italien vs. EU-Kommission: Warum ein Defizitverfahren kontraproduktiv wäre: Heimberger P. , 2019
\item The current economic downturn in Europe must be seen in the context of a wider problem of economic polarisation: Heimberger P. , in Vienna Institute for International Economic, 2019
\item Unemployment in Europe: What should be done?: Heimberger P. , 2019
\item What to do about divergence between EU countries? The problem of structural polarization: Heimberger P., Kapeller J. , 2019
\item Überwachungskapitalismus: Das Internet als totalitärer Markt: Ötsch W. , 2019
\item ‘Output gap nonsense': Understanding the budget conflict between the EC and Italy’s government: Heimberger P. , 2019
\item Die Krise als Katalysator für den Aufschwung des Rechtspopulismus: Pühringer S. , Vol. 47, 2018
\item Soll der Staat bei Bildung, Gesundheit und Sozialem kürzen? Austeritätspolitik seit der Finanzkrise im Vergleich: Heimberger P. , 2017
\item Vorsicht bei Ländervergleichen – insbesondere bei Staatsausgaben!: Heimberger P. , 2017
\item Weniger Staatsausgaben: Abbau des Sozialstaats und Vertiefung von Wirtschaftskrisen: Heimberger P. , 2017
\item Wie ein makroökonomisches Modell die Spaltung der Eurozone befördert: Heimberger P., Kapeller J. , 2017
\item Österreichs Bildungs-, Gesundheits- und Sozialausgaben im europäischen Vergleich: Wenn der Staat spart, kann das für private Haushalte teuer werden: Heimberger P. , 2017
\item Österreichs Staatsausgabenstrukturen im europäischen Vergleich: Heimberger P. , 2017
\item Das europäische Schattenbankensystem: Bestandsaufnahme und gegenwärtige Entwicklungen: Beyer K., Bräutigam L. , 2016
\item Mehr öffentliche Investitionen sind sinnvoll und erforderlich: Heimberger P. , 2016
\item Verankerung wohlstandorientierter Politik. Working Paper der Kammer für Arbeiter und Angestellte für Wien, Reihe „Materialien zu Wirtschaft und Gesellschaft“, Nr. 165: Griesser M., Brand U. , 2016
\item Ökonomisches Denken in der Krise: Pühringer S. , 2016
\item Nachfrageseitige Ursachen der Expansion des Schattenbankensystems: Beyer K. , 2015
\item Verteilungstendenzen im Kapitalismus: Globale Perspektiven: Kapeller J., Schütz B. , 2015
\item Von freien zu zivilisierten Märkten. Ein New Deal für die europäische Handelspolitik: Kapeller J., Schütz B., Tamesberger D. , 2015
\item Die Finanzkrise 2007-2009 als Krise von Schattenbanken. Eine einführende institutionelle Analyse: Ötsch W., Beyer K., Mader L. , Delft, 2014
\item Die Risiken im Schatten des Systems: Beyer K. , 2014
\item From Free to Civilized Markets: First steps towards Eutopia: Kapeller J., Schütz B., Tamesberger D. , Bremen, 2014
\item Ökonomische Krisen als Krankheiten und Katastrophen?: Pühringer S. , 2014
\item Ahnungslos, aber nicht tatenlos – Wie ÖkonomInnen seit der Finanzkrise Politik mach(t)en: Pühringer S. , 2013
\item Die Umstrukturierung der LPGen in Thüringen ab 1990.: Hirte K. , in Landeszentrale für politische Bildung Thüringen., Druckerei Sömmerda GmbH, Erfurt, 2012
\item Die Evolution des ökonomischen Wissens und des Wissens über den Kapitalismus. Performativity als Analyseinstrument: das Beispiel der Fabian Society, der Mont Pèlerin Society und der Chicagoer Schule: Ötsch W., Hirte K., Nordmann J. , 2010
\end{itemize}
