\subsection*{2024}
\paragraph{Vortrag auf einer Tagung (referiert)}
\begin{enumerate}
	\item Rath J.: Unravelling the social-epistemological costs of short-term contracts in Austrian research funding schemes. EAEPE 2024, Österreich, 21.11.2024
	\item Pühringer S.: Promoters of academic competition? The role of academic social networks and platforms in the competitization of science. Universität Kassel, Kassel, Deutschland, 04.04.2024
	\item Porak L.: Openness only thrives where fairness survives: The geo-economic turn in EU competitiveness discourse. EAEPE, Spanien, 05.09.2024
	\item Porak L.: Openness only thrives where fairness survives: The geo-economic turn in EU competitiveness discourse. , HPMA Conference, Österreich, 10.10.2024
	\item Maad M.: Wealth Inequality and Media Narratives: Exploring Redistribution Policy Debates on Taxation in German-Speaking Countries. Momentum Kongress: Alternativen, Österreich, 18.10.2024
	\item Hager T., Pühringer S.: Gendered Competitive Practices in Economics. A Multi-Layer Model of Women’s Underrepresentation, Österreich, 26.09.2024
	\item Hager T., Mellacher P., Rath M.: Gender Norms and Network Structure: A Model of the Intra-Household Division of Labor. 36th Annual EAEPE Conference, Spanien, 04.09.2024
	\item Forde S., Theine H.: Make big tech pay their fair share? A comparative analysis of news-based policies and the underlying economic imaginaries that support them. International Conference on Historical-Materialist Policy Analysis, Österreich, 03.10.2024
	\item Eder J.: Identifying and overcoming the challenges of the Austrian mobility transition. Konferenz der European Association for Evolutionary Political Economy, Spanien, 04.09.2024
	\item Cserjan L., Eder J., Hornykewycz A., Porak L., Pühringer S.: Identifying and overcoming challenges for the Austrian mobility transition. European Association for Evolutionary Political Economy, Bilbao, Spanien, 04.09.2024
	\item Cserjan L., Eder J., Hornykewycz A., Porak L., Pühringer S.: Economic Preconditions and Impacts of the Mobility Transition in Austria's Railway Sector. European Association for Evolutionary Political Economy, Bilbao, Spanien, 06.09.2024
	\item Cserjan L.: Mobility Transition in Austria’s Railway Sector:Economic Preconditions and Impact. Forum for Macroeconomics and Macroeconomic Policies, Berlin, Deutschland, 24.10.2024
	\item Azevedo S., Hager T., Porak L.: Unraveling Explicit and Implicit Rules of Competition: An Interdisciplinary Framework for Analysis. 36th Annual SASE Conference, Irland, 29.06.2024
	\item Azevedo S., Hager T., Porak L.: Kritische Utopien als Methode und Praxis: Der Forschungsstandort Österreich, Österreich, 18.10.2024
	\item Aigner E., Aistleitner M., Pühringer S.: The social and epistemic structure of research on Socio-ecological Transformation. Challenging the economic mainstream?. Society for the Advancement of Socio-Economics, Limerick, Irland, 28.06.2024
	\item Aigner E., Aistleitner M., Pühringer S.: The social and epistemic structure of research on Socio-ecological Transformation. Challenging the economic mainstream?. European Association for Evolutionary Political Economy, Bilbao, Spanien, 04.09.2024
\end{enumerate}
\paragraph{Eingeladener Vortrag an Universität}
\begin{enumerate}
	\item Rath J.: Institutional Evolutionary Economics: Introduction. Plurale Ökonomik InnsbruckLV VU Exploring Economics (Universität Innsbruck), Universität Innsbruck, Österreich, 03.12.2024
	\item Pühringer S.: Welche Universität braucht gute Wissenschaft?. Vernetzungstreffen ULV und NUWiss an der TU Graz, Österreich, 26.02.2024
	\item Pühringer S.: Netzwerke von Überreichtum in Österreich. Wirtschaftspolitische Akademie, WU Wien, WU Wien, Österreich, 11.04.2024
	\item Pühringer S.: Sozial-ökologische Transformation braucht eine andere Wirtschaft. FH Oberösterreich, Campus Wels, Österreich, 12.11.2024
	\item Porak L.: The European green transition and the role of industrial policy. Selbstorganisierte Ringvorlesung WU, Österreich, 21.05.2024
	\item Porak L.: Woher kommt Schokolade und wer macht eigentlich mein Handy?. JKU, Linz, ScienceHolidays Workshop, Österreich, 22.07.2024
	\item Kapeller J.: Pluralism in Economics. Universität Zürich, Zürich, Schweiz, 01.02.2024
	\item Kapeller J.: Orthodox vs. Heterodox understandings of wealth dynamics. SummerSchool „Wealth Inequality Research in  Africa“, Makerere University, Kampala, Uganda, 16.09.2024
	\item Kapeller J.: Pluralism in economics and the split between heterodoxy and orthodoxy. SummerSchool „Wealth Inequality Research in  Africa“, Makerere University, Kampala, Uganda, 17.09.2024
	\item Kapeller J.: Pluralism in Economics. , Universität Köln, Deutschland, 01.10.2024
	\item Gräbner-Radkowitsch C.: Pluralismus und Komplexität: Ein Gegenentwurf zur neoklassischen Gleichgewichtsökonomik?. Universität Hamburg, Deutschland, 19.01.2024
	\item Gräbner-Radkowitsch C.: Agentenbasiertes Modellieren und Komplexitätsökonomik. Universität Siegen, Deutschland, 24.01.2024
	\item Altreiter C., Pühringer S.: How capitalist academia needs to change to assist structured socio-ecological transformation. Gender, Work and Organization Workshops Resisting Business-as-Usual: Organizing for Hope and Justice, Deutschland, 23.02.2024
\end{enumerate}
\paragraph{Hauptvortrag / Eingeladener Vortrag auf einer Tagung}
\begin{enumerate}
	\item Pühringer S.: Wie viel Wettbewerb wollen wir (uns leisten)? Die „Kosten“ kompetitiver Drittmittelvergabe an österreichischen Unis. Universitätspersonalrechtliche Gespräche, Linz, Österreich, 29.05.2024
	\item Kapeller J.: The climate crisis from a pluralist perspective. pluralumn Conference, Keynote, Universität Duisburg-Essen, Deutschland, 06.08.2024
	\item Hager T.: Podiumsdiskussion zum Thema \glqq Mehr als Geld: Was braucht gute Wissenschaft?\grqq{}. Unterbau Linz, Österreich, 14.03.2024
	\item Eder J.: The green transformation in Europe: who pays the price?. Workshop der Young Scholar Initiative im Rahmen der Young Economists Conference der Arbeiterkammer Wien, Österreich, 25.09.2024
	\item Cserjan L., Hager T., Porak L., Pühringer S.: Die Macht des ökonomischen Denkens: Narrative, Akteur*innen und Netzwerke. Beyond Growth Conference Austria, Österreich, 15.05.2024
\end{enumerate}
\paragraph{Eingeladener Vortrag an anderen Institutionen}
\begin{enumerate}
	\item Pühringer S.: Superreichen-Netzwerke in Österreich. AK Salzburg, Österreich, 28.08.2024
	\item Porak L.: Die grüne Transformation des EU-Verkehrssektors. AK OberösterreichAK EU-Länderreferent*innentagung, AK Oberösterreich, Österreich, 20.06.2024
	\item Porak L.: Degrowth als neue Wirtschaftsordnung?. FH OÖ, Campus Wels, Österreich, 21.11.2024
	\item Kapeller J.: Ökonomische Polarisierung in Europa. , Zentrum für Schulqualität und Lehrerbildung, Deutschland, 11.06.2024
	\item Kapeller J.: Woher kommt das Geld?. , Beitrag im Rahmen der Abschlussveranstaltung zur Ausstellung „Wofür das Geld?“, Hall in Tirol, Österreich, 15.10.2024
	\item Kapeller J.: Ökonomische Polarisierung in Europa. , Zentrum für Schulqualität und Lehrerbildung, Deutschland, 06.11.2024
	\item Hirte K.: Konversionsstrategien in der Landwirtschaft: Agrarische sozial-ökologische Transformation und die Landwirte. Universität Göttingen, Deutschland, 26.02.2024
	\item Eder J.: Vortrag bei Lernwerkstatt der Gewerkschaftsschule Oberösterreich zur Einführung in wissenschaftliches Schreiben \glqq Von der Idee zum Schreiben\grqq{}, Österreich, 05.10.2024
	\item Cserjan L., Rath J.: Herausforderungen für eine Wirtschaft im 21. Jahrhundert. JKU LinzSchüler*innen Workshop an der JKU, Österreich, 06.02.2024
	\item Cserjan L., Eder J., Hornykewycz A., Porak L., Pühringer S.: Studienpräsentation \glqq Mobilitätswende produzieren\grqq{}.. AK Wien, Wien, AK Wien, Österreich, 27.05.2024
\end{enumerate}
\paragraph{Andere Vorträge und Präsentationen}
\begin{enumerate}
	\item Hornykewycz A., Sommer M.: Gestalte das perfekte Wirtschaftssystem, Österreich, 29.11.2024
	\item Hornykewycz A.: JKU Science Holidays: Gestalte das perfekte Wirtschaftssystem, Österreich, 16.07.2024
	\item Hornykewycz A.: JKU Science Holidays: Gestalte das perfekte Wirtschaftssystem, Österreich, 23.07.2024
	\item Hornykewycz A.: JKU Science Holidays: Gestalte das perfekte Wirtschaftssystem, Österreich, 24.07.2024
	\item Eder J.: Input bei Research Seminar des ICAE: Identifying and overcoming the challenges of the Austrian mobility, Österreich, 05.11.2024
	\item Eder J.: Input bei Konferenz \glqq Monster verstehen" in Schiene III zu "Wer verfolgt welche Transformationsstrategien in Deutschland und Europa? Und was wäre eine progressive Strategie für den wirtschaftlichen Umbau?\grqq{}, Deutschland, 16.11.2024
	\item Cserjan L., Porak L.: Wirtschaftliche Herausforderungen im 21. Jahrhundert. JKU Linz, Schüler*innenworkshop, Österreich, 18.06.2024
\end{enumerate}
\subsection*{2023}
\paragraph{Andere Vorträge und Präsentationen}
\begin{enumerate}
	\item Rath J.: Kapitalismus und Demokratie im Wandel: Marx, Polanyi und Schumpeter im Dialog. Entwicklungspolitische Hochschulwochen Linz 2023Entwicklungspolitische Hochschulwochen Linz 2023, Österreich, 20.11.2023
	\item Pühringer S.: Exzellent prekär. NUWiss Gründungsevent, Universität Wien, Teilnahme Podiumsdiskussion, Österreich, 22.04.2023
	\item Pühringer S.: Unternehmenslobbys drängen immer stärker in die Schulen. Expert*innen diskutierten über eine zukunftsfähige Wirtschaftsbildung. Attac Österreich, Podiumsdiskussion, Österreich, 24.11.2023
	\item Kapeller J.: Kurt Rothschild Preisverleihung 2022. Interview mit Jakob Kapeller als Juryvorsitzender, Renner Institut, Österreich, 23.11.2023
	\item Hager T.: NUWiss vor Ort. Die lokalen Unterbauinitiativen: Unterbau Linz. NUWiss Gründungsevent, Universität Wien, Österreich, 22.04.2023
	\item Aistleitner M.: Beschäftigungspolitische Implikationen einer sozial-ökologischen Transformation am Beispiel der Fahrzeuglieferindustrie in OÖ. Arbeitnehmer*innen-Kurie des AMS, AK OÖ Linz, Österreich, 20.11.2023
\end{enumerate}
\paragraph{Vortrag auf einer Tagung (referiert)}
\begin{enumerate}
	\item Rath J.: Ranking the best of Research? Values and Valuation in Science. Workshops Value in Economics, Universität LissabonWorkshop „Value and Valuation“, Portugal, 23.06.2023
	\item Rath J.: Employment Relationships in the Age of Digitalisation: Analysing Policy Measures in an Agent-Based Framework. EAEPE 2023, Vereinigtes Königreich, 14.09.2023
	\item Porak L., Reinke R.: ‘The charm of emission trading’: Ideas of German economists on economic policy in times of crises. EAEPE Annual Conference, Universität LeedsEAEPE Annual Conference, Vereinigtes Königreich, 12.09.2023
	\item Porak L., Reinke R.: ‘The charm of emission trading’: Ideas of German economists on economic policy in times of crises. Annual Conference of the Political Economy Section of the German Association for Political SciencesAnnual Conference of the Political Economy Section of the German Association for Political Sciences, Vereinigtes Königreich, 25.09.2023
	\item Kapeller J.: Heating up houses instead of the climate: A transparent and realistic pathway for transforming the German housing sector. EAEPE Annual Conference, University of Leeds, Vereinigtes Königreich, 13.09.2023
	\item Hornykewycz A., Kapeller J.: Heating up Houses instead of the Climate A transparent and realistic pathway for transforming the German housing sector. Young Economists Conference 2023, AK, LInz, Österreich, 06.10.2023
	\item Hager T., Pühringer S.: Gendered Competitive Practices in Economics A Multi-Layer Model of Women’s Underrepresentation. 8. Österreichischer Workshop Feministischer Ökonom*innen (FemÖk), Österreich, 05.05.2023
	\item Hager T., Pühringer S.: Gendered Competitive Practices in Economics. A Multi-Layer Model of Women’s Underrepresentation. Value and Valuation. Challenges in Political Economy Analysis., Portugal, 23.06.2023
	\item Hager T., Pühringer S.: Gendered Competitive Practices in Economics. A Multi-Layer Model of Women’s Underrepresentation. 35th Annual SASE Conference, Rio de Janeiro, online, Virtuell, 11.07.2023
	\item Gräbner-Radkowitsch C., Hager T., Hornykewycz A.: Competing for sustainability? An institutionalist analysis of the new development model of the European Union. ICAPE 2023 Conference, Loyola University, New Orleans, Vereinigte Staaten, 05.01.2023
	\item Gräbner-Radkowitsch C., Hager T., Hornykewycz A.: Competing for sustainability? An institutionalist analysis of the new development model of the European Union, Vereinigtes Königreich, 14.09.2023
	\item Gräbner-Radkowitsch C.: Degrowth and the Global South? How institutionalism can complement a timely discourse on ecologically sustainable development in an unequal world. ASSA, New Orleans, Vereinigte Staaten, 11.01.2023
	\item Aistleitner M., Pühringer S.: What money can buy. Networks of the superriches in Austria. EAEPE Annual Conference, University of Leeds, Vereinigtes Königreich, 13.09.2023
\end{enumerate}
\paragraph{Eingeladener Vortrag an anderen Institutionen}
\begin{enumerate}
	\item Rath J.: Herausforderungen für eine Wirtschaft im 21. Jahrhundert. JKU LinzSchüler*innen Workshop an der JKU, Österreich, 30.03.2023
	\item Pühringer S.: What economic education is missing: the real world. Workshops des DFG-Netzwerks Economic Criticism, PH Linz, Österreich, 07.03.2023
	\item Pühringer S.: Wissenschaft und Wettbewerb. Netzwerk Unterbau Wissenschaft (NUWiss), Österreich, 16.03.2023
	\item Pühringer S.: Wettbewerb in der Wissenschaft. Mittelbau-Vollversammlung, JKU Linz, Linz, Österreich, 16.03.2023
	\item Pühringer S.: Sustainable Socio-Economic Transition and Economic Reasoning. Hearing zum FWF-START Preis 2023, Wien, Österreich, 16.06.2023
	\item Pühringer S.: Konsequenzen der Wettbewerbsorientierung von und an Universitäten. GÖD-Dialogforum Arbeitsplatz Universtität, Wien, Österreich, 14.11.2023
	\item Pühringer S.: Welche Wirtschaft(swissenschaft) braucht Sozialökologische Transformation?. Kulturhaus Amstetten, Österreich, 14.12.2023
	\item Pühringer S.: Sozialökologische Transformation braucht andere Wirtschaft. Kulturhof Amstetten, Österreich, 14.12.2023
	\item Porak L.: Implicit and explicit rules of competition. Österreichische Akademie der Wissenschaft, Preisverleihung ÖAW Stipendium, Österreich, 08.11.2023
	\item Maesse J., Pühringer S., Rossier T.: Power and Influence of Economists: Contributions to the Social Studies of Economics. Economic Research Council London, online, Virtuell, 09.02.2023
	\item Kapeller J.: Wirtschaft anders denken! Was ist plurale Ökonomik?. ZSL Baden-Württemberg, Deutschland, 05.02.2023
	\item Kapeller J.: Rethinking the economy – The contribution of pluralist economics. ZSL Baden-Württemberg, Deutschland, 06.02.2023
	\item Kapeller J.: Wirtschaft anders denken? – Der Beitrag einer pluralen Ökonomie. Kulturhof Amstetten, Österreich, 04.05.2023
	\item Kapeller J.: Can a European wealth tax close the green investment gap?. Ars Electronica Festival Linz, Linz, Österreich, 08.09.2023
	\item Hornykewycz A., Rath J.: Herausforderungen für eine Wirtschaft im 21. Jahrhundert. JKU LinzSchüler*innen Workshop an der JKU, Österreich, 20.06.2023
	\item Hornykewycz A., Rath J.: Herausforderungen für eine Wirtschaft im 21. Jahrhundert. JKU LinzSchüler*innen Workshop an der JKU, Österreich, 19.12.2023
	\item Hager T., Hornykewycz A.: Herausforderungen für eine Wirtschaft im 21. Jahrhundert. JKU LinzSchüler*innen Workshop an der JKU, Österreich, 17.04.2023
	\item Aistleitner M., Pühringer S.: Netzwerke der Superreichen in Österreich. AK Österreich, Wien, Österreich, 01.08.2023
\end{enumerate}
\paragraph{Eingeladener Vortrag an Universität}
\begin{enumerate}
	\item Pühringer S.: Konsequenzen der Wettbewerbsorientierung von und an Universitäten. Eine Perspektive der interdisziplinären Wettbewerbsforschung. Universtität Wien, Österreich, 12.01.2023
	\item Pühringer S.: Konsequenzen der Wettbewerbsorientierung von und an Universitäten. Universität Wien, Präsidiumssitzung des Universitätslehrerverbandes ULV, Österreich, 23.09.2023
	\item Pühringer S.: Wissensproduktion und Publikationskulturen in der “Wettbewerbswissenschaft”. Universität Jena, Thüringer Universitäts- und LandesbibliothekKonferenz Wissenschaftliche Publikationskulturen im Zeitalter von Open Access, Deutschland, 30.11.2023
	\item Kapeller J.: Pluralismus in der Ökonomik. Universität Zürich, Schweiz, 09.09.2023
	\item Kapeller J.: Pluralismus in der Ökonomik. Universität Köln, Deutschland, 13.12.2023
	\item Hirte K.: Elitetheorien und Demokratieverständnis. Auffassungen und Vorgänge vor und nach 1945. Universität Passau, Bereich Soziologie mit Schwerpunkt Techniksoziologie und nachhaltige Entwicklung, Passau, Deutschland, 18.04.2023
	\item Hirte K.: Zur aktuellen agrarischen Transformationsforschung – Profile, Zugänge, Kernauffassungen. Universität Passau, Bereich Soziologie mit Schwerpunkt Techniksoziologie und nachhaltige Entwicklung, Passau, Deutschland, 21.06.2023
	\item Hirte K.: Agrarische sozio-ökonomische Transformation und Transformationswissen. Universität Passau, Bereich Soziologie mit Schwerpunkt Techniksoziologie und nachhaltige Entwicklung, Passau, Deutschland, 04.07.2023
	\item Hirte K.: Bodendaten und Bodenpolitik im Agrarbereich: Bodendaten (Reichsbodenschätzung), Bodennutzungsdaten (INVEKOS; BMEL und Testbetriebsnetz) sowie Bodenbelastung (BMVU). Universität Passau, Deutschland, 13.12.2023
	\item Gräbner-Radkowitsch C.: Agentenbasierte Modellierung und Komplexitätsökonomik. Universität Siegen, Deutschland, 23.01.2023
	\item Gräbner-Radkowitsch C.: Agentenbasierte Modelle \& Komplexitätsökonomik. Universität Siegen, Deutschland, 26.01.2023
	\item Aistleitner M.: A world full of nails: Über die eigenwillige (Nicht-)Thematisierung Sozial-Ökologischer Transformation in den Wirtschaftswissenschaften. Universität Linz, Entwicklungspolitische Hochschulwochen Linz 2023, Deutschland, 21.11.2023
\end{enumerate}
\paragraph{Hauptvortrag / Eingeladener Vortrag auf einer Tagung}
\begin{enumerate}
	\item Kapeller J.: Why are there so many power laws in economics?. Young Economists Conference 2023, AK Wien, OÖAK Young Economists Conference 2024, Österreich, 06.10.2023
\end{enumerate}
\paragraph{Vortrag auf einer Tagung (nicht referiert)}
\begin{enumerate}
	\item Gräbner-Radkowitsch C., Hager T., Hornykewycz A.: Competing for Sustainability? An institutionalist analysis of the new development model of the EU, Deutschland, 27.06.2023
\end{enumerate}
\subsection*{2022}
\paragraph{Vortrag auf einer Tagung (referiert)}
\begin{enumerate}
	\item Rath J.: EU Policy-making and the construction of economic indicators: the case of Competitiveness in the European Semester. 34th Annual EAEPE Conference, Neapel, Neapel, Italien, 08.09.2022
	\item Rath J.: EU Policy-making and the construction of economic indicators: the case of Competitiveness in the European Semester. 34th Annual EAEPE Conference 2022, Neapel, Italien, 08.09.2022
	\item Rath J.: The political economy of academic publishing: On the commodification of a public good. Young Economist Conference 2022, AK Wien, Österreich, 07.10.2022
	\item Pühringer S.: Winning urban competition with a social agenda. The competition imaginary in Viennese urban development plans. 34th Annual SASE Conference, Amsterdam, Niederlande, 10.07.2022
	\item Pühringer S.: Competitive Performativity of Academic Social Networks. The Subjectification of Competition on Researchgate, Twitter and Google Scholar. 34th Annual EAEPE Conference, Neapel, Italien, 07.09.2022
	\item Pühringer S.: Wettbewerb in akademischen, politischen und öffentlichen Diskursen. SPACE-Abschlusskonferenz,  Universität Wien, Österreich, 28.09.2022
	\item Kapeller J., Wildauer R.: Can the European Green Deal Contribute to a Socio-Ecological Transformation?. 34th Annual SASE Conference, Amsterdam, Niederlande, 10.07.2022
	\item Hornykewycz A., Rath J.: Shaping Employment Relationships in the Age of Digitalisation: Analysing Policy Measures in an Agent-Based Framework. 34th Annual SASE Conference, Amsterdam, Amsterdam, Niederlande, 11.07.2022
	\item Hornykewycz A.: Modelling the Demand for Heterogeneous Consumption Products. A Macroeconomic Stock-Flow Consistent Agent-Based Approach. 34th Annual EAEPE Conference, Neapel, Neapel, Italien, 08.09.2022
	\item Gräbner-Radkowitsch C., Heimberger P., Kapeller J.: Technological capabilities, globalisation and economic growth. 34th Annual SASE Conference, Amsterdam, Niederlande, 11.07.2022
	\item Gräbner-Radkowitsch C., Hager T.: The Toll it Takes: On Technological Competitiveness and Joining the European Union. 34th Annual SASE Conference, Amsterdam, Niederlande, 11.07.2022
	\item Gräbner-Radkowitsch C., Hager T.: Curse or Blessing: On Technological Competitiveness and Joining the European Union. 34th Annual EAEPE Conference, Neapel, Neapel, Italien, 09.09.2022
	\item Gräbner-Radkowitsch C., Hager T.: Bane or Boon: On Technological Competitiveness and Joining the European Union. Young Economists Conference 2022, Wien, Österreich, 09.10.2022
	\item Gräbner-Radkowitsch C.: The heterogeneous effects of digitalization on macroeconomic developments -- a comparative perspective. 34th Annual EAEPE Conference, Neapel, Italien, 06.09.2022
	\item Aistleitner M., Griesebner T., Pühringer S.: Networks of the super-rich in Austria. 34th Annual EAEPE Conference, Neaples, Italien, 08.09.2022
	\item Aistleitner M.: Development and Interdisciplinarity: re-examining the \glqq economics silo\grqq{}. 34th Annual SASE Conference, Amsterdam, Amsterdam, Niederlande, 11.07.2022
\end{enumerate}
\paragraph{Eingeladener Vortrag an anderen Institutionen}
\begin{enumerate}
	\item Pühringer S., Rath J.: The political economy of academic publishing: On the commodification of a public good. Jahreskonferenz der Kritischen Kommunikationswissenschaften (KRIKOWI), WU Wien, Österreich, 13.05.2022
	\item Pühringer S.: Universitäten und Forschung im Wettbewerb. Präsidiumssitzung des ULV, online, Österreich, 29.04.2022
	\item Pühringer S.: Konsequenzen der Wettbewerbsorientierung von und an Universitäten. Jahresversammlung des ULV, Universität Wien, Österreich, 25.06.2022
	\item Pühringer S.: Reichennetzwerke in Österreich. AK Wien, Österreich, 14.07.2022
	\item Pühringer S.: Konsequenzen der Wettbewerbsorientierung von und an Universitäten. 2. Vernetzungsworkshops der School for Social Science and Humanities, JKU Linz, Österreich, 16.11.2022
	\item Kapeller J.: Pluralismus in der Ökonomie (und die Klimafrage). Heinrich-Böll-Stiftung, Deutschland, 02.11.2022
	\item Hornykewycz A., Rath J.: Arbeitsbeziehungen im Wandel. Der Effekt von Digitalisierung auf die Ausgestaltung von Arbeitsverträgen und Kooperation. AK Vorarlberg, SchaffareiTechnikfolgenabschätzung aus Arbeitnehmer:innenperspektive, Österreich, 25.11.2022
	\item Hager T., Pühringer S.: Verzerrte Expertise. Zur (Frei)handelsdebatte in den Wirtschaftswissenschaften. Entwicklungstagung 2022: Ungleiche Möglichkeiten, Österreich, 12.11.2022
	\item Hager T., Hornykewycz A., Pühringer S.: Neoliberale Think Tanks und ihre Verbindungen zu Reichen-Netzwerken. 4. Österreichische Reichtumskonferenz, Österreich, 17.10.2022
	\item Hager T., Hornykewycz A.: Herausforderungen für eine Wirtschaft im 21. Jahrhundert. JKU LinzSchüler*innen Workshop an der JKU, Österreich, 29.06.2022
	\item Hager T., Heck I., Rath J.: Competition in Transitional Processes: Polanyi \& Schumpeter. Pluralumni Netzwerk, online, Österreich, 10.01.2022
\end{enumerate}
\paragraph{Andere Vorträge und Präsentationen}
\begin{enumerate}
	\item Pühringer S.: Wettbewerb in der Wissenschaft: Ambivalenzen und (un-)in-tendierte Folgen. Teilnahme Podiumsdiskussion, SPACE-Abschlusskonferenz, Universität Wien, Österreich, 28.09.2022
	\item Pühringer S.: Konsequenzen der Wettbewerbsorientierung von und an Universitäten. Mittelbau-Vollversammlung, Universität Wien, Österreich, 24.11.2022
	\item Kapeller J., Wildauer R.: EU-Vermögenssteuer für ein grünes und gerechtes Europa. Diskurs.Das WissenschaftlernetzDiskurs.Das Wissenschaftlernetz, Österreich, 02.06.2022
	\item Kapeller J.: Wechselseitige Kritik ist nur möglich mit nachvollziehbarer Forschung. ZBW Leibniz Informationszentrum Wirtschaft, Interview, Deutschland, 29.12.2022
	\item Hornykewycz A., Rath J.: The Translation of Social Relations into Computer-Based Models: New Possibilities and Familiar Limitations. Thomas Gegenhuber und Barbara StiglbauerGastvortrag MA \glqq digital society", LV "Interdisziplinäres Projekt: Digitalisierung interdisziplinär\grqq{} (LVA-Nr. 232477 2022W), Österreich, 14.11.2022
	\item Aistleitner M., Porak L., Pühringer S., Rath J.: Wettbewerb und Wettbewerbsfähigkeit im öffentlichen, politischen und akademischen Diskurs. Präsentation Forschungsergebnisse, SPACE-Abschlusskonferenz, Universität Wien, Österreich, 28.09.2022
\end{enumerate}
\paragraph{Vortrag auf einer Tagung (nicht referiert)}
\begin{enumerate}
	\item Pühringer S.: Competitive Performativity of Academic Social Networks. The Subjectification of Competition on Researchgate, Twitter and Google Scholar. Workshop „Rankings and the structure of Economics”, JKU Linz, Österreich, 21.07.2022
	\item Gräbner-Radkowitsch C., Hager T.: The Political Economy of Measuring Competitiveness. SPACE-Abschlusskonferenz, Universität WienMission Competition. Transdisziplinäre Perspektiven auf Wettbewerbsgesellschaften., Österreich, 28.09.2022
\end{enumerate}
\paragraph{Eingeladener Vortrag an Universität}
\begin{enumerate}
	\item Pühringer S.: Machtstrukturen in der Ökonomie und der performative Einfluss auf Politik und Gesellschaft. Universtität Kiel, online, Deutschland, 01.03.2022
	\item Kapeller J.: Pluralism in economics (\& the role of heterodoxy). University of Durham, Vereinigtes Königreich, 07.02.2022
	\item Kapeller J.: Konzernmacht in globalen Güterketten. Johannes Kepler Universität Linz, Österreich, 03.05.2022
	\item Kapeller J.: Pluralismus in der Ökonomie: Philosophische Perspektiven. Research seminar of the Departmant of Philosophy at University of Duisburg-Essen, Deutschland, 09.06.2022
	\item Kapeller J.: Pluralismus in der Ökonomie: Philosophische Perspektiven. Universität Bochum, Deutschland, 25.10.2022
	\item Hirte K.: Agrarpolitik und Agrarökonomie. Universität Passau, Bereich Soziologie mit Schwerpunkt Techniksoziologie und nachhaltige Entwicklung, Österreich, 14.06.2022
	\item Hirte K.: Performativity: Vom Entstehen von agrarischen Verarbeitungsstrukturen am Beispiel Schlachthöfe. Universität Passau, Österreich, 15.06.2022
	\item Hirte K.: Verschweigen als gewolltes Nichtwissen in der Wissenschaft: Strategien in Umbruchzeiten. Universität Passau, Deutschland, 26.10.2022
	\item Hirte K.: Die deutsche universitäre Agrarsoziologie als Wissenschaftsdisziplin. Universität Passau, Bereich Soziologie mit Schwerpunkt Techniksoziologie und nachhaltige Entwicklung, Deutschland, 06.12.2022
	\item Gräbner-Radkowitsch C.: Liegt die Wahrheit irgendwo dazwischen? Eine plurale Perspektive auf globale Ungleichheit. Universität Tübingen, Deutschland, 02.05.2022
	\item Gräbner-Radkowitsch C.: Was ist plurale Ökonomik?. Universität Kiel, Deutschland, 19.10.2022
	\item Gräbner-Radkowitsch C.: Kompexitätsökonomik. FU Berlin, Deutschland, 03.11.2022
	\item Gräbner-Radkowitsch C.: Komplexitätsökonomik. Zeppelin Universität Friedrichshafen, Deutschland, 17.11.2022
\end{enumerate}
\paragraph{Hauptvortrag / Eingeladener Vortrag auf einer Tagung}
\begin{enumerate}
	\item Pühringer S.: Konsequenzen der Wettbewerbsorientierung von und an Universitäten: Eine Perspektive der interdisziplinären Wettbewerbsforschung. Tagung des Zentralausschusses Universitätslehrerinnen und Universitätslehrer beim Bundesministerium für Bildung, Wissenschaft und Forschung, Österreich, 28.01.2022
	\item Kapeller J.: Dilemmata marktliberaler Globalisierung. Evangelische Akademie TutzingGlobalisierungs- und Wachstumsgrenzen, Tutzing, Deutschland, 09.03.2022
	\item Kapeller J.: Das Menschenbild moderner Ökonomie. Universität Wien, Österreich, 25.05.2022
	\item Kapeller J.: Wirtschaft anders denken! Was ist plurale Ökonomik?. ZSL Baden-Württemberg, Österreich, 30.09.2022
	\item Kapeller J.: Socio-ecological transition for the sustainable development of well-being. 6th Trade Union related Economists (TUREC) meeting, Athen, Griechenland, 13.10.2022
	\item Gräbner-Radkowitsch C.: Applied economic methodology: oxymoron or ideal?. Second Philosophy \& Economics Conference, Vienna, Österreich, 29.09.2022
	\item Aistleitner M., Griesebner T., Pühringer S.: Networks of the super-rich in Austria. Young Economists Conference 2022, Wien, Österreich, 10.10.2022
\end{enumerate}
\subsection*{2021}
\paragraph{Eingeladener Vortrag an anderen Institutionen}
\begin{enumerate}
	\item Schütz B.: Nach „Koste es was es wolle“: Eine neue Ära in der Ökonomie?. Club Research, in Kooperation mit der AK-Wien und der Stadt Wien, Teilnahme Podiumsdiskussion, Österreich, 27.01.2021
	\item Pühringer S.: Neoliberale Denkmuster als Basis des degressiven Arbeitslosengeldes. Aktionskonferenz Arbeitslosengeld rauf!, Wien, Katamaran, Österreich, 18.09.2021
	\item Kapeller J.: Past and future of pluralism in economics. Economic Commission for Latin America and the Caribbean (ECLAC), online (geplant in CHL), Österreich, 29.01.2021
	\item Kapeller J.: Heterodox Economics and Economic Criticism. , DFG Research Network on Economic Criticism, Deutschland, 08.07.2021
	\item Kapeller J.: A European Wealth Tax. , Brüssel, S\&D FISC Working Group on Wealth Tax, Belgien, 07.09.2021
	\item Kapeller J.: A European Wealth Tax? Context, methods, revenues. Conference “Fiscal Matters”, European Parliament, Brüssel, EU-Parlament, Belgien, 28.09.2021
	\item Heimberger P.: Outputlücken-Schätzungen und Fiskalpolitik im Kontext der Covid19-Krise. Bundesministerium für Wirtschaft und Energie Deutschland, Berlin, Deutschland, 18.02.2021
	\item Heimberger P.: Corona-Krise und EU-Wiederaufbauplan: Ein ‚fiskalpolitischer Durchbruch‘ für Europa?. SPD Bayern, München, Deutschland, 20.02.2021
	\item Heimberger P.: Fiskalpolitik und europäischer Integrationsprozess. JEF Münster NRW, Online (geplant in DEU), Österreich, 11.05.2021
	\item Heimberger P.: Italy -- Europe’s basket case?. Forum New Economy, Berlin, Deutschland, 15.06.2021
	\item Heimberger P.: Italy and Europe will rise or fall together Participation in a panel discussion. International Weekly Seminar, Le Grand Continent, Österreich, 24.06.2021
	\item Heimberger P.: Why the fiscal rules should be reformed: Output gap estimates and fiscal policy in Italy. Event on the EU’s fiscal rules followed by a panel discussion with representatives of different parties from the Draghi government, Rome, Italien, 12.07.2021
	\item Heimberger P.: How should we deal with the budgetary costs due to the Covid-19 crisis?. Austrian trade union federation, Invited participation in a panel discussion, Österreich, 25.08.2021
	\item Heimberger P.: Italy myths and the future of Europe. Austrian embassy in Rome, Rom, Italien, 23.09.2021
	\item Heimberger P.: How to reform the EU’s fiscal rules?. New Economics Foundation and the European Trade Union Confederation“Fiscal 2.0: A roadmap for the future?”, Teilnahme Panel-Diskussion, Österreich, 30.09.2021
	\item Heimberger P.: Budgetpolitik im Kontext der Covid-19-Krise. FSG-Tagung der Gewerkschaft GPA, Wien, Österreich, 12.10.2021
	\item Heimberger P.: Budgetpolitik im Kontext der Covid-19-Krise. Chamber of Labour Upper Austria, Linz, Österreich, 12.10.2021
	\item Heimberger P.: Fiscal policy and the Covid-19 crisis. Belgium Central Economic Council, Belgien, 24.11.2021
	\item Heimberger P.: The political importance of technical details in the European Semester. European Trade Union Institut, Belgien, 29.11.2021
	\item Gräbner-Radkowitsch C.: Country capabilities, product complexity and finance in the EU. Rebuilding Macroeconomics Globalization Hub Conference, Vereinigtes Königreich, 05.01.2021
\end{enumerate}
\paragraph{Vortrag auf einer Tagung (referiert)}
\begin{enumerate}
	\item Rath J.: The Political Economy of Academic Publishing: On the Commodification of a Public Good. EAEPE Conference 2021, Österreich, 03.09.2021
	\item Pühringer S., Rath J.: The political economy of academic publishing: On the commodification of a public good. SASEAnnual Conference of the Society for the Advancement of Socio-Economics (SASE), virtuell (geplant: DEU), Österreich, 03.07.2021
	\item Porak L.: Governing the Ungovernable. Recontextualizations of European Policy Discourse. The International Max Planck Research SchoolFirst Doctoral Conference on the Social and Political Constitution of the Economy, Köln, Deutschland, 26.03.2021
	\item Porak L.: Governing the Ungovernable. Recontextualizations of European Policy Discourse. Economic Policy Conference, Online, Österreich, 27.05.2021
	\item Porak L.: Governing the Ungovernable. Recontextualizations of Competition in European Policy Discourse. 33rd Annual SASE Meeting, Online, Österreich, 04.07.2021
	\item Porak L.: Governing the Ungovernable. Recontextualizations of Competition in European Policy Discourse. 33rd Annual EAEPE Conference, Online, Österreich, 03.09.2021
	\item Porak L.: Warum müssen wir noch immer arbeiten?. Momentum-Kongress 2021, Österreich, 16.10.2021
	\item Porak L.: Governing the Ungovernable. Recontextualizations of Competition in European Policy Discourse. Young Economists Conference 2021, Österreich, 07.11.2021
	\item Kapeller J.: Author Meets Critics: “Intangible Flow Theory in Economics: Human Participation in Economic and Societal Production”. 33rd Annual SASE Meeting, Participant in panel discussion -- online (planned in DEU), Österreich, 05.07.2021
	\item Kapeller J.: Tracing the invisible rich: The rank correction approach to inferring tails in survey data. Conference of the European Economic Association, Österreich, 25.08.2021
	\item Kapeller J.: Tracing the invisible rich. Conference of the European Economic Association, University of Copenhagen + digital, Dänemark, 25.08.2021
	\item Kapeller J.: (How) can heterodox economics inform the study of institutions?. University of GrazWorkshop \glqq Institutions\grqq{}, Graz University, Österreich, 16.10.2021
	\item Hornykewycz A., Rath J.: Shaping Sustainable Employment Relationships in the Age of Digitalisation: Analysing Policy Measures in an Agent-Based Framework. Momentum2021: Arbeit | Momentum Kongress, Hallstatt, Österreich, 16.10.2021
	\item Hornykewycz A., Rath J.: Shaping Sustainable Employment Relationships in the Age of Digitalisation: Analysing Policy Measures in an Agent-Based Framework. Young Economicsts Conference 2021Young Economists Conference 2021, Österreich, 06.11.2021
	\item Heimberger P.: Does economic globalisation promote economic growth? A meta analysis Presentation. 13th FIW-Research Conference ‘International Economics, Online, Österreich, 19.02.2021
	\item Heimberger P.: Still learning to catch-up? Technological capabilities, globalisation and economic growth. 33 rd Annual EAEPE Conference, Österreich, 03.09.2021
	\item Heimberger P.: Do corporate tax cuts boost growth?. 25th FMM Conference 2021, Deutschland, 30.10.2021
	\item Hager T., Heck I., Rath J.: Competition in Transitional Processes: Polanyi \& Schumpeter. Conference of the International Schumpeter Society, online, Österreich, 09.07.2021
	\item Hager T., Heck I., Rath J.: Competition in Transitional Processes: Polanyi and Schumpeter. 24th Annual Conference of the European Society of the History of Economic Thought, online, Österreich, 09.10.2021
	\item Gräbner-Radkowitsch C., Hornykewycz A., Schütz B.: The Emergence of Debt and Secular Stagnation in an unequal Society: A stock-flow consistent agent-based Approach. 33rd Annual EAEPE ConferenceEAEPE Annual Conference, Online, Österreich, 03.09.2021
	\item Gräbner-Radkowitsch C., Hornykewycz A., Schütz B.: The Emergence of Debt and Secular Stagnation in an unequal Society: A stock-flow consistent agent-based Approach. Momentum KongressEAEPE Annual Conference, Hallstatt, Österreich, 15.10.2021
	\item Gräbner-Radkowitsch C., Hornykewycz A., Schütz B.: The Emergence of Debt and Secular Stagnation in an unequal Society: A stock-flow consistent agent-based Approach. 25th FMM Conference 2021, Deutschland, 29.10.2021
	\item Gräbner-Radkowitsch C., Heimberger P., Kapeller J., Landesmann M., Schütz B.: The evolution of debtor-creditor relationships within a monetary union:
 Trade imbalances, excess reserves and economic policy. 25th FMM Conference, Berlin, Deutschland, 29.10.2021
	\item Gräbner-Radkowitsch C., Heimberger P., Kapeller J., Landesmann M.: The evolution of debtor-creditor relationships within a monetary union:
 Trade imbalances, excess reserves and economic policy. EAEPE Annual Conference, Online, Österreich, 04.09.2021
	\item Gräbner-Radkowitsch C., Hager T.: (Mis)Measuring Competitiveness: The Quantification of a Malleable Concept in the European Semester. 33rd Annual EAEPE Conference, Online, Österreich, 03.09.2021
	\item Gräbner-Radkowitsch C., Hager T.: (Mis)Measuring Competitiveness:  The Quantification of a Malleable Concept in the European Semester. Young Economists Conference 2021, Linz, Österreich, 06.11.2021
	\item Beyer K., Pühringer S.: \glqq Für die „Leistungsträger“ und „uns Österreicher“: Eine Mediendiskursanalyse zu Sozialreformen der ÖVP/FPÖ-Regierung 2017-2019 in Österreich\grqq{}. Momentum KongressMomentum: Arbeit, Hallstatt, Österreich, 16.10.2021
	\item Altreiter C., Gräbner-Radkowitsch C., Pühringer S., Rogojanu A., Wolfmayr G.: Theorizing Competition: An Interdisciplinary Framework. 33rd Annual SASE Meeting, Online (geplant in DEU), Österreich, 04.07.2021
	\item Aistleitner M., Pühringer S.: Top economists as free-trade cheerleaders in the public. EAEPE33rd Annual EAEPE Conference, virtuell (geplant in ITA, Neapel), Österreich, 03.09.2021
	\item Aistleitner M., Pühringer S.: Power structures in economics and their impact on trade policies: An SSE approach. Universität Neapel, Neapel (hybrid), Italien, 06.10.2021
	\item Aistleitner M., Kapeller J., Kronberger D.: The Authors of Economics Journals Revisited: Evidence from a Large-Scale Replication of Hodgson \& Rothman (1999). SASE 2021 Conference: After Covid? Critical Conjunctures and Contingent Pathways of Contemporary Capitalism, Virtual Congress, Österreich, 02.07.2021
	\item Aistleitner M., Kapeller J., Kronberger D.: The Authors of Economics Journals Revisited: Evidence from a Large-Scale Replication of Hodgson \& Rothman (1999). EAEPE Conference 202133rd Annual EAEPE Conference, Online, Österreich, 03.09.2021
	\item Aistleitner M.: The industrial policy research discourse (2000-2020) – a topic modelling approach. Young Economists Conference 2021: Industrial policy for sustainable work and a green transformation, Österreich, 22.11.2021
\end{enumerate}
\paragraph{Vortrag auf einer Tagung (nicht referiert)}
\begin{enumerate}
	\item Pühringer S.: Trade narratives of economists: A CDA approach. Workshop: Discourse Analysis in Economics, JKU Linz, Österreich, 21.10.2021
	\item Gräbner-Radkowitsch C., Pühringer S.: Freihandelsdebatten in der Ökonomie. Universität Linz, Österreich, 30.11.2021
\end{enumerate}
\paragraph{Eingeladener Vortrag an Universität}
\begin{enumerate}
	\item Pühringer S.: Handelsnarrative in der ökonomischen Debatte. Entwicklungspolitische HochschulwochenEntwicklungspolitische Hochschulwochen, PH OÖ, Linz (virtuell), Österreich, 30.11.2021
	\item Porak L.: Social Limits to Growth. , Universität Lüneburg, Deutschland, 10.06.2021
	\item Porak L.: Kulturelle Wachstumskritik. Universität Kiel, Deutschland, 02.11.2021
	\item Kapeller J.: Pandemic pushes polarization: Macroeconomic divergence in the Eurozone. Käte Hamburger Kolleg, online (geplant in DEU), Österreich, 23.04.2021
	\item Kapeller J.: The micro-macro link in heterodox economics. University of Tübingen, Tübingen, Deutschland, 17.11.2021
	\item Hirte K.: Plurale ökonomische Ideengeschichte. Initiative „Mehr ökonomische Vielfalt erreichen“, online (geplant in DEU), Österreich, 29.04.2021
	\item Hirte K.: Plurale volkswirtschaftliche Ideengeschichte. Universität Lüneburg, Deutschland, 29.04.2021
	\item Heimberger P.: Reforming the EU’s fiscal rules: Output gap estimates and fiscal policy in Belgium and beyond. Joint Vienna Institute, Österreich, 26.11.2021
	\item Gräbner-Radkowitsch C., Strunk B.: Warum braucht es Pluralismus in den Wirtschaftswissenschaften?. , Universität Wien, Österreich, 09.06.2021
	\item Gräbner-Radkowitsch C.: Komplexitätsökonomik – Alternative zur Gleichgewichtsökonomik?. Zeppelin Universität, Deutschland, 17.11.2021
	\item Gräbner-Radkowitsch C.: Komplexitätsökonomik – eine Einführung. Freie Universität Berlin, Deutschland, 03.12.2021
	\item Aistleitner M.: Wissenschaftstheoretische Ansätze in den WiWi und ihre Auswirkungen auf gesellschaftliche Prozesse (Teil 1). 18. SOLV an der WU WienWissensproduktion: Unis als Wissensfabriken und ihre Rolle in der Gesellschaft, Österreich, 20.04.2021
	\item Aistleitner M.: Wissenschaftstheoretische Ansätze in den WiWi und ihre Auswirkungen auf gesellschaftliche Prozesse (Teil 2). 18. SOLV an der WU WienWissensproduktion: Unis als Wissensfabriken und ihre Rolle in der Gesellschaft, Österreich, 28.04.2021
\end{enumerate}
\paragraph{Hauptvortrag / Eingeladener Vortrag auf einer Tagung}
\begin{enumerate}
	\item Pühringer S.: German neoliberalism in crisis?. University of Utrecht, Market Makers ProjectVarieties of Neoliberalization, Utrecht University, virtuell, Niederlande, 18.03.2021
	\item Kapeller J.: Pluralism in economics. , University of Zuerich, Schweiz, 07.10.2021
	\item Kapeller J.: Structural Change in times of increasing openness. University of LjubljanaSix Decades of the Past, Six Decades of the Future, online, Slowenien, 08.10.2021
	\item Kapeller J.: Economic Polarization in Europe. Six Decades of the Past, Six Decades of the Future, Centre of the International Relations of the University of Ljubljana, Slowenien, 09.10.2021
	\item Kapeller J.: What is heterodox economics?. , University of The Hague, Niederlande, 27.10.2021
	\item Heimberger P., Paczos W.: Comments on “Can the state be a good investor?”. Polish Economic Institute, Warsaw, Polen, 06.07.2021
	\item Gräbner-Radkowitsch C.: Komplexitätsökonomik -- ein Teaser. Kritische Wirtschaftswissenschaftler Berlin, Deutschland, 05.01.2021
\end{enumerate}
\paragraph{Andere Vorträge und Präsentationen}
\begin{enumerate}
	\item Kapeller J.: Conversations for the Future of Europe. nstitute on the Future of Taxation in Europe, Robert Schuman Centre for Advanced Studies, Participant in panel discussion -- online (planned in ITA), Österreich, 19.05.2021
	\item Hornykewycz A., Rath J.: Shaping Sustainable Employment Relationships in the Age of Digitalisation: Analysing Policy Measures in an Agent-Based Framework, Österreich, 02.12.2021
	\item Hager T., Hornykewycz A., Jonjic M., Porak L., Rath J.: „Hinter jeder erfolgreichen Frau steht ein Mann, der ihr den Rücken stärkt.“. Momentum-Kongress, Hallstatt, Österreich, 16.10.2021
	\item Hager T., Heck I., Rath J.: Competition in Transitional Processes: Polanyi and Schumpeter. Center for the History of Political Economy, Duke UniversityDuke University CHOPE Summer Institute 2021, online event (planned in USA), Österreich, 04.06.2021
	\item Gräbner-Radkowitsch C., Hornykewycz A., Schütz B.: The Emergence of Debt and Secular Stagnation in an unequal Society: A stock-flow consistent agent-based approach, Österreich, 18.11.2021
\end{enumerate}
\subsection*{2020}
\paragraph{Vortrag auf einer Tagung (referiert)}
\begin{enumerate}
	\item Pühringer S., Wolfmayer G.: Theorizing Competition: An interdisciplinary approach to the genesis of a contested concept. EAEPEEAEPE Online Conference, online aufgrund von COVID-19 (geplant: ESP), Österreich, 04.09.2020
	\item Pühringer S., Rath J.: Talking about Competition: Discursive shifts in the economic imaginary of competition in public debates. SASE 2020, online aufgrund von COVID-19 (geplant: NLD), Österreich, 22.07.2020
	\item Pühringer S., Rath J.: Talking about Competition. Discursive shifts in the economic imaginary of competition in public debates. EAEPE 2020, online aufgrund von COVID-19 (geplant: ESP), Österreich, 03.09.2020
	\item Pühringer S.: Theorizing Competition: An interdisciplinary approach to the genesis of a contested concept. Society for the Advancement of Socio-EconomicsSASE Annual Conference, online aufgrund von COVID-19 (geplant: NLD / Amsterdam), Österreich, 18.07.2020
	\item Porak L.: Wohin steuert die Europäische Union? Ein Klärungsversuch der strategischen Ausrichtung der EU seit Lissabon. Momentum 2020, Online, Österreich, 16.10.2020
	\item Porak L.: Allbetroffenheit in der Pandemie? Ein soziologischer Blick auf das Erleben der Auswirkungen der Corona-Krise. Momentum 2020, Österreich, 17.10.2020
	\item Hirte K.: Zettelkasten ist nicht gleich Zettelkasten. Zum ontologischen Problem in der Netzwerkforschung mit Fokus auf Mark Lombardi und Niklas Luhmann. Tagung: Warum Netzwerkforschung, Deutsche Gesellschaft für Netzwerkforschung, Darmstadt, Deutschland, 24.03.2020
	\item Hager T., Heck I., Rath J.: Competition in Transformational Processes: Polanyi \& Schumpeter. EAEPE, Online, Österreich, 03.09.2020
	\item Hager T., Heck I., Rath J.: Competition in Transformational Processes: Polanyi \& Schumpeter. Momentum-Kongress, Online, Österreich, 16.10.2020
	\item Bértola L., Kapeller J., Pezzini M.: Presente y futuro del desarrollo económico desde una perspectiva Latinoamericana. Economic Commission for Latin America and the Carribean (ECLAC, Chile, 02.10.2020
	\item Aistleitner M., Pühringer S.: Exploring the trade (policy) narratives in economic elite discourse. EAEPE annual conference 2020, online aufgrund von COVID-19 (geplant: ESP), Österreich, 03.09.2020
\end{enumerate}
\paragraph{Andere Vorträge und Präsentationen}
\begin{enumerate}
	\item Pühringer S.: Monopolies in Science Publishing: A black hole for public spending. SPACE-Kickoff, WU Wien, Österreich, 30.06.2020
\end{enumerate}
\paragraph{Eingeladener Vortrag an Universität}
\begin{enumerate}
	\item Kapeller J.: Was ist heterodoxe Ökonomie?. Universität WienSelbst-Organisierte Lehrveranstaltung: \glqq Alternative Wirtschaftstheorien -- Heterodoxe Ökonomie\grqq{}, Wien, Österreich, 19.03.2020
	\item Kapeller J.: Finanzialisierung und globale Ungleichheit. , Universität Wien, Österreich, 18.12.2020
	\item Gräbner-Radkowitsch C.: Pluralismus in der Ökonomik -- Wissenschaftstheoretische Hintergründe. Rethinking Economics Tübingen, Deutschland, 15.12.2020
\end{enumerate}
\paragraph{Hauptvortrag / Eingeladener Vortrag auf einer Tagung}
\begin{enumerate}
	\item Kapeller J.: Refeudalisierung als Gefahr für die Demokratie. Armutskonferenz SalzburgArmutskonferenz, Österreich, 10.03.2020
	\item Kapeller J.: Which are the most inspiring economic questions of our time?. , online -- YSI-Konferenz (INET), Vereinigtes Königreich, 11.11.2020
	\item Kapeller J.: Past and future of pluralism in economics. , University of Cambridge, Vereinigtes Königreich, 23.11.2020
\end{enumerate}
\subsection*{2019}
\paragraph{Andere Vorträge und Präsentationen}
\begin{enumerate}
	\item Schütz B.: Menschen- und Umweltrechte in globalen Beschaffungsketten. Workshop Global denken, global handeln: Über Globalisierung, Johannes Kepler Universität Linz, Österreich, 18.01.2019
	\item Pühringer S., Schütz B.: Warum gibt es die Staatsschuldenkrise?. SchülerInnenworkshop, Johannes Kepler Universität Linz, Österreich, 11.02.2019
	\item Pühringer S.: Machtstrukturen Neoliberaler Think Tanks in Österreich. Evangelische Bildungsakademie Oberösterreich, Linz, Österreich, 30.04.2019
\end{enumerate}
\paragraph{Vortrag auf einer Tagung (referiert)}
\begin{enumerate}
	\item Schütz B.: Creating a Pluralist Paradigm: An Application to the Minimum Wage Debate. 31st EAEPE Annual Conference, Warsaw School of Economics, Polen, 14.09.2019
	\item Schütz B.: Ein pluralistisches Paradigma? Anwendungsbeispiel Mindestlohn. Momentum19: Widerspruch, Hallstatt, Österreich, 11.10.2019
	\item Rath J.: Widersprüche im technologischen Fortschritt: am Beispiel der Blockchain Technologie. Momentum19: Widerspruch, Hallstatt, Kongresszentrum, Österreich, 12.10.2019
	\item Pühringer S., Ötsch W.: Market and society in Neo-liberalism and right-wing populism: A “reversed embeddedness”?. International Karl Polanyi Conference, Wien, ÖFSE, Österreich, 04.05.2019
	\item Pühringer S.: Why Economics is still Political Economy. Economists as enlightened experts and/or political proclaimers. 2. Vienna Conference on Pluralism in Economics, BOKU Wien, Österreich, 15.04.2019
	\item Pühringer S.: Divided We Stand? Professional Consensus and Political Conflict in Academic Economics. Annual Conference Society for the Advancement of Socio-Economics (SASE), New York, New School for Social Research, Vereinigte Staaten, 29.06.2019
	\item Pühringer S.: Divided We Stand? Professional Consensus and Political Conflict in Academic Economics. EAEPE Annual Conference, Warsaw School of Economics, Polen, 13.09.2019
	\item Porak L.: Der Wert des Widerspruchs für die demokratische Praxis. Momentum 2019, Österreich, 11.10.2019
	\item Kapeller J., Piiroinen P., Raghavendra S., Schütz B.: The conflict over income in a capitalist society: A stock-flow consistent approach. 23rd Conference of the Forum for Macroeconomics and Macroeconomic Policies (FMM), Berlin, Deutschland, 25.10.2019
	\item Kapeller J.: Pluralist and interdisciplinary research in academic economics. GSÖBW-Meeting: \glqq crossing borders, embracing pluralism\grqq{}, Duisburg, Deutschland, 21.02.2019
	\item Kapeller J.: Economics \& Philosophy. Pre-Conference at European Association for Evolutionary Economy, Warschau, Polen, 12.09.2019
	\item Kapeller J.: Convergence and Polarization Dynamics in Europa and globally. , Warschau, Polen, 14.09.2019
	\item Kapeller J.: Economic Polarization in Europe. Momentum 2019: Widerspruch, Hallstatt, Kongresszentrum, Österreich, 12.10.2019
	\item Hornykewycz A.: Models of Capability Accumulation. European Association for Evolutionary Political Economy, Polen, 13.09.2019
	\item Hirte K.: Netzwerkanalysen zur Entwicklung der professoralen Agrarpolitiker und Agrarökonomen Deutschlands 1933 bis 2000. , Aachen, RWTH, Deutschland, 29.03.2019
	\item Hirte K.: Too big to fail…? Subventionierte Strukturplan-Politik in Deutschland zur Forcierung von Massentierhaltung, Österreich, 26.09.2019
	\item Hirte K.: \glqq Alte Bekannte im neuen Gewand\grqq{} – Zum Phänomen kontextualer Neuinterpretation von Wissen. , Salzburg, Universität, Österreich, 27.09.2019
	\item Hirte K.: Relationalität in der Soziologie und „Poppers Fluch“. , Salzburg, Universität, Österreich, 28.09.2019
	\item Heimberger P.: Structural change in times of increasing openness: Assessing path dependency in European economic integration. 2nd Vienna Conference on Pluralism in Economics, Wien, Österreich, 16.04.2019
	\item Heimberger P.: The power of economic models: The case of the EU's fiscal regulation framework. SASE 2019, New York City, Vereinigte Staaten, 28.06.2019
	\item Heimberger P.: Is the Eurozone disintegrating? Macroeconomic divergence, structural polarization. EAEPE Conference 2019, Warschau, Polen, 13.09.2019
	\item Heimberger P.: Does economic globalization affect income inequality? A meta-analysis. Young Economists Conference 2019, Wien, Österreich, 01.10.2019
	\item Heimberger P.: Does economic globalization affect income inequality? A meta-analysis. , Wien, Österreich, 05.12.2019
	\item Gräbner-Radkowitsch C., Strunk B.: Harvesting the Benefits of Different Perspectives: Theoretical and Practical Reflections on Pluralistic Education in Economics. Grenzen überwinden, Pluralismus wagen. Perspektiven sozio*ökonomischer Hochschullehre, Duisburg, Deutschland, 22.02.2019
	\item Gräbner-Radkowitsch C., Kapeller J.: Structural change in times of increasing openness: Assessing path dependency in European Economic integration. Jahrestagung des Vereins für Socialpolitik, Leipzig, Deutschland, 23.09.2019
	\item Gräbner-Radkowitsch C., Heimberger P., Kapeller J., Schütz B.: Is the Eurozone disintegrating? Macroeconomic divergence, structural polarization, trade and fragility. FMM Conference 2019, Berlin, Deutschland, 26.10.2019
	\item Gräbner-Radkowitsch C.: Kommentar zu Komplexität und Evolution in der Ökonomik. Jahrestagung Ausschuss für Evolutorische Ökonomik (VfS), Wien, Österreich, 05.07.2019
	\item Gräbner-Radkowitsch C.: Unrealistic models and how to identify them: on accounts of model realisticness. Annual Conference of the European Association for Evolutionary Political Economy, Warschau, Polen, 14.09.2019
	\item Aistleitner M., Pühringer S.: Exploring the trade narrative in the ‘top 5’ journals in economics. 2. Vienna Conference on Pluralism in Economics, BOKU Wien, Österreich, 16.04.2019
	\item Aistleitner M., Grimm C., Kapeller J.: Auftragsvergabe, Leistungsqualität und Kostenintensität im Schienenpersonenverkehr. Momentum 2019, Österreich, 12.10.2019
	\item Aistleitner M.: (Towards) exploring the genesis of competition in economic thought. Young Economists Conference 2019, Österreich, 02.10.2019
\end{enumerate}
\paragraph{Hauptvortrag / Eingeladener Vortrag auf einer Tagung}
\begin{enumerate}
	\item Pühringer S.: Neoliberalismus und „Rechtspopulismus“ Analogien und Widersprüche. Politische Akademie TutzingTagung Neoliberalismus und Rechtspopulismus: Zusammenhänge, Strategien und Umgangsformen, Tutzing (München), Deutschland, 19.12.2019
	\item Kapeller J.: The future of pluralism in economics. Still Rethinking? The Need for Pluralism in Economics, London, Vereinigtes Königreich, 31.03.2019
	\item Kapeller J.: Philosophy and economics. Still Rethinking? The Need for Pluralism in Economics, London, Vereinigtes Königreich, 31.03.2019
	\item Kapeller J.: Development and globalization: a pluralist perspective. 2nd Vienna Conference on Pluralism in Economics, Wien, Österreich, 16.04.2019
	\item Kapeller J.: Der Feind im Inneren: Das alte Dilemma des Liberalismus. Top Management Symposium \glqq Open Society and its Enemies\grqq{}, Udine, Italien, Italien, 31.05.2019
	\item Kapeller J.: Wirtschaftliche Polarisierung in Europa. Wirtschaftskammer KärtnenAktuelle Gegensätze \& ökonomische Betrachtungen, Kärntnen, Österreich, 05.12.2019
	\item Heimberger P.: Fiscal policies in international perspective. Applied economics seminar at the Joint Vienna Institute, Wien, Österreich, 04.06.2019
	\item Heimberger P.: 'Output gap nonsense': Reforming the EU's fiscal rules. ETUC Policy Committee, Österreich, 10.10.2019
	\item Gräbner-Radkowitsch C.: Pluralism, complexity and the effective triangulation of methods. Summer School on Complexity Economics, Behavioural Economics, and Data Science, Bochum, Deutschland, 02.09.2019
	\item Gräbner-Radkowitsch C.: Country capabilities, product complexity, and finance in the EU: An AB-SFC multi country model for policy analysis. Royal Economic SocietyHow Can Interdisciplinary Research Enhance the Policy Relevance of Macroeconomics?, Edinburgh, Vereinigtes Königreich, 19.09.2019
	\item Gräbner-Radkowitsch C.: Getting the best of both worlds: potentials for triangulating agent-based and equilibrium-based analysis. Agent-based economics, Graz, Österreich, 30.10.2019
\end{enumerate}
\paragraph{Eingeladener Vortrag an anderen Institutionen}
\begin{enumerate}
	\item Kapeller J.: Gesellschaftliche Verantwortung der Wissenschaft. Friedrich-Ebert-Stiftung BerlinFachkonferenz Wissenschaft in Verantwortung, Berlin, Deutschland, 21.11.2019
	\item Kapeller J.: Der Feind im Inneren – Das alte Dilemma des Liberalismus. 18. IT- \& Beratertag der Wirtschaftskammer Österreich, Österreich, 04.12.2019
	\item Hirte K.: Funktionale Konstrukte von Bevölkerung im Kontext der „Agrar- versus Industriestaatsdebatte. Charité Berlin, Deutschland, 28.10.2019
	\item Grimm C.: Paradigmen und Politik. Bestandsaufnahme der deutschen Ökonomik im Vergleich zu den USA. , Duisburg, Deutschland, 22.02.2019
\end{enumerate}
\paragraph{Vortrag auf einer Tagung (nicht referiert)}
\begin{enumerate}
	\item Hirte K.: Ökonomisches Denken selbst bewertet – Muster, Mechanismen und Dynamiken in der ökonomischen Dogmenhistorie. Tagung des DFG-Netzwerkes \glqq Soziologie des ökonomischen Denkens\grqq{}, Universität Kassel, Deutschland, 16.05.2019
\end{enumerate}
\subsection*{2018}
\paragraph{Vortrag auf einer Tagung (referiert)}
\begin{enumerate}
	\item Schütz B.: Employment and the minimum wage: A pluralist approach. 20th Anniversary Conference of the Association for Heterodox Economics, University of Leicester, Vereinigtes Königreich, 13.07.2018
	\item Pühringer S.: The Anti-democratic Logic of Right-wing Populism and Neoliberal Market-fundamentalism. Association for Social Economics (ASE)ASSA Annual Conference, Philadelphia, Vereinigte Staaten, 07.01.2018
	\item Pühringer S.: What economics education is missing: the real world. EAEPE, Nice -- Sophia Antipolis, Frankreich, 06.09.2018
	\item Pühringer S.: Divided we stand? Die politischen Orientierungen ‚öffentlicher ÖkonomInnen‘ in den USA. Universität Hamburg, World Economics Association, AK Politische Ökonomie10 Jahre nach der Weltfinanzkrise, Universität Hamburg, Deutschland, 16.11.2018
	\item Pühringer S.: Polit-ökonomische Machtstrukturen unter „öffentlichen ÖkonomInnen“. Momentum 2018: Klasse, Hallstatt, Österreich, 19.12.2018
	\item Kapeller J., Schütz B.: Government Policies and Financial Crises: Mitigation, Postponement or Prevention?. ASSA Annual Meeting 2018, Philadelphia, Vereinigte Staaten, 05.01.2018
	\item Kapeller J.: Structural change in time of increasing openness: assessing path dependency in European economic integration. Annual Meeting of the German Economic Association, Leipzig, DE, Deutschland, 24.09.2018
	\item Kapeller J.: The focus of academic economics. 22. FMM-Konferenz, Berlin, Deutschland, 28.10.2018
	\item Hirte K.: Entitäten und Relationen – Die relationale Forschungsperspektive. Tagung: Das Paradigma der Relationalität, Deutsche Gesellschaft für Netzwerkforschung, Deutschland, 04.12.2018
	\item Hirte K.: Intended and unintended non-knowledge – a neglected area in the debate on economic pluralism. Forms of Power in Economics: New perspectives for the Social Studies of Economics between networks, discourses and fields, Universität Giessen, Deutschland, 07.12.2018
	\item Heimberger P.: Do labor market rigidities increase unemployment? Evidence for 23 OECD countries over 1985-2013. Young Economists Conference 2018, Wien, Österreich, 10.10.2018
	\item Heimberger P.: Strukturelle Polarisierung: Warum Europa trotz Aufschwungs ökonomisch auseinanderdriftet. Momentum 2018, Hallstatt, Österreich, 22.10.2018
	\item Gräbner-Radkowitsch C., Heimberger P., Kapeller J.: The Kaldor paradox revisited: New evidence in times of increasing openness“. Annual Meeting of the European Association for Evolutionary Economy, Nice, Frankreich, 07.09.2018
	\item Gräbner-Radkowitsch C., Heimberger P., Kapeller J.: Structural change in times of increasing openness: Assessing path dependency in European economic integration. EAEPE Conference 2018, Nizza, Frankreich, 08.09.2018
	\item Gräbner-Radkowitsch C.: Die Bedeutung von Vertrauen und sozialer Kontrolle für die Funktion informeller Wert-Transfer Systems. Tagung des Evolutorischen Ausschusses des Vereins für Socialpolitik, TU Dresden, Deutschland, 13.07.2018
	\item Elsner W., Gräbner-Radkowitsch C.: To trust or to control -- Informal value transfer systems and computational analysis in institutional economics. Allied Social Sciences Associations Meeting, Philadelphia, PA, Vereinigte Staaten, 07.01.2018
	\item Beyer K., Pühringer S.: Why Economics is Still Political Economy: Economists as enlightened experts and/or political proclaimers. Forms of Power in Economics, Universität Giessen, Deutschland, 06.12.2018
	\item Aistleitner M.: The Focus of Academic Economics -- Before and After the Crisis. EAEPE 2018: Evolutionary foundations at a crossroad: Assessments, outcomes and implications for policymakers, Nice -- Sophia Antipolis, Frankreich, 07.09.2018
\end{enumerate}
\paragraph{Eingeladener Vortrag an Universität}
\begin{enumerate}
	\item Schütz B.: Geld -- Eine post-keynesianische Perspektive. VW-Zentrum für Studierende, WU Wien, Österreich, 18.04.2018
	\item Pühringer S.: Wie denken zukünftige Ökonom\_innen?. Universität Mannheim; Plurale ÖkonomikDiskussionsveranstaltung zur Zukunft der deutschen Ökonomik, Mannheim, Universität, Deutschland, 20.03.2018
	\item Kapeller J.: Ökonomische Diskurse: Historische und philosophische Perspektiven. WU WienEinführung zu SOLVXII -- Debatten in der Ökonomie, Wien, Österreich, 11.04.2018
	\item Kapeller J.: Die Spaltung als Modell: Europas Zerfall als Krise der Wirtschaftstheorie. Universität TübingenRingvorlesung 10 Years After the Crash, Tübingen, Deutschland, 04.07.2018
	\item Hirte K.: Vergessen – Verkennen – Vermeiden: Das Problem Andersdenkende in den Wissenschaften. Universität Jena, Deutschland, 20.06.2018
	\item Hirte K.: Performativität und Ökonomik. Universität Jena, Deutschland, 21.06.2018
	\item Heimberger P.: Modellierung von Fiskalpolitik: Theorie und Empirie anhand von aktuellen Debatten. SOLV XII – Debatten der Ökonomie, WU Wien, Österreich, 09.05.2018
\end{enumerate}
\paragraph{Hauptvortrag / Eingeladener Vortrag auf einer Tagung}
\begin{enumerate}
	\item Pühringer S., Ötsch W.: „Netzwerke des Marktes“ in ihrem Einfluss auf Gesellschaft. Reshaping Economics, Tutzing (München), Deutschland, 28.04.2018
	\item Gräbner-Radkowitsch C.: Theory development through agent-based modelling Lessons from economics. Theory Development Through Agent-Based Modeling, Hannover, Deutschland, 14.06.2018
	\item Bäuerle L., Pühringer S.: VWL-Studium als alltägliche Lebensrealität. Reshaping Economics, Tutzing, Evang. Akademie, Deutschland, 27.04.2018
\end{enumerate}
\paragraph{Andere Vorträge und Präsentationen}
\begin{enumerate}
	\item Pühringer S.: Verteilungsgerechtigkeit und ökonomische Ungleichheit. HAK Traun, Traun, Österreich, 10.04.2018
	\item Kapeller J., Tamesberger D.: Zivilisierte Märkte. Pressegespräch, Ausweichquartier Parlament, Österreich, 18.07.2018
	\item Kapeller J., Schütz B.: Ökonomische Effekte der Verkehrsreform des Landes Tirol. LH-Stv. Ingrid Felipe, Landhaus Innsbruck, Österreich, 04.06.2018
	\item Hirte K.: Ernährungswende gemeinsam gestalten. Bio Austria Linz, Teilnahme Podiumsdiskussion, Österreich, 06.03.2018
	\item Aigner E., Pühringer S.: Politik und Paradigmen in der Ökonomie in Deutschland und den USA. FGW Forschungsinstitut für gesellschaftliche WeiterentwicklungFGW Vernetzungstreffen, Düsseldorf, Deutschland, 16.03.2018
\end{enumerate}
\paragraph{Eingeladener Vortrag an anderen Institutionen}
\begin{enumerate}
	\item Pühringer S.: 10 Jahre Krise: Neoliberale Denkmuster, wirtschaftspolitische Debatten und der Einfluss von Think Tanks. AK KärntenÖkonomie: Was ist eigentlich neoliberal?, Villach, Österreich, 09.02.2018
	\item Pühringer S.: Machtstrukturen Neoliberaler Think Tanks. KSÖ, ÖGB, KAB, Linz, Cardijn Haus, Österreich, 15.10.2018
	\item Pühringer S.: Freiheitliche Flügelkämpfe? (Historische) Konfliktlinien in der FPÖ. Kurswechsel-Heftpräsentation, Linz, Central, Österreich, 18.12.2018
	\item Kapeller J.: Verteilungsgerechtigkeit im Familienrecht: Eine philosophische Perspektive. Association of Austrian Judges in Family Lay31. FamilienrichterInnen -- Tagung, Schladming, Österreich, 07.06.2018
	\item Kapeller J.: Widersprüche in der Ökonomie. , Velden, Österreich, 24.08.2018
	\item Hirte K.: Performativität und Ökonomik – Worum sorgen sich Ökonom*innen?. „Plurale Ökonomik“, Universität JenaVorlesungsreihe „Plurale Ökonomik“, Universität Jena, Deutschland, 21.06.2018
	\item Heimberger P.: Österreichs Staatsausgabenstrukturen im europäischen Vergleich. Margit SchratzenstallerBudget Jour Fixe WIFO, WIFO, Wien, Österreich, 10.01.2018
	\item Heimberger P.: Fiscal policies in international perspective. wiiw, Joint Vienna Institute, Wien, Österreich, 05.06.2018
	\item Gräbner-Radkowitsch C., Schütz B.: Country capabilities, product complexity, and finance in the EU:  An AB-SFC multi country model for policy analysis. Rebuilding Macroeconomics, National Institute of Economic and Social Research, London, Vereinigtes Königreich, 15.11.2018
\end{enumerate}
\paragraph{Habilitationsvortrag}
\begin{enumerate}
	\item Hirte K.: Zur Transformation der ostdeutschen Agrarstrukturen 1990/1991 sowie zu ihrer neuerlichen Transformation. Universität Jena, Universität Jena, Deutschland, 31.01.2018
\end{enumerate}
\paragraph{Vortrag auf einer Tagung (nicht referiert)}
\begin{enumerate}
	\item Hirte K.: Andersdenkende als Vordenker im wirtschaftswissenschaftlichen Feld. Tagung des DFG-Netzwerkes \glqq Soziologie des ökonomischen Denkens\grqq{}, Max-Planck-Institut Köln, Deutschland, 19.10.2018
\end{enumerate}
\subsection*{2017}
\paragraph{Andere Vorträge und Präsentationen}
\begin{enumerate}
	\item Schütz B.: Wie wird gerechter Handel möglich?. Öffentlicher ExpertInnenworkshop der Allianz gerechter Handel, Linz, Österreich, 09.06.2017
	\item Kapeller J.: Globalization and Free Trade. Europäisches Forum AlpbachForum Alpbach 2017, Alpbach, Tirol, Österreich, 29.08.2017
	\item Hirte K.: Wirtschaftswissenschaften und sozial-ökologische Transformation. Institut für ökologische Wirtschaftsforschung (IÖW) Berlin, Diskussionsteilnahme, Deutschland, 06.11.2017
\end{enumerate}
\paragraph{Hauptvortrag / Eingeladener Vortrag auf einer Tagung}
\begin{enumerate}
	\item Pühringer S., Ötsch W.: Zur zentralen Rolle ordoliberaler Netzwerke in wirtschaftspolitischen Reformprozessen in Deutschland. Gesellschaft für Sozioökomischen Bildung und Wissenschaft (GSÖBW)Gründungskonferenz der Gesellschaft für Soziökomischen Bildung und Wissenschaft (GSÖBW), Tutzing (München), Deutschland, 17.03.2017
	\item Pühringer S.: Bilder von ÖkonomInnen zur Finanzkrise 2008. Frankenakademie BambergMacht der Bilder, Macht der Sprache, Bamberg, Deutschland, 25.05.2017
	\item Heimberger P.: Fiscal consolidation and disintegration tendencies in Europe: Is there a link?. „Europe at a crossroads": Member seminar of the Vienna Institute for International Economic Studies, Wien, Österreich, 30.03.2017
	\item Heimberger P.: Die Eurokrise: Wettbewerbsdenken vs. koordinierte Wirtschaftspolitik. , Linz, Österreich, 19.05.2017
	\item Heimberger P.: Ungleichheit in den USA: Ökonomische und gesellschaftliche Auswirkungen. , Wien, Österreich, 22.11.2017
	\item Gräbner-Radkowitsch C.: Verification and validation of agent-based models: Epistemological importance and theoretical challenges. Prof. Dr. Sylvie GeisendorfAgent-based modeling in Ecological Economics -- From toy model to verified tool of analysis, ESCP Europe Berlin, Deutschland, 19.05.2017
\end{enumerate}
\paragraph{Vortrag auf einer Tagung (referiert)}
\begin{enumerate}
	\item Pühringer S., Stelzer-Orthofer C.: Neoliberale Think Tanks als Katalysatoren gesellschaftlicher Reformdiskurse.. ESPAnet Austria: 1. Forschungskonferenz Sozialpolitik, Wien, WU, Österreich, 21.04.2017
	\item Pühringer S.: Selling the “economic truth” to the public. Herbert Giersch and Hans-Werner Sinn as “public intellectuals". University of BelfastLanguage and Economics, Belfast, Irland, 17.07.2017
	\item Kapeller J.: Citation metrics and the development of economics. European Society for the History of Economic Thought (ESHET)The Fragmentation of Economics and the New Role of the History of Economic Thought, University of Torino Campus Luigi Einaudi, Italien, 15.09.2017
	\item Kapeller J.: Citation metrics and the development of economics. Institute for New Economic Thinking2017 INET Plenary Conference, Edinburgh, Scotland, Vereinigtes Königreich, 24.10.2017
	\item Hirte K.: Mendelejews Traum – Faktuale und fiktionale Komponenten einer Wissenschaftsgeschichte. Universität Augsburg, 11. Workshop Historische Netzwerkforschung, Deutschland, 26.05.2017
	\item Heimberger P.: Enforcing economic liberalism in European fiscal policy-making: On the role of the European Commission’s poential output model. A Great Transformation?, Linz, Österreich, 11.01.2017
	\item Heimberger P.: Did fiscal consolidation cause the double-dip recession in the euro area?. Keynes-Tagung „Keynes, Geld und Finanzen“, Wien, Österreich, 20.02.2017
	\item Heimberger P.: The dynamic effects of fiscal consolidation episodes on income inequality: Evidence for 17 OECD countries over 1978-2013. ECINEQ Conference 2017, New York City, Vereinigte Staaten, 19.07.2017
	\item Heimberger P.: The dynamic effects of fiscal consolidation episodes on income inequality: Evidence for 17 OECD countries over 1978-2013. European Integration at a Crossroads, Österreich, 12.10.2017
	\item Heimberger P.: The dynamic effects of fiscal consolidation episodes on income inequality: Evidence for 17 OECD countries over 1978-2013. FMM Conference 2017, Deutschland, 10.11.2017
	\item Gräbner-Radkowitsch C., Heinrich T.: The dance of Godzilla and the earthquake: on the sectoral and structural foundations of macroeconomic dynamics. Conference on Complex Systems, Cancun, Mexiko, Mexiko, 20.09.2017
	\item Gräbner-Radkowitsch C.: The Nature of Institutions: A Computational Perspective. Allied Social Sciences Associations (ASSA) Annual Meeting, San Francisco, CA, Vereinigte Staaten, 07.01.2017
	\item Gräbner-Radkowitsch C.: The Complementary Relationship Between Evolutionary-Institutional and Complexity Economics. Allied Social Sciences Associations (ASSA) Annual Meeting, Chicago, IL, Vereinigte Staaten, 07.01.2017
	\item Gräbner-Radkowitsch C.: The complexity approach to economic development as a new rationale for industrial policy?. EAEPE Symposium on Development Economics, DIW, Berlin, DE, Österreich, 11.06.2017
	\item Gräbner-Radkowitsch C.: On the many ways a model can be unrealistic – and still useful. What to make of highly unrealistic models?, Academy of Finland Centre of Excellence in the Philosophy of Social Sciences (TINT), Finnland, 12.10.2017
	\item Griesser M.: Uneven waves of commodification, decommodification, and recommodification. Karl Polanyi and the Analysis of Welfare State Transformation. A Great Transformation? Global Perspectives on Contemporary Capitalisms, JKU Linz, Österreich, 13.01.2017
	\item Griesser M.: Arbeitsmarktpolitik als Gesellschaftspolitik. ESPAnet Austria: 1. Forschungskonferenz Sozialpolitik, Wirtschaftsuniversität Wien, Österreich, 20.04.2017
	\item Ghorbani A., Gräbner-Radkowitsch C.: Towards a computational understanding of institutions – A review and a new proposition. Jahrestagung des World Interdisciplinary Network for Institutional Research (WINIR), Utrecht Universiy, Utrecht, NL, Niederlande, 15.09.2017
	\item Ghorbani A., Gräbner-Radkowitsch C.: On defining institutions and how to study them. Annual Conference of the European Association for Evolutionary Political Economy (EAEPE), Corvinus University Budapest, Ungarn, 21.10.2017
	\item Elsner W., Gräbner-Radkowitsch C.: Sources of cooperation, performance, and stability in informal value transfer systems. Annual Conference of the European Association for Evolutionary Political Economy (EAEPE), Corvinus University Budapest, Ungarn, 19.10.2017
	\item Beyer K., Pühringer S.: The political consequences of paradigmatic monism in economics. Evidences from a comparative analysis of German and US economics. EAEPEAnnual Conference European Ass. for Evol. Pol. Economy (EAEPE), Budapest, Corvinus University, Ungarn, 20.10.2017
	\item Aistleitner M.: Prospects of a sustainable automotive industry. A Great Transformation? Global Perspectives on Contemporary Capitalisms, 10-13.01.2017, Österreich, 11.01.2017
	\item Aistleitner M.: In cars we trust. Automobile manufacturing as part of an integrative European industrial policy.. Young Economists Conference 2017 (‘European integration at a crossroads’), Österreich, 13.10.2017
	\item Aistleitner M.: Citation Patterns in Economics and Beyond: Assessing the Peculiarities of Economics from Two Scientometric Perspectives. Momentum 2017: Vielfalt, Österreich, 20.12.2017
\end{enumerate}
\paragraph{Eingeladener Vortrag an Universität}
\begin{enumerate}
	\item Pühringer S.: Diskursive und politische Wirkmächtigkeit ökonomischen Denkens. Institut für Institutionelle und Heterodoxe Ökonomie, WU Wien, WU Wien, Österreich, 17.01.2017
	\item Pühringer S.: Profil der deutschsprachigen Volkswirtschaftslehre im internationalen Vergleich. Universität HamburgHeimann-Colloquiums an der Universität Hamburg, Universität Hamburg, Deutschland, 29.06.2017
	\item Pühringer S.: Ordoliberalismus als politisches Gestaltungswissen:  ein deutscher Sonderweg. Universität JenaBedingungen und politische Funktionen exakter Wissenschaften, Uni Jena, Deutschland, 27.10.2017
	\item Pühringer S.: Economists' Narratives and Metaphors of the Financial Crisis. Universität Wien, Österreich, 30.11.2017
	\item Kapeller J.: Paradigm lost? Herausforderungen der Ökonomik nach der Krise, Deutschland, 26.04.2017
	\item Kapeller J.: Economics and Philosophy: Models and Pluralism, Deutschland, 27.04.2017
	\item Kapeller J.: Verteilungstendenzen im globalen Kapitalismus. Eine plurale Perspektive., Deutschland, 04.05.2017
	\item Kapeller J.: Wissenschaftstheorie und Ökonomie. , Universität Bochum, Deutschland, 08.05.2017
	\item Kapeller J.: Wissenschaftstheorie und Ökonomie. Universität TübingenRingvorlesung \glqq Frontiers of Economics – Die Wirtschaftswissenschaft zwischen Krise und Aufbruch\grqq{}, Universität Tübingen, DE, Deutschland, 20.12.2017
	\item Gräbner-Radkowitsch C.: Komplexitätsökonomik. Ringvorlesung \glqq Ökonomische Denkschulen und Grundlagen der Wissenschaftstheorie\grqq{}, Ruhr Universität Bochum, Deutschland, 12.06.2017
	\item Gräbner-Radkowitsch C.: Institutionenökonomik, Komplexitätsökonomik und Pluralismus. Kritische Wirtschaftswissenschaftler FU BerlinRingvorlesung \glqq Denkschulen und aktuelle Kontroversen der Ökonomik\grqq{}, Freie Universität Berlin, Deutschland, 28.11.2017
\end{enumerate}
\paragraph{Eingeladener Vortrag an anderen Institutionen}
\begin{enumerate}
	\item Pühringer S.: Ordoliberale Netzwerke und ihre Wirkmächtigkeit. Akademie für Politische Bildung Tutzingder(wirtschafts-)politische Einfluss des Ordoliberalismus, Tutzing (München), Deutschland, 10.10.2017
	\item Kapeller J., Schütz B., Tamesberger D.: Moralität, Wettbewerb und internationaler Handel: Eine europäische Perspektive. Österreichische ArbeiterkammerAK LänderreferentInnentagung EU und Internationales, Wien, Österreich, 04.04.2017
	\item Kapeller J.: The Political Economy of Bank Regulatory Evasion. Vienna International Summer SchoolSummer School, WU Wirtschaftsuniversität Wien, Österreich, 06.09.2017
	\item Kapeller J.: Agenten, Auktionen, Ausschreibungen: Von Nobelpreisen und Vergabeverfahren. ParlamentBahnenquete, Parlament Wien, Österreich, 13.09.2017
	\item Griesser M.: Wohlstandsorientierte Politik: Möglichkeiten für eine gesellschaftliche Verankerung?. , Kammer für Arbeiter und Angestellte Wien, Österreich, 16.03.2017
	\item Griesser M.: MigrantInnen als Zielgruppe der österreichischen Gewerkschaftsbewegung. Präsentation von Heft 2/2017 der Österreichischen Zeitschrift für Soziologie, Forschungs- und Beratungsstelle Arbeitswelt (FORBA), Wien, Österreich, 14.06.2017
\end{enumerate}
\paragraph{Vortrag auf einer Tagung (nicht referiert)}
\begin{enumerate}
	\item Kapeller J.: Pluralist and heterodox economics: Conceptual clarification and exemplary applications. COST-Workshops, Irland, 15.03.2017
	\item Hirte K.: The two transitions from algebraic to analytical thinking in economics in the 18th and 19th century. Tagung des DFG-Netzwerkes \glqq Soziologie des ökonomischen Denkens\grqq{}, Universität München, Deutschland, 09.02.2017
	\item Grimm C.: Paradigmatische Homogenität? Aktueller Stand und  Zukunftsperspektiven der Ökonomik in Deutschland und den USA. Momentum 2017: Vielfalt, Hallstatt, Österreich, 20.10.2017
	\item Griesser M.: Ansätze einer Verankerung wohlstandsorientierter Politik in Österreich. Kongress „Gutes Leben für alle“, Wirtschaftsuniversität Wien, Österreich, 11.02.2017
\end{enumerate}
\paragraph{Posterpräsentation}
\begin{enumerate}
	\item Gräbner-Radkowitsch C., Heinrich T.: Beyond Equilibrium: Revisting Two-Sided Markets from an Agent-Based Perspective. Conference on Complex Systems, Tempe, Arizona, Vereinigte Staaten, 29.09.2017
\end{enumerate}
\subsection*{2016}
\paragraph{Vortrag auf einer Tagung (referiert)}
\begin{enumerate}
	\item Pühringer S., Ötsch W.: Zur zentralen Rolle ordoliberaler Netzwerke in wirtschaftspolitischen Reformprozessen in Deutschland. Jahrestagung des Ausschusses für Evolutorische Ökonomik im Verein für Socialpolitik an der Leibniz Universität Hannover, Universität Hannover, Deutschland, 02.07.2016
	\item Pühringer S.: Still the Queens of Social Sciences. (Post-)Crisis power balances  of “public economists” in Germany. University of Crete, Faculty of Social, Economic and Political SciencesCrisis and the Social Sciences: New Challenges and Perspectives, Crete, Rythymno, Griechenland, 11.06.2016
	\item Pühringer S.: The “Performative Footprint” of economists political and societal impact of post-war German economists. Annual Conference \glqq European Association for Evolutionary and Political Economy\grqq{}, Manchester, Metropolitan University, Vereinigtes Königreich, 04.11.2016
	\item Liedl B., Pühringer S.: Strategien einer permanenten neoliberalen Diskurshegemonie?. Momentum Kongress 2016: Macht, Hallstatt, Österreich, 14.10.2016
	\item Kapeller J.: The performativity of potential output: Pro-cyclicality and path-dependency in coordinating European fiscal policies. Conference on Complex Systems (CCS), Amsterdam, NL, Niederlande, 20.09.2016
	\item Kapeller J.: Spezialisierung, Stratifikation und internationale Wirtschaft. Momentum16: Macht, Hallstatt, Österreich, 15.10.2016
	\item Hirte K.: Zeitlichkeit und Tauschfähigkeit bei Rosa Luxemburg und Joseph Alois Schumpeter. SPP 1688 der DFG, Universität Gießen., Tagung: Aisthetische Eigenzeiten von Tausch und Gabe. SPP 1688 der DFG, Universität Gießen., Deutschland, 24.09.2016
	\item Hirte K.: Ökonomen-Netzwerke – Zu historischen und aktuellen Netzwerken der deutschen VWL-Professorinnen und Professoren. Schader-Forum, Tagung: Der Stand der Netzwerkforschung, Darmstadt, Deutschland, 05.12.2016
	\item Heimberger P., Kapeller J.: The performativity of potential output: Pro-cyclicality and path dependency in coordinating European fiscal policies. FMM conference -- Towards Pluralism in Macroeconomics?, Berlin, Deutschland, 21.10.2016
	\item Heimberger P., Kapeller J.: The performativity of potential output: Pro-cyclicality and path dependency in coordinating European fiscal policies. The 28th annual EAEPE conference, Manchester, Vereinigtes Königreich, 05.11.2016
	\item Gräbner-Radkowitsch C.: Computational Game Theory and the Micro Foundations of Institutions. Institutions -- Emergence or Design?, Niederlande, 08.09.2016
	\item Gräbner-Radkowitsch C.: The effect of selection and reputation mechanisms on the emergence of cooperation,. Annual Conference of the European Association for Evolutionary Political Economy (EAEPE), Manchester Metropolitan University, Manchester, UK, Vereinigtes Königreich, 04.11.2016
	\item Griesser M., Pühringer S.: Discursive Shifts in the Transition from an “Active” to an “Activating” Labour Market Policy (LMP). Interpretive Policy Analysis Conference, University of Hull, Vereinigtes Königreich, 07.07.2016
	\item Aistleitner M.: Perspektiven für eine nachhaltige Automobilindustrie. Momentum 2016: Macht, Österreich, 15.10.2016
\end{enumerate}
\paragraph{Andere Vorträge und Präsentationen}
\begin{enumerate}
	\item Pühringer S.: Wirkungen ökonomischer Lehre. Treffen AG Ökonomie und Studierende, Universität Frankfurt am Main, Deutschland, 27.07.2016
	\item Pühringer S.: Podiumsdiskussion: Was kommt nach der Krise. im Anschluss an die Filmpremiere \glqq Zero Crash\grqq{}, Linz, Movimiento Kino, Österreich, 19.09.2016
	\item Hubmann G., Pühringer S.: Workshop: Think Tanks und Macht. ICAE-Sommerakademie „Globalisierung und Konzernmacht“, Wissensturm Linz, Österreich, 03.06.2016
	\item Griesser M.: Workshop \glqq Armut und soziale Ausgrenzung\grqq{}. Tagung „Helfe sich wer kann! Unser Sozialsystem im Umbruch“, FH Oberösterreich, Fakultät für Medizintechnik und Angewandte Sozialwissenschaften, Österreich, 05.11.2016
	\item Brand U., Griesser M.: Studienpräsentation \glqq Verankerung wohlstandsorientierter Politik". Workshop "Verankerung wohlstandsorientierter Politik\grqq{}, AK Wien, Österreich, 12.12.2016
	\item Beyer K., Bräutigam L.: Das europäische Schattenbankensystem – Typologisierung und Bewertung regulatorischer Initiativen auf europäischer Ebene. Arbeiterkammer Wien, Arbeitskreis für europäische Wirtschaftspolitik, Arbeiterkammer Wien, Österreich, 12.05.2016
\end{enumerate}
\paragraph{Eingeladener Vortrag an Universität}
\begin{enumerate}
	\item Pühringer S.: Krisen als Krankheiten und Katastrophen: Zur diskursiven Wirkmächtigkeit ökonomischen Denkens. New Design University St. PöltenLectures and Debates zur \glqq Sprache des Geldes\grqq{}, St. Pölten, Österreich, 06.04.2016
	\item Pühringer S.: Ökonomisches Denken in der Krise?. Ksoe und Institut für Institutionelle und Heterodoxe Ökonomie, WU WienWirtschaft, menschengerecht gedacht?, WU Wien, Österreich, 23.05.2016
	\item Pühringer S.: Diskursive und politische Wirkmächtigkeit ökonomischen Denkens. Universität WienRingvorlesung „Theorie, Modell, Wirklichkeit“ an der Universität Wien, Universität Wien, Österreich, 21.06.2016
	\item Pühringer S.: The role of democracy in the history of liberal (economic) thought. Summerschool Alternative Economic and Monetary Systems, BOKU Wien, Österreich, 01.08.2016
	\item Pühringer S.: Workshop Verteilung und Gerechtigkeit: Ökonomische und politphilosophische Perspektiven. Cusanus Hochschule BernkastelHerbstakademie der Cusanus Hochschule, Cusanus Hochschule, Deutschland, 07.09.2016
	\item Pühringer S.: Pluralismus in der Ökonomie? Wissenschaftstheoretische Grundlagen und wissenschaftspolitische Konsequenzen. Universität TübingenRingvorlesung \glqq Rethinking Economics\grqq{}, Tübingen, Österreich, 26.10.2016
	\item Kapeller J.: The social philosophy of globalized markets. , Universität Graz, Österreich, 07.04.2016
	\item Kapeller J.: Spezifische Aspekte ökonomischen Modelldenkens. , Universität Wien, Österreich, 26.04.2016
	\item Kapeller J.: Die Spaltung als Modell: Europas Zerfall als Krise der Wirtschaftstheorie. , Leipzig, DE, Deutschland, 10.11.2016
	\item Kapeller J.: Was ist heterodoxe Ökonomie?. , Universität Innsbruck, Österreich, 15.12.2016
	\item Gräbner-Radkowitsch C.: Komplexitätsökonomik. Ringvorlesung \glqq Denkschulen und aktuelle Kontroversen der Ökonomik\grqq{}, Freie Universität Berlin, Deutschland, 09.02.2016
	\item Gräbner-Radkowitsch C.: Eine Einführung in die Institutionenökonomik(en). Ringvorlesung \glqq Denkschulen und aktuelle Kontroversen der Ökonomik\grqq{}, Freie Universität Berlin, Deutschland, 21.11.2016
	\item Gräbner-Radkowitsch C.: The complexity challenge: Systemist thinking, computational modeling, and pluralism in economics. Verleihung des WIWA Nachwuchspreises für Plurale Ökonomik, Universität Witten-Herdecke, Deutschland, 12.12.2016
	\item Grimm C.: Pluralismus in der Ökonomik? Zur Frage von Machtverhältnissen und Homogenität in der deutschsprachigen Ökonomik.. Kritische Wirtschaftswissenschaften Göttingen, Göttingen, Deutschland, 07.11.2016
\end{enumerate}
\paragraph{Hauptvortrag / Eingeladener Vortrag auf einer Tagung}
\begin{enumerate}
	\item Pühringer S.: Think Tank networks of German Neoliberalism. WZB BerlinMore Roads from Mont Pèlerin. Neoliberalism Studies, Berlin, Deutschland, 20.03.2016
	\item Kapeller J.: Wissenschaftstheorie und Ökonomie. Inst. für Makroökonomie und  Konjunkturforschung (IMK) in der Hans Böckler-Striftung2. IMK-Workshop \glqq Plurale Ökonomik\grqq{}, Inst. für Makroökonomie und Konjunkturforschung (IMK) in der Hans Böckler-Stiftung, Berlin, Deutschland, 13.08.2016
	\item Kapeller J.: The future of economic teaching. Paneldiskussion im Rahmen der Jahrestagung des Vereins für Sozialpolitik (VfS), Augsburg, DE, Deutschland, 05.09.2016
	\item Kapeller J.: Epistemology of Economics. PreConference-session at the annual conference of the European Association for Evolutionary Political Economy (EAEPE), Manchester, Vereinigtes Königreich, 03.11.2016
	\item Kapeller J.: Selbstverwirklichung vs. Selbstoptimierung: Das ambivalente Erbe der Aufklärung. 5. Internationale Hartheim-Konferenz, Hartheim, Österreich, 18.11.2016
	\item Kapeller J.: Citation metrics and the development of economics. Governing Economics: Institutional Changes, New Frontiers and the State of Pluralism, Amiens, FR, Frankreich, 07.12.2016
	\item Gräbner-Radkowitsch C.: Netzwerke und Komplexität in der Ökonomik. Ringvorlesung \glqq Alternative Mikroökonomie\grqq{}, Wirtschaftsuniversität Wien, Österreich, 10.05.2016
	\item Gräbner-Radkowitsch C.: Computations, Mechanisms, and Socio-Ecological Systems: A meta-theoretical appraisal of ABM. Prof. Dr. Sylvie GeisendorfAgent-based modeling in Ecological Economics -- A useful tool or just a fancy gadget?, ESCP Europe Berlin, Deutschland, 20.05.2016
\end{enumerate}
\paragraph{Eingeladener Vortrag an anderen Institutionen}
\begin{enumerate}
	\item Pühringer S.: Workshop zu: Philosophien der Verteilungsgerechtigkeit. Kreuzschwesternschule Linz, Linz, Österreich, 07.01.2016
	\item Pühringer S.: Vortrag und Workshop zur Finanzkrise und deren wirtschaftspolitischen Folgen. HAK Traun, Traun, Österreich, 01.03.2016
	\item Pühringer S.: Wie denken zukünftige Ökonom\_innen?. FGW -- Forschungsstelle für gesellschaftliche Weiterentwicklung, Düsseldorf, Deutschland, 27.09.2016
	\item Pühringer S.: Netzwerke, Paradigmen, Attitüden -- Der deutsche Sonderweg im Fokus. FGW -- Forschungsstelle für gesellschaftliche WeiterentwicklungFGW-Vernetzung und Projektpräsentation, Düsseldorf, Deutschland, 27.09.2016
	\item Kapeller J.: Humanistische Handelspolitik und inklusives Wachstum. , Bertelsmann-Stiftung, Berlin, Deutschland, 26.01.2016
	\item Kapeller J.: Geschichte und Praxis der Austeritätspolitik. , Friedrich-Ebert-Stiftung, Wiesbaden, Deutschland, 05.02.2016
	\item Kapeller J.: Wie liberal ist der Neoliberalismus?. , Hans Böckler Stiftung, Wien, Österreich, 23.02.2016
	\item Kapeller J.: Der Kapitalismus und die schöne neue Welt: Eine ambivalente Beziehung. Jahrestagung der Sozialplattform OÖ, Linz, Österreich, 11.10.2016
	\item Aigner E., Pühringer S.: Netzwerke, Paradigmen, Attitüden -- Der deutsche Sonderweg im Fokus. FGW -- Forschungsstelle für gesellschaftliche WeiterentwicklungFGW-Vernetzung und Projektpräsentation, Düsseldorf, Deutschland, 27.09.2016
\end{enumerate}
\paragraph{Vortrag auf einer Tagung (nicht referiert)}
\begin{enumerate}
	\item Hirte K.: Entstehen und Vergehen der deutschen Heterodoxen ab den 1970ern. DFG-Netzwerk „Soziologie des ökonomischen Denkens“Tagung des DFG-Netzwerkes \glqq Soziologie des ökonomischen Denkens\grqq{}, Universität Frankfurt am Main, Universität Frankfurt am Main, Deutschland, 11.02.2016
	\item Heimberger P.: Die Macht ökonomischer Modelle am Beispiel des \glqq Potential-Output\grqq{}-Modells der Europäischen Kommission. Momentum -- Macht, Österreich, 14.10.2016
	\item Grimm C.: Performativität, Machtverhältnisse und Ökonomik. Zur Frage der Homogenität in der deutschsprachigen Ökonomik. Momentum 2016: Macht, Hallstatt, Österreich, 14.10.2016
	\item Grimm C.: Profil der deutschsprachigen Volkswirtschaftslehre. Paradigmatische Ausrichtung und politische Wirkmächtigkeit deutschsprachiger Ökonom\_innen. FGW Jahrestagung, Düsseldorf, Deutschland, 05.11.2016
\end{enumerate}
\subsection*{2015}
\paragraph{Vortrag auf einer Tagung (nicht referiert)}
\begin{enumerate}
	\item Pühringer S.: Philosophien der Verteilungsgerechtigkeit. 6. Sommerakademie des ICAE: Kapitalismus und Gerechtigkeit -- Die Rolle der Ungleichheit im 21. Jahrhundert, Linz, Österreich, 20.06.2015
	\item Kapeller J.: Piketty verstehen. 6. Sommerakademie des ICAE, Universität Linz, Österreich, 20.06.2015
	\item Heimberger P.: Did Fiscal Consolidation Cause the Double Dip Recession in the Euro Area?. The Spectre of Stagnation? Europe in the World Economy, Berlin, Deutschland, 23.10.2015
	\item Heimberger P.: Austeritätspolitik in Zeiten der Eurokrise: Wachstumseffekte der fiskalischen Konsolidierungsmaßnahmen 2011-2013. Ökonomie! Welche Ökonomie? Zu Stand und Status der Wirtschaftswissenschaft, Linz, Österreich, 05.12.2015
	\item Beyer K., Bräutigam L.: Die Rolle der Politik in der Entwicklung des Schattenbankensektors. Ökonomie! Welche Ökonomie? Zu Stand und Status der Wirtschaftswissenschaft, Österreich, 04.12.2015
\end{enumerate}
\paragraph{Vortrag auf einer Tagung (referiert)}
\begin{enumerate}
	\item Pühringer S.: Markets as „ultimate judges“ of economic policies. Angela Merkel´s interpretation of the economic crisis and her conclusions for European crisis policies. 1st Vienna Conference on Pluralism in Economics, Österreich, 11.04.2015
	\item Pühringer S.: The narrow concept of “economic normality” in economists public crisis discourse. 10th Conference of Interpretive Policy Analysis., Lille, Frankreich, 08.07.2015
	\item Pühringer S.: Krise und ökonomische Politikberatung. 7. Wintertagung des ICAE: Ökonomie! Welche Ökonomie? Zu Stand und Status der Wirtschaftswissenschaft, Linz, Österreich, 05.12.2015
	\item Kapeller J.: Beyond Foundations: Systemism in Economic Thinking. ASSA Converence 2015, Boston, Vereinigte Staaten, 03.01.2015
	\item Kapeller J.: Two perspectives on citation metrics in economics. Conference of the Institute for New Economic Thinking (INET) "Egalite, Liberte, Fragilite, Paris, Frankreich, 12.04.2015
	\item Hirte K.: Ökonom*innen und Politik. Analyse zur politischen Einflußnahme deutschsprachiger Ökonom*innen und Ökonomie.. , Workshop Netzwerk „Soziologie der Ökonomie“, Universität Freiburg, Deutschland, 16.07.2015
	\item Hirte K.: Agrarsoziologie ohne Aufgaben?  Zum vergangenen und zukünftigem Aufgaben-verständnis in der deutschsprachigen Agrarsoziologie. , Soziologiekongress 2015, Universität Innsbruck, Österreich, 01.10.2015
	\item Hirte K.: Politik und ihre Ad-hoc- Gremien in Krisenzeiten: Public Management zwischen Demokratie und Wirkungsorientierung?. , Momentum 2015: Kritik, Track 2: Public Management, Hallstatt, Österreich, 23.10.2015
	\item Hirte K.: Exklusionen in der Ökonomie? Die Situation der Heterodoxen in Deutschland und ihre Ursachen. , Konferenz „Teaching Economics in the 21st Century, HWR Berlin, Deutschland, 27.11.2015
	\item Hirte K.: „Landnahme“. Zur Ökonomie Rosa Luxemburgs und dem Defizit-Problem in der Ökonomie.. , Wintertagung ICAE, Universität Linz, Österreich, 05.12.2015
	\item Gräbner-Radkowitsch C., Heinrich T.: Beyond Equilibrium: Revisting Two-Sided Markets from an Agent-Based Perspective. Annual Conference of the European Association for Evolutionary Political Economy (EAEPE), University of Genoa, Italien, Italien, 19.08.2015
	\item Gräbner-Radkowitsch C.: Agent-Based Computational Models: A Useful Heuristic for Institutional Pattern Modeling?. Allied Social Sciences Associations (ASSA) Annual Meeting, Boston, MA, Vereinigte Staaten, 03.01.2015
	\item Gräbner-Radkowitsch C.: The Nature of Institutions: A Computational Perspective. Allied Social Sciences Associations (ASSA) Annual Meeting, San Francisco, CA, Italien, 18.09.2015
	\item Gräbner-Radkowitsch C.: Eine algorithmische Perspektive auf die Evolution von Institutionen. Buchenbachworkshop, Buchenbach, DE, Deutschland, 01.10.2015
	\item Grimm C.: Wirtschaftspolitische Ausrichtungen in österreichischen Parteiprogrammen. 7. Wintertagung des ICAE: Ökonomie! Welche Ökonomie? Zu Stand und Status der Wirtschaftswissenschaft, Wissensturm, Linz, Österreich, 05.12.2015
	\item Griesser M., Pühringer S.: Economics return to a dismal science? The changing role of economic thought in German labour market reforms from the 1960s to the 2000s. Inaugural Conference on Cultural Political Economy, Lancaster University, Vereinigtes Königreich, 02.09.2015
	\item Griesser M., Pühringer S.: Economics return to a dismal science? A CPE approach to the role of economic thought in German labour market reforms from the 1960s to the 2000s. Inaugural Conference of Cultural Political Economy, Lancaster University, Vereinigtes Königreich, 02.09.2015
	\item Griesser M.: Funktionen, Policies, Diskurse. Ein Beitrag zur Erklärung (sozial-)staatlicher Transformation. Tag der Politikwissenschaft 2015, Universität Salzburg, Österreich, 28.11.2015
	\item Flechtner S., Gräbner-Radkowitsch C.: It’s not only about inequality: the role of unmet aspirations and speculative capital for explaining social protest. Annual Conference of the European Association for Evolutionary Political Economy (EAEPE), University of Genoa, Italien, Italien, 18.09.2015
\end{enumerate}
\paragraph{Eingeladener Vortrag an anderen Institutionen}
\begin{enumerate}
	\item Pühringer S.: Workshop zu: Philosophien der Verteilungsgerechtigkeit. HAK Traun, Österreich, 17.11.2015
	\item Kapeller J.: Krise der Wirtschaftswissenschaften -- Chancen für eine plurale Ökonomik?. Friedrich Ebert Stiftung (FES), Berlin, Deutschland, 23.02.2015
	\item Kapeller J.: The crisis as a game changer? Der Einfluss der Krise auf die Entwicklung ökonomischer Forschung [The crisis and the development of economic research]. Conference on the OECD's \glqq New Approaches to Economic Challenges\grqq{}-Program, Bundeskanzleramt Wien, Österreich, 22.04.2015
	\item Kapeller J.: Some implications of traditional philosophy for current economics. UN-ECLAC, Santiago de Chile, Chile, 03.09.2015
	\item Hirte K., Thieme S.: Zum performativen Fußabdruck der ÖkonomInnen 1955-1992. 2te pluralistische Ergänzungsveranstaltung des Netzwerkes Real World Economics2te pluralistische Ergänzungsveranstaltung des Netzwerkes Real World Economics, Münster, Deutschland, 07.09.2015
	\item Hirte K.: Zum Entstehen und „Vergehen“ der deutschen universitären Agrarsoziologie. 79. Tagung AG Ländliche Sozialforschung Österreich, 79. Tagung AG Ländliche Sozialforschung Österreich, Österreich, 20.03.2015
	\item Griesser M.: MigrantInnen als Zielgruppe?. Studienpräsentation zu solidarischen Beratungs- und Unterstützungsangeboten von ArbeitnehmerInnenorganisationen in Österreich, Bildungszentrum der AK Wien, Österreich, 20.05.2015
\end{enumerate}
\paragraph{Andere Vorträge und Präsentationen}
\begin{enumerate}
	\item Kapeller J., Landesmann M., Mohr F., Schütz B.: Government policies and financial crises: On the mitigation, postponement and prevention of crisis within a Minsky-Veblen cycle. MitarbeiterInnenseminar des Insituts für VWL der JKU, St. Agatha, Österreich, 28.11.2015
	\item Kapeller J.: Verteilungstendenzen im Kapitalismus -- Nationale und globale Perspektiven. BEIGEWUMPräsentation Kurswechsel 2/15: Vermögensungleichheit, Kapitalismus und Demokratie, WU Wien, Österreich, 03.11.2015
	\item Hirte K.: Moderation zu: Wirtschaftspolitische Herausforderungen einer starken Vermögenskonzentration. Wissensturm LinzVortragsreihe \glqq Eigentum\grqq{}, Linz, Österreich, 22.04.2015
	\item Hirte K.: Zur Finanzkrise und ihren forschungsseitigen Herausforderungen. HAK Ried, Österreich, 01.07.2015
\end{enumerate}
\paragraph{Eingeladener Vortrag an Universität}
\begin{enumerate}
	\item Kapeller J.: Pluralism in Economics: Perspectives from Philosophy of Science. University of Maastricht, Maastricht, NL, Niederlande, 05.02.2015
	\item Kapeller J.: Die Rolle pluraler Ökonomik in der ökonomischen Bildung. , Universität Tübingen, DE, Deutschland, 05.05.2015
	\item Kapeller J.: Pluralismus in der Ökonomie: Eine wissenschaftstheoretische Perspektive. , Universität Erfurt, DE, Deutschland, 01.06.2015
	\item Kapeller J.: Pluralismus in der Ökonomie: Eine wissenschaftstheoretische Perspektive. , Universität Halle, DE, Deutschland, 09.06.2015
	\item Kapeller J.: Pluralism in Economics. , Santiago de Chile, Universidad de Chile, Chile, 02.09.2015
	\item Kapeller J.: Pluralismus in der Ökonomie: Eine wissenschaftstheoretische Perspektive. , Universität Tübingen, DE, Deutschland, 04.11.2015
	\item Hirte K.: Denkschulenentwicklungen in der deutschen VWL nach 1945. FB Wirtschaftswissenschaften der Universität Frankfurt am MainKolloquium Dogmenhistorie, Frankfurt am Main, Deutschland, 24.04.2015
	\item Hirte K.: Europäische Agrarpolitik. , Ringvorlesung Ernährungssouveränität. BOKU Wien, Österreich, 22.06.2015
	\item Gräbner-Radkowitsch C.: Pluralismus, Komplexität und Mikroökonomik. , Universität Erfurt, DE, Deutschland, 13.01.2015
\end{enumerate}
\paragraph{Hauptvortrag / Eingeladener Vortrag auf einer Tagung}
\begin{enumerate}
	\item Kapeller J.: Ökonomische Forschung und die Finanzkrise. 7. Wintertagung des ICAE: Ökonomie! Welche Ökonomie? Zu Stand und Status der Wirtschaftswissenschaft, Wissensturm, Linz, Österreich, 05.12.2015
	\item Hirte K.: Der Einfluss europäischer Agrarpolitikmaßnahmen auf die Arbeit im Agrarsektor. , 23. Witzenhäuser Konferenz „Frohes Schaffen? – Arbeit in der Landwirtschaft“, Universität Kassel, Deutschland, 01.12.2015
	\item Aistleitner M.: Die Dominanz des ökonomischen Mainstreams aus szientometrischer Perspektive: Befunde zum \glqq Matthäus-Effekt\grqq{} in den Wirtschaftswissenschaften. 5. Studentischer Soziologiekongress, Tübingen, Deutschland, 02.10.2015
	\item Aistleitner M.: Die Macht der Wissenschaftsstatistik und die Entwicklung der Ökonomie. Momentum 15: Kritik, Hallstatt, Kongresszentrum, Österreich, 23.10.2015
	\item Aistleitner M.: Der Einfluss der Wissenschaftsstatistik auf den ökonomischen Diskurs. 7. Wintertagung des ICAE: Ökonomie! Welche Ökonomie? Zu Stand und Status der Wirtschaftswissenschaft, Wissensturm Linz, Österreich, 05.12.2015
\end{enumerate}
\subsection*{2014}
\paragraph{Vortrag auf einer Tagung (nicht referiert)}
\begin{enumerate}
	\item Ötsch W.: Moderation einer Diskussion zur Sozialen Markwirtschaft. Institut für die Gesamtanalyse der WirtschaftMarkt! Welcher Markt? Der interdisziplinäre Diskurs um Märkte und Marktwirtschaft, Linz, Österreich, 11.12.2014
	\item Ötsch W.: “Der” Markt: Genese und Wirkungsweise eines vieldeutigen Begriffs. Institut für die Gesamtanalyse der WirtschaftMarkt! Welcher Markt? Der interdisziplinäre Diskurs um Märkte und Marktwirtschaft, Linz, Österreich, 12.12.2014
	\item Pühringer S.: „Der Markt als Richter“. Marktdisziplin und Austeritätspolitik. Institut für die Gesamtanalyse der WirtschaftMarkt! Welcher Markt? Der interdisziplinäre Diskurs um Märkte und Marktwirtschaft, Linz, Österreich, 12.12.2014
	\item Griesser M.: Gewerkschaftliche Erneuerung durch Bewegungen der Migration?. Workshop „(Kritische) Gewerkschaftsforschung in Österreich“, Universität Wien, Österreich, 29.11.2014
	\item Bräutigam L.: Financial Markets: A Critical Approach to their Functions and Impact on Economy.. EAEPE (European Association of Evolutionary Political Economy)Unemployment and Austerity in Mediterranean Countries, Nikosia und Aya Napa, Zypern, 19.11.2014
	\item Bräutigam L.: Markt, Marktakteure und die Rolle des Staates im Wirtschaftsleben. Institut für die Gesamtanalyse der WirtschaftMarkt! Welcher Markt? Der interdisziplinäre Diskurs um Märkte und Marktwirtschaft, Linz, Österreich, 12.12.2014
	\item Beyer K.: Emanzipation bei Marx und seine Kritik an Proudhon und dessen ideengeschichtlichen Nachfahren. Momentum-Kongress 2014, Hallstatt, Österreich, 18.10.2014
\end{enumerate}
\paragraph{Eingeladener Vortrag an Universität}
\begin{enumerate}
	\item Ötsch W.: Mechanismen der Finanzmärkte, Globalisierung und Wirtschaftskrise. Masterlehrgang Management und Leadership für Frauen, Linz, Österreich, 17.04.2014
	\item Ötsch W.: Die unsichtbare Hand des Marktes: Wahrheit oder Glaubenslehre. Fachhochschule Wieselburg, Wieselburg, Deutschland, 06.05.2014
	\item Ötsch W.: Vom mittelalterlichen Dienst an Gott zum neoklassischen Produktionsfaktor. Ein Überblick über die Kulturgeschichte von Arbeit und Zeit. Wirtschaftsuniversität Wien, Österreich, 07.05.2014
	\item Ötsch W.: Wie Bilder und Metaphern Wirtschaft beeinflussen: mit Beispielen vom 18. Jahrhundert bis zum Neoliberalismus. Wirtschaftsuniversität Wien, Wien, Österreich, 23.10.2014
	\item Ötsch W.: Podiumsdiskussion über das Programm von \glqq Agenda Austria\grqq{}. Johannes Kepler Universität Linz, Linz, Österreich, 19.11.2014
	\item Ötsch W.: Keynotespeaker zum Thema Mythos Markt – Die Wichtigkeit von Narrativen und von diskursiven Netzwerken. Wirtschaftsuniversität Wien, Wien, Österreich, 21.11.2014
	\item Ötsch W.: The Neoliberal Utopia as Negation of All Other Utopias. Institut für Systematische Theologie der Katholisch-Theologischen FakultätUnterwegs zu einer neuen Zivilisation geteilter Genügsamkeit. Perspektiven utopischen Denkens 25 Jahre nach dem Tod Ignacio Ellacurías, Universität Wien, Österreich, 05.12.2014
	\item Kapeller J.: Pluralismus in der Ökonomie: Eine wissenschaftstheoretische Perspektive. , Universität Graz, Österreich, 21.10.2014
	\item Kapeller J.: Pluralismus in der Ökonomie: Eine wissenschaftstheoretische Perspektive. , Universität Hamburg, DE, Deutschland, 23.10.2014
	\item Kapeller J.: Relevanz, Realität, Modelle. Was braucht die Ökonomie?. Wirtschaftsuniversität Wien, Wien, Österreich, 29.10.2014
	\item Kapeller J.: Pluralismus in der Ökonomie: Eine wissenschaftstheoretische Perspektive. , Universität Göttingen, DE, Deutschland, 05.11.2014
	\item Kapeller J.: Die Rückkehr des Rentiers. Ansichten zu Thomas Pikettys ‘Capital in the 21st century’. Johannes Kepler Universität Linz, Linz, Österreich, 12.11.2014
	\item Kapeller J.: Ranglisten und andere Evaluationskriterien – die Macht der Wissenschaftsstatistik. , Universität Hamburg, DE, Deutschland, 18.12.2014
	\item Hirte K.: Kritische Betrachtung der Entwicklung der EU-Agrarpolitik. Universität Kassel, Ringvorlesung Ernährungssouveränität, FB Ökologische Agrarwissenschaften, Deutschland, 12.11.2014
	\item Hirte K.: Europäische Agrarpolitik. Institutionen, Innovationen, Interpretationen.. Universität Kassel, Kassel, Deutschland, 12.11.2014
	\item Hirte K.: ÖkonomInnen und Ökonomie – Theorie, Erhebungen, Zugänge. Universität KasselWorkshop „Zur Pluralität der wissenschaftlichen Lehre in Deutschland“, Kassel, Deutschland, 01.12.2014
\end{enumerate}
\paragraph{Eingeladener Vortrag an anderen Institutionen}
\begin{enumerate}
	\item Ötsch W.: Demokratie und Finanzkapitalismus. Vortragsreihe Demokratie von ATTAC Wels, Wels, Österreich, 04.02.2014
	\item Ötsch W.: Der Papst als Kritiker des Wirtschaftssystems. Vorstand der Katholischen Arbeitnehmerbewegung und der Betriebsseelsorge Oberösterreich, Linz, Österreich, 29.03.2014
	\item Ötsch W.: Ökonomische Theorie und öffentliche Meinung. Kueser-­‐Akademie Alfter bei Bonn, Alfter bei Bonn, Deutschland, 09.07.2014
	\item Pühringer S.: Podiumsdiskussion zum Film Too big to Tell von Johanna Tschautscher. Voest Alpine, Linz, Österreich, 09.07.2014
	\item Nordmann J.: Die Funktionslogik von Think-Tanks in neoliberalen Machtapparaten. , Wissenschaftszentrum Berlin, Deutschland, 04.02.2014
	\item Kapeller J.: Die Verteilung von Einkommen und Vermögen in Europa und International. , Arbeiterkammer Oberösterreich Linz, Österreich, 10.09.2014
	\item Kapeller J.: Wirtschaftspolitik, Verteilungsgerechtigkeit und Demokratie. Arbeiterkammer BurgenlandJahreshauptversammlung der Arbeiterkammer Burgenland, Österreich, 07.11.2014
	\item Hirte K.: Europäische Agrarpolitik – eine kritische Sicht. Nyeleni-Forum 2014, Salzburg, Österreich, 14.04.2014
	\item Hirte K.: ÖkonomInnen in der Finanzkrise -- Wie performativ ist Wissenschaft. AK Politische Ökonomie und WEA, Frankfurt am Main, Deutschland, 19.10.2014
	\item Hirte K.: Internationaler Freihandel auf Basis komparativer Kostenvorteile? -- zu Theorie und Praxis eines Versprechens. Arbeiterkammer LinzWorkshop Wirtschaftstheorie und Politik: Freihandel, Linz, Österreich, 24.11.2014
	\item Elsner W., Gräbner-Radkowitsch C., Heinrich T.: Workshop: Einführung in die Komplexitätsökonomik. IMK Berlin, HBS BerlinPluralismus in der Ökonomik, IMK Berlin, Deutschland, 09.08.2014
\end{enumerate}
\paragraph{Hauptvortrag / Eingeladener Vortrag auf einer Tagung}
\begin{enumerate}
	\item Pühringer S.: Vom „Über-die-Verhältnisse-leben“ in der Krise. AK WienWirtschaftspolitik in der EU – das Scheitern des neoklassischen Paradigmas, Österreich, 02.06.2014
	\item Kapeller J.: Wissenschaftstheorie und Ökonomie. , Inst. für Makroökonomie und Konjunkturforschung (IMK) in der Hans Böckler-Stiftung, Berlin, Deutschland, 08.08.2014
	\item Kapeller J.: Wie können alternative ökonomische Forschungsansätze an Universitäten gefördert werden?\glqq . IMK-Workshop "Pluralismus in der Ökonomie\grqq{}, Inst. für Makroökonomie und Konjunkturforschung (IMK) in der Hans Böckler-Striftung, Berlin, Deutschland, 10.08.2014
	\item Kapeller J.: Inequality before and after the Crisis. European Forum Alpbach, Alpbach, Österreich, 27.08.2014
\end{enumerate}
\paragraph{Vortrag auf einer Tagung (referiert)}
\begin{enumerate}
	\item Kapeller J., Schütz B., Tamesberger D.: From free to civilised markets. First steps towards Eutopia. WINIR conference \glqq Institutions that change the world\grqq{}, Old Royal Naval College, London, Vereinigtes Königreich, 12.09.2014
	\item Kapeller J., Schütz B.: From Free to Civilized Markets: First Steps towards Eutopia. Progressive Economy Forum, Europäisches Parlament Brüssel, Belgien, 06.03.2014
	\item Hirte K.: Zur Performativität der Ökonomik und aktuellen ÖkonomInnen-Netzwerken in Deutschland, Österreich, 17.07.2014
	\item Hirte K.: Märkte und die Anerkennung von Arbeit – schlechte und ungleiche Bezahlung als nur eine geschlechterspezifische Frage? Ein erkenntnistheoretischer Zugang zur Frage der schlechten Bezahlung bestimmter Tätigkeiten. Institut für die Gesamtanalyse der WirtschaftMarkt! Welcher Markt? Der interdisziplinäre Diskurs um Märkte und Marktwirtschaft, Linz, Österreich, 12.12.2014
	\item Gräbner-Radkowitsch C.: About Implicit Assumptions in Formal Modelling – The Case of DSGE and ABM in Economics. Annual Conference of the European Association for Evolutionary Political Economy (EAEPE), University of Cyprus, Nicosia, Zypern, Zypern, 07.11.2014
	\item Bräutigam L., Pühringer S.: Crisis resistance of inequality. The Future of Capitalism Development, Un(der)employment and Inequality, Wien, Österreich, 24.09.2014
\end{enumerate}
\paragraph{Andere Vorträge und Präsentationen}
\begin{enumerate}
	\item Beyer K.: Emanzipation bei Marx und seine Kritik an Proudhon.. Momentum 14 -- Ökonomik und Emanzipation, Hallstadt, Österreich, 18.10.2014
\end{enumerate}
\subsection*{2013}
\paragraph{Vortrag auf einer Tagung (nicht referiert)}
\begin{enumerate}
	\item Ötsch W.: Einführung in das Schattenbankensystem. Workshop Schattenbanken, Arbeiterkammer Oberösterreich Linz, Österreich, 01.03.2013
	\item Ötsch W.: Finanzkrise und Staatschuldenkrise. Bezirkskonferenz des SPÖ-Pensionistenverbandes, Eferding, Österreich, 20.03.2013
	\item Ötsch W.: Finanzkrise und Staatschuldenkrise. Bezirkskonferenz des SPÖ-Pensionistenverbandes, Mattighofen, Österreich, 27.03.2013
	\item Ötsch W.: Finanzkapitalismus: was ist das? Strukturen eines Machtsystems. 9. Internationales Open Space „Geld oder Leben – Symposium für eine lebensfreudige Finanzwirtschaft“, Schrems, Österreich, 18.05.2013
	\item Ötsch W.: Die Finanzkrise 2007-2009 als Krise von Schattenbanken. Eine institutionelle Analyse. Jahrestagung des Ausschusses Evolutorische Ökonomik im Verein für Socialpolitik, Delft, Niederlande, 06.07.2013
	\item Ötsch W.: Populismus und Demagogie. Inhalte und Muster eines gefährlichen Denkens, mit Beispielen bis zur Tea Party und Frank Stronach. Arbeiterkammer Kärnten, der Fachhochschule und Pädagogischen Hochschule KärntenAlltagsrassismus, Klagenfurt, Österreich, 24.10.2013
	\item Ötsch W.: Eigene Bilder von Identität und Zugehörigkeit erfahren und verändern. Arbeiterkammer Kärnten, der Fachhochschule und Pädagogischen Hochschule KärntenAlltagsrassismus, Klagenfurt, Österreich, 24.10.2013
	\item Plaimer W.: Bürgerbeteiligungsmodelle. Bezirksparteiausschuss der SPÖ, Steyr, Österreich, 15.04.2013
	\item Nordmann J.: Podiumsdiskussion: Zur Lage der Demokratie. Demokratiewerkstatt, Linz, Österreich, 24.01.2013
	\item Beyer K., Bräutigam L.: Vortrag CDOs -- A Critical Phenomenon of the Financial System in the Crisis. Arbeiterkammer WienThe Economy in Crisis and the Crisis in Economics, Wien, Österreich, 10.09.2013
\end{enumerate}
\paragraph{Eingeladener Vortrag an Universität}
\begin{enumerate}
	\item Ötsch W.: Ökonomik als soziale Physik – Von der Moralwissenschaft zum Naturalismus. Universität Bayreuth, Bayreuth, Deutschland, 29.10.2013
	\item Pühringer S.: Ökonomie und ÖkonomInnen -- Interdependenz von Wirtschaftstheorie und ihrer Praxis. Universität Bayreuth, VWL IG — Professur für International GovernanceRingvorlesung Ökonomik zwischen Modell und Wirklichkeit, Uni Bayreuth, Deutschland, 26.11.2013
	\item Nordmann J.: Allegorien des Lesens. Paul de Man im Kontext. Universität Flensburg, Flensburg, Deutschland, 21.06.2013
	\item Hirte K.: Europäische Agrarpolitik. Ernährungssouveränität, BOKU Wien, Österreich, 12.03.2013
\end{enumerate}
\paragraph{Eingeladener Vortrag an anderen Institutionen}
\begin{enumerate}
	\item Ötsch W.: Bankenrettungen. QuerdenkRe(ihe) der Bürogemeinschaft E20, Salzburg, Österreich, 10.02.2013
	\item Pühringer S., Ötsch W.: Was ist eigentlich liberal?. Grünen Wirtschaft Oberösterreich, Keplersalon Linz, Linz, Österreich, 06.11.2013
	\item Plaimer W.: Bürgerbeteiligungsmodelle. Bezirksparteiausschuss  der SPÖ, Ried, Österreich, 12.06.2013
	\item Plaimer W.: Möglichkeiten vom Bürgerbeteiligungsmodellen in Mehrheits- und Minderheitsgemeinden. Bezirksparteiausschuss der SPÖ Kirchdorf, Kirchdorf, Österreich, 01.07.2013
	\item Plaimer W.: Möglichkeiten zur Bürgerbeteiligung in Steyr. Stadtteilverband  der SPÖ Steyr, Steyr, Österreich, 28.08.2013
	\item Plaimer W.: Bürgerbeteiligungsmodelle. Ortsparteiausschuss  der SPÖ Vorchdorf, Vorchdorf, Österreich, 29.08.2013
	\item Plaimer W.: Bürgerbeteiligungsmodelle. Ortsparteiausschuss der SPÖ Wolfern, Wolfern, Österreich, 07.10.2013
	\item Plaimer W.: Möglichkeiten vom Bürgerbeteiligungsmodellen in Mehrheits- und Minderheitsgemeinden. Bezirksparteiausschuss der SPÖ Vöcklabruck, Vöcklabruck, Österreich, 28.11.2013
	\item Plaimer W.: Möglichkeiten zur Bürgerbeteiligung in Vorchdorf. Ortspartei  der SPÖ Vorchdorf, Vorchdorf, Österreich, 09.12.2013
	\item Nordmann J.: Was ist Demokratie?. ATTAC Wels, Wels, Österreich, 24.04.2013
	\item Hirte K., Pühringer S.: ÖkonomInnen in der Finanzkrise -- Diskurse, Netzwerke, Initiativen. AK LinzWorkshop der AK Linz, Arbeiterkammer Oberösterreich Linz, Österreich, 05.12.2013
\end{enumerate}
\paragraph{Vortrag auf einer Tagung (referiert)}
\begin{enumerate}
	\item Pühringer S.: The implementation of the European Fiscal Compact in Austria as a post-democratic phenomenon. Prof. Arne Heise1st World Keynes Conference, Izmir, Türkei, 27.06.2013
	\item Pühringer S.: Tsunami, Earthquake, Fever. Economists’ framing of the financial crisis in the public economic discourse. AK WienThe Economy in Crisis and the Crisis of Economics, Austrian Chamber of Labor, Vienna, Österreich, 10.09.2013
	\item Pühringer S.: Aus den Vorhöfen der Macht in die Medien hin zur eigenen Partei. Formen der der Einflussnahme von ÖkonomInnen auf Politik und Wirtschaft im Zuge der Finanz- und Wirtschaftskrisenpolitik.. Österreichische Gesellschaft für Soziologie, JKU Linz, Österreich, 26.09.2013
	\item Plaimer W., Pühringer S.: Die Implementierung des Fiskalpakts als postdemokratisches Phänomen. Arbeitskreis Internationale Politische ÖkonomieDominanz der Wirtschaft, Wiedererwachen der Politik, internationale Politische Ökonomie?, Deutschland, 22.03.2013
	\item Plaimer W., Pühringer S.: Der Fiskalpakt und seine Implementation in Österreich. Momentum 2013: Fortschritt, Hallstatt, Kongresszentrum, Österreich, 18.10.2013
	\item Hirte K., Pühringer S.: Zur Performativität der Ökonomik als Wissenschaft in Verantwortung. ICAE, Uni Linz; ZÖSS, Uni HamburgWissen! Welches Wissen? Wahrheit, Theorie und Glauben in der ökonomischen Theorie, Wissensturm Linz, Österreich, 14.12.2013
	\item Hirte K.: Das Problem der Anerkennung von Arbeit. Workshop Feministische ÖkonomInnen, Wien, Österreich, 18.01.2013
	\item Gräbner-Radkowitsch C.: Simulationen in der Evolutorischen Ökonomik. Buchenbachworkshop, Buchenbach, DE, Deutschland, 03.10.2013
	\item Gräbner-Radkowitsch C.: Formal Foundations of Agent-Based Models and Simulations. Annual Conference of the European Association for Evolutionary Political Economy (EAEPE), University of Paris 13, Paris, Frankreich, Frankreich, 08.11.2013
\end{enumerate}
\paragraph{Hauptvortrag / Eingeladener Vortrag auf einer Tagung}
\begin{enumerate}
	\item Plaimer W.: Bürgerbeteiligungsmodelle. Gemeindevertreterverband der SPÖ, Braunau, Österreich, 10.06.2013
\end{enumerate}
\paragraph{Andere Vorträge und Präsentationen}
\begin{enumerate}
	\item Beyer K.: Workshop zum Schattenbankensystem. ATTAC Sommer-Akademie „Reset Finance! Wege zu einem gesellschaftlich kontrollierten Finanz- und Bankensystem“,, Eisenstadt, Österreich, 20.07.2013
	\item Beyer K.: Diskussionsbeitrag zum Vortrag Modellierungskulturen in der Ökonomik: Vom Disziplinierungsinstrument zum Treiber von Theoriepluralismus?. ICAE-Wintertagung: Wissen! Welches Wissen, Linz, Österreich, 13.12.2013
\end{enumerate}
\subsection*{2012}
\paragraph{Andere Vorträge und Präsentationen}
\begin{enumerate}
	\item Ötsch W.: Diskussion zum Film Inside Job. Arbeiterkammer LinzDokumentarfilmreihe Die Politik in der Krise, Linz, Österreich, 12.09.2012
	\item Plaimer W.: Mehr Beteiligung durch direkte Demokratie?. Renner Institut Oberösterreich, Österreich, 03.12.2012
	\item Hirte K., Nordmann J.: Kommentar zum Vortrag \glqq Rebooting Is Not An Option: Toward Equitable Social and Economic Development\grqq{} von Stephanie Seguino. , Wissensturm Linz, Österreich, 29.02.2012
	\item Hirte K.: Moderation zu: „’Rebooting’ is not an option: Toward equitable social and economic development” von Stephanie Seguino (University of Vermont/USA), (via skype), Wissensturm Linz.. Vortragsreihe „Ökonomia“ -- Wirtschaft aus feministischer Sicht„Ökonomia“ -- Wirtschaft aus feministischer Sicht, Wissensturm Linz, Österreich, 29.02.2012
	\item Hirte K.: Moderation zum Vortrag \glqq Geld regiert die Welt! Wer regiert das Geld?\grqq{} von Prof. Dr. Margit Kennedy. , Wissensturm Linz, Österreich, 07.03.2012
	\item Hirte K.: Das Konzept einer Gesamtanalyse der Wirtschaft. 1te pluralistische Ergänzungsveranstaltung zur Jahrestagung des Vereins für Socialpolitik1te pluralistische Ergänzungsveranstaltung zur Jahrestagung des Vereins für Socialpolitik, Göttingen, Deutschland, 11.09.2012
\end{enumerate}
\paragraph{Vortrag auf einer Tagung (nicht referiert)}
\begin{enumerate}
	\item Ötsch W.: Krise in Europa. Forschungsplattform Politik -- Religion -- Kunst InnsbruckFrühjahrstagung 2012: Katastrophen, Universität Innsbruck, Österreich, 12.04.2012
	\item Ötsch W.: Ratingagenturen in der neoliberalen Wirtschaft. software-systems.atLänder- und Regionenrating, Österreichische Kontrollbank Wien, Österreich, 26.04.2012
	\item Ötsch W.: Finanzmärkte und Postdemokratie. Katholisches Bildungswerk Wien, Groß-Enzersdorf, Großenzersdorf, Österreich, 27.08.2012
	\item Ötsch W.: Theoriegeschichte als (kritische) Kulturgeschichte. Vereins für Socialpolitik des Netzwerkes Real World Economics1. pluralistischen Ergänzungsveranstaltung zur Jahrestagung, Göttingen, Deutschland, 10.09.2012
	\item Ötsch W.: Hintergründe der Staatsschuldenkrise – wie denken die handelnden Akteure?. Management Club Salzburg, Salzburg, Österreich, 17.09.2012
	\item Ötsch W.: Lenken im Hamsterrad? Politisches Handeln in Zeiten von Finanz-Spekulationen, Schuldenkrisen und allmächtigen Konzernen. Werkstatt-Gespräche Bludenz, Bludenz, Österreich, 25.09.2012
	\item Ötsch W.: Wieviel Sturm verträgt der Euro-Rettungsschirm?. Gesellschaft für Kommunikation, Entwicklung und dialogische Bildung, Salzburg, Österreich, 12.11.2012
	\item Ötsch W.: Eröffnung der 4. Jahrestagung Die Politische Ökonomie von Regulierungsoasen. Institut für die Gesamtanalyse der WirtschaftDie Politische Ökonomie von Regulierungsoasen, Wissensturm Linz, Österreich, 29.11.2012
	\item Ötsch W.: The Political Economy of Offshore Jurisdictions: Its Neoliberal Background. Institut für die Gesamtanayse der WirtschaftDie Politische Ökonomie von Regulierungsoasen, Wissensturm Linz, Österreich, 30.11.2012
	\item Pühringer S.: How liberalism lost its concept of democracy.. Momentum Kongress 2012 im Panel “Recht, Freiheit, Demokratie”., Österreich, 28.09.2012
	\item Pühringer S.: Diskussionsbeitrag zu: Ötsch, Silke: Our banking secrecy is a strong castle. Institut für die Gesamtanalyse der WirtschaftDie Politische Ökonomie von Regulierungsoasen, Wissensturm Linz, Österreich, 30.11.2012
	\item Plaimer W.: Postdemokratie in Österreich. Momentum Kongress 2012 im Panel “Recht, Freiheit, Demokratie”, Österreich, 28.09.2012
	\item Hirte K.: Das Institut für die Gesamtanalyse der Wirtschaft. Vereins für Socialpolitik des Netzwerkes Real World Economics1. pluralistischen Ergänzungsveranstaltung zur Jahrestagung, Göttingen, Deutschland, 11.09.2012
\end{enumerate}
\paragraph{Eingeladener Vortrag an Universität}
\begin{enumerate}
	\item Ötsch W.: Hintergründe zur Staatsschuldenkrise: wie denken die handelnden Akteure?. Österreichischen Hochschülerschaft Linz, Universität Linz, Österreich, 09.05.2012
	\item Ötsch W.: Folgt nach der Finanz- die Demokratiekrise?. Bürgerforum Europa 2020, Universität Linz, Österreich, 15.06.2012
	\item Ötsch W.: Wider die Marktgläubigkeit. Ökonomie neu denken lernen (Publikumsdiskussion). Alanus Hochschule für Kunst und GesellschaftÖkonomisch-philosophischen Herbstakademie 2012: Die Ökonomien des Gemeinsamen. Neue Orte ökonomischer Bildung, Alfter bei Bonn, Deutschland, 18.09.2012
	\item Ötsch W.: Neoliberalismus, ökonomische Theorien und Propaganda. Alanus Hochschule für Kunst und GesellschaftÖkonomisch-philosophischen Herbstakademie 2012: Die Ökonomien des Gemeinsamen. Neue Orte ökonomischer Bildung, Alfter bei Bonn, Deutschland, 18.09.2012
	\item Ötsch W.: Der Verlust der Kategorien von Moral und der Gesellschaft in der Geschichte der Nationalökonomie. Alanus Hochschule für Kunst und GesellschaftÖkonomisch-philosophischen Herbstakademie 2012: Die Ökonomien des Gemeinsamen. Neue Orte ökonomischer Bildung, Alfter bei Bonn, Deutschland, 18.09.2012
	\item Nordmann J.: Theorien der liberalen Gesellschaft. , Universität Flensburg, Deutschland, 15.06.2012
	\item Nordmann J.: Grenzen der Krisendebatte. Zum Verhältnis von Sach- und Grundsatzdiskussionen in den Printmedien und in den Sozialwissenschaften. Sprachliche Konstruktionen sozial- und wirtschaftspolitischer \glqq Krisen\grqq{} in der BRD, Universität Trier, Deutschland, 16.07.2012
	\item Hirte K.: Alte und neue Erklärungsansätze zum Aufstieg und Fall von Eliten. Johannes Kepler Universität LinzLunch Lectures, Johannes Kepler Universität, Österreich, 14.11.2012
\end{enumerate}
\paragraph{Hauptvortrag / Eingeladener Vortrag auf einer Tagung}
\begin{enumerate}
	\item Ötsch W.: Ausgeliefert? Aufgaben und Chancen der Politik angesichts der \glqq Macht der Märkte\grqq{} und einer ausgehöhlten Demokratie. Katholisches Bildungswerk Wien / Katholische Sozialakademie Östereich, Wien, Stephansplatz, Österreich, 15.03.2012
\end{enumerate}
\paragraph{Eingeladener Vortrag an anderen Institutionen}
\begin{enumerate}
	\item Ötsch W.: Staatsschulden: Wer ist schuld, wer zahlt?. , Rathaus Wien, Österreich, 19.03.2012
	\item Ötsch W.: Muss Forschung ökonomisch sein? CityScienceTalk der Langen Nacht der Forschung. , Linz, Österreich, 27.04.2012
	\item Ötsch W.: Ausgeliefert? Aufgaben und Chancen der Politik angesichts der 'Macht der Märkte' und einer ausgehöhlten Demokratie. Armutsnetzwerk Vöcklabruck, Vöcklabruck, Österreich, 31.05.2012
	\item Ötsch W.: Geschichte und Ausprägungen des Neoliberalismus. , Linz, Österreich, 18.06.2012
	\item Nordmann J.: Moderation zum Vortrag "West-End. Das Scheitern der Moderne als kapitalistisches Patriarchat und die Logik der Alternativen von Univ.Prof. Claudia von Werlhof. , Wissensturm Linz, Österreich, 14.03.2012
	\item Nordmann J.: Moderation zum Vortrag \glqq Wem gehört die Welt? Zur Wiederentdeckung der Gemeingüter\grqq{} von Silke Helfrich. , Wissensturm Linz, Österreich, 21.03.2012
	\item Nordmann J.: Moderation zu Werner Rügemer: Ratingagenturen. , Wissensturm Linz, Österreich, 11.05.2012
	\item Nordmann J.: Think-Tanks und die Institutionen der EU. Eine kritische Analyse aus demokratietheoretischer Sicht. IPA-Konferenz, Tilburg, Niederlande, 08.07.2012
\end{enumerate}
\paragraph{Vortrag auf einer Tagung (referiert)}
\begin{enumerate}
	\item Hirte K., Pühringer S.: Economists and Economics -- Discourse Profiles of Exonomists in the Financial Crisis. Joint Conference of AHE, IIPPE and FAPEJoint Conference of AHE, IIPPE and FAPE, Sorbonne Paris, Frankreich, 07.07.2012
\end{enumerate}
\subsection*{2011}
\paragraph{Vortrag auf einer Tagung (nicht referiert)}
\begin{enumerate}
	\item Ötsch W.: Mechanismen der Finanzmärkte, Globalisierung und Wirtschaftskrise. Management und Leadership für Frauen, Österreich, 28.05.2011
	\item Ötsch W.: Reasons behind the actual cris of Euro. World Festival der International Union of Socialist Youth, Weißenbach am Attersee, Österreich, 30.07.2011
	\item Ötsch W.: Gesellschaft bei Marx und im Neoliberalismus: die Gesellschaft \glqq des Marktes". Arbeitskreis Politische Ökonomie zum Thema "Karl Marx 2011\grqq{}, Trier, Deutschland, 14.10.2011
	\item Pühringer S.: Gleichheit versus Vielfalt. Ein konstruierter Widerspruch?. Momentum 4, Track 5: Was ist Gleichheit?, Hallstadt, Österreich, 27.10.2011
	\item Plaimer W.: Postdemokratie in Österreich? Am Beispiel von politischen Entscheidungsprozessen. Demokratie! Welche Demokratie?, Wissensturm Linz, Österreich, 02.12.2011
	\item Nordmann J., Plaimer W.: Veränderung von Machtverhältnissen in politischen Entscheidungsprozessen. Momentum 4, Track 5: Was ist Gleichheit?, Hallstadt, Österreich, 28.10.2011
	\item Nordmann J.: Keine Alternative. Neoliberale Positionen in den Printmedien nach dem Crash. Dreiländerkongress der Deutschen Gesellschaft für Soziologie, der Österreichischen Gesellschaft für Soziologie und der Schweizerischen Gesellschaft für Soziologie, Innbruck, Österreich, 29.09.2011
	\item Nordmann J.: Die neoliberale Oligarchie. Zum aktuellen Verhältnis von Besitz und Macht in der Demokratie. Demokratie! Welche Demokratie?, Wissensturm Linz, Österreich, 02.12.2011
\end{enumerate}
\paragraph{Vortrag auf einer Tagung (referiert)}
\begin{enumerate}
	\item Ötsch W.: Akteur und Markt, Subjekt und Ordnung in der ökonomischen Theorie. , Geisteswissenschaftliches Zentrum Moderner Orient (ZMO), Berlin, Deutschland, 24.02.2011
	\item Hirte K.: Die professorale Scientific Community der deutschen Agrarökonomie und Agrarpolitik vor und nach 1945. Prosopographische Arbeiten. Inhalte -- Methoden -- Erfahrungen -- Desiderata, Herder-Institut, Marburg, Deutschland, 05.05.2011
	\item Hirte K.: Geographische Bezüge in professoralen Netzwerken -- universitäre Lieferbeziehungen in der deutschen Agrarpolitik und Agrarökonomie. , Universität Sarbrücken, Deutschland, 27.05.2011
	\item Hirte K.: Performativity -- ein theoretischer Ansatz zur Wirkungsanalyse der Ökonomie auf die Gesellschaft?. Universität KasselTagung: Krise des Kapitalismus und die Zukunft der ökonomischen Wissenschaft, Deutschland, 28.09.2011
	\item Hirte K.: Konkurrierende Vergangenheit -- Geschichte und Gegengeschichte zur Vergangenheit der Nachkriegsgeneration der deutschen Agrarpolitikprofessoren. Dreiländerkongress der Deutschen Gesellschaft für Soziologie, der Österreichischen Gesellschaft für Soziologie und der Schweizerischen Gesellschaft für Soziologie, Innsbruck, Österreich, 29.09.2011
	\item Hirte K.: Netzwerke im Internet -- eine neue kritische Öffentlichkeit? Das Beispiel Guttenberg. Dreiländerkongress der Deutschen Gesellschaft für Soziologie, der Österreichischen Gesellschaft für Soziologie und der Schweizerischen Gesellschaft für Soziologie, Innsbruck, Österreich, 01.10.2011
	\item Hirte K.: Crowdsourcing -- temporäre virtuelle Gemeinschaften und ihr Regelbezug. Der Fall Guttenplag.. 41. Jahrestagung der Gesellschaft für Informatik, Berlin, Deutschland, 04.10.2011
	\item Hirte K.: Gleichheit und Vielfalt als normative Konzeptionen? -- zu den philosophischen Implikationen bei Simone de Beauvoir. Momentum 4, Track 5, Hallstadt, Österreich, 29.10.2011
\end{enumerate}
\paragraph{Eingeladener Vortrag an Universität}
\begin{enumerate}
	\item Ötsch W.: Wirtschaftskrise, Krise der Mainstream-Ökonomik -- Chance für die Sozialökonomie?. Eduard-Heimann-Colloquiumsreihe: Nach der Krise -- Folgen für Wirtschaft, Wissenschaft und Politik, Universität Hamburg, Deutschland, 09.02.2011
	\item Ötsch W.: Ist der Euro am Ende?. , Universität Linz, Österreich, 22.11.2011
	\item Ötsch W.: Politik und Wirtschaft, insbesondere zum Neoliberalismus. , Donau-Universität Krems, Österreich, 16.12.2011
	\item Nordmann J.: Schreibreflexion und Subjekt im Spätwerk von Roland Barthes. , Universität Flensburg, Deutschland, 21.01.2011
	\item Nordmann J.: Foucault. Eine Einführung in sein Denken, Österreich, 18.05.2011
	\item Nordmann J.: Die Österreichische Schule im Neoliberalismus. , Universität Bayreuth Bayreuth, Deutschland, 23.05.2011
	\item Nordmann J.: Thomas Kuhns wissenschaftliche Revolution. Grenzen der Paradigmenlehre. , Universität Flensburg, Deutschland, 27.05.2011
	\item Nordmann J.: Krisen und Alternativen des neoliberalen Modells. , Universität Linz, Österreich, 15.11.2011
	\item Nordmann J.: Der Gesellschaftsvertrag von Locke bis Rawls. , Universität Flensburg, Deutschland, 25.11.2011
	\item Hirte K.: Performativity -- ein neues Konzept zur Analyse des Einflusses von Ökonomen?. , Universität Hamburg, Deutschland, 17.11.2011
\end{enumerate}
\paragraph{Eingeladener Vortrag an anderen Institutionen}
\begin{enumerate}
	\item Ötsch W.: Kapitalismus und Moral. Max-Delbrück-Forum, Max-Delbrück-Gymnasium, Berlin, Deutschland, 23.02.2011
	\item Ötsch W.: Wer ist schuld an der Staatsschuld?. , Kepler Salon Linz, Österreich, 07.11.2011
	\item Ötsch W.: Retten wir €uropa! Die Frage ist nur, wie?. Grüne Wirtschaft OÖ, Landgraf Loft Linz, Österreich, 28.11.2011
	\item Ötsch W.: Podiumsdiskussion aus Anlass des 90. Geburtstages von Prof. Kazimierz Laski. Arbeitsgemeinschaft für Wissenschaftliche Wirtschaftspolitik, Österreichisches Gesellschafts- und Wirtschaftsmuseum Wien, Österreich, 06.12.2011
	\item Hirte K.: Das Performativity-Konzept und seine soziologietheoretische Fundierung. Zentrum für Agrarlandschaftsforschung (ZALF) Müncheberg, Müncheberg, Deutschland, 05.10.2011
	\item Hirte K.: Arbeit bei Marx. Arbeitskreis Politische Ökonomie zum Thema \glqq Karl Marx 2011\grqq{}, Trier, Deutschland, 14.10.2011
\end{enumerate}
\paragraph{Hauptvortrag / Eingeladener Vortrag auf einer Tagung}
\begin{enumerate}
	\item Nordmann J.: Liberalismus und Neoliberalismus. , Volkshochschule Linz Linz, Österreich, 15.10.2011
\end{enumerate}
\paragraph{Andere Vorträge und Präsentationen}
\begin{enumerate}
	\item Hirte K.: Ernährungssouveränität. , Teilnahme Podiumsdiskussion, Universität für Bodenkultur Wien, Österreich, 09.06.2011
\end{enumerate}
\subsection*{2010}
\paragraph{Vortrag auf einer Tagung (nicht referiert)}
\begin{enumerate}
	\item Ötsch W.: Nach der Wirtschaftskrise: was haben wir gelernt?. Podiumsdiskussion, Hotel Park Inn, Linz, Österreich, 18.02.2010
	\item Ötsch W.: Arbeitsmarkt der Zukunft, Zukunft der Sozialwirtschaft. Sozialplattform OberösterreichSozialwirtschaft. Wandel-Zukunft-Chancen, Linz, Österreich, 25.02.2010
	\item Ötsch W.: Was ist Grün. Die GrünenAlternative gefragt?! Politische Theorie und Geschichte der Grünen, Wien, Österreich, 13.03.2010
	\item Ötsch W.: Mythos Markt. Marktradikale Propaganda und ökonomische Theorie. Bündnis für Eine Welt/ÖIE mit ATTAC-Kärnten, Klagenfurt, Österreich, 22.04.2010
	\item Ötsch W.: Wirtschaftskrise, Sparpakete und politische Optionen. Klausur der Landtagklubs der SPÖ, Bad Ischl, Österreich, 22.06.2010
	\item Ötsch W.: Der Begriff \glqq der Markt" im Diskurs um die Wirtschaft -- Die Tiefenbedeutung von "Markt\grqq{}. Ein Schlüssel zum Verständnis der neoliberal-marktradikalen Gesellschaft. , Hallstadt, Österreich, 22.10.2010
	\item Nordmann J.: Trash, Skandale und Ratschläge statt Aufklärung und politische Bildung. Über das Zusammenspiel von kommerzialisierten Medien und gemachter Meinung in der neoliberalen Gesellschaft. 3. Momentum Kongress, Hallstadt, Hallstatt, Österreich, 22.10.2010
	\item Nordmann J.: Gibt es eine neoliberale Gesellschaft? Theoretische Überlegungen zur Analyse eines ambivalenten Phänomens. , Tagung: Gesellschaft! Welche Gesellschaft?, Österreich, 03.12.2010
\end{enumerate}
\paragraph{Eingeladener Vortrag an Universität}
\begin{enumerate}
	\item Ötsch W.: Keynesianismus und Neoliberalismus. Zur Geschichte des Wirtschaftssystems und Finanzmarktkrise und Lösungsvorschläge. Zur aktuellen Situation des Kapitalismus. Institut für Lebensbegleitendes LernenAktuelle Fragen der Wirtschaftspolitik, Pädagogische Hochschule Salzburg, Österreich, 23.02.2010
	\item Ötsch W.: Wie Wirtschaft Welt bewegt. Die großen ökonomischen Modelle auf dem Prüfstand. Junge WirtschaftPodiumsdiskussion zum Buch von Hans Bürger und K.W. Rothschild, Johannes Kepler Universität, Österreich, 12.04.2010
	\item Ötsch W.: Wie pluralistisch ist die Volkswirtschaftslehre an der JKU wirklich?. Podiumsdiskussion Initiative kritischer Studierender, Johannes Kepler Universität JKU Linz, Österreich, 18.05.2010
	\item Nordmann J., Ötsch W.: Globale Tendenzen der Weltwirtschaft im 20. Jahrhundert: von der Krise der 70er Jahre zur Großen Krise ab 2008. Ringvorlesung: Global History -- Society and Governance, Johannes Kepler Universität, Österreich, 27.04.2010
	\item Hirte K.: Performativität -- ein tragfähiger Ansatz, um den Einfluss von ökonomischen Theorien auf reale Wirtschaftsabläufe zu analysieren?. Forschungskolloquiums, Johannes Kepler Universität, Österreich, 29.04.2010
	\item Hirte K.: Macht- und Elitetheorien – alte und neue Ansätze. ICAE, JKU LinzSommerakademie 2010, Österreich, 05.06.2010
	\item Hirte K.: Netzwerkanalyse und Zeitverläufe. Forschungskolloquiums, Johannes Kepler Universität, Österreich, 24.06.2010
\end{enumerate}
\paragraph{Hauptvortrag / Eingeladener Vortrag auf einer Tagung}
\begin{enumerate}
	\item Ötsch W.: Der Markt-Begriff im Neoliberalismus. Prof. Reisinger, Katholisch-Theologische Privatuniversität Linz, Österreich, 30.11.2010
\end{enumerate}
\paragraph{Eingeladener Vortrag an anderen Institutionen}
\begin{enumerate}
	\item Ötsch W.: Dem Volk auf's Maul g'schaut? -- Politische Sprache in der aktuellen Asyldebatte. , Wissensturm Linz, Österreich, 28.01.2010
	\item Ötsch W.: Globale Wirtschaft -- Globale Krise. Institut für FortbildungLehrerfortbildung, Private Pädagogische Hochschule der Diözese Linz, Österreich, 22.02.2010
	\item Hirte K.: Historie der professoralen Agrarökonomen Deutschlands 1933 – 1955. Forschungskolleg ICAE, Universität Linz, Österreich, 24.06.2010
	\item Hirte K.: Das neoklassische Freihandelsmodell -- Fundament für Entwicklungszusammenarbeit oder Zementierung globaler Ungleichheiten?. 3. Momentum Kongress, Hallstadt, Hallstadt, Österreich, 22.10.2010
\end{enumerate}
\paragraph{Vortrag auf einer Tagung (referiert)}
\begin{enumerate}
	\item Pühringer M., Pühringer S.: Solidarität im Kapitalismus. 3. Momentum Kongress in Hallstatt, Hallstatt, Österreich, 22.10.2010
	\item Knierim A.: Aktionsforschung – ein Weg zum Design institutioneller Neuerungen. Fachtagung „Anpassung an den Klimawandel – regional umsetzen!“, Darmstadt, Deutschland, 10.06.2010
	\item Hirte K., Nordmann J., Ötsch W.: Die Evolution ökonomischen Wissens und des Wissens über den Kapitalismus. Performativity als Analyseinstrument: das Beispiel der Fabian Society, der Mont Pelerin Society und der Chicagoer Schule. Jahrestagung des Ausschusses Evolutorische Ökonomik, Verein für Socialpolitik, Linz, Österreich, 03.07.2010
	\item Hirte K.: Netzwerkanalytische Betrachtungen zu historischen Verläufen -- Vorteile und neue Erkenntnisse? (am Beispiel der Historie der professoralen Agrarökonomen Deutschlands 1933 bis 1955). Workshop Historische Netzwerkforschung, Universität Essen, Deutschland, 30.05.2010
	\item Hirte K.: Vielschichtigkeit in der Realität -- Stringenz in der Analyse? Das Problem der Datenbewältigung bei der Erstellung von Ego-Netzwerken.. Methodenworkshop Netzwerkforschung in den Geistes- und SozialwissenschaftenMethodenworkshop Netzwerkforschung in den Geistes- und Sozialwissenschaften, Universität Trier, Deutschland, 18.09.2010
	\item Hirte K.: Ego-Netzwerke 1933 bis nach 1945 im Zeitverlauf -- Brüche? Kontinuitäten? Typica?. , Universität Wien, Österreich, 13.11.2010
\end{enumerate}
\paragraph{Andere Vorträge und Präsentationen}
\begin{enumerate}
	\item Hirte K., Nordmann J., Ötsch W.: Moderation zu: Neue Machtsysteme in der Postdemokratie. Sommerakademie 2010, BauAkademie Lachstatt, Steyr, Österreich, 04.06.2010
\end{enumerate}
\subsection*{2009}
\paragraph{Vortrag auf einer Tagung (referiert)}
\begin{enumerate}
	\item Hirte K.: Institutionalisierungsprozesse im Ökologischen Landbau in den Neuen Bundesländern. Die GÄA e. V. und Biopark.. 10. Wissenschaftstagung Ökolandbau, Zürich, Schweiz, 11.02.2009
\end{enumerate}
\paragraph{Eingeladener Vortrag an anderen Institutionen}
\begin{enumerate}
	\item Hirte K.: Preistheorien und Preisgestaltung – eine systematisierende Hinterfragung (Beispiel Milchmarkt und Milchmarktregelungen).. dito Milcherzeugergemeinschaft Milch BoardMilchtagung 2009, Hardehausen, Deutschland, 03.03.2009
\end{enumerate}
