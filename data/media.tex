\subsection*{2024}
\paragraph{Präsentation in Radio/TV}
\begin{enumerate}
	\item Pühringer S.: Pressekonferenz zur Studie: "Netzwerke der Superreichen in Österreich". AK Wien, Wien, Österreich, 26.11.2024
	\item Pühringer S.: Die Nobelpreise 2024: Wirtschaft. , Ö1 Dimensionen, Österreich, 05.12.2024
	\item Hornykewycz A.: Junge Forschung \#9 mit Anna Hornykewycz. Podcast "In der Wirtschaft", Deutschland, 01.03.2024
	\item Hager T.: Forschen mit Ablaufdatum, Österreich, 07.11.2024
	\item Aistleitner M., Pühringer S.: Wirtschaftsmodelle ohne Klimakrise. , Ö1 Dimensionen, Österreich, 06.03.2024
\end{enumerate}
\subsection*{2023}
\paragraph{Präsentation in Radio/TV}
\begin{enumerate}
	\item Pühringer S., Ötsch W.: Wie funktioniert die Ökonomie als Wissenschaft?. dorfttv.at, Reihe: Denken hilft, Folge 14Radiosendung, Österreich, 21.08.2023
	\item Pühringer S.: Denkanstöße liefern - Interview mit START-Preisträger Stephan Pühringer. Die Presse, Interview, Österreich, 28.07.2023
	\item Pühringer S.: Verteilung von Reichtum, Finanz- und Wirtschaftsbildung sowie neoliberale Denkweisen. FM4-Sendung "Auf Laut"FM4-Sendung "Auf Laut", Österreich, 08.11.2023
	\item Hirte K.: Geschichte der Wirtschaftspolitik. Radio Wissen Bayern2, Interviewbeitrag, Deutschland, 19.01.2023
\end{enumerate}
\subsection*{2022}
\paragraph{Präsentation in Radio/TV}
\begin{enumerate}
	\item Pühringer S., Rogojanu A.: Competition Universalism und einige Zusammenhänge zwischen Wettbewerb und Raum. cultural broadcasting archive (cba), cba.fro.at, 10te Podcast-Folge zum Forschungsprojekt SPACE, Österreich, 29.11.2022
	\item Pühringer S.: Prekäre Wissenschaft: Warum Unis reformiert werden müssen. mosaik-Redaktion, Österreich, 21.12.2022
	\item Plehwe D., Pühringer S.: Think-Tanks: Neue Akteure im politischen Diskurs. Ö1, Österreich, 02.02.2022
	\item Kapeller J.: Interview zu: Ist die Preisexplosion begründet. KronenZeitung, Österreich, 08.06.2022
	\item Gräbner-Radkowitsch C., Kapeller J.: MAN Steyr und der internationale Wettbewerb. cultural broadcasting archive (cba), cba.fro.at, 8te Podcast-Folge zum Forschungsprojekt SPACE, Österreich, 26.06.2022
	\item Gräbner-Radkowitsch C.: Agent Based Modelling. Podcast Vielfalt in den Wirtschaftswissenschaften, Deutschland, 27.05.2022
	\item Aistleitner M.: Ökonomie: Enge Verflechtung verzerrt den Wettbewerb der Ideen. Marlene Nowotny (Ö1), Österreich, 14.10.2022
\end{enumerate}
\subsection*{2021}
\paragraph{Präsentation in Radio/TV}
\begin{enumerate}
	\item Rath J.: Wie wird über Wettbewerb gesprochen?. cultural broadcasting archive (cba), Österreich, 02.03.2021
	\item Pühringer S.: Wettbewerb infrage - neue Podcastreihe. Orange 94, Österreich, 05.01.2021
	\item Pühringer S.: Wettbewerb als Ordnungsprinzip. cultural broadcasting archive (cba), Österreich, 05.01.2021
	\item Pühringer S.: Private Verlage profitieren von Forschungsgeldern. ORF.at, Wissen aktuell, Österreich, 01.02.2021
	\item Pühringer S.: Der Wert, zitiert zu werden. cultural broadcasting archive (cba), Österreich, 04.05.2021
	\item Pühringer S.: Sozialer Schein, aber wirtschaftsliberale Positionen. mdr, Deutschland, 02.12.2021
	\item Kapeller J.: Kosten und Nutzen des „billigen“ Geldes. Deutschlandfunk Kultur, Deutschland, 09.03.2021
	\item Kapeller J.: EU taxation capabilities and the way forward towards institutional progress in Europe. World Economic Association, Vereinigte Staaten, 05.07.2021
	\item Kapeller J.: Der Soziologe Robert K. Merton: Alles misslingt nach Plan. SWR2 WISSEN, Interview mit Jakob Kapeller, Sendung von Michael Reitz, Deutschland, 19.11.2021
	\item Hirte K.: Geschichte des Neoliberalismus. Deutschlandfunk Kultur, Feature-Zeitfragen, Deutschland, 28.12.2021
	\item Gräbner-Radkowitsch C., Hager T., Porak L.: Wettbewerb in und rund um die Europäische Union. cultural broadcasting archive (cba), Österreich, 06.07.2021
\end{enumerate}
\subsection*{2020}
\paragraph{Präsentation in Radio/TV}
\begin{enumerate}
	\item Pühringer S.: Wo lassen denken? Die unsichtbare Macht von Think Tanks. oe1.orf Sendereihe Dimensionen, Österreich, 20.02.2020
	\item Kapeller J.: Solidarität – Wandel eines starken Begriffs. Ö1, Participant panel discussion, Österreich, 08.01.2020
	\item Kapeller J.: City Science Talk: Solidarität. Ö1, ö1  -Sendung, Österreich, 08.01.2020
\end{enumerate}
\subsection*{2019}
\paragraph{Präsentation in Radio/TV}
\begin{enumerate}
	\item Pühringer S.: Wie funktioniert Wettbewerb?. schroedingerskatze.at - Der österreichische Wissenschaftsblog, Österreich, 12.07.2019
	\item Kapeller J.: Interview zu pluraler Ökonomik, Teil I. Future Histories Podcast, Deutschland, 28.07.2019
	\item Kapeller J.: Interview zu pluraler Ökonomik, Teil II. Future Histories Podcast, Deutschland, 11.08.2019
	\item Kapeller J.: Vermögen wächst auch mit Vermögenssteuer weiter. Moment.at, Interview, Österreich, 23.10.2019
\end{enumerate}
\subsection*{2018}
\paragraph{Präsentation in Radio/TV}
\begin{enumerate}
	\item Heimberger P.: Das ABC der Finanzwelt - G wie Great Depression B. Ö1 Radiokolleg, Österreich, 12.12.2018
	\item Heimberger P.: 10 Fragen an Philipp Heimberger oder: Auf den Spuren der NAIRU. dezernatzukunft.org, Deutschland, 20.12.2018
	\item Beyer K.: Silvio Gesell und die Freiwirtschaftslehre. Ö1 Radiokelleg, Interview, Österreich, 18.07.2018
\end{enumerate}
\subsection*{2017}
\paragraph{Präsentation in Radio/TV}
\begin{enumerate}
	\item Ötsch W.: Aus der moralischen Entrüstung gehen. archiv.arbeit-wirtschaft.at/, Interviewaufzeichnung, Österreich, 16.10.2017
	\item Schütz B.: EU-Krise: Neue Chancen für Europa?. Radio FRO, Beitrag von Bernhard Schütz; zus. mit Franziskus Forster und Christian Diabl, Österreich, 11.05.2017
	\item Schütz B.: Die vielen Facetten des Reichtums in Österreich. oe1.orf.at, Österreich, 23.11.2017
	\item Schütz B.: Ab wann ist jemand reich?. science.orf.at, Österreich, 23.11.2017
	\item Kapeller J.: Reichtum in Österreich. kontrast.at, Österreich, 30.10.2017
\end{enumerate}
\subsection*{2016}
\paragraph{Präsentation in Radio/TV}
\begin{enumerate}
	\item Ötsch W.: Kann man soziale Verantwortung lehren?. Oxiblog.de, Interviewaufzeichnung, Deutschland, 28.07.2016
	\item Ötsch W.: Populismus, Demagogie und die „Wut von unten“. Soziologieblock.hypotheses.org, Interview mit Walter O. Ötsch, Österreich, 21.12.2016
	\item Pühringer S.: Think Tanks, Ökonomen und Politik. okto TV im Rahmen der Interviewreihe "Eingeschenkt" mit Martin Schenk, Wien, Österreich, 31.03.2016
	\item Pühringer S.: Die Denkfabrik der Millionäre. okto.tv, Österreich, 12.05.2016
	\item Kapeller J.: Panama Papers. Servus TV, "Talk im Hangar 7", Salzburg, Österreich, 07.04.2016
	\item Kapeller J.: Großer Betrug am kleinen Bürger: Warum zahlen wir noch Steuern?. Talk im Hangar-7, Österreich, 08.04.2016
	\item Kapeller J.: Globalisierung und Konzernmacht – Zur Moral des Profits im 21. Jahrhundert. cultural broadcasting archive (cba), cba.fro.at, Interview, Österreich, 05.09.2016
\end{enumerate}
\subsection*{2015}
\paragraph{Präsentation in Radio/TV}
\begin{enumerate}
	\item Pühringer S.: Märkte als Richter – Angela Merkels Sprechen über die Krise. Radio Dreyeckland, Freiburg im Breisgau, Deutschland, 19.10.2015
	\item Pühringer S.: "Märkte als Richter" - Stephan Pühringer über Merkels Sprechen über die Krise. rdl.de, Focus Europa, Interview, Deutschland, 19.10.2015
	\item Pühringer S.: Welche Ökonomie wollen wir?. Radio FRO, FROZINE, Österreich, 27.11.2015
\end{enumerate}
\subsection*{2013}
\paragraph{Präsentation in Radio/TV}
\begin{enumerate}
	\item Kapeller J.: Wie reich ist Österreich? Studienautor Kapeller live im Studio. orf.at, Österreich, 05.08.2013
\end{enumerate}
