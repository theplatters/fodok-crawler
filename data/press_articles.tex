\subsection*{2024}
\begin{enumerate}
    	 \item Pühringer S., Ötsch W.: Wie man die AfD kleinkriegt, in SPIEGEL, 2024
	 \item Pühringer S., Altreiter C.: The (self-imposed) Limits of Science, in Economists For Future, Serie 2023-24 Annual Debate Series, 2024
	 \item Pühringer S., Aistleitner M., Cserjan L., Hieselmayr S.: Das diskrete Netzwerk der Reichsten in Österreich, in Profil, 2024
	 \item Pühringer S.: Portrait zur Langen Nacht der Forschung, in Instagram, 2024
	 \item Pühringer S.: Eco vom 05.12.2024, in ORF, in Eco, 2024
	 \item Gräbner-Radkowitsch C., Strunk B.: Quelle place pour le Sud global dans la décroissance?, in The Conversation, 2024
\end{enumerate}
\subsection*{2023}
\begin{enumerate}
    	 \item Pühringer S., Partheymüller J.: Prekäre Arbeit an Universitäten kann man nicht wegrechnen, in Der Standard, 2023
	 \item Pühringer S., Altreiter C.: Die (selbstauferlegten) Grenzen der Wissenschaft, in Makronom, 2023
	 \item Hirte K.: Euphemistische Ökonomik, in Blog Postwachstum, 2023
	 \item Gräbner-Radkowitsch C., Strunk B.: Degrowth und der globale Süden, in Blog Postwachstum, 2023
	 \item Gräbner-Radkowitsch C., Strunk B.: Degrowth and the Global South: remarks on the twin problem of structural interdependencies, in Developing economics blog, 2023
\end{enumerate}
\subsection*{2022}
\begin{enumerate}
    	 \item Kapeller J.: Wechselseitige Kritik ist nur möglich mit nachvollziehbarer Forschung, in ZBW Leibniz Informationszentrum Wirtschaft, Seite(n) 1-2, 2022
	 \item Hirte K.: Strukturkonzentrationen in der Schlachthofbranche und die Rolle von Ökonomen, in Oxi - Wirtschaft anders denken, Nr. 01/22, Seite(n) 8, 2022
	 \item Heimberger P., Schütz B.: Das wackelige Fundament der Schuldenbremse, in Makronom, 2022
	 \item Gräbner-Radkowitsch C., Lage J., Wiese F.: Zur ökonomischen Bedeutung von Suffizienz, in Makronom, 2022
	 \item Breth L., Part F., Pühringer S., Schlitz N., Sperner P., Völkl Y.: Arbeitsplatz Universität: Gefangen in der Teilzeit, in Der Standard, 2022
	 \item Aistleitner M.: Verzerrter Wettbewerb in der Forschung, in science.orf.at, 2022
\end{enumerate}
\subsection*{2021}
\begin{enumerate}
    	 \item Porak L.: Und ewig lockt der Wettbewerb, in Makronom, 2021
	 \item Hirte K.: Ökonomie und Ideologie, in OXI - Wirtschaft anders denken, Seite(n) 16-17, 2021
	 \item Heimberger P., Kowall N.: Sieben „überraschende“ Fakten zu Italien, in Makronom, 2021
	 \item Heimberger P., Gechert S.: Erhöhen Unternehmenssteuersenkungen das Wirtschaftswachstum?, in Ökonomenstimme, 2021
	 \item Heimberger P.: Wie stark der globale Steuerwettbewerb tatsächlich ist, in Makronom, 2021
	 \item Heimberger P.: Wachstum durch Unternehmensteuersenkungen? Die FDP weckt übertriebene Hoffnungen, in Handelsblatt, 2021
	 \item Heimberger P.: Höhere Staatsschulden = weniger Wachstum?, in Makronom, 2021
	 \item Heimberger P.: Fakten-Mensch: Ökonom Philipp Heimberger kritisiert die aus seiner Sicht verzerrte Debatte über die EU, in Süddeutsche Zeitung, 2021
	 \item Heimberger P.: Eine globale Mindeststeuer stoppt die Steuerflucht der Konzerne, in Handelsblatt, 2021
	 \item Heimberger P.: Drei Gründe, warum Staatsschulden nicht zwingend problematisch sind, in Handelsblatt, 2021
	 \item Heimberger P.: Draghi darf das Sparen nicht übertreiben, in Handelsblatt, 2021
	 \item Heimberger P.: Die deutschen Inflationssorgen speisen sich aus einem verzerrten Geschichtsbild, in Handelsblatt, 2021
	 \item Heimberger P.: Die Finanzglobalisierung verschärft die Einkommensungleichheit, in Handelsblatt, 2021
	 \item Heimberger P.: Die EU-Anleihen sind ein Zukunftsmodell für Europa, in Handelsblatt, 2021
	 \item Heimberger P.: Der Nobelpreis für David Card hat die Mindestlohn-Befürworter in Deutschland gestärkt, in Handelsblatt, 2021
	 \item Heimberger P.: Beschäftigungsschutz erhöht die Arbeitslosigkeit nicht, in Ökonomenstimme, 2021
	 \item Heimberger P.: Arbeitnehmerrechte sind kein Jobkiller, in Handelsblatt, 2021
	 \item Gräbner C., Heimberger P., Kapeller J.: Ökonomische Offenheit: Die Vermessung der Globalisierung?, in Ökonomenstimme, 2021
	 \item Altreiter C., Rogojanu A., Gräbner C., Pühringer S., Wolfmayr G.: 20 Jahre gute Absichten in der Wissenschaftspolitik, in Die Presse, 2021
	 \item Altreiter C., Gräbner C., Pühringer S., Rogojanu A., Wolfmayr G.: Wie befristete Uni-Stellen Innovation verhindern, in Der Standard, 2021
\end{enumerate}
\subsection*{2020}
\begin{enumerate}
    	 \item Schütz B.: Gewohnter Umsatz wird auf sich warten lassen, in Kronen Zeitung, Interview (von Elisabeth Rathenböck) mit Bernhard Schütz, 2020
	 \item Kapeller J.: Wahrheiten über Wertfreiheit – eine Replik auf den Gastkommentar ‚Werte, Wahrheit, Wirtschaftsforschung‘ von Harald Oberhofer, in Der Standard, 2020
	 \item Hirte K.: „Um welches Ziel es in der Ökonomie geht, ist also bestimmbar“, in Agora42 (Philosophisches Wirtschaftsmagazin), 2020
	 \item Hirte K.: Agrarsubventionen als Preis der Marktwirtschaft, in Wege für eine Bäuerliche Zukunft – Zeitschrift der ÖBV/ Via Campesina Austria, Vol. 43, Nr. 1 (361), Wien, Seite(n) 10-11, 2020
	 \item Heimberger P., Truger A.: Der Outputlücken-Nonsense gefährdet Deutschlands Erholung von der Corona-Krise, in Makronom, 2020
	 \item Heimberger P., Krahé M., van't Klooster J., Ponattu D.: Gleichwertige Lebensverhältnisse im Euroraum, in Frankfurter Allgemeine Zeitung, 2020
	 \item Heimberger P., Kapeller J.: Wie die ‚Herren der Modelle‘ versuchen, den Outputlücken-Nonsense zu rechtfertigen, in Makronom, 2020
	 \item Heimberger P.: Zu niets doen voor Z-Europa is gevaarlijk (Nichts für Südeuropa zu tun ist gefährlich), in NRC Handelsblad, 2020
	 \item Heimberger P.: Wie ein Rechenfehler zu fatalen Konsequenzen für die Haushalte der EU-Staaten führen kann, in Handelsblatt, 2020
	 \item Heimberger P.: Wie die EU-Kommission Deutschlands Budgetsituation schlechtrechnet, in Handelsblatt, 2020
	 \item Heimberger P.: Kein Weg in die „Schuldenunion“, in Der Standard, 2020
	 \item Heimberger P.: Enttäuschender EU-Gipfel: Zeit für eine „Neue Südpolitik”, in Makronom, 2020
	 \item Heimberger P.: Einigung der Eurogruppe: Bestenfalls ein erster Schritt, in Makronom, 2020
	 \item Heimberger P.: Die gefährliche Geschichtsverdrehung des Hans-Werner Sinn, in Handelsblatt, 2020
	 \item Heimberger P.: Beschäftigungsschutz ist kein Jobkiller, in Makronom, 2020
	 \item Gräbner C., Strunk B.: Kritik an der Pluralen Ökonomik – Was ist dran und warum ist das wichtig?, in Ökonomenstimme, 2020
	 \item Gräbner C., Heimberger P., Kapeller J.: Ökonomische Offenheit: Die Vermessung der Globalisierung, in Makronom, 2020
\end{enumerate}
\subsection*{2019}
\begin{enumerate}
    	 \item Pühringer S., Grimm C.: Männlich, mikroökonomisch, Mainstream? Eine Untersuchung der VWL-Lehrstühle in Deutschland, in Makronom, 2019
	 \item Heimberger P., Pekanov A.: Den wirtschaftlichen Abschwung bekämpfen!, in Der Standard, 2019
	 \item Heimberger P., Kapeller J.: Auch Deutschland wird zum Opfer des "Outputlücken-Nonsens", in Makronom, 2019
	 \item Heimberger P., Gräbner C., Kapeller J.: Eine Strategie gegen die ökonomische Polarisierung Europas, in Makronom, 2019
	 \item Heimberger P.: Wirtschaftlicher Abschwung: Relevanz und Gegenmaßnahmen, in Momentum Newsletter, 2019
	 \item Heimberger P.: Welche Rolle spielt der „Outputlücken-Nonsense“?, in Makronom, 2019
	 \item Heimberger P.: The danger of  'nonsense Output gaps', in Financial Times, 2019
\end{enumerate}
\subsection*{2018}
\begin{enumerate}
    	 \item Hirte K., Poppinga O.: Was Märkte (nicht) mit Demokratie zu tun haben, in Wege für eine Bäuerliche Zukunft – Zeitschrift der ÖBV/ Via Campesina Austria, Vol. 41, Nr. 3 (353), Wien, Seite(n) 4-6, 2018
	 \item Hirte K.: Die Diskrepanz zwischen Reden und Handeln ist im ökonomischen Bereich besonders groß, in Agora42 (Philosophisches Wirtschaftsmagazin), 2018
	 \item Heimberger P., Gräbner C., Kapeller J.: Warum Europa trotz Aufschwung ökonomisch weiter auseinander driftet, in Makronom, 2018
	 \item Heimberger P.: Vier europäische Lehren aus den Turbulenzen in Italien, in Makronom, 2018
	 \item Heimberger P.: Hilf dir selbst, dann hilft dir Deutschland: Fortschritt bei der Reform der Eurozone, in Der Standard, 2018
\end{enumerate}
\subsection*{2017}
\begin{enumerate}
    	 \item Pühringer S.: Migration: It's still the economy, stupid!, in Der Standard, 2017
	 \item Heimberger P.: Zerfällt Europa? Trump als Entscheidungsbeschleuniger, in Der Standard, 2017
	 \item Heimberger P.: Europa am Scheidepunkt: Österreichs wichtige Rolle, in Der Standard, 2017
	 \item Heimberger P.: Die EU braucht einen wirtschaftspolitischen Kurswechsel, in Der Standard, 2017
\end{enumerate}
\subsection*{2016}
\begin{enumerate}
    	 \item Ötsch W.: Ökonomisches Denken, Rechtspopulismus und Rechtsextremismus, in Makroskop, 2016
	 \item Ötsch W.: Markt-Glauben, Klima-Krise und Katastrophen-Leugnung, Teil 1-4, in Makroskop, 2016
	 \item Ötsch W.: Der Blick von oben und der Blick von unten, in Makroskop, 2016
	 \item Heimberger P.: Trumps Team: Politik von und für Vermögende, in Die Presse, 2016
	 \item Heimberger P.: Investitionen gegen die Dauerkrise im Euroland, in Der Standard, 2016
	 \item Heimberger P.: Austeritätspolitik in der Eurozone: Ein Schuss ins eigene Knie, in Makronom, 2016
\end{enumerate}
\subsection*{2015}
\begin{enumerate}
    	 \item Pühringer S.: Märkte als Richter: Zur Dominanz neoliberaler Krisennarrative, in Ksoe-Dossier (Ksoe Nachrichten der Katholischen Sozialakademie), Nr. 1, Seite(n) 1-3, 2015
\end{enumerate}
\subsection*{2013}
\begin{enumerate}
    	 \item Ötsch W.: Das Team Stronach: Die österreichische Tea Party, in Die Presse, 2013
\end{enumerate}
