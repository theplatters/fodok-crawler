\subsection*{2024}
\paragraph{Eingeladener Vortrag an Universität}
\begin{enumerate}
	\item Theine H.: The climate crisis portrayal in Austrian media:  A Cross-Paradigm approach combining structural topic modelling and critical discourse analysis. Research Seminar, ICAE, Österreich, 15.10.2024
\end{enumerate}
\paragraph{Andere Vorträge und Präsentationen}
\begin{enumerate}
	\item Terhorst C.: Inequalities in green transitions: The neglected role of collective infrastructures and institutions. ICAE Research Seminar, Österreich, 17.12.2024
	\item Rath J.: Short-term Employment and Social Stratification in Austrian Academia. ICAE Research Seminar 2023/24, Österreich, 16.01.2024
\end{enumerate}
\subsection*{2023}
\paragraph{Andere Vorträge und Präsentationen}
\begin{enumerate}
	\item Theine H.: Catering for the interests of the richest or staunch fighters for social justice? A long-term perspective on media discourses of economic inequality in the US press. Research Seminar, ICAE, Österreich, 28.03.2023
	\item Rath M.: Backlash or Progress? - Exploring the long-time effect of Covid-policies on the division of care work. Research Seminar, ICAE, Österreich, 21.03.2023
	\item Porak L.: Wettbewerbsfähige Nachhaltigkeit: Eine Historisch-Materialistische Analyse der Europäischen grünen Transformation. Research Seminar, ICAE, Österreich, 09.05.2023
	\item Hirte K.: Eliten und Demokratie — zur Etablierung der funktionalistischen Elite- und Demokratieauffassung nach 1945 und die Rolle von C. J. Friedrich. Open Research Seminar, ICAE, Österreich, 02.05.2023
	\item Gräbner-Radkowitsch C., Pühringer S.: Competitization and Quantification in academic Economics: Origins, Evolution and Implications. Research Seminar, ICAE, Österreich, 31.01.2023
	\item Ferschli B.: Underdemand or undersupply of labour? Automation and employment in the Austrian automotive sector. A value chain perspective. Research Seminar, ICAE, Österreich, 24.01.2023
	\item Delgado P.: Supply, Demand, and Preferences: Determinants of Meat Consumption in Spain: 1964 – 2019. Research Seminar, ICAE, Österreich, 17.01.2023
	\item Aistleitner M.: Forschung über Wissenschaft. Research Seminar, ICAE, Österreich, 06.02.2023
\end{enumerate}
\subsection*{2022}
\paragraph{Andere Vorträge und Präsentationen}
\begin{enumerate}
	\item Theine H.: Klimakrise, klimasoziale Politik und die Rolle der Superreichen. Research Seminar, ICAE, Österreich, 05.05.2022
	\item Schnitzer H.: Criticizing Economics for its Lack of Realism?. Research Seminar, ICAE, Österreich, 15.11.2022
	\item Pühringer S., Wolfmayr G.: The Performativity of Competitiveness in Academia. Research Seminar, ICAE, Österreich, 09.06.2022
	\item Porak L., Reinke R.: How public economists in Germany are thinking about the economy, economic policy, and economics. Research Seminar, ICAE, Österreich, 13.12.2022
	\item Porak L.: Re-Inventing Competitiveness — tracing the development of competitiveness as European governmental problem. Research Seminar, ICAE, Österreich, 02.06.2022
	\item Neujeffski M.: European Cohesion Policy in Times of Austerity: A Regional Analysis of Germany and Portugal. Research Seminar, ICAE, Österreich, 18.10.2022
	\item Mellacher P.: Opinion Dynamics under Conflicting Interests. Research Seminar, ICAE, Österreich, 30.06.2022
	\item Lang S.: Dedicated Innovation Systems for the Transformation towards Sustainability. Research Seminar, ICAE, Österreich, 23.06.2022
	\item Hornykewycz A.: Introducing heterogeneous consumer products into a single-country, SFC macro-ABM framework. Research Seminar, ICAE, Österreich, 28.04.2022
	\item Hornykewycz A.: Employment Relationships in the Age of Digitalization. Research Seminar, ICAE, Österreich, 08.11.2022
	\item Hirte K.: Euphemismen in der Ökonomik: Das Beispiel „Arbeitgeber“ – Herkunft und Etablierung eines Begriffs. Open Research Seminar, ICAEOpen Research Seminar, ICAE, Österreich, 07.04.2022
	\item Harroche A.: The implementation of policies for excellence in French higher education. Research Seminar, ICAE, Österreich, 22.11.2022
	\item Hager T.: Bane or Boon: On Technological Competitiveness and Joining the European Union. Open Research Seminar ICAE, Österreich, 20.12.2022
	\item Gartiser M.: Don’t fear the reaper — Tax policy frames in German print media. Research Seminar, ICAE, Österreich, 29.11.2022
	\item Eder J.: Mit neuen Eigentumsformen durch die sozial-ökologische Transformation? Die europäische Industrie zukunftsfit machen. Open Research Seminar, ICAE, Österreich, 27.01.2022
	\item Eckert G.: Konfliktlinien in den Wirtschaftswissenschaften und ihre Bedeutung für ‚Bildung und nachhaltige Entwicklung‘ (BNE). Open Research Seminar, ICAE, Österreich, 24.02.2022
	\item Bäuerle L.: Wirtschaft als transformativer Prozess. Grundlagen und Implikationen für die Wirtschaftswissenschaften. Research Seminar, ICAE, Österreich, 21.04.2022
	\item Arbogast M.: Modern Central Banks: Navigating between 'the stars' and a state-led growth model. Open Research Seminar, ICAE, Österreich, 13.01.2022
	\item Aistleitner M., Pühringer S.: Netzwerke der Superreichen in Österreich. Research Seminar, ICAE, Österreich, 25.10.2022
	\item Aistleitner M., Hager T., Porak L., Pühringer S., Rath J.: Battle over Indicators or a battle of ideas?. Research Seminar, ICAE, Österreich, 06.12.2022
	\item Aistleitner M.: Development and Interdisciplinarity: a comment on Mitra et al. (2020). Research Seminar, ICAE, Österreich, 12.05.2022
\end{enumerate}
\subsection*{2021}
\paragraph{Andere Vorträge und Präsentationen}
\begin{enumerate}
	\item Hirte K.: Oligopole Strukturen in der Schlachthofindustrie und ihre Herkunft. Research Seminar ICAE, Universität LinzResearch Seminar, ICAE, Österreich, 27.05.2021
	\item Hager T.: Lobbyismus und gesamtwirtschaftliche Entwicklung – ein Literaturüberblick zu Mancur Olsons Theorie. Open Research Seminar ICAE, Österreich, 09.12.2021
\end{enumerate}
\subsection*{2020}
\paragraph{Andere Vorträge und Präsentationen}
\begin{enumerate}
	\item Strohmaier R.: The Role of Public Procurement for Driving Corporate Social Innovation in Global IT Supply Chains. Open Research Seminar ICAE, Universität Linz, Österreich, 12.11.2020
	\item Schütz B.: The conflict over income in a capitalist society: A stock-flow consistent approach. Open Research Seminar ICAE, Universität Linz, Österreich, 09.01.2020
	\item Pühringer S., Rath J.: Monopolies in Science Publishing: A black hole for public spending?. Open Research Seminar ICAE, Universität Linz, Österreich, 14.05.2020
	\item Pühringer S., Rath J.: Talking about Competition: Discursive shifts in the economic imaginary of competition. Open Research Seminar ICAE, Universität Linz, Österreich, 16.07.2020
	\item Pühringer S.: Diskursanalyse zu Schwarz-Blauen Arbeitsmarkt- und Sozialpolitikreformen. Open Research Seminar ICAE, Universität Linz, Österreich, 23.01.2020
	\item Porak L.: Die strategische Ausrichtung der Europäischen Union. Open Research Seminar ICAE, Universität Linz, Österreich, 09.07.2020
	\item Kennedy M.: Narrative Economics: Introduction to Literary and Media Representations of the Economy. Open Research Seminar ICAE, Universität Linz, Österreich, 29.10.2020
	\item Janischweski A.: Inequality in a zero growth economy. Open Research Seminar ICAE, Universität Linz, Österreich, 23.07.2020
	\item Hornykewycz A., Schütz B.: Minsky-Veblen Cycles: A stock-flow consistent, agent-based approach. Open Research Seminar ICAE, Universität Linz, Österreich, 05.11.2020
	\item Hirte K.: Zettelkasten ist nicht gleich Zettelkasten: Eine Kontrastierung  der Arbeiten von Mark Lombardi und Niklas Luhmann. Universität LinzOpen Research Seminar ICAE, Universität Linz, Österreich, 23.04.2020
	\item Heimberger P.: Corporate Tax Competition: A meta-analysis. Open Research Seminar ICAE, Universität Linz, Österreich, 26.11.2020
	\item Hager T., Heck I., Rath J.: Competition in Transformational Processes: Polanyi \& Schumpeter. Open Research Seminar ICAE, Universität Linz, Österreich, 17.12.2020
	\item Gräbner-Radkowitsch C., Pühringer S.: Theorizing Competition: An interdisciplinary approach to the genesis of a contested concept. Open Research Seminar ICAE, Universität Linz, Österreich, 22.10.2020
	\item Gräbner-Radkowitsch C.: The Emergence of Core-Periphery Structures in the European Trade Network. Open Research Seminar ICAE, Universität Linz, Österreich, 18.06.2020
	\item Griesebner T.: Global Child Chain - Indische Leihmutterschaft im Kontext globaler Güterkettenforschung. Open Research Seminar ICAE, Universität Linz, Österreich, 19.11.2020
	\item Gerdes L.: Evolutionary political economic policies in a multisector, multi-regional agent-based model of a global value chain, resource extraction and climate change. Open Research Seminar ICAE, Universität Linz, Österreich, 10.12.2020
	\item Aistleitner M., Gräbner-Radkowitsch C., Hornykewycz A.: Theory and empirics of capability accumulation: Implications for macroeconomic modelling. Open Research Seminar ICAE, Universität Linz, Österreich, 13.02.2020
\end{enumerate}
\subsection*{2019}
\paragraph{Andere Vorträge und Präsentationen}
\begin{enumerate}
	\item Rath J.: Solving the Hold-up-Problem: Mechanism Design vs Social Norms. Open Research Seminar ICAE, Universität Linz, Österreich, 27.02.2019
	\item Heimberger P.: Does employment protection affect unemployment? A meta-study. Open Research Seminar ICAE, Universität Linz, Österreich, 21.11.2019
	\item Eder J.: Prestons Community Wealth Building-Ansatz – Eine lokale Entwick-lungsstrategie (nicht nur) für Orte, die ‚nichts zählen‘. Open Research Seminar ICAE, Universität Linz, Österreich, 05.12.2019
	\item Aistleitner M.: Towards exploring the genesis of competition in economic thought. Open Research Seminar ICAE, Universität Linz, Österreich, 14.11.2019
\end{enumerate}
