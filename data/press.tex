\subsection*{2025}
  \begin{enumerate}
  	 \item Theine, Hendrik, Medienförderung: unabhängige, vielfältige und demokratische Medienordnung sichern,  Diskurs. Das Wissenschaftsnetz 2025-11-28
	 \item Theine, Hendrik, Am Kipppunkt?,  Der Falter 2025-11-28
	 \item Theine, Hendrik, Wissenschafter schlagen Alarm: Österreichs Mediensystem "am Kipppunkt",  Der Standard 2025-11-28
	 \item Theine, Hendrik, Medienwissenschafter sehen Journalismus am "Kipppunkt",  Kleine Zeitung 2025-11-28
	 \item Theine, Hendrik, A Paramount–Warner Bros. Discovery merger could give Trump even more influence over US media – shaping the news and culture Americans watch and stream,   2025-09-22
	 \item Theine, Hendrik, Wie Trump und Musk die Medien in Europa kontrollieren wollen,   2025-01-20
	 \item Theine, Hendrik, Rückverteilung: "Systemische Fragen wurden eher ausgeblendet",  Malte Kornfeld 2025-05-15
	 \item Theine, Hendrik // Verita, Carlotta, "Guter Rat für Rückverteilung". Wie Medien über Reichtum berichten,   2025-05-21
	 \item Theine, Hendrik, Umverteilung braucht Medienmacht,   2025-10-16
	 \item Hieselmayr, Sophie zitiert in Was die Zölle für ein Industriebundesland bedeuten,  Kronen Zeitung 2025-07-30
	 \item Hieselmayr, Sophie zitiert in "Wir legen einfach Zölle fest": Trump verschickt Briefe an Handelspartner,  Verena Mitterlechner 2025-07-11
	 \item Cserjan, Lukas // Eder, Julia Theresa // Hornykewycz, Anna // Porak, Laura // Pühringer, Stephan zitiert in Österreichs Bahnindustrie: Stille Heldin der Wirtschaft,  VOL.AT 2025-05-07
	 \item Cserjan, Lukas // Eder, Julia Theresa // Hornykewycz, Anna // Porak, Laura // Pühringer, Stephan zitiert in Österreichs Bahnindustrie: Stille Heldin der Wirtschaft,  VIENNA.AT 2025-05-07
	 \item Cserjan, Lukas // Eder, Julia Theresa // Hornykewycz, Anna // Porak, Laura // Pühringer, Stephan zitiert in Bahnindustrie für Anderl “stille Heldin” der Volkswirtschaft,  Vorarlberger Nachrichten 2025-05-07
	 \item Cserjan, Lukas // Eder, Julia Theresa // Hornykewycz, Anna // Porak, Laura // Pühringer, Stephan zitiert in Bahnindustrie für Anderl "stille Heldin" der Volkswirtschaft,  Salzburger Nachrichten 2025-05-07
	 \item Cserjan, Lukas // Eder, Julia Theresa // Hornykewycz, Anna // Porak, Laura // Pühringer, Stephan zitiert in "Die stille Heldin der heimischen Volkswirtschaft",  Elisabeth PrechtlOÖ Nachrichten 2025-05-07
	 \item Cserjan, Lukas // Eder, Julia Theresa // Hornykewycz, Anna // Porak, Laura // Pühringer, Stephan zitiert in Kann der Bahnsektor Österreichs Industrie retten?,  Franziska Schwarzprofil 2025-05-07
	 \item Cserjan, Lukas // Eder, Julia Theresa // Hornykewycz, Anna // Porak, Laura // Pühringer, Stephan zitiert in Auch die Eisenbahnindustrie fürchtet sich vor China,  Tiroler Tageszeitung 2025-05-07
	 \item Cserjan, Lukas // Eder, Julia Theresa // Hornykewycz, Anna // Porak, Laura // Pühringer, Stephan zitiert in Heimische Bahn - Industrie ist Weltspitze, doch Chinesen holen auf,  Krone 2025-05-07
	 \item Cserjan, Lukas // Eder, Julia Theresa // Hornykewycz, Anna // Porak, Laura // Pühringer, Stephan zitiert in Nicht nur bei E-Autos: Auch die Bahnindustrie fürchtet sich vor China,  Nicolas DworakDer Standard 2025-05-07
	 \item Pühringer, Stephan zitiert in Superreiche sind gut vernetzt – über ihr Vermögen ist aber oft wenig bekannt,  Standard 2025-07-10
	 \item Bäuerle, Lukas, Zukunft: Das wird gut,  Petra Pinzler, Stefan SchmittZEIT Online 2025-06-29
	 \item Cserjan, Lukas // Hornykewycz, Anna // Kapeller, Jakob // Schütz, Bernhard zitiert in Studie zu CO2-neutralen Gebäuden: Sozial gerechte Sanierung,  Jura Forum 2025-03-26
	 \item Cserjan, Lukas // Hornykewycz, Anna // Kapeller, Jakob // Schütz, Bernhard zitiert in Study on CO2 neutral buildings: Socially fair refurbishment.,  Informationsdienst Wissenschaft 2025-03-26
	 \item Cserjan, Lukas // Hornykewycz, Anna // Kapeller, Jakob // Schütz, Bernhard zitiert in Studie zu CO2-neutralen Gebäuden: Sozial gerechte Sanierung.,  Informationsdienst Wissenschaft 2025-03-26
	 \item Cserjan, Lukas // Hornykewycz, Anna // Kapeller, Jakob // Schütz, Bernhard zitiert in CO₂ neutral buildings: Study calls for accelerated energy renovation in building sector to achieve fossil-free goals.,  Phys.org 2025-03-26
	 \item Cserjan, Lukas // Hornykewycz, Anna // Kapeller, Jakob // Schütz, Bernhard zitiert in Studie zu CO2-neutralen Gebäuden: Sozial gerechte Sanierung.,  openPR 2025-03-26
	 \item Hornykewycz, Anna // Kapeller, Jakob, A socio-ecological transformation of the building sector.,  Earth4All 2025-05-06
	 \item Hornykewycz, Anna // Kapeller, Jakob, Zur Dekarbonisierung des deutschen Gebäudesektors.,  ifso Blog 2025-01-27
	 \item Bäuerle, Lukas, Akademie aktuell: Zukunftsfähiges Wirtschaften,  NDR infoNDR info 2025-05-11
	 \item Cserjan, Lukas // Eder, Julia Theresa // Hornykewycz, Anna // Porak, Laura, Aktive Industriepolitik schafft gute Arbeitsplätze,  Zeitschrift Wirtschaft und Umwelt 2025-03-15
	 \item Cserjan, Lukas, Superreiche: Wer zieht die Fäden?,  Christian BunkeArbeit & Wirtschaft 2025-03-04
  \end{enumerate}
\subsection*{2024}
  \begin{enumerate}
  	 \item Kapeller, Jakob, Was Vermögen steuern? Neue wissenschaftliche Befunde zu Vermögens- und Erbschaftssteuern,  Diskurs. Das Wissenschaftsnetz 2024-09-11
	 \item Gräbner-Radkowitsch, Claudius // Strunk, Birte, Quelle place pour le Sud global dans la décroissance?,  The Conversation 2024-03-07
	 \item Altreiter, Carina // Pühringer, Stephan zitiert in The (self-imposed) Limits of Science,  Economists For Future 2024-06-04
	 \item Aistleitner, Matthias // Cserjan, Lukas // Pühringer, Stephan zitiert in Das diskrete Netzwerk der Reichsten in Österreich,  Profil 2024-11-30
	 \item Pühringer, Stephan zitiert in Portrait zur Langen Nacht der Forschung,  Instagram 2024-01-24
	 \item Ötsch, Walter // Pühringer, Stephan zitiert in Wie man die AfD kleinkriegt,  SPIEGEL 2024-01-22
  \end{enumerate}
\subsection*{2019}
  \begin{enumerate}
  	 \item Gräbner-Radkowitsch, Claudius // Heimberger, Philipp // Kapeller, Jakob, Eine Strategie gegen die ökonomische Polarisierung Europas,  Makronom 2019-07-09
	 \item Heimberger, Philipp zitiert in Wirtschaftlicher Abschwung: Relevanz und Gegenmaßnahmen,  Momentum Newsletter 2019-09-11
	 \item Heimberger, Philipp // Kapeller, Jakob, Auch Deutschland wird zum Opfer des "Outputlücken-Nonsens",  Makronom 2019-10-14
	 \item Heimberger, Philipp, Den wirtschaftlichen Abschwung bekämpfen!,  Der Standard 2019-10-10
	 \item Heimberger, Philipp zitiert in The danger of 'nonsense Output gaps',  Financial Times 2019-06-13
	 \item Grimm, Christian // Pühringer, Stephan, Männlich, mikroökonomisch, Mainstream? Eine Untersuchung der VWL-Lehrstühle in Deutschland,  Makronom 2019-02-04
	 \item Heimberger, Philipp, Welche Rolle spielt der „Outputlücken-Nonsense“?,  Makronom 2019-06-06
  \end{enumerate}
\subsection*{2018}
  \begin{enumerate}
  	 \item Gräbner-Radkowitsch, Claudius // Heimberger, Philipp // Kapeller, Jakob, Warum Europa trotz Aufschwung ökonomisch weiter auseinander driftet,  Makronom 2018-03-15
	 \item Heimberger, Philipp, Hilf dir selbst, dann hilft dir Deutschland: Fortschritt bei der Reform der Eurozone,  Der Standard 2018-06-27
	 \item Heimberger, Philipp, Vier europäische Lehren aus den Turbulenzen in Italien,  Makronom 2018-06-07
	 \item Hirte, Katrin zitiert in Die Diskrepanz zwischen Reden und Handeln ist im ökonomischen Bereich besonders groß,  Agora42 (Philosophisches Wirtschaftsmagazin) 2018-01-12
	 \item Hirte, Katrin zitiert in Was Märkte (nicht) mit Demokratie zu tun haben,  Wege für eine Bäuerliche Zukunft – Zeitschrift der ÖBV/ Via Campesina Austria 2018-07-01
  \end{enumerate}
\subsection*{2017}
  \begin{enumerate}
  	 \item Heimberger, Philipp, Europa am Scheidepunkt: Österreichs wichtige Rolle,  Der Standard 2017-11-30
	 \item Heimberger, Philipp, Zerfällt Europa? Trump als Entscheidungsbeschleuniger,  Der Standard 2017-02-15
	 \item Heimberger, Philipp zitiert in Die EU braucht einen wirtschaftspolitischen Kurswechsel,  Der Standard 2017-01-01
	 \item Pühringer, Stephan, Migration: It's still the economy, stupid!,  Der Standard 2017-04-28
  \end{enumerate}
\subsection*{2023}
  \begin{enumerate}
  	 \item Gräbner-Radkowitsch, Claudius // Strunk, Birte, Degrowth and the Global South: remarks on the twin problem of structural interdependencies,  Developing economics blog 2023-11-06
	 \item Altreiter, Carina // Pühringer, Stephan, Die (selbstauferlegten) Grenzen der Wissenschaft,  Makronom 2023-11-13
	 \item Hirte, Katrin, Euphemistische Ökonomik,  Blog Postwachstum 2023-07-04
	 \item Gräbner-Radkowitsch, Claudius // Strunk, Birte, Degrowth und der globale Süden,  Blog Postwachstum 2023-10-23
	 \item Pühringer, Stephan, Prekäre Arbeit an Universitäten kann man nicht wegrechnen,  Der Standard 2023-03-31
  \end{enumerate}
\subsection*{2021}
  \begin{enumerate}
  	 \item Heimberger, Philipp, Beschäftigungsschutz erhöht die Arbeitslosigkeit nicht,  Ökonomenstimme 2021-01-11
	 \item Heimberger, Philipp, Die EU-Anleihen sind ein Zukunftsmodell für Europa,  Handelsblatt 2021-01-08
	 \item Heimberger, Philipp zitiert in Die Finanzglobalisierung verschärft die Einkommensungleichheit,  Handelsblatt 2021-06-22
	 \item Altreiter, Carina // Gräbner-Radkowitsch, Claudius // Pühringer, Stephan zitiert in Wie befristete Uni-Stellen Innovation verhindern,  Der Standard 2021-07-01
	 \item Heimberger, Philipp, Erhöhen Unternehmenssteuersenkungen das Wirtschaftswachstum?,  Ökonomenstimme 2021-06-21
	 \item Heimberger, Philipp zitiert in Die deutschen Inflationssorgen speisen sich aus einem verzerrten Geschichtsbild,  Handelsblatt 2021-05-17
	 \item Heimberger, Philipp zitiert in Der Nobelpreis für David Card hat die Mindestlohn-Befürworter in Deutschland gestärkt,  Handelsblatt 2021-11-09
	 \item Porak, Laura, Und ewig lockt der Wettbewerb,  Makronom 2021-09-09
	 \item Heimberger, Philipp, Höhere Staatsschulden = weniger Wachstum?,  Makronom 2021-12-01
	 \item Heimberger, Philipp zitiert in Wachstum durch Unternehmensteuersenkungen? Die FDP weckt übertriebene Hoffnungen,  Handelsblatt 2021-08-03
	 \item Altreiter, Carina // Gräbner-Radkowitsch, Claudius // Pühringer, Stephan, 20 Jahre gute Absichten in der Wissenschaftspolitik,  Die Presse 2021-08-22
	 \item Hirte, Katrin, Ökonomie und Ideologie,  OXI - Wirtschaft anders denken 2021-09-01
	 \item Heimberger, Philipp zitiert in Drei Gründe, warum Staatsschulden nicht zwingend problematisch sind,  Handelsblatt 2021-08-31
	 \item Heimberger, Philipp zitiert in Fakten-Mensch: Ökonom Philipp Heimberger kritisiert die aus seiner Sicht verzerrte Debatte über die EU,  Süddeutsche Zeitung 2021-03-11
	 \item Heimberger, Philipp zitiert in Arbeitnehmerrechte sind kein Jobkiller,  Handelsblatt 2021-02-07
	 \item Heimberger, Philipp, Wie stark der globale Steuerwettbewerb tatsächlich ist,  Makronom 2021-02-04
	 \item Heimberger, Philipp, Sieben „überraschende“ Fakten zu Italien,  Makronom 2021-02-15
	 \item Heimberger, Philipp zitiert in Draghi darf das Sparen nicht übertreiben,  Handelsblatt 2021-04-05
	 \item Gräbner-Radkowitsch, Claudius // Heimberger, Philipp // Kapeller, Jakob, Ökonomische Offenheit: Die Vermessung der Globalisierung?,  Ökonomenstimme 2021-02-22
	 \item Heimberger, Philipp zitiert in Eine globale Mindeststeuer stoppt die Steuerflucht der Konzerne,  Handelsblatt 2021-03-07
  \end{enumerate}
\subsection*{2022}
  \begin{enumerate}
  	 \item Kapeller, Jakob zitiert in Wechselseitige Kritik ist nur möglich mit nachvollziehbarer Forschung,  ZBW Leibniz Informationszentrum Wirtschaft 2022-12-29
	 \item Gräbner-Radkowitsch, Claudius, Zur ökonomischen Bedeutung von Suffizienz,  Makronom 2022-06-20
	 \item Aistleitner, Matthias zitiert in Verzerrter Wettbewerb in der Forschung,  science.orf.at 2022-10-14
	 \item Pühringer, Stephan, Arbeitsplatz Universität: Gefangen in der Teilzeit,  Der Standard 2022-11-27
	 \item Heimberger, Philipp // Schütz, Bernhard, Das wackelige Fundament der Schuldenbremse,  Makronom 2022-11-08
	 \item Hirte, Katrin, Strukturkonzentrationen in der Schlachthofbranche und die Rolle von Ökonomen,  Oxi - Wirtschaft anders denken 2022-01-21
  \end{enumerate}
\subsection*{2016}
  \begin{enumerate}
  	 \item Heimberger, Philipp, Austeritätspolitik in der Eurozone: Ein Schuss ins eigene Knie,  Makronom 2016-11-08
	 \item Ötsch, Walter, Der Blick von oben und der Blick von unten,  Makroskop 2016-08-08
	 \item Ötsch, Walter, Ökonomisches Denken, Rechtspopulismus und Rechtsextremismus,  Makroskop 2016-06-23
	 \item Heimberger, Philipp zitiert in Investitionen gegen die Dauerkrise im Euroland,  Der Standard 2016-11-17
	 \item Ötsch, Walter, Markt-Glauben, Klima-Krise und Katastrophen-Leugnung, Teil 1-4,  Makroskop 2016-10-03
	 \item Heimberger, Philipp, Trumps Team: Politik von und für Vermögende,  Die Presse 2016-12-19
  \end{enumerate}
\subsection*{2015}
  \begin{enumerate}
  	 \item Pühringer, Stephan zitiert in Märkte als Richter: Zur Dominanz neoliberaler Krisennarrative,  Ksoe-Dossier (Ksoe Nachrichten der Katholischen Sozialakademie) 2015-01-01
  \end{enumerate}
\subsection*{2013}
  \begin{enumerate}
  	 \item Ötsch, Walter, Das Team Stronach: Die österreichische Tea Party,  Die Presse 2013-10-20
  \end{enumerate}
\subsection*{2020}
  \begin{enumerate}
  	 \item Hirte, Katrin zitiert in Agrarsubventionen als Preis der Marktwirtschaft,  Wege für eine Bäuerliche Zukunft – Zeitschrift der ÖBV/ Via Campesina Austria 2020-03-01
	 \item Heimberger, Philipp // Kapeller, Jakob, Wie die ‚Herren der Modelle‘ versuchen, den Outputlücken-Nonsense zu rechtfertigen,  Makronom 2020-01-06
	 \item Heimberger, Philipp, Einigung der Eurogruppe: Bestenfalls ein erster Schritt,  Makronom 2020-04-10
	 \item Heimberger, Philipp zitiert in Gleichwertige Lebensverhältnisse im Euroraum,  Frankfurter Allgemeine Zeitung 2020-04-01
	 \item Heimberger, Philipp, Enttäuschender EU-Gipfel: Zeit für eine „Neue Südpolitik”,  Makronom 2020-04-27
	 \item Heimberger, Philipp, Die gefährliche Geschichtsverdrehung des Hans-Werner Sinn,  Handelsblatt 2020-12-17
	 \item Schütz, Bernhard zitiert in Gewohnter Umsatz wird auf sich warten lassen,  Kronen Zeitung, Interview (von Elisabeth Rathenböck) mit Bernhard Schütz 2020-05-17
	 \item Heimberger, Philipp, Kein Weg in die „Schuldenunion“,  Der Standard 2020-05-01
	 \item Heimberger, Philipp, Der Outputlücken-Nonsense gefährdet Deutschlands Erholung von der Corona-Krise,  Makronom 2020-06-02
	 \item Heimberger, Philipp zitiert in Zu niets doen voor Z-Europa is gevaarlijk (Nichts für Südeuropa zu tun ist gefährlich),  NRC Handelsblad 2020-05-01
	 \item Heimberger, Philipp, Beschäftigungsschutz ist kein Jobkiller,  Makronom 2020-12-03
	 \item Hirte, Katrin, „Um welches Ziel es in der Ökonomie geht, ist also bestimmbar“,  Agora42 (Philosophisches Wirtschaftsmagazin) 2020-09-06
	 \item Kapeller, Jakob zitiert in Wahrheiten über Wertfreiheit – eine Replik auf den Gastkommentar ‚Werte, Wahrheit, Wirtschaftsforschung‘ von Harald Oberhofer,  Der Standard 2020-08-01
	 \item Gräbner-Radkowitsch, Claudius // Heimberger, Philipp // Kapeller, Jakob, Ökonomische Offenheit: Die Vermessung der Globalisierung,  Makronom 2020-10-01
	 \item Heimberger, Philipp, Wie ein Rechenfehler zu fatalen Konsequenzen für die Haushalte der EU-Staaten führen kann,  Handelsblatt 2020-10-12
	 \item Heimberger, Philipp, Wie die EU-Kommission Deutschlands Budgetsituation schlechtrechnet,  Handelsblatt 2020-11-16
	 \item Gräbner-Radkowitsch, Claudius // Strunk, Birte, Kritik an der Pluralen Ökonomik – Was ist dran und warum ist das wichtig?,  Ökonomenstimme 2020-11-16
  \end{enumerate}
