\subsection*{2026}
\begin{enumerate}
	\item Bäuerle, L., Barkey, M., Sagvosdkin, V., \& Semb, J. (2026). “We need long-term thinking, predictability, and reliability.”: Imagined futures of the German heating and housing transition. Futures, 175, Artikel 103736. https://doi.org/10.1016/j.futures.2025.103736
	\item Sagvosdkin, V., Bäuerle, L., Semb, J., \& Praetorius, B. (2026). To build or not to build? Sustainability narratives between growth-oriented ‘eco-markets’ and regulative ‘sufficiency’ in Germany's heat transition. Energy Research and Social Science, 131, 104527. Artikel 104527. https://doi.org/10.1016/j.erss.2025.104527
\end{enumerate}
\subsection*{2025}
\begin{enumerate}
	\item Aistleitner, M. (2025). World Development and interdisciplinarity: re-examining the economics silo. In A. Yalçıntaş, \& A. Heise (Hrsg.), Decolonial Narratives in Economics: Alternative and Underrepresented Voices (1 Aufl., S. 16-36). Edward Elgar Publishing. https://doi.org/10.4337/9781035329649.00010
	\item Altreiter, C., Gräbner-Radkowitsch, C., Pühringer, S., \& Rogojanu, A. et al. (2025). Theorizing competition: An interdisciplinary framework. Competition \& Change, 29(5), 593-614. https://doi.org/10.1177/10245294251330343
	\item Braganza, O., \& Kapeller, J. (2025). Reappraising consumption nudging—on liberty in the age of climate catastrophe. Humanities and Social Sciences Communications, 12(1), Artikel 251. https://doi.org/10.1057/s41599-024-04246-0
	\item Bäuerle, L., \& Groenewald, M. M. (2025). From monoculture to pluricultures. Recent trends in economic education. In The Routledge Handbook of Heterodox Economics: Volume 2: Dynamics and Alternatives (2 Aufl., Band 2, S. 353-368). Routledge Taylor \& Francis Group. https://doi.org/10.4324/9781003687054-28
	\item Bäuerle, L., Barkey, M., Sagvosdkin, V., \& Semb, J. (2025). “We Need Long-Term Thinking, Predictability, and Reliability”: Imagined Futures of the German Heating and Housing Transition. https://doi.org/10.2139/ssrn.5180335
	\item Drahoss, E., \& Theine, H. (2025). “Taxing the Palaces”: Dissecting the Media Discourse of Red Vienna’s Wealth Tax Policy. Journalism History. Vorzeitige Online-Publikation. https://doi.org/10.1080/00947679.2025.2540342
	\item Eder, J. (2025). Mobilitätswende produzieren – Eine Allianz für den europäischen Bahnausbau schaffen. EU-Infobrief, 2025(3), 18-25. https://emedien.arbeiterkammer.at/viewer/image/AC05712646\_2025\_3/18/\#topDocAnchor
	\item Eder, J., \& Porak, L. (im Druck). Die EU zwischen strategischer Autonomie und Unterordnung unter die USA. In Multiple Krise und neue Konstellationen des Kapitalismus: Elemente einer Bestandsaufnahme (1 Aufl.). Verlag Westfälisches Dampfboot. 
	\item Eder, J., \& Rammer, J. (2025). Greening the European economy at the expense of other world regions? Tracing the EU’s quest for green hydrogen in Chile. In Critical Political Economy of the European Polycrisis (1 Aufl., S. 209-224). Edward Elgar Publishing. https://doi.org/10.4337/9781035347940\_14
	\item Eder, J., Pühringer, S., \& Cserjan, L. (2025). Mobilitätswende produ­zieren: Warum die Bahn­industrie jetzt im Zentrum stehen muss. Blog Beitrag
	\item Gerold, S., Gruszka, K., Sardadvar, K., \& Theine, H. et al. (2025). Outsourcing Domestic Work in the Crisis of Social Reproduction: Platform-Mediated Cleaning and the Role of Clients. New Technology, Work and Employment, 1-12. Vorzeitige Online-Publikation. https://doi.org/10.1111/ntwe.70015
	\item Hager, T., \& Pühringer, S. (2025). Gendered Competitive Practices in Economics: A Multi-Layer Model of Women's Underrepresentation. In S. Pühringer, J. Maesse, \& T. Rossier (Hrsg.), The Power of Rankings in Economics and Research Organizations: Contributions to the Social Studies of Economics (1 Aufl., S. 78 -- 100) https://doi.org/10.4324/9781032637310-7
	\item Hager, T., Mellacher, P., \& Rath, M. (2025). Gender Norms and Network Structure: A Model of the Intrahousehold Division of Labor. Feminist Economics, 31(3), 237-274. https://doi.org/10.1080/13545701.2025.2549406
	\item Hofbauer, J., Kreissl, K., \& Pühringer, S. (2025). „Der Weg zur Zerstörung ist mit guten Absichten gepflastert.”: Über die ökonomischen, sozialen und epistemischen Kosten von Reformen im Wissenschafts- und Universitätssystem. Kurswechsel, (2), 12-22.
	\item Hornykewycz, A., Kapeller, J., Weber, J. D., \& Schütz, B. et al. (2025). Carbon neutrality in the residential sector: a general toolbox and the case of Germany. npj Climate Action, 4(1). https://doi.org/10.1038/s44168-025-00229-2
	\item Hornykewycz, A., Kapeller, J., Weber, J. D., \& Schütz, B. et al. (2025). Toolbox Buildings. Software
	\item Polkowski, J., Theine, H., \& Krüger, U. (2025). Gemeinwohl oder Privateigentum? Die Positionen der deutschsprachigen Presse in der Debatte über eine Freigabe der Corona-Impfstoffpatente. Studies in Communication and Media, 14(2), 234-260. https://doi.org/10.5771/2192-4007-2025-2-234
	\item Popiel, P., Ali, C., Theine, H., \& Forde, S. L. (2025). Performative competition: The U.S. wireless communication market and the T-Mobile/Sprint merger. International Communication Gazette. Vorzeitige Online-Publikation. https://doi.org/10.1177/17480485251405601
	\item Porak, L. (2025). Displacing ordoliberalism in favour of EU sovereignty: An analysis of green EU industrial policy from a Cultural Political Economy perspective. Der moderne Staat, 18(1), 17-34. https://doi.org/10.3224/dms.v18i1.02
	\item Porak, L., \& Reinke, R. (2025). The Charm of Emission Trading. Journal of Economic Issues.
	\item Pühringer, S., \& Wolfmayr, G. (2025). Organizers and promoters of academic competition? The role of (academic) social networks and platforms in the competitization of science. In The Power of Rankings in Economics and Research Organizations : Contributions to the Social Studies of Economics  (1 Aufl., S. 192-210). Routledge Taylor \& Francis Group. https://doi.org/10.4324/9781032637310-14
	\item Pühringer, S., Maesse, J., \& Rossier, T. (2025). The power of rankings in economics and research organizations. an Introduction. In The Power of Rankings in Economics and Research Organizations: Contributions to the Social Studies of Economics (1 Aufl., S. 3-14). Routledge Taylor \& Francis Group. 
	\item Reinke, R., \& Porak, L. (2025). On the Nexus of Economic Knowledge Production and Expertise in Policymaking. Historical Social Research, 50(2), 376-404. https://doi.org/10.12759/hsr.50.2025.28
	\item Reinke, R., \& Porak, L. (2025). “The Charm of Emission Trading”: Ideas of German Economists on Economic Policy in Times of Crises. Journal of Economic Issues, 59(4), 1191-1211. https://doi.org/10.1080/00213624.2025.2575141
	\item Sachverständigenkommission für den Vierten Gleichstellungsbericht der Bundesregierung (2025). Gleichstellung in der sozial-ökologischen Transformation: Gutachten für den Vierten Gleichstellungsbericht der Bundesregierung. Bundesstiftung Gleichstellung. https://www.gleichstellungsbericht.de/wp-content/uploads/2025/04/GutachtenVierterGleichstellungsbericht\_WEB\_20250303\_bf\_V2.pdf
	\item Semb, J., Bäuerle, L., \& Sagvosdkin, V. (2025). System-induced transition inertia in the transformation of the German heating and housing sector. https://doi.org/10.2139/ssrn.5406252
	\item Sevignani, S., \& Theine, H. (2025). Media Property: New Explorations in Media and Communication Studies. International Journal of Communication, 19, 797-803. https://ijoc.org/index.php/ijoc/article/download/24428/4920
	\item Sevignani, S., Theine, H., \& Tröger, M. (2025). Towards Media Environment Capture: A Theoretical Contribution on the Influence of Big Tech on News Media. International Journal of Communication, 19, 804-824. https://ijoc.org/index.php/ijoc/article/download/21987/4921
	\item Teebken, J., Klüh, U., Kleinod, S., \& Theine, H. (2025). Unterschätzte Synergien zwischen Vermögensbesteuerung und Klimawandelanpassung: Klimapolitik: Drei Fliegen mit einer Klappe. Ökologisches Wirtschaften, 40(4), 45-50. https://doi.org/10.14512/OEW400445
	\item Theine, H., \& Verita, C. (2025). Informing Philadelphia -- Strengths and Gaps in Local Media’s Coverage of Critical Information Needs. https://www.asc.upenn.edu/sites/default/files/2025-09/Informing\%20Philadelphia\%20White\%20Paper.pdf
	\item Theine, H., Bartsch, J., \& Tröger, M. (2025). Does Media Ownership Matter for Journalistic Content? A Systematic Scoping Review of Empirical Studies. International Journal of Communication, 19, 865-888. https://ijoc.org/index.php/ijoc/article/download/21986/4924
	\item Theine, H., Friedrich, J., Grabner, D., \& Ferschli, B. (2025). Die Krise intensiviert sich? Medienökonomische Perspektiven auf die \glqq vierte Gewalt" in Österreich: Medienökonomische Perspektiven auf die "vierte Gewalt\grqq{} in Österreich. Wirtschaft und Gesellschaft, 50(4), 21-52. https://doi.org/10.59288/wug504.264
	\item Tröger, M., \& Theine, H. (2025). On the Impact of the Trump Administration on Media and Media Regulation in Europe. Journalism Research, 8(1), 94-106. https://doi.org/10.1453/2569-152X-12025-15004-en
	\item Vaughan, M., Theine, H., Schieferdecker, D., \& Waitkus, N. (2025). Communication about economic inequality: a systematic review. Annals of the International Communication Association, 49(3), 147-158. https://doi.org/10.1093/anncom/wlaf006
	\item Wansleben, L., \& Terhorst, C. (2025). Ungleichheiten in der grünen Transformation: Die prädistributive Macht von Infrastrukturen. Zeitschrift für Soziologie, 54(3), 342-361. https://doi.org/10.1515/zfsoz-2025-2023
	\item Barbutev, L., Theine, H., Mast, T., \& Spannuth, J. S. (2025). Media, Telecommunication, and Internet Concentration in Germany 2019-2023. (Global Media and Internet Concentration Project). https://doi.org/10.22215/gmicp/2025.12.276
	\item Cserjan, L., Eder, J. T., Hornykewycz, A., \& Porak, L. et al. (2025). Mobilitätswende produzieren: Produktionsbedingungen der österreichischen Bahnindustrie und industrielle Potenziale durch den Ausbau des öffentlichen Verkehrs. (Verkehr und Infrastruktur; Band 73). Verlag Arbeiterkammer Wien. https://wien.arbeiterkammer.at/interessenvertretung/umweltundverkehr/verkehr/unsere-bahnen/Mobilitaetswende\_produzieren.pdf
	\item Eder, J., \& Porak, L. (2025). Die EU zwischen strategischer Autonomie und Unterordnung unter die USA. (ICAE Working Papers; Band 165).
	\item Eder, J., \& Rammer, J. (2025). Greening the European economy at the expense of other world regions? Tracing the EU's quest for green hydrogen in Chile. (ICAE Working Paper; Band 168). https://www.jku.at/fileadmin/gruppen/108/ICAE\_Working\_Papers/wp168.pdf
	\item Guerriero, A. Z., \& Kapeller, J. (2025). The Global Perspective on Income Inequality. In Global Inequality: Rethinking Sociology in the 21st Century (1 Aufl., S. 69-95). (Studies in Critical Social Sciences; Band 329). Brill. https://doi.org/10.1163/9789004733626\_004
	\item Kapeller, J., \& Özgür, M. E. (Hrsg.) (2025). Economic Ideas Across Borders: A History of German Influence on Turkish Economics. (1 Aufl.) (Routledge Studies in the History of Economics). Routledge. https://doi.org/10.4324/9781003622000
	\item Pühringer, S., Rossier, T., \& Maesse, J. (Hrsg.) (2025). The Power of Rankings in Economics and Research Organizations: Contributions to the Social Studies of Economics. (1 Aufl.) (Frontiers in Political Economy). Routledge Taylor \& Francis Group. https://doi.org/10.4324/9781032637310
	\item Theine, H., Verita, C., \& Gartiser, M. (2025). Über Reichtum berichten. Der \glqq Gute Rat für Rückverteilung\grqq{} in den Medien. (OSB-Arbeitspapier; Nr. 79). https://www.otto-brenner-stiftung.de/ueber-reichtum-berichten/
\end{enumerate}
\subsection*{2024}
\begin{enumerate}
	\item Altreiter, C., \& Litschauer, K. (2024). Strategies of Capital Accumulation in Times of Land Scarcity. A Field Perspective on Social Housing Construction in Vienna. Forum for Social Economics, 53(2), 216-232. https://doi.org/10.1080/07360932.2022.2125423
	\item Altreiter, C., Pühringer, S., \& Völkl, Y. (2024). Politiken der Verwettbewerblichung: Einblicke ins im österreichischen Wissenschaftssystem und darüber hinaus. In BdWi (Hrsg.), Umkämpfte Wissenschaftsfreiheit – Verhältnis von Wissenschaft \& Politik: 14. Studienheft des Bundes demokratischer Wissenschaftlerinnen und Wissenschaftler (Band 14, S. 29-32)
	\item Azevedo, S., Hager, T., \& Porak, L. (2024). Kritische Utopien als Methode und Praxis: Der Forschungsstandort Österreich. In Momentum Kongress Paper 
	\item Buyinza, F., Kapeller, J., Senono, V., \& Anber, M. (2024). Differential impacts of electricity access on educational outcomes: Evidence from Uganda. The Electricity Journal, 37(1), 1-10. Artikel 107362. https://doi.org/10.1016/j.tej.2023.107362
	\item Bäuerle, L., \& Graupe, S. (2024). Enacting Economic Transformation. The Transformative Economic Capabilities (TEC) Approach. World Futures, 81(4), 225-254. https://doi.org/10.1080/02604027.2024.2417040
	\item Cserjan, L., Hieselmayr, S., Aistleitner, M., \& Pühringer, S. (2024). Netz­wer­ke der Super­rei­chen: Zur poli­ti­schen Öko­no­mie der Ver­mö­gens­kon­zen­tra­ti­on in Österreich. Kurswechsel, 2024(3). https://www.beigewum.at/kurswechsel/netzwerke-der-superreichen-zur-politischen-oekonomie-der-vermoegenskonzentration-in-oesterreich/
	\item Deindl, R., Hager, T., \& Schäfer, K. (2024, Okt). Zerrissen, überlastet, prekär – Mittelbau zwischen Doktoratsstudium und universitärer Lehre.
	\item Disslbacher, F., Hornykewycz, A., \& Kapeller, J. (2024, Nov). Vermögen in Österreich: stark konzen­triert, gering be­steuert.
	\item Finkeldey, J., Fischer, T., Theine, H., \& Bohnenberger, K. (2024). Erratum to: The Politics of Germany’s eco-social transformation. Zeitschrift für Politikwissenschaft. https://doi.org/10.1007/s41358-024-00391-9
	\item Finkeldey, J., Fischer, T., Theine, H., \& Bohnenberger, K. (2024). The Politics of Germany’s eco-social transformation. Zeitschrift für Politikwissenschaft. https://doi.org/10.1007/s41358-024-00389-3
	\item Gampe, A., Hubmann, G., \& Kapeller, J. (2024). Sozialer Fortschritt in offenen Gesellschaften des 21. Jahrhunderts: Unrealistische Utopie oder notwendige Möglichkeit? In Gampe, Anja; Söylemez, Seçkin (Hrsg.), Weltoffenheit, Toleranz und Gemeinsinn. Chancen und Herausforderungen in der Gegenwartsgesellschaft (S. 13-38). transcript. 
	\item Griesser, M., Beyer, K., \& Pühringer, S. (2024). Für die "Leistungsträger“ und „uns Österreicher“: Eine Mediendiskursanalyse zu Sozialpolitikreformen der ÖVP/FPÖ-Regierung 2017–2019 in Österreich. Momentum Quarterly, 13(1), 33-55.
	\item Gruszka, K., Pillinger, A., Gerold, S., \& Theine, H. (2024). (In)visible by Design: An Analysis of a Domestic Labor Platform. Critical Sociology. https://doi.org/10.1177/08969205241276803
	\item Kapeller, J. (2024). Die klassische ökonomische Verhaltenstheorie und ihre Grenzen. In Wirtschaft plural: Perspektiven Pluraler Ökonomik in der politischen Bildung Bundeszentrale für politische Bildung. https://www.bpb.de/system/files/dokument\_pdf/TuM\_Wirtschaft\_plural\_Handreichung.pdf
	\item Kapeller, J. (2024, Jän). Vermögensriesen und ein Heer von Zwergen.
	\item Kapeller, J., \& Gräbner-Radkowitsch, C. (2024). Development models in the EU: opportunities and challenges. In U. Glassmann, \& C. Gräbner-Radkowitsch (Hrsg.), Jahrbuch Ökonomie und Gesellschaft (Band 35, S. 49-80)
	\item Kapeller, J., Gräbner-Radkowitsch, C., \& Hornykewycz, A. (2024). Corporate power and global value chains: current approaches for conceptualizing the power of multinationals. Review of Evolutionary Political Economy, 5(2), 371-397. https://doi.org/10.1007/s43253-024-00121-5
	\item Kapeller, J., Hornykewycz, A., Weber, J., \& Cserjan, L. (2024). Dekarbonisierung des Gebäudesektors als Teil einer sozial-ökologischen Transformation. Ein Gestaltungsvorschlag. ifso expertise, 25, 1-23. https://www.uni-due.de/imperia/md/images/soziooekonomie/ifsoexp25\_khwc2024\_dekarbonisierung\_gebaeudesektor\_v2.1.pdf
	\item Kapeller, J., Pühringer, S., Rath, J., \& Aistleitner, M. (2024). Proxy failures in practice: Examples from the sociology of science. Behavioral and Brain Sciences, 47, Artikel e76. https://doi.org/10.1017/S0140525X23002856
	\item Porak, L. (2024, Feb). Globale geo-ökonomische Unordnung: Europa braucht industriepolitische Antworten.
	\item Porak, L., \& Reinke, R. (2024). The contribution of qualitative methods to economic research in an era of polycrisis. Review of Evolutionary Political Economy, 5(1), 31-49. https://doi.org/10.1007/s43253-024-00116-2
	\item Porak, L., Gräbner-Radkowitsch, C., Kapeller, J., \& Rath, J. (2024). Pluralismus in der Volkswirtschaftslehre und seine Relevanz für die Debatte um Armut. In Beiträge für eine zukunftsfähige Wirtschafts- und Finanzbildung (S. 75-86)
	\item Porak, L., Premrov, T., \& Witzani-Haim, D. (2024). Nach der Wahl: Ist mehr Wohl­stand trotz dro­hen­der Aus­teri­tät möglich? Der Kurswechsel.
	\item Pühringer, S., \& Wolfmayr, G. (2024). Competitive Performativity of Academic Social Networks. The Subjectivation of Competition on ResearchGate? Research Evaluation, 33, Artikel rvae048. https://doi.org/10.1093/reseval/rvae048
	\item Rath, J., Hornykewycz, A., \& Burnazoglu, M. (2024). Power in Economics without Power in Economics? Review of Evolutionary Political Economy, 5, 301-328.
	\item Schlaile, M. P., Hector, V., Peters, L., \& Bäuerle, L. et al. (2024). Innovation Amidst Turmoil: A SenseMaker Study of Managerial Responses to the COVID-19 Crisis in Germany. Journal of Innovation Economics and Management, 43(1), 285-318. https://doi.org/10.3917/jie.pr1.0154
	\item Solveign, D., Müller, H., Porak, L., \& Winkin, M. R. (2024). Vergesellschaftung zukunftsweisend gestalten. In Vergesellschaftung und die sozial- ökologische Frage Wie wir unsere Gesellschaft gerechter, zukunftsfähiger und resilienter machen können 
	\item Theine, H., \& Sevignani, S. (2024). Introduction to the special issue: Media transformation and the challenge of property. European Journal of Communication, 39(5), 407–411. https://doi.org/10.1177/02673231241274347
	\item Theine, H., \& Sevignani, S. (2024). Media property: Mapping the field and future trajectories in the digital age. European Journal of Communication, 39(5), 412–425.
	\item Ötsch, W. (Hrsg.) (2024). Theoriegeschichte der Ökonomie als Imaginationsgeschichte. In Ötsch Walter O., Priddat Birger P., Groß Steffen W. (Hrsg.), Das Imaginative der Politischen Ökonomie. Metropolis Verlag. 
	\item Hager, T. (2024). Lobbying and Macroeconomic Development. In Mause, Karsten; Polk, Andreas (Hrsg.), The Political Economy of Lobbying (Band 43, S. 77-99). (Studies in Public Choice). Springer. 
	\item Heck, I., Hornykewycz, A., Kapeller, J., \& Wildauer, R. (2024). Vermögensverteilung in Österreich: Eine Analyse auf Basis des HFCS 2021/22. (1 Aufl.) (S. 1-34). (Materialien zu Wirtschaft und Gesellschaft -- Working Paper-Reihe der AK Wien). https://emedien.arbeiterkammer.at/viewer/image/AC17296950/
	\item Pühringer, S., Aistleitner, M., Cserjan, L., \& Hieselmayr, S. et al. (2024). Idiosyncrasies of the superrich: On the political economy of wealth concentration in Austria. (Materialien zu Wirtschaft und Gesellschaft -- Working Paper-Reihe der AK Wien). https://emedien.arbeiterkammer.at/viewer/image/AC17367868/
	\item Hirte, K. (2024). Performativität: Geplante Landwirtschaftsstrukturen – das Beispiel Böckenhoff-Plan. In Gruber, Holle; Henkel, Anna; Scheel, Laura (Hrsg.), Land: Digitalisierung, Agrarwandel, Energiewende – soziologische Perspektiven zu ländlichen Räumen im Umbruch (S. 87-100). (10 Minuten Soziologie). trancript. https://doi.org/10.14361/9783839465233-007
	\item Ötsch, W., \& Hilt, A. (Hrsg.) (2024). Das Imaginative der Politischen Ökonomie. (Kritische Studien zu Markt und Gesellschaft). Metropolis Verlag.
\end{enumerate}
\subsection*{2023}
\begin{enumerate}
	\item Aistleitner, M., \& Pühringer, S. (2023). Biased Trade Narratives and Its Influence on Development Studies: A Multi-level Mixed-Method Approach. European Journal of Development Research, 35(6), 1322-1346. https://doi.org/10.1057/s41287-023-00583-z
	\item Aistleitner, M., \& Pühringer, S. (2023). L´expertise parziale dell´economics: il caso della ricerca (delle politiche) sul commercio. In Nicoletta, Gerardo C.; di Carlo, Michele S.; Ventrone, Oreste (Hrsg.), Economisti e Società. Nuove sociologie dell'expertise economica (S. 25-40). Liguori Editore. 
	\item Aistleitner, M., \& Pühringer, S. (2023). The Social Field of Elite Trade Economists: A Quantitative Social Studies of Economics Perspective. Oeconomia, 13(2), 475-515. https://doi.org/10.4000/oeconomia.15808
	\item Aistleitner, M., Kapeller, J., \& Kronberger, D. (2023). The authors of economics journals revisited: evidence from a large-scale replication of Hodgson and Rothman (1999). Journal of Institutional Economics, 19(1), 86–101. https://doi.org/10.1017/S174413742200025X
	\item Altreiter, C., Azevedo, S., Porak, L., \& Pühringer, S. et al. (2023). Winning city competition with a social agenda. The competition imaginary in Viennese urban development plans. Urban Research \& Practice, 17(2), 240-259. https://doi.org/10.1080/17535069.2022.2161834
	\item Benz, P., Maesse, J., Pühringer, S., \& Rossier, T. (2023). Economics and Power. In Macknight, Viski; Medvecky, Fabien (Hrsg.), Making Economics Public (S. 18-25). Routledge. 
	\item Benz, P., Maesse, J., Pühringer, S., \& Rossier, T. (2023). Il potere e l'economics. In Nicoletta, Gerardo C.; di Carlo, Michele S.; Ventrone, Oreste (Hrsg.), Economisti e Società. Nuove sociologie dell'expertise economica (S. 17-24). Liguori Editore. 
	\item Bäuerle, L., \& Graupe, S. (2023). Reframing economic agency in times of uncertainty. International Journal of Pluralism and Economics Education, 14(1), 31-46. https://doi.org/10.1504/IJPEE.2023.133642
	\item Dammerer, Q., Hubmann, G., \& Theine, H. (2023). Wealth taxation in the Austrian Press from 2005 to 2020: a critical political economy analysis. Cambridge Journal of Economics, 47(3), 633-665. https://doi.org/10.1093/cje/bead011
	\item Graupe, S., \& Bäuerle, L. (2023). Die Spirale transformativen Lernens. In M.-A. Heidelmann, V. Storozenko, \& S. Wieners (Hrsg.), Forschungsdiskurs und Etablierungsprozess der Organisationspädagogik: Theorien, Methodologien und Methodiken im pluralen Diskurs einer erziehungswissenschaftlichen Subdisziplin (S. 223-241). Springer Fachmedien Wiesbaden. https://doi.org/10.1007/978-3-658-40997-5\_15
	\item Gräbner-Radkowitsch, C. (2023). Elements of an evolutionary approach to comparative economic studies: Complexity, Systemism, and Path-Dependent Development. In Casagrande, Sara; Dallago, Bruno (Hrsg.), The Routledge Handbook of Comparative Economic Systems (S. 81-102). Routledge. https://doi.org/10.4324/9781003144366-6
	\item Gräbner-Radkowitsch, C., \& Strunk, B. (2023). Degrowth and the Global South: The Twin Problem of Global Dependencies. Ecological Economics, 213, Artikel 107946. https://doi.org/10.1016/j.ecolecon.2023.107946
	\item Gräbner-Radkowitsch, C., \& Strunk, B. (2023). Degrowth and the Global South? How Institutionalism can Complement a Timely Discourse on Ecologically Sustainable Development in an Unequal World. Journal of Economic Issues, 57(2), 476-483. https://doi.org/10.1080/00213624.2023.2201640
	\item Gräbner-Radkowitsch, C., Hager, T., \& Hornykewycz, A. (2023). Competing for Sustainability? An Institutionalist Analysis of the New Development Model of the European Union. Journal of Economic Issues, 57(2), 676-683. https://doi.org/10.1080/00213624.2023.2203637
	\item Hager, T. (2023). Lobbyismus und gesamtwirtschaftliche Entwicklung. In Andreas Polk, Karsten Mause (Hrsg.), Handbuch Lobbyismus (S. 817–842). Springer VS. 
	\item Hager, T., \& Pühringer, S. (2023, Jän). Im Netz der Einfluss-Reichen.
	\item Hager, T., Mellacher, P., \& Rath, M. (2023). Endogenous Heterogeneous Gender Norms and the Distribution of Paid and Unpaid Work in an Intra-Household Bargaining Model.
	\item Hubmann, G., \& Kapeller, J. (2023). Rezension zu Markus Marterbauer/Martin Schürz: Angst und Angstmacherei. WISO -- Wirtschafts- und sozialpolitische Zeitschrift, 46(1), 102-109.
	\item Kapeller, J., Wildauer, R., \& Leitch, S. (2023). Can a European wealth tax close the green investment gap? Ecological Economics, 209(209), Artikel 107849. https://doi.org/10.1016/j.ecolecon.2023.107849
	\item Litschauer, K., Kumnig, S., Kohout, R., \& Wolfmayr, G. et al. (2023). Die Bereitstellung von leistbarem Wohnraum in Zeiten der Wohnkrise: Deutungen und Praktiken der Wiener Gemeinnützigkeit. DISP, 59(3), 33-49. https://doi.org/10.1080/02513625.2023.2288448
	\item Ornetzeder, M., Pichler, M., Madner, V., \& Görg, C. et al. (2023). Kapitel 10: Integrierte Perspektiven auf Strukturbedingungen. In C. Görg, V. Madner, A. Muhar, A. Novy, A. Posch, K. W. Steininger, \& E. Aigner (Hrsg.), APCC Special Report: Strukturen für ein klimafreundliches Leben (1 Aufl., S. 347-349). Springer Spektrum. https://doi.org/10.1007/978-3-662-66497-1\_14
	\item Porak, L. (2023). Political sovereignty in tension with global capitalist accumulation: the case of the European socio-economic strategy. Critical Policy Studies, 18(3), 490-513. https://doi.org/10.1080/19460171.2023.2274542
	\item Porak, L. (2023). Wettbewerbsfähige Nachhaltigkeit: eine Historisch-Materialistische Analyse der Ideen, Institutionen und Machtverhältnisse in der europäischen grünen Transformation. Momentum Quarterly, 12(1), 65-83. https://doi.org/10.15203/momentumquarterly.vol12.no1.p65-83
	\item Porak, L., \& Schamberger, K. (2023). Islands in the Privately Dominated Sea of Capitalist Media. In Güney, Selma; Hille, Lina; Pfeiffer, Juliane; Porak, Laura; Theine, Hendrik (Hrsg.), Eigentum, Medien, Öffentlichkeit (S. 443-449). Westend Verlag. 
	\item Pühringer, S. (2023). Soziale und ökologische Probleme müssen zusammen betrachtet werden. In Renner Institut (Hrsg.), Ein aktiver Staat, der die Menschen stärkt und schützt (S. 24-27)
	\item Pühringer, S. (2023, Mai). Man ist akademische Einzelunternehmer*in.
	\item Pühringer, S., \& Altreiter, C. (2023, Jän). Woran scheitert transformative Wissensproduktion?
	\item Rieder, M., \& Theine, H. (2023). Breaking Down the Discourse, Exposing Power in Economic Journalism: Critical Discourse Analysis. In H. Silke, F. Quinn, \& M. Rieder (Hrsg.), How to Read Economic News: A Critical Approach to Economic Journalism (S. 192-225). Routledge. https://doi.org/10.4324/9781003154747-11
	\item Steffestun, T., \& Ötsch, W. (2023). Economization: The (re-)organization of knowledge and ignorance according to ‘the market’. ephemera, 23(1), 133-159.
	\item Theine, H. (2023). Economic Imaginaries, Economics Theories and the Role of Economic Journalism. In H. Silke, F. Quinn, \& M. Rieder (Hrsg.), How to Read Economic News: A Critical Approach to Economic Journalism (S. 27-51). Routledge. https://doi.org/10.4324/9781003154747-3
	\item Theine, H., \& Regen, L. (2023). Kapitel 20. Mediendiskurse und -strukturen. In C. Görg, V. Madner, A. Muhar, A. Novy, A. Posch, K. W. Steininger, \& E. Aigner (Hrsg.), APCC Special Report: Strukturen für ein klimafreundliches Leben (S. 547–566). Springer. https://doi.org/10.1007/978-3-662-66497-1\_24
	\item Ötsch, W. (2023). Wert und Werte in der Ökonomik. Agora42 (Philosophisches Wirtschaftsmagazin), (02/2023), 9-13.
	\item Bäuerle, L. (2023). Transformation, Agency and the Economy: The Case for a Grounded Economics. (Economics and Humanities). Routledge. https://doi.org/10.4324/9781003371687
	\item Kapeller, J., \& Hubmann, G. (2023). Dilemmata marktliberaler Globalisierung. In Sturn, Richard; Klüh, Ulrich (Hrsg.), Wachstums- und Globalisierungsgrenzen (Band 20). (Jahrbuch Normative und institutionelle Grundfragen der Ökonomik). Metropolis Verlag. 
\end{enumerate}
\subsection*{2022}
\begin{enumerate}
	\item Aistleitner, M. (2022). Development and Interdisciplinarity: re-examining the “economics silo”.
	\item Aistleitner, M., Kapeller, J., \& Kronberger, D. (2022). The Authors of Economics Journals Revisited: Evidence from a Large-Scale Replication of Hodgson \& Rothman (1999).
	\item Altreiter, C., Azevedo, S., Porak, L., \& Pühringer, S. et al. (2022). Winning urban competition with a social agenda. The competition imaginary in Viennese urban development plans.
	\item Beyer, K. M., \& Pühringer, S. (2022). Divided We Stand? On the Political Engagement of U.S. Economists. Journal of Economic Issues, 56(3), 883-903. https://doi.org/10.1080/00213624.2022.2093581
	\item Derndorfer, J., Hoffmann, R., \& Theine, H. (2022). Integrating environmental issues within party manifestos. Exploring trends across European welfare states. In Towards Sustainable Welfare States in Europe (S. 80-107). Edward Elgar Publishing Ltd.. 
	\item Grisold, A., \& Theine, H. (2022). Debating inequality: The case of Piketty's capital in the 21st century. In The Media and Inequality (S. 176-191). Taylor and Francis. https://doi.org/10.4324/9781003104476-16
	\item Gräbner, C., \& Hornykewycz, A. (2022). Capability accumulation and product innovation: an agent-based perspective. Journal of Evolutionary Economics, 32(1), 87-121. https://doi.org/10.1007/s00191-021-00732-9
	\item Gräbner-Radkowitsch, C. (2022). Elements of an evolutionary approach to comparative economic studies: complexity, systemism, and path dependent development.
	\item Gräbner-Radkowitsch, C., \& Hafele, J. (2022). Why Fostering Socio-economic Convergence in the EU Is Necessary for Successful Climate Change Mitigation. In Heinrich Böll Foundation, ZOE-Institute for Future-Fit Economies and Finanzwende Recherche (Hrsg.), Making the great turnaround work: Economic policy for a green and just transition (S. 104-114)
	\item Gräbner-Radkowitsch, C., Hager, T., \& Hornykewycz, A. (2022). Competing for sustainability? An institutionalist analysis of the new development model of the European Union.
	\item Gräbner-Radkowitsch, C., Heimberger, P., Kapeller, J., \& Landesmann, M. et al. (2022). The evolution of debtor-creditor relationships within a monetary union: Trade imbalances, excess reserves and economic policy. Structural Change and Economic Dynamics, 62, 262-289. https://doi.org/10.1016/j.strueco.2022.05.004
	\item Gräbner-Radkowitsch, C., Hornykewycz, A., \& Schütz, B. (2022). The emergence of debt and secular stagnation in an unequal society: a stock- flow consistent agent-based approach.
	\item Gräbner-Radkowitsch, C., Hornykewycz, A., \& Schütz, B. (2022). The emergence of debt and secular stagnation in an unequal society: a stockflow consistent agent-based approach.
	\item Gräbner-Radkowitsch, C., Tamesberger, D., Heimberger, P., \& Kapelari, T. et al. (2022). Trade Models in the European Union. Economic Annals, 67(235), 7-36. https://doi.org/10.2298/EKA2235007G
	\item Hager, T., Heck, I., \& Rath, J. (2022). Polanyi and Schumpeter: Transitional processes via societal spheres. European Journal of the History of Economic Thought, 29(6), 1089–1110. https://doi.org/10.1080/09672567.2022.2131865
	\item Heimberger, P., \& Schütz, B. (2022). Die Budgetsemielastizität und ihre Auswirkungen auf Verschuldungsspielräume im Rahmen der Schuldenbremse. Wirtschaftsdienst, 102(11), 834-837. https://doi.org/10.1007/s10273-022-3325-y
	\item Heimberger, P., \& Schütz, B. (2022). Evaluierung des Zusammenhangs von Produktionspotenzial und Budgetsemielastizität im Rahmen der deutschen Schuldenbremse.
	\item Hirte, K., Ötsch, W., \& Pühringer, S. (2022). Die Netzwerkanalyse und der Umgang mit ihren Forschungsergebnissen. Eine kritische Replik zum Beitrag von Nico Sonntag. Berliner Journal für Soziologie, 32(1), 153-163. https://doi.org/10.1007/s11609-022-00462-0
	\item Kapeller, J., \& Hubmann, G. (2022). Notizen zum ökonomischen Element in der politischen Doktrinbildung. Zeitschrift für Wirtschafts- und Unternehmensethik, 23, 34-37.
	\item Kapeller, J., \& Huwe, V. (2022). Critical junctures of hope: how to bridge the gap between the necessary and the feasible? GAIA -- Ecological Perspectives for Science and Society, 31(1), 10-13. https://doi.org/10.14512/gaia.31.1.4
	\item Kapeller, J., \& Wildauer, R. (2022). Tracing the invisible rich: a new approach to modelling Pareto tails in survey data. Labour Economics, 75(102145), Artikel 102145. https://doi.org/10.1016/j.labeco.2022.102145
	\item Kapeller, J., Pühringer, S., \& Grimm, C. (2022). Paradigms and Policies: The state of economics in the German-speaking countries. Review of International Political Economy, 29(4), 1183-1210. https://doi.org/10.1080/09692290.2021.1904269
	\item Koch, L. T., Ötsch, W., \& Graupe, S. (2022). Wissenschaftstheoretische Grundlagen. In Lehmann-Waffenschmidt, Marco; Peneder, Michael (Hrsg.), Evolutorische Ökonomik. Konzepte, Wegbereiter und Anwendungsfelder (S. 349-359). Metropolis Verlag. 
	\item Maesse, J., Pühringer, S., Rossier, T., \& Benz, P. (2022). The role of power in the Social Studies of Economics: an introduction. In Jens Maesse, Stephan Pühringer, Thierry Rossier,  Pierre Benz (Hrsg.), Power and Influence of Economists: Contributions to the Social Studies of Economics. Routledge. 
	\item Porak, L. (2022). Wettbewerbsfähige Nachhaltigkeit – Die Lösung unserer Probleme? In Momentum Kongress Paper (S. 1-14)
	\item Pühringer, S., \& Beyer, K. (2022). Divided We Stand? Professional Consensus and Political Conflict in Academic Economics. Journal of Economic Issues. https://doi.org/10.2139/ssrn.3425768
	\item Pühringer, S., \& Beyer, K. (2022). Who are the economists Germany listens to? What it needs to have academic, public or political impact. In Maesse, Jens; Pühringer, Stephan; Rossier, Thierry; Benz, Pierre (Hrsg.), Power and Influence of Economists: Contributions to the Social Studies of Economics. (S. 147-169). Routledge. 
	\item Reiner, C. (2022). It’s the End of Globalization as We Know It! Zeitgemäße Betrachtungen zur politischen Ökonomik der Globalisierungskrise.
	\item Reiner, C., \& Bellak, C. (2022). Hat die ökonomische Macht von Unternehmen in Österreich zugenommen?
	\item Theine, H., \& Grisold, A. (2022). Die Medienberichterstattung zur Vermögens- und Erbschaftsbesteuerung in Deutschland. Zeitschrift für Politikwissenschaft, 32(1). https://doi.org/10.1007/s41358-022-00314-6
	\item Theine, H., \& Porak, L. (2022). Introduction: Doing pluralism. International Journal of Pluralism and Economics Education, 13(1), 2-8. http://www.scopus.com/inward/record.url?scp=85135757403\&partnerID=8YFLogxK
	\item Theine, H., Humer, S., Moser, M., \& Schnetzer, M. (2022). Emissions inequality: Disparities in income, expenditure, and the carbon footprint in Austria. Ecological Economics, 197, Artikel 107435. https://doi.org/10.1016/j.ecolecon.2022.107435
	\item Ötsch, W. (2022, Mai). Klima, Markt und Zukunftsbilder.
	\item Pühringer, S., Aistleitner, M., \& Griesebner, T. (2022). Networks of the super-rich in Austria. Evidence from an explorative case study. (Materialien zu Wirtschaft und Gesellschaft --  Working Paper-Reihe der AK Wien). https://emedien.arbeiterkammer.at/viewer/image/AC16689081/
	\item Bäuerle, L. (2022). Ökonomie – Praxis – Subjektivierung: Eine praxeologische Institutionenforschung am Beispiel ökonomischer Hochschulbildung (1 Aufl.). [Dissertation, Europa-Universität Flensburg]. transcript Verlag.
	\item Hilt, A., Bäuerle, L., \& Steffestun, T. (2022). Die Hochschule für Gesellschaftsgestaltung als Ort gesellschaftlichen Lernens in und aus Krisen – ein Erfahrungsbericht aus dem ersten Corona-Semester an der Cusanus Hochschule für Gesellschaftsgestaltung. In L.-M. Schröder, H. Hantke, T. Steffestun, \& R. Hedtke (Hrsg.), In Krisen aus Krisen lernen: Sozioökonomische Bildung und Wissenschaft im Kontext sozial-ökologischer Transformation (1 Aufl., S. 127-149). (Sozioökonomische Bildung und Wissenschaft). Verlag für Sozialwissenschaften. 
\end{enumerate}
\subsection*{2021}
\begin{enumerate}
	\item Aistleitner, M., \& Pühringer, S. (2021). The Trade (Policy) Discourse in Top Economic Journals. New Political Economy, 26(5), 748-764. https://doi.org/10.1080/13563467.2020.1841145
	\item Aistleitner, M., Gräbner, C., \& Hornykewycz, A. (2021). Theory and Empirics of Capability Accumulation: Implications for Macroeconomic Modelling. Research Policy, 50(6), Artikel 104258. https://doi.org/10.1016/j.respol.2021.104258
	\item Bäuerle, L. (2021). Beyond indifference – an economics for the future. In E. Fullbrook, \& J. Morgan (Hrsg.), Post-Neoliberal Economics (1 Aufl., S. 175-208). World Economics Association. 
	\item Bäuerle, L. (2021). The power of economics textbooks: Shaping meaning and identity. In Power and Influence of Economists: Contributions to the Social Studies of Economics (S. 53-69). Taylor and Francis. https://doi.org/10.4324/9780367817084-5
	\item Cordes, C., Elsner, W., Gräbner, C., \& Heinrich, T. et al. (2021). The collapse of cooperation: The endogeneity of institutional break-up and its asymmetry with emergence. Journal of Evolutionary Economics, 31(4), 1291-1315. https://doi.org/10.1007/s00191-021-00739-2
	\item Eder, J. (2021). Decreasing Dependency through Self-Reliance: Strengthening Local Economies through Community Wealth Building.
	\item Griesser, M., Beyer, K., \& Pühringer, S. (2021). Für die „Leistungsträger“ und „uns Österreicher“: Eine Mediendiskursanalyse zu Sozialreformen der ÖVP/FPÖ-Regierung 2017-2019 in Österreich. In Momentum Kongress Paper 
	\item Gräbner, C., \& Hager, T. (2021, Sep). (Mis)Measuring Competitiveness: The Quantification of a Malleable Concept in the European Semester.
	\item Gräbner, C., \& Heinrich, T. (2021). Introduction to the symposium: The Complexity of Institutions: Theory and Computational Models. Forum for Social Economics, 50(2), 153-156. https://doi.org/10.1080/07360932.2020.1752765
	\item Gräbner, C., \& Kapeller, J. (2021). Konzernmacht in globalen Güterketten. In Karin Fischer, Christian Reiner und Cornelia Staritz (Hrsg.), Globale Güterketten und ungleiche Entwicklung. Arbeit, Kapital, Natur und Konsum (S. 195-217). Mandelbaum. 
	\item Gräbner, C., \& Pühringer, S. (2021, Mai). Competition Universalism: Its Historical Origins and Timely Alternatives.
	\item Gräbner, C., \& Strunk, B. (2021). Pluralism in economics – its critiques and their lessons. Developing Economics.
	\item Gräbner, C., Elsner, W., \& Lascaux, A. (2021). Trust and Social Control. The Sources of Stability in Informal Value Transfer Systems. Computational Economics, (58), 1077-1102. https://doi.org/10.1007/s10614-020-09994-0
	\item Gräbner, C., Heimberger, P., Kapeller, J., \& Springholz, F. (2021). Understanding economic openness: A review of existing measures. Review of World Economics, 157(1), 87-120. https://doi.org/10.1007/s10290-020-00391-1
	\item Gräbner, C., Hornykewycz, A., \& Schütz, B. (2021). The emergence of debt and secular stagnation in an unequal society: A stock-flow consistent agent-based approach. In Momentum Kongress Paper 
	\item Gräbner-Radkowitsch, C., \& Pühringer, S. (2021). Competition universalism: Its historical origins and timely alternatives.
	\item Gräbner-Radkowitsch, C., Heimberger, P., Kapeller, J., \& Landesmann, M. et al. (2021). The evolution of debtor-creditor relationships within a monetary union: Trade imbalances, excess reserves and economic policy.
	\item Hager, T., Hornykewycz, A., Jonjic, M., \& Porak, L. et al. (2021). „Hinter jeder erfolgreichen Frau steht ein Mann, der ihr den Rücken stärkt.“. In Momentum Kongress Paper 
	\item Heimberger, P. (2021). Corporate tax competition: A meta-analysis. European Journal of Political Economy, 69(1020002), Artikel 102002. https://doi.org/10.1016/j.ejpoleco.2021.102002
	\item Heimberger, P. (2021). Does economic globalisation promote economic growth? The World Economy. https://doi.org/10.1111/twec.13235
	\item Heimberger, P. (2021). Does economic globalization affect government spending? A meta-analysis. Public Choice, 187(3-4), 349-374. https://doi.org/10.1007/s11127-020-00784-8
	\item Heimberger, P. (2021). Does employment protection affect unemployment? A meta-analysis. Oxford Economic Papers, 73(3), 982-1007. https://doi.org/10.1093/oep/gpaa037
	\item Heimberger, P. (2021). Keynes, die Outputlücke und Probleme mit den Fiskalregeln. Blog der Keynes-Gesellschaft.
	\item Heimberger, P. (2021). What is structural about unemployment in OECD countries? Review of Social Economy, 79(2), 380-412. https://doi.org/10.1080/00346764.2019.1678067
	\item Heimberger, P. (2021, Apr). Verschwenderisches, reformfaules Italien? Warum gängige Mythen falsch und gefährlich sind.
	\item Heimberger, P. (2021, Dez). The push for a global minimum corporate tax rate.
	\item Heimberger, P. (2021, Feb). Draghi government: Seven ‘surprising’ facts about Italy.
	\item Heimberger, P. (2021, Jul). Budgetkürzungen durch „Outputlücken-Nonsens“.
	\item Heimberger, P. (2021, Jän). EU bonds are a model for the future of Europe.
	\item Heimberger, P. (2021, Mai). European fiscal rules: reform urgently needed.
	\item Heimberger, P. (2021, Mai). Keynes, output gap nonsense and the EU’s fiscal rules.
	\item Heimberger, P. (2021, Mär). Beeld over Italiaanse economie klopt niet.
	\item Heimberger, P. (2021, Mär). Financial globalisation has increased income inequality.
	\item Heimberger, P. (2021, Mär). Fiscal austerity and the rise of the Nazis.
	\item Heimberger, P. (2021, Mär). Keynes, the output gap and the EU’s fiscal rules.
	\item Heimberger, P., \& Gechert, S. (2021, Jul). Corporate tax cuts do not boost growth.
	\item Heimberger, P., \& Kowall, N. (2021, Feb). Il governo Draghi: sette fatti sorprendenti sull’Italia.
	\item Hirte, K. (2021). Unternehmenskonzentrationen in der Fleischbranche und die performative Rolle der Agrarökonomik. ÖZS -- Österreichische Zeitschrift für Soziologie, 46(2), 187-206. https://doi.org/10.1007/s11614-021-00448-x
	\item Hornykewycz, A., \& Rath, J. (2021). Shaping sustainable employment relationships in the age of Digitalisation: analysing policy measures in an agent-based framework. In Momentum Kongress Paper 
	\item Kapeller, J. (2021). Intangible Flow Theory: A New Way for Conceptualizing Embeddedness? Accounting, Economics and Law, 14(1), 159-164.
	\item Kapeller, J. (2021). Polarisierung oder Konvergenz? Zur ökonomischen Zukunft des vereinten Europas. WISO direkt -- Analysen und Konzepte zur Wirtschafts- und Sozialpolitik.
	\item Kapeller, J. (2021). Ökonomische Polarisierung in Europa. Zeitschrift Bürger und Staat, 71(4), 246-251.
	\item Kapeller, J., \& Gräbner, C. (2021). Standortwettbewerb und Deindustrialisierung: Das Beispiel MAN als Lehrbuchfall. WISO -- Wirtschafts- und sozialpolitische Zeitschrift, 44(4), 34-52.
	\item Kapeller, J., \& Gräbner-Radkowitsch, C. (2021). Konzernmacht in globalen Güterketten. In Globale Warenketten und ungleiche Entwicklung. Arbeit, Kapital, Konsum, Natur. Mandelbaum Verlag. 
	\item Kapeller, J., \& Gräbner-Radkowitsch, C. (2021). Standortwettbewerb und Deindustrialisierung: Das Beispiel MAN als Lehrbuchfall.
	\item Kapeller, J., \& Rehm, M. (2021). Vom empiristischen Humanismus zum partizipativen Sozialismus – Review von Thomas Piketty ‚Kapital und Ideologie‘. Soziologische Revue, 44(1), 25-33.
	\item Kapeller, J., \& Wildauer, R. (2021). A Fitting Pareto tails to wealth survey data: A practitioners’ guide. Journal of Income Distribution. https://doi.org/10.25071/1874-6322.40447
	\item Kapeller, J., \& Wildauer, R. (2021, Apr). Eine europäische Vermögenssteuer für das Klima.
	\item Kapeller, J., Leitch, S., \& Wildauer, R. (2021). A European Wealth Tax for a Fair and Green Recovery.
	\item Kapeller, J., Leitch, S., \& Wildauer, R. (2021). A European wealth tax for a fair and green recovery. Policy Study.
	\item Kapeller, J., Leitch, S., \& Wildauer, R. (2021). Is a 10 trillion euro European climate investment initiative fiscally sustainable?
	\item Kapeller, J., Leitch, S., \& Wildauer, R. (2021). Is a € 10 Trillion European climate investment initiative fiscally sustainable? Policy Study.
	\item Kapeller, J., Leitch, S., \& Wildauer, R. (2021, Apr). A European wealth tax.
	\item Maesse, J., Pühringer, S., Rossier, T., \& Benz, P. (2021). Power and Influence of Economists: Contributions to the Social Studies of Economics. Taylor and Francis. https://doi.org/10.4324/9780367817084
	\item Porak, L. (2021). Governing the Ungovernable -- Recontextualizations of ‘Competition’ in European Policy Discourse.
	\item Porak, L. (2021). Warum müssen wir (noch immer) arbeiten? Eine hegemonietheoretische Analyse der Bedeutung und des Wertes von Lohnarbeit für den modernen Staat. In Momentum Kongress Paper 
	\item Porak, L., \& Schröter, G. (2021). Strategien für einen Wandel der ökonomischen Lehre. Forschungsjournal Soziale Bewegungen, 34(4), 718-729.
	\item Porak, L., Pühringer, S., \& Rath, J. (2021, Mär). So denken Ökonom*innen über Wettbewerb – eine Kritische Analyse des österreichischen Expert*innendiskurses.
	\item Pühringer, S. (2021). Zur Pluralität der ökonomischen Politikberatung in Deutschland.
	\item Pühringer, S. (2021). Zur Pluralität in der ökonomischen Politikberatung in Deutschland. Eine empirische Untersuchung. Leviathan -- Berliner Zeitschrift für Sozialwissenschaft, 49, 243-265.
	\item Pühringer, S., \& Beyer, K. M. (2021). Who are the economists Germany listens to? The social structure of influential German economists. In Power and Influence of Economists: Contributions to the Social Studies of Economics (S. 147-169). Taylor and Francis. 
	\item Pühringer, S., \& Rath, J. (2021). Monopolies in Science Publishing. A Black Hole for Public Spending? Journal of Management Information and Decision Sciences, 24(6), 1-5.
	\item Pühringer, S., Beyer, K., \& Kronberger, D. (2021). Soziale Rhetorik, neoliberale Praxis. Beuteler Extradienst.
	\item Pühringer, S., Porak, L., \& Rath, J. (2021). Talking about competition? Discursive shifts in the economic imaginary of competition in public debates.
	\item Pühringer, S., Rath, J., \& Griesebner, T. (2021). The political economy of academic publishing: On the commodification of a public good. PLOS One, 16(6), Artikel e0253226. https://doi.org/10.1371/journal.pone.0253226
	\item Schütz, B. (2021). \glqq Koste es, was es wolle\grqq{}. Eine neue Ära der Ökonomie? Economy.
	\item Schütz, B. (2021). Creating a pluralist paradigm: An application to the minimum wage debate. Journal of Economic Issues, 55(1), 103-124. https://doi.org/10.1080/00213624.2021.1874786
	\item Tamesberger, D., \& Theurl, S. (2021). Design and Take Up of Austria’s Coronavirus Short Time Work Model.
	\item Theine, H. (2021). Economists in public discourses: The case of wealth and inheritance taxation in the German press. In Power and Influence of Economists: Contributions to the Social Studies of Economics (S. 188-206). Taylor and Francis. 
	\item Ötsch, W. (2021). Narration und Imagination. Die Rolle von imaginierten Bildern in der Geschichte der Wirtschaftstheorie. In Künzel, Christine; Priddat, Birger (Hrsg.), Fiktion und Narration in der Ökonomie. Interdisziplinäre Perspektiven auf den Umgang mit ungewisser Zukunft (S. 241-267). Metropolis. 
	\item Ötsch, W., \& Graupe, S. (2021). Vorwort. Fifty-fifty Verlag.
	\item Ötsch, W., \& Graupe, S. (2021). Walter Lippmann: Die Illusion von Wahrheit oder die Erfindung der Fake News. Fivty-fivty Verlag, Edition Buchkomplizen.
	\item Ötsch, W., \& Pühringer, S. (2021). Ordoliberalismus. In Michael G. Festl (Hrsg.), Handbuch Liberalismus (S. 372-378). J.B. Metzler. 
	\item Ötsch, W., \& Steffestun, T. (Hrsg.) (2021). Wissen und Nichtwissen der ökonomisierten Gesellschaft. Aufgaben einer neuen Politischen Ökonomie. Metropolis Verlag.
	\item Ötsch, W., \& Wodak, R. (2021). Populismus. In Ferstl, Michael G. (Hrsg.), Handbuch Liberalismus (S. 535-541). J.B. Metzler. 
	\item Strohmaier, R. (2021). Agent of Sustainable Change -- Der unternehmerische Staat und sozial-ökologische Transformation. In Klüh, Ulrich; Sturn, Richard (Hrsg.), Der Staat in der großen Transformation. Jahrbuch Normative und institutionelle Grundfragen der Ökonomik (S. 169-192). (Jahrbuch Normative und institutionelle Grundfragen der Ökonomik). Metropolis. 
	\item Bäuerle, L. (2021). Das vermeintliche Wissen der ökonomischen Lehrbuchwissenschaft. Ein Essay. In I. De Gennaro, S. Kazmierski, R. Lüfter, \& R. Simon (Hrsg.), Ökonomie als Problem (1 Aufl., Band 5, S. 11-34). (Elementa Oeconomica). Verlag Karl Alber. 
	\item Gräbner-Radkowitsch, C., Heimberger, P., Kapeller, J., \& Landesmann, M. et al. (2021). The evolution of debtor-creditor relationships within a monetary union: Trade imbalances, excess reserves and economic policy. (Ifso working paper; Nr. 10).
	\item Heimberger, P. (2021). Do Higher Public Debt Levels Reduce Economic Growth? (wiiw Working Papers; Nr. 211).
	\item Heimberger, P., \& Gechert, S. (2021). Do Corporate Tax Cuts Boost Economic Growth? (wiiw Working Papers; Nr. 201). forthcoming, doi/full/10.1080/09692290.2021.1904269.
	\item Pühringer, S., Beyer, K., \& Kronberger, D. (2021). Soziale Rhetorik, neoliberale Praxis: Eine Analyse der Wirtschafts- und Sozialpolitik der AfD. (OBS Arbeitspapier; Nr. 52). Otto Brenner Stiftung.
\end{enumerate}
\subsection*{2020}
\begin{enumerate}
	\item Aistleitner, M., \& Pühringer, S. (2020). Exploring the trade (policy) narratives in economic elite discourse.
	\item Aistleitner, M., Gräbner-Radkowitsch, C., \& Hornykewycz, A. (2020). Theory and Empirics of Capability Accumulation: Implications for Macroeconomic Modelling.
	\item Altreiter, C., Gräbner-Radkowitsch, C., Pühringer, S., \& Rogojanu, A. et al. (2020). Theorizing Competition: an interdisciplinary approach.
	\item Altreiter, C., Gräbner-Radkowitsch, C., Pühringer, S., \& Rogojanu, A. et al. (2020). Theorizing competition: an interdisciplinary framework.
	\item Beyer, K., Griesser, M., \& Pühringer, S. (2020, Okt). Türkis-blaue Arbeitsmarkt- und Sozialpolitik revisited: zwischen Meritokratie und Wohlfahrtschauvinismus.
	\item Bäuerle, L. (2020). An essay on the putative knowledge of textbook economics. real-world economics review, 91, 53-69. https://www.paecon.net/PAEReview/issue91/Bauerle91.pdf
	\item Bäuerle, L. (2020). Reproduction, deconstruction, imagination on three possible modi operandi of economic education. Journal of Social Science Education, 19(3), 22-39. https://doi.org/10.4119/jsse-3378
	\item Bäuerle, L., Ötsch, W., \& Pühringer, S. (2020). Wirtschaft(lich) studieren. Erlebniswelten von Studierenden der Volkswirtschaftslehre. Springer. https://doi.org/10.1007/978-3-658-30057-9
	\item Grabner, D., Grisold, A., \& Theine, H. (2020). Stagnation, social tensions, unfairness. Economic inequality as a problem. In Economic Inequality and News Media: Discourse, Power, and Redistribution (S. 144-168). Oxford University Press. https://doi.org/10.1093/oso/9780190053901.003.0008
	\item Grisold, A., \& Theine, H. (2020). Media and economic inequality: Review of prior research. In Economic Inequality and News Media: Discourse, Power, and Redistribution (S. 70-88). Oxford University Press. https://doi.org/10.1093/oso/9780190053901.003.0004
	\item Grisold, A., \& Theine, H. (2020). “Now, What Exactly is the Problem?“ Media Coverage of Economic Inequalities and Redistribution Policies. The Piketty Case. Journal of Economic Issues, 54(4), 1071-1094. https://doi.org/10.1080/00213624.2020.1829905
	\item Grisold, A., Rieder, M., \& Theine, H. (2020). Is this feasible? Policy proposals such as a global wealth tax. In Economic Inequality and News Media: Discourse, Power, and Redistribution (S. 169-188). Oxford University Press. https://doi.org/10.1093/oso/9780190053901.003.0009
	\item Gräbner, C., \& Strunk, B. (2020). Pluralism in economics: its critiques and their lessons. Journal of Economic Methodology, 27(4), 311-329. https://doi.org/10.1080/1350178X.2020.1824076
	\item Gräbner, C., Heimberger, P., \& Kapeller, J. (2020). Pandemic pushes polarisation: the Corona crisis and macroeconomic divergence in the Eurozone. Journal of Industrial and Business Economics, 47(3), 425-438. https://doi.org/10.1007/s40812-020-00163-w
	\item Gräbner, C., Heimberger, P., Kapeller, J., \& Schütz, B. (2020). Is the Eurozone disintegrating? Macroeconomic divergence, structural polarization, trade and fragility. Cambridge Journal of Economics, 44(3), 647-669. https://doi.org/10.1093/cje/bez059
	\item Gräbner, C., Heimberger, P., Kapeller, J., \& Schütz, B. (2020). Structural change in times of increasing openness: assessing path dependency in European economic integration. Journal of Evolutionary Economics, 30(5), 1467-1495. https://doi.org/10.1007/s00191-019-00639-6
	\item Gräbner-Radkowitsch, C., \& Hafele, J. (2020). The emergence of core-periphery structures in the European Union: a complexity perspective.
	\item Gräbner-Radkowitsch, C., \& Hornykewycz, A. (2020). Capability accumulation and product innovation: an agent-based perspective.
	\item Gräbner-Radkowitsch, C., Heimberger, P., \& Kapeller, J. (2020). Do the “smart kids” catch up? Technological capabilities, globalisation and economic growth.
	\item Gräbner-Radkowitsch, C., Heimberger, P., \& Kapeller, J. (2020). Pandemic pushes polarisation: The Corona crisis and macroeconomic divergence in the Eurozone.
	\item Hager, T., Heck, I., \& Rath, J. (2020). Competition in Transformational Processes: Polanyi \& Schumpeter. In Momentum (Hrsg.), Momentum Kongress Paper (S. 1-25)
	\item Heimberger, P. (2020). Does economic globalisation affect income inequality? A meta-analysis. The World Economy, (11), 1-23. https://doi.org/10.1111/twec.13007
	\item Heimberger, P. (2020). Structural polarisation and path dependent development models in the EU. Blog Developing Economics.
	\item Heimberger, P. (2020). The dynamic effects of fiscal consolidation episodes on income inequality: Evidence for 17 OECD countries over 1978-2013. Empirica, 47(1), 53-81. https://doi.org/10.1007/s10663-018-9404-z
	\item Heimberger, P. (2020, Aug). EU-Wiederaufbaufonds als Kernstück europäischer Krisenbekämpfung: Progressiver Durchbruch oder Enttäuschung?
	\item Heimberger, P. (2020, Dez). Hyperinflation and the Rise of the Nazis.
	\item Heimberger, P. (2020, Feb). Budgetpolitik im Wirtschaftsabschwung: erhebliche Spielräume vorhanden.
	\item Heimberger, P. (2020, Mär). Wie die ökonomische Globalisierung die Einkommensungleichheit beeinflusst.
	\item Heimberger, P. (2020, Okt). Budgetpolitik in der Corona-Krise: Reform der Budgetregeln erforderlich.
	\item Heimberger, P., \& Kapeller, J. (2020). ‘Output gap nonsense’ and the EU’s fiscal rules: A response to the European Commission’s economists. Blog Institute for New Economic Thinking.
	\item Heimberger, P., \& Kowall, N. (2020, Jun). Seven ’surprising’ facts about the Italian economy.
	\item Heimberger, P., Huber, J., \& Kapeller, J. (2020). The power of economic models: The case of the EU's fiscal regulation framework. Socio-Economic Review, 18(2), 337-366. https://doi.org/10.1093/ser/mwz052
	\item Heimberger, P., Krahé, M., Ponattu, D., \& van't Klooster, J. (2020, Mai). Keeping the promise of eurozone convergence.
	\item Hirte, K. (2020). Das doppelte Reflektionsproblem. In Hochmann, Lars (Hrsg.), Economists4future (S. 43-58). Murmann Verlag. 
	\item Hirte, K. (2020). Friedman’s Instrumentalismus und das Problem von Kopernikus. In Pühringer, Stephan; Graupe, Silja; Hirte, Katrin; Kapeller, Jakob; Panther, Stephan (Hrsg.), Jenseits der Konventionen (S. 97-122). Metropolis Verlag. 
	\item Hirte, K. (Hrsg.) (2020). Friedman’s Instrumentalismus und das Problem von Kopernikus. Zur zentralen Rolle von Ausgangsannahmen in Theorien. In Pühringer, Stephan; Graupe, Silja; Hirte, Katrin; Kapeller, Jakob; Panther, Stephan (Hrsg.), Jenseits der Konventionen: Alternatives Denken zu Wirtschaft, Gesellschaft und Politik. (S. 97-122). Metropolis Verlag. 
	\item Kapeller, J. (2020, Jul). Polarisierung oder Konvergenz? Zur ökonomischen Zukunft des vereinten Europa.
	\item Kapeller, J., \& Gräbner-Radkowitsch, C. (2020). Konzernmacht in globalen Güterketten.
	\item Kapeller, J., \& Pühringer, S. (2020). Paradigmen und Politik. Der derzeitige Stand der Ökonomie. In Jakob Kapeller, Stephan Pühringer, Silja Graupe, Kathrin Hirte, Stephan Panther (Hrsg.), Jenseits der Konventionen: Alternatives Denken zu Wirtschaft, Gesellschaft und Politik (S. 221-252). Metropolis. 
	\item Kapeller, J., Wildauer, R., \& Heck, I. (2020). Vermögenskonzentration in Österreich: Ein Update auf Basis des HFCS 2017. Wirtschaft und Gesellschaft, (206), 1-38.
	\item Piétron, D., Porak, L., \& Thieme, S. (2020). Plurale Ökonomik -- Eine kurze Einführung. In Thielscher, Christian (Hrsg.), Wirtschaftswissenschaften verstehen (S. 189-205). Springer Gabler. 
	\item Porak, L. (2020). Der größte ‘Trumpf’ Europas: Eine Analyse des ‘economic imaginary’ der Europäischen Kommission.
	\item Porak, L. (2020). Miteinander und voneinander lernen. Vielfalt in der ökonomischen Lehre. In Hochmann, Lars (Hrsg.), Economists4future (S. 127-142). Murmann Verlag. 
	\item Porak, L. (2020). Wohin steuert die Europäische Union? Ein Klärungsversuch der strategischen Ausrichtung der EU seit Lissabon. In Momentum Kongress Paper (S. 1-22)
	\item Porak, L., \& Neuffer, S. (2020). Die moderne Lehrbuchwissenschaft als Zombiewissenschaft. Agora42 (Philosophisches Wirtschaftsmagazin).
	\item Pühringer, S. (2020). Think Tank Networks of German Neoliberalism. Power Structures in Economics and Economic Policies in Post-War Germany. In Mirowski, Philip; Plehwe, Dieter; Slobodian, Quinn (Hrsg.), Nines Lives of Neoliberalism (S. 283-306). Verso Books. 
	\item Pühringer, S., \& Griesser, M. (2020). From the ʻplanning euphoriaʼ to the ʻbitter economic truthʼ: The Transmission of economic ideas into German Labour Market Policies in the 1960s and 2000s. Critical Discourse Studies, 17(5), 476-493. https://doi.org/10.1080/17405904.2019.1681283
	\item Pühringer, S., Graupe, S., Hirte, K., \& Kapeller, J. (Hrsg.) et al. (2020). Jenseits der Konventionen. Alternatives Denken zu Wirtschaft, Gesellschaft und Politik. Festschrift für Walter Ötsch. Metropolis.
	\item Pühringer, S., Graupe, S., Hirte, K., \& Kapeller, J. et al. (2020). Vorwort. Jenseits der Konventionen. In Pühringer, Stephan; Graupe, Silja; Hirte, Katrin; Kapeller, Jakob; Panther, Stephan (Hrsg.), Jenseits der Konventionen. Eine Festschrift für Walter Ötsch (S. 9-16). Metropolis. 
	\item Pühringer, S., Rath, J., \& Griesebner, T. (2020). The political economy of academic publishing: On the commodification of a public good.
	\item Rieder, M., Silke, H., \& Theine, H. (2020). Media coverage of economic inequality: The empirical study-an initial overview. In Economic Inequality and News Media: Discourse, Power, and Redistribution (S. 106-123). Oxford University Press. https://doi.org/10.1093/oso/9780190053901.003.0006
	\item Schulmeister, S. (2020). Fixing long-term price paths for fossil energy – the optimal incentive for limiting global warming.
	\item Schütz, B. (2020). Die Auswirkung von Mindestlöhnen auf die Arbeitslosigkeit: Ein Paradigmenvergleich. In Stephan Pühringer, Silja Graupe, Katrin Hirte, Jakob Kapeller, Stephan Panther (Hrsg.), Jenseits der Konventionen: Alternatives Denken zu Wirtschaft, Gesellschaft und Politik (S. 157-173). Metropolis. 
	\item Steffestun, T. (Hrsg.) (2020). The Constitution of Ignorance -- zur Bedeutung von Nichtwissen in der Verhaltensökonomie. In Ötsch, Walter O.; Steffestun, Theresa (Hrsg.), Wissen und Nichtwissen der ökonomisierten Gesellschaft. (S. 85-132). Metropolis Verlag. 
	\item Theine, H., \& Grabner, D. (2020). Trends in economic inequality and news mediascape. In Economic Inequality and News Media: Discourse, Power, and Redistribution (S. 21-47). Oxford University Press. https://doi.org/10.1093/oso/9780190053901.003.0002
	\item Vogel, L., Jühlke, R., Porak, L., \& Quinz, H. (2020). Allbetroffenheit in der Pandemie? Ein soziologischer Blick auf das Erleben der Auswirkungen der Corona-Krise. In Momentum Kongress Paper (S. 1-19)
	\item Wildauer, R., Leitch, S., \& Kapeller, J. (2020). How to boost the European Green Deal’s scale and ambition.
	\item Ötsch, W. (2020). Ist der Neoliberalismus am Ende? In Schmidinger, Thomas; Weidenholzer, Josef (Hrsg.), Virenregime. Wie die Coronakrise unsere Welt verändert. Befunde, Analyse, Anregungen. bahoe books. 
	\item Ötsch, W. (2020). Wissen, Selbstwissen und Nichtwissen der marktfundamentalen Ökonomie. In Ötsch, Walter O.; Steffestun, Theresa (Hrsg.), Wissen und Nichtwissen der ökonomisierten Gesellschaft (S. 85-131). Metropolis Verlag. 
	\item Ötsch, W., \& Graupe, S. (Hrsg.) (2020). Imagination und Bildlichkeit in der Ökonomie – eine Einführung. In Imagination und Bildlichkeit der Wirtschaft. Zur Geschichte und Aktualität imaginativer Fähigkeiten in der Ökonomie (S. 1-33). Springer. 
	\item Ötsch, W., \& Steffestun, T. (Hrsg.) (2020). Zur Einführung: Wissen und Nichtwissen in einer ökonomisierten Gesellschaft. Konturen einer neuen Politischen Ökonomie. In Wissen und Nichtwissen der ökonomisierten Gesellschaft. (S. 7–35). Metropolis Verlag. 
	\item Bäuerle, L. (2020). Reproduzieren, Dekonstruieren, Imaginieren.: Ein existenzpädagogischer Blick auf (mögliche) Konventionen ökonomischer Bildung. In S. Pühringer, S. Graupe, K. Hirte, J. Kapeller, \& S. Panther (Hrsg.), Jenseits der Konventionen -- Alternatives Denken zu Wirtschaft, Gesellschaft und Politik: Eine Festschrift für Walter O. Ötsch (1 Aufl., S. 337-370). (Kritische Studien zu Markt und Gesellschaft). Metropolis Verlag. 
	\item Bäuerle, L. (2020). Ökonomisierte ökonomische Bildung: Zur Dominanz struktureller Bedingungen wirtschaftswissenschaftlicher Studiengänge. In C. Fridrich, R. Hedtke, \& W. O. Ötsch (Hrsg.), Grenzen überschreiten, Pluralismus wagen (1 Aufl., Band 3, S. 111-134). (Sozioökonomische Bildung und Wissenschaft). Verlag für Sozialwissenschaften. https://doi.org/10.1007/978-3-658-29642-1\_7
	\item Bäuerle, L., Hantke, H., Schröder, L.-M., \& Urban, J. (2020). Wirtschaft neu lehren – eine Einleitung. In J. Urban, L.-M. Schröder, H. Hantke, \& L. Bäuerle (Hrsg.), Wirtschaft neu lehren: Erfahrungen aus der pluralen sozioökonomischen Hochschulbildung (1 Aufl., S. 1-16). (Sozioökonomische Bildung und Wissenschaft). Verlag für Sozialwissenschaften. https://doi.org/10.1007/978-3-658-30920-6\_1
	\item Urban, J., Schröder, L.-M., Hantke, H., \& Bäuerle, L. (Hrsg.) (2020). Wirtschaft neu lehren: Erfahrungen aus der pluralen sozioökonomischen Hochschulbildung. (1 Aufl.) (Sozioökonomische Bildung und Wissenschaft). Verlag für Sozialwissenschaften. https://doi.org/10.1007/978-3-658-30920-6
	\item Aistleitner, M., \& Pühringer, S. (2020). Exploring the trade (policy) narratives in economic elite discourse. (SSRN).
	\item Aistleitner, M., Gräbner-Radkowitsch, C., \& Hornykewycz, A. (2020). Theory and empirics of capability accumulation: implications for macroeconomics modelling. (S. 1-29). (Rebuilding Macroeconomics Working Paper Series; Nr. 6).
	\item Gräbner-Radkowitsch, C., \& Hafele, J. (2020). Emergence of Core-Periphery Structures in the European Union: A Complexity Perspective. (S. 1-21). (Rebuilding Macroeconomics Working Paper Series; Nr. 17).
	\item Hornykewycz, A., \& Gräbner-Radkowitsch, C. (2020). Capability Accumulation and Product Innovation: Agent-Based Perspective. (S. 1-22). (Rebuilding Macroeconomics Working Paper Series; Nr. 9).
	\item Kapeller, J., Wildauer, R., \& Leitch, S. (2020). How to boost the European Green Deal’s scale and ambition. (FEPS Policy Paper).
\end{enumerate}
\subsection*{2019}
\begin{enumerate}
	\item Aistleitner, M., Grimm, C., \& Kapeller, J. (2019). Auftragsvergabe, Leistungsqualität und Kostenintensität im Schienenpersonenverkehr. In Momentum Kongress Paper (S. 1-56)
	\item Aistleitner, M., Kapeller, J., \& Steinerberger, S. (2019). Citation Patterns in Economics and Beyond: Assessing the Peculiarities of Economics from Two Scientometric Perspectives. Science in Context, 32(4), 361-380. https://doi.org/10.1017/S0269889720000022
	\item Beyer, K. (2019). Antonino Palumbo and Alan Scott (2018): Remaking Market Society: A Critique of Social Theory and Political Economy in Neoliberal Times. ÖZS -- Österreichische Zeitschrift für Soziologie, 44(2), 249-252.
	\item Beyer, K., \& Pühringer, S. (2019). Divided we stand? Professional consensus and political conflict in academic economics.
	\item Bäuerle, L. (2019). Optische Grenzgänge: Konstellationen des Sehens bei Nikolaus von Kues und Jeremy Bentham. Freiburger Zeitschrift für Philosophie und Theologie, 66, 92-117. https://doi.org/10.5169/seals-869321
	\item Bäuerle, L. (2019). Warum VWL studieren?: Sinnangebote ökonomischer Lehrbuchliteratur. Zeitschrift für Diskursforschung = Journal for Discourse Studies, 2018(3), 306-332.
	\item Flechtner, S., \& Gräbner, C. (2019). The heterogeneous relationship between income and inequality: a panel co-integration approach. Economics Bulletin, 39(4), 2540–2549.
	\item Flechtner, S., \& Gräbner-Radkowitsch, C. (2019). The heterogeneous relationship between income and inequality: a panel co-integration approach.
	\item Graupe, S., Ötsch, W., \& Rommel, F. (2019). Spielräume des Denkens. Zur Einführung. In Graupe, Silja; Ötsch, Walter O.; Rommel, Florian (Hrsg.), Spielräume des Denkens (S. 9-32). Metropolis. 
	\item Graupe, S., Ötsch, W., \& Rommel, F. (Hrsg.) (2019). Spielräume des Denkens. Metropolis.
	\item Griesser, M. (2019). Deutungsrahmen der aktiven Arbeitsmarktpolitik: ein deutsch-österreichischer Vergleich von diskursiven Frames aus Anlass von 50 Jahren Arbeits(markt)förderungsgesetz. Momentum Quarterly, 8(3), 166-182. https://doi.org/10.15203/momentumquarterly.vol8.no3.p166-182
	\item Grimm, C. (2019). Ideas have Consequences: Eine vergleichende Analyse zur transformativen Rolle von Ideen. Momentum Quarterly, 8(4), 183-247. https://doi.org/10.15203/momentumquarterly.vol8.no4.p215-229
	\item Grimm, C. (2019, Apr). Der Einfluss des Neoliberalismus auf österreichische Parteiprogramme.
	\item Gräbner, C., Bale, C. S. E., Furtado, B. A., \& Alvarez-Pereira, B. et al. (2019). Getting the Best of Both Worlds? Developing Complementary Equation-Based and Agent-Based Models. Computational Economics, 53(2), 763-782. https://doi.org/10.1007/s10614-017-9763-8
	\item Gräbner, C., Kapeller, J., \& Heimberger, P. (2019, Jul). Economic Polarisation in Europe: Causes and Policy Options.
	\item Gräbner-Radkowitsch, C., Tamesberger, D., Heimberger, P., \& Kapelari, T. et al. (2019). Trade Models in the European Union.
	\item Heimberger, P. (2019). Arbeitsmarktinstitutionen, Kapitalakkumulation und Arbeitslosigkeit in OECD-Ländern, Wirtschaft und Gesellschaft. Wirtschaft und Gesellschaft, 45(1).
	\item Heimberger, P. (2019, Dez). How much space for fiscal expansion? Germany falls victim to 'output gap nonsense’.
	\item Heimberger, P. (2019, Jul). Italien vs. EU-Kommission: Warum ein Defizitverfahren kontraproduktiv wäre.
	\item Heimberger, P. (2019, Jun). ‘Output gap nonsense': Understanding the budget conflict between the EC and Italy’s government.
	\item Heimberger, P. (2019, Mai). Arbeitslosigkeit in Europa: Was man tun könnte.
	\item Heimberger, P. (2019, Mai). Unemployment in Europe: What should be done?
	\item Heimberger, P. (2019, Sep). The current economic downturn in Europe must be seen in the context of a wider problem of economic polarisation.
	\item Heimberger, P., \& Kapeller, J. (2019, Mai). What to do about divergence between EU countries? The problem of structural polarization.
	\item Heimberger, P., \& Pekanov, A. (2019, Sep). Dem wirtschaftlichen Abschwung entgegenwirken: Zur wichtigen Rolle der Fiskalpolitik.
	\item Heinrich, T., \& Gräbner, C. (2019). Beyond equilibrium: revisiting two-sided markets from an agent-based modelling perspective. International Journal of Computational Economics and Econometrics, 9(3), 153-180. https://doi.org/10.1504/ijcee.2019.100558
	\item Hirte, K. (2019). Das dritte gossensche Gesetz. In Hochmann, Lars; Graupe, Silja; Korbun, Thomas; Panther, Stephan; Schneidewind, Uwe (Hrsg.), Möglichkeits¬wissen¬schaften. Ökonomie mit Möglichkeitssinn (S. 133-176). Metropolis Verlag. 
	\item Hirte, K. (2019). Die deutsche Agrarpolitik und Agrarökonomik. Entstehung und Wandel zweier ambivalenter Disziplinen. Springer.
	\item Hirte, K., \& Poppinga, O. (2019). Das Gemeine an der Gemeinwohldebatte. Wege für eine bäuerliche Zukunft – Zeitschrift der ÖBV/ Via Campesina Austria, 42(3 (358)), 7-9.
	\item Kapeller, J. (2019). Humankapital. In von Braunmühl, Claudia; Gerstenberger, Heide; Ptak, Ralf; Wichterich, Christa (Hrsg.), ABC der globalen Unordnung. Von »Anthropozän« bis »Zivilgesellschaft« (S. 120-121). VSA Verlag. 
	\item Kapeller, J. (2019). Pluralism in Economics: Epistemological Rationales and Pedagogical Implementation. In Decker, Samuel; Elsner, Wolfram; Flechtner, Svenja (Hrsg.), Advancing Pluralism in Teaching Economics (S. 55-77). Routledge. 
	\item Kapeller, J., \& Ferschli, B. (2019). Hans Albert und die Kritik am Modell-Platonismus in den Wirtschaftswissenschaften. In Franco, Giuseppe (Hrsg.), Handbuch Karl Popper (S. 733-749). Springer Fachbuch. 
	\item Kapeller, J., \& Meyer, D. (2019). Introduction: change and persistence in contemporary economics. Science in Context, 32(4), 357-360. https://doi.org/10.1017/S0269889720000010
	\item Kapeller, J., Gräbner, C., \& Heimberger, P. (2019). Wirtschaftliche Polarisierung in Europa: Ursachen und Handlungsoptionen. Friedrich-Ebert Stiftung.
	\item Kapeller, J., Gräbner, C., \& Heimberger, P. (2019, Sep). Holding Together what Belongs Together: A Strategy to Counteract Economic Polarisation in Europe.
	\item Kapeller, J., Gräbner-Radkowitsch, C., \& Heimberger, P. (2019). Economic Polarisation in Europe: Causes and Policy Options.
	\item Kapeller, J., Schütz, B., \& Ferschli, B. (2019). Exkurs: Die Macht der internationalen Vermögensverwalter: Das Beispiel BlackRock. In Globale Ungleichheit: Über Zusammenhänge von Kolonialismus, Arbeitsverhältnissen und Naturverbrauch Mandelbaum Verlag. 
	\item Kapeller, J., Schütz, B., \& Ferschli, B. (2019). Finanzialisierung und Globale Ungleichheit. In Globale Ungleichheit: Über Zusammenhänge von Kolonialismus, Arbeitsverhältnissen und Naturverbrauch. Mandelbaum Verlag. 
	\item Porak, L. (2019). Der Wert des Widerspruchs für die demokratische Praxis. In Momentum Kongress Paper, Track 10 (S. 1-15)
	\item Pühringer, S. (2019). The 'eternal character' of austerity measures in European crisis policies: Evidence from the Fiscal Compact discourse in Austria. In Discourse Analysis and Austerity: Critical Studies from Economics and Linguistics (S. 237-252). Taylor and Francis. 
	\item Pühringer, S. (2019). The “eternal character” of austerity measures in European crisis policies. Evidences from the Fiscal Compact discourse in Austria. In Power, Kate; Ali, Tanweer; Lebduskova, Eva (Hrsg.), Discourse Analysis and Austerity: Critical Studies from Economics and Linguistics (S. 134-158). Routledge. 
	\item Pühringer, S., \& Bäuerle, L. (2019). What economics education is missing: the real world. International Journal of Social Economics, 46(8), 977-991. https://doi.org/10.1108/IJSE-04-2018-0221
	\item Pühringer, S., \& Ötsch, W. (2019). Die Wirkmacht der „Liebe zum Markt”: Zum anhaltenden Einfluss ordoliberaler ÖkonomInnen-Netzwerke in Politik und Gesellschaft.
	\item Rieder, M., \& Theine, H. (2019). ‘Piketty is a genius, but … ’: an analysis of journalistic delegitimation of Thomas Piketty’s economic policy proposals. Critical Discourse Studies, 16(3), 248-263. https://doi.org/10.1080/17405904.2019.1573148
	\item Schütz, B. (2019). Creating a pluralist paradigm: An application to the minimum wage debate. In Momentum Kongress Paper (S. 1-41)
	\item Theine, H., \& Rieder, M. (2019). 'The billionaires' boot boys start screaming'-A critical analysis of economic policy discourses in reaction to Piketty's Capital in the Twenty-First Century. In Critical Policy Discourse Analysis (S. 169-192). Edward Elgar Publishing Ltd.. 
	\item Wildauer, R., \& Kapeller, J. (2019). Rank Correction: A New Approach to Differential Non-Response in Wealth Survey Data.
	\item Ötsch, W. (2019). Wissen und Nichtwissen angesichts ‚des Marktes‘. Das Konzept von Hayek. In Graupe, Silja; Ötsch, Walter O.; Rommel, Florian (Hrsg.), Spielräume des Denkens (S. 311-339). Metropolis. 
	\item Ötsch, W. (2019). Ökonomie als Lachnummer. In Krauß, Dietrich (Hrsg.), Die Rache des Mainstreams an sich selbst. 5 Jahre »Die Anstalt« (S. 260-271). Westend Verlag. 
	\item Ötsch, W. (2019, Apr). Überwachungskapitalismus: Das Internet als totalitärer Markt.
	\item Ötsch, W., \& Pühringer, S. (2019). Marktfundamentalismus als Kollektivgedanke. Mises und die Ordoliberalen. In Richard Sturn, Nenad Pantelic (Hrsg.), Dem Markt vertrauen? Beiträge zur Tiefenstruktur neoliberaler Regulierung. (S. 185-210). Metropolis. 
	\item Ötsch, W., \& Pühringer, S. (2019). Was ist eine Krise? Ein Rückblick auf die Wirtschafts- und Finanzkrisen 2008 und 2010. Blickpunkt WISO.
	\item Ötsch, W., Graupe, S., \& Loske, R. (2019). ’Erkühne Dich, weise zu sein!‘ Grundlegung einer Gemeinsinn-Ökonomie. GWP -- Gesellschaft, Wirtschaft, Politik, 68(2), 243-250.
	\item Heimberger, P. (2019). Does economic globalisation affect income inequality? A meta-analysis. (wiiw Working Papers; Nr. 165).
	\item Heimberger, P. (2019). The Impact of Labour Market Institutions and Capital Accumulation on Unemployment: Evidence for the OECD, 1985-2013. (wiiw Working Papers; Nr. 164).
	\item Beyer, K., \& Pühringer, S. (2019). Divided we stand? Professional consensus and political conflict in academic economics. (Working Paper Serie ök; Nr. 51).
	\item Pühringer, S., \& Ötsch, W. (2019). Die Wirkmacht der \glqq Liebe zum Markt\grqq{}: Zum anhaltenden Einfluss ordoliberaler ÖkonomInnenNetzwerke in Politik und Gesellschaft. (Working Paper Serie ök; Nr. 51).
	\item Ötsch, W., \& Pühringer, S. (2019). The anti-democratic logic of right-wing Populism and neoliberal market-fundamentalism. (Working Paper Serie ök; Nr. 48).
	\item Bäuerle, L., Pühringer, S., \& Ötsch, W. (2019). \glqq Ohne Effizienz geht es nicht\grqq{}. Ergebnisse einer qualitativ-empirischen Erhebung unter Studierenden der Volkswirtschaftslehre. (FGW-Studien; Nr. 13). FGW. https://www.ssoar.info/ssoar/handle/document/67301
	\item Dimmelmeier, A., Hafele, J., \& Theine, H. (2019). „Die Daten sind nun einmal die Daten“ Legitimationsmuster und Wissenschaftsverständnisse in der Pluralismusdebatte. In D. J. Petersen, D. Willers, E. M. Schmitt, R. Birnbaum, J. H. E. Meyerhoff, S. Gießler, \& B. Roth (Hrsg.), Perspektiven einer pluralen Ökonomik (S. 25-41). (Wirtschaft + Gesellschaft). Springer Verlag. https://doi.org/10.1007/978-3-658-16145-3\_2
\end{enumerate}
\subsection*{2018}
\begin{enumerate}
	\item Aistleitner, M., Kapeller, J., \& Steinerberger, S. (2018). The Power of Scientometrics and the Development of Economics. Journal of Economic Issues, 52(3), 816-834. https://doi.org/10.1080/00213624.2018.1498721
	\item Beyer, K., \& Pühringer, S. (2018). Freiheitliche Flügelkämpfe? (Historische) Konfliktlinien in der FPÖ. Kurswechsel, (3), 19-27.
	\item Beyer, K., Grimm, C., Kapeller, J., \& Pühringer, S. (2018). Netzwerke, Paradigmen, Attitüden. Der deutsche Sonderweg im Fokus. Paradigmatische Ausrichtung und politische Orientierung von deutschen und US-amerikanischen Ökonomi\_nnen im Vergleich. FGW.
	\item Ferschli, B., Kapeller, J., \& Wildauer, R. (2018). Zur Verteilung und Klassenstruktur der Österreichischen Vermögen. In Momentum Kongress Paper (S. 1-55)
	\item Graupe, S., \& Ötsch, W. (Hrsg.) (2018). Macht der Bilder, Macht der Sprache. Band 37 der Schriftenreihe der Freien Akademie Falkensee. Angelika Lenz Verlag.
	\item Griesser, M., \& Brand, U. (2018). Wachstum? Wohlstand und Lebensqualität! Momentum Quarterly, 7(2), 53-72.
	\item Griesser, M., \& Hofmann, J. (2018). Editorial: Freie Fahrt für reiche Burschen? Schwarz-Blau ist zurück! Kurswechsel, (3).
	\item Grimm, C. (2018). Wirtschaftspolitische Positionen österreichischer Parteien im historischen Verlauf. Die Ausgestaltung österreichischer Parteiprogrammatiken hinsichtlich neoliberalen Gedankenguts. Momentum Quarterly, 7(3), 136-154. https://doi.org/10.15203/momentumquarterly.vol7.no3.p136-154
	\item Grimm, C., Kapeller, J., \& Pühringer, S. (2018). Paradigms and Policies: The state of economics in the german-speaking countries.
	\item Gräbner, C. (2018). Formal Approaches to Socio-economic Analysis -- Past and Perspectives. Forum for Social Economics, 47(1), 32-63. https://doi.org/10.1080/07360932.2015.1042491
	\item Gräbner, C. (2018). How to Relate Models to Reality? An Epistemological Framework for the Validation and Verification of Computational Models. Journal of Artificial Societies and Social Simulation, 21(3), Artikel 8. https://doi.org/10.18564/jasss.3772
	\item Gräbner, C., Elsner, W., \& Lascaux, A. (2018). To trust or to control: Informal value transfer systems and computational analysis in institutional economics. Journal of Economic Issues, 52(2), 559-569. https://doi.org/10.1080/00213624.2018.1469936
	\item Gräbner, C., Heinrich, T., Kudic, M., \& Vermeulen, B. (2018). The dynamics of and on networks: an introduction. International Journal of Computational Economics and Econometrics, 8(3/4), 229-241.
	\item Gräbner-Radkowitsch, C., \& Strunk, B. (2018). Pluralism in economics: its critiques and their lessons.
	\item Gräbner-Radkowitsch, C., Heimberger, P., Kapeller, J., \& Schütz, B. (2018). Structural change in times of increasing openness: assessing path dependency in European economic integration.
	\item Gräbner-Radkowitsch, C., Heimberger, P., Kapeller, J., \& Springholz, F. (2018). Measuring Economic Openness: A review of existing measures and empirical practices.
	\item Hager, T., Rath, J., \& Wimmler, L. (2018). Work or Die? How Wage dependency determines the production process. In Momentum Kongress Paper (S. 1-37)
	\item Heimberger, P. (2018). Fiscal multipliers, unemployment and debt. Wirtschaftsuniversität Wien.
	\item Heimberger, P. (2018). The dynamic effects of fiscal consolidation episodes on income inequality: Evidence for 17 OECD Countries over 1978-2013.
	\item Hirte, K. (2018). Zeitlichkeit und Tauschfähigkeit bei Rosa Luxemburg und Joseph A. Schumpeter. In Bies, Michael; Giacovelli, Sebastian; Langenohl, Andreas (Hrsg.), Ästhetische Eigenzeiten von Tausch und Gabe Wehrhahn Verlag. 
	\item Hirte, K., \& Thieme, S. (2018). Heterodoxie in der Ökonomik. In Schetsche, Michael; Schmied-Knittel, Ina (Hrsg.), Heterodoxie. Konzepte, Traditionen, Figuren der Abweichung Halem Verlag. 
	\item Kapeller, J. (2018). The Top Journals Club in Economics. Institute for New Economic Thinking (INET), Commentaries.
	\item Kapeller, J., \& Dobusch, L. (2018). Open strategy-making with crowds and communities: Comparing Wikimedia and Creative Commons. Long Range Planning, 51(4), 561-579. https://doi.org/10.1016/j.lrp.2017.08.005
	\item Kapeller, J., Böck, M., Schütz, B., \& Zens, G. (2018). Ökonomische Effekte der Verkehrsreform des Landes Tirol. Johannes Kepler Universität.
	\item Kapeller, J., Ferschli, B., Schütz, B., \& Wildauer, R. (2018). Wie viel bringt die Vermögenssteuer? Neue Aufkommensschätzungen für Österreich. Wirtschafts- und Sozialwissenschaftliche Zeitschrift, 40(1), 146-160.
	\item Landesmann, M., Kapeller, J., Mohr, F. X., \& Schütz, B. (2018). Government policies and financial crises: mitigation, postponement or prevention? Cambridge Journal of Economics, 42(2), 309-330. https://doi.org/10.1093/cje/bew073
	\item Porak, L. (2018). Die Positionierung von Studierenden an öffentlichen Hochschulen. In Momentum Kongress Paper (S. 1-18)
	\item Pühringer, S. (2018). Politische und gesellschaftliche Wirkmächtigkeit von ÖkonomInnen-Netzwerken. In Momentum Kongress Paper (S. 1-10)
	\item Pühringer, S. (2018). The “eternal character” of austerity measures in European crisis policies.
	\item Pühringer, S. (2018, Jul). Die Krise als Katalysator für den Aufschwung des Rechtspopulismus.
	\item Pühringer, S., \& Egger, J. (2018). Krisenbilder von ÖkonomInnen in der Presse. In Walter Ötsch; Silja Graupe (Hrsg.), Macht der Bilder, Macht der Sprache (S. 75-86)
	\item Pühringer, S., \& Grimm, C. (2018). Die deutschsprachige Volkswirtschaftslehre. beigewum.at.
	\item Pühringer, S., \& Liedl, B. (2018). Ökonomische Expertise und polit-ökonomische Machtstrukturen. In AK Kärnten (Hrsg.), Welt aus den Fugen. Wie der Neoliberalismus unser Leben verändert. (S. 41-56). ÖGB Verlag. 
	\item Pühringer, S., \& Ötsch, W. (2018). Neoliberalism and Right-wing Populism: conceptual analogies. Forum for Social Economics, 47(2), 192-203. https://doi.org/10.1080/07360932.2018.1451765
	\item Soder, M., Niedermoser, K., \& Theine, H. (2018). Beyond growth: new alliances for socio-ecological transformation in Austria. Globalizations, 15(4), 520-535. https://doi.org/10.1080/14747731.2018.1454680
	\item Spash, C. L., \& Theine, H. (2018). Voluntary individual carbon trading. Friend or foe? In The Cambridge Handbook of Psychology and Economic Behaviour, Second Edition (S. 595-624). Cambridge University Press 2005. https://doi.org/10.1017/9781316676349.021
	\item Ötsch, W. (2018). Bilder des Rechtspopulismus. In Ötsch, Walter O.; Graupe, Silja (Hrsg.), Macht der Bilder, Macht der Sprache (S. 113-128). Angelika Lenz Verlag. 
	\item Ötsch, W. (2018). Es wächst das Bewusstsein einer multiplen Krise. Agora42 (Philosophisches Wirtschaftsmagazin).
	\item Ötsch, W. (2018). Rechtspopulismus: Ein Gesellschaftsbild mit eskalierender Wirkung. Salzburger Theologische Zeitschrift, 21(1), 7-22.
	\item Ötsch, W., \& Graupe, S. (2018). Einführung: Die Bedeutung von Bildern für das Denken. In Ötsch, Walter O.; Graupe, Silja (Hrsg.), Macht der Bilder, Macht der Sprache (S. 9-19). Angelika Lenz Verlag. 
	\item Ötsch, W., \& Graupe, S. (Hrsg.) (2018). Einführung: Die Bedeutung von Bildern für Denken und Sprechen. In Macht der Bilder, Macht der Sprache. Band 37 der Schriftenreihe der Freien Akademie Falkensee. (S. 75–86). Angelika Lenz Verlag. 
	\item Ötsch, W., \& Graupe, S. (Hrsg.) (2018). Vorwort zu Walter Lippmann: Die öffentliche Meinung. Wie sie entsteht und manipuliert wird. In Die öffentliche Meinung. Wie sie entsteht und manipuliert wird Westend Verlag. 
	\item Ötsch, W., \& Pühringer, S. (2018). Was ist eine Krise? Wie ökonomische Theorien Wahrnehmung formen. Kurswechsel, (4/2018), 7-17.
	\item Ötsch, W., Pühringer, S., \& Hirte, K. (2018). Netzwerke des Marktes : Ordoliberalismus als Politische Ökonomie. Springer VS.
	\item Ötsch, W. (2018). Mythos Markt. Mythos Neoklassik. Das Elend des Marktfundamentalismus. (Kritische Studien zu Markt und Gesellschaft). Metropolis.
	\item Pühringer, S., \& Bäuerle, L. (2018). What economics education is missing: The real world. (Working Paper Serie ök; Nr. 37).
	\item Ötsch, W. (2018). Bilder in der Geschichte der Ökonomie: Das Beispiel der Metapher von der Wirtschaft als Maschine. (Working Paper Serie ök; Band 42).
	\item Ötsch, W. (2018). Wissen und Nicht-Wissen angesichts \glqq des Marktes\grqq{}: Das Konzept von Hayek. (Working Paper Serie ök; Band 43).
	\item Ötsch, W., \& Graupe, S. (2018). Der vergessene Lippmann: Politik, Propaganda und Markt. (Working Paper Serie ök; Nr. 39).
	\item Ötsch, W., \& Pühringer, S. (2018). Marktfundamentalismus als Kollektivgedanke. Mises und die Ordoliberalen. (Working Paper Serie ök; Nr. 41).
	\item Aigner, E., Aistleitner, M., Glötzl, F., \& Kapeller, J. (2018). The Focus of Academic Economics: Before and After the Crisis. (Institute for New Economic Thinking Working Paper Series).
	\item Gräbner, C., Heimberger, P., Kapeller, J., \& Schütz, B. (2018). Desintegration in Europa? Makroökonomische Divergenz und strukturelle Polarisierung. In Momentum Kongress Paper (S. 1-32). (Momentum quarterly).
\end{enumerate}
\subsection*{2017}
\begin{enumerate}
	\item Aistleitner, M., Kapeller, J., \& Steinerberger, S. (2017). Citation Patterns in Economics and Beyond: Assessing the Peculiarities of Economics from Two Scientometric Perspectives.
	\item Aistleitner, M., Kapeller, J., \& Steinerberger, S. (2017). Citation Patterns in Economics and Beyond: Assessing the Peculiarities of Economics from Two Scientometric Perspectives. In Momentum Kongress Paper (S. 1-22)
	\item Bäuerle, L. (2017). Clockwork Economics: Ontologische Grundlagen in der Ökonomik. [Dissertation, Universität Bayreuth]. Universität Bayreuth. https://epub.uni-bayreuth.de/id/eprint/3296/
	\item Bäuerle, L. (2017). Die ökonomische Lehrbuchwissenschaft: Zum disziplinären Selbstverständnis der Volkswirtschaftslehre. Momentum Quarterly, 6(4), 252-270. https://doi.org/10.15203/momentumquarterly.vol6.no4.p252-270
	\item Bäuerle, L. (2017). Fohrmann, Oliver: Im Spiegel des Geldes. Bildung und Identität in Zeiten der Ökonomisierung. Rezension. Sociologia internationalis, 55(2), 310-312.
	\item Ferschli, B., Kapeller, J., Schütz, B., \& Wildauer, R. (2017). Bestände und Konzentration privater Vermögen in Österreich.
	\item Ferschli, B., Kapeller, J., Schütz, B., \& Wildauer, R. (2017). Bestände und Konzentration privater Vermögen in Österreich. In Materialien zu Wirtschaft und Gesellschaft 
	\item Gerhartinger, P., Haunschmid, P., \& Tamesberger, D. (2017). How to explain Wage Growth Slowdown in Austria?
	\item Griesser, M. (2017). Images and imaginaries of unemployed people. Discursive shifts in the transition from active to activating labour market policies in Germany. Critical Social Policy, (2), [online first]. https://doi.org/10.1177/0261018317727481
	\item Griesser, M. (2017). Rezension von „Monika Burmester, Emma Dowling \& Norbert Wohlfahrt (Hg.) (2017): Privates Kapital für soziale Dienste? Wirkungsorientiertes Investment und seine Folgen für die Soziale Arbeit“. Soziales Kapital, 264-267.
	\item Griesser, M. (2017). sezonieri.at: Kollektive Handlungsfähigkeit von ErntearbeiterInnen in Österreich. In Schmidjell, Cornelia; Sedmak, Clemens; Koch, Andreas; Kapferer, Elisabeth; Gaisbauer, Helmut P.; Bogner, Stefan; Wimmer, Bernd (Hrsg.), Lesebuch Soziale Ausgrenzung III (S. 89-92). Mandelbaum Verlag. 
	\item Griesser, M., \& Sauer, B. (2017). Von der sozialen Neuzusammensetzung zur gewerkschaftlichen Erneuerung? MigrantInnen als Zielgruppe der österreichischen Gewerkschaftsbewegung. ÖZS -- Österreichische Zeitschrift für Soziologie, (2), 147-166. https://doi.org/10.1007/s11614-017-0262-x
	\item Grimm, C. (2017). Paradigmatische Homogenität? Aktueller Status und Zukunftsperspektiven der Ökonomik in Deutschland und den USA. In Momentum Kongress Paper (S. 1-16)
	\item Grisold, A., \& Theine, H. (2017). How come we know? The media coverage of economic inequality. International Journal of Communication, 11, 4265-4284.
	\item Gräbner, C. (2017). Dealing adequately with the political element in formal modelling. In Katsikides, Savas and Hanappi, Hardy and Scholz-Wäckerle, Manuel (Hrsg.), Theory and Method of Evolutionary Political Economy (S. 236-254). Routledge. 
	\item Gräbner, C. (2017). Die Rolle des Gleichgewichtskonzepts in der mikroökonomischen Ausbildung. In Till van Treek, Janina Urban (Hrsg.), Wirtschaft neu denken (S. 60-73). iRIGHTS media. 
	\item Gräbner, C. (2017). The Complementary Relationship Between Institutional and Complexity Economics: The Example of Deep Mechanismic Explanations. Journal of Economic Issues, 51(2), 392-400. https://doi.org/10.1080/00213624.2017.1320915
	\item Gräbner, C. (2017). The Complexity of Economies and Pluralism in Economics. Journal of Contextual Economics, 137(3), 193-225.
	\item Gräbner, C., \& Heinrich, T. (2017). Von Onlineplattformen und mittelalterlichen Märkten -- Gleichgewichtsmodelle und agentenbasierte Modellierung zweiseitiger Märkte. TATuP -- Zeitschrift für Technikfolgenabschätzung in Theorie und Praxis, 26(3), 23-29.
	\item Gräbner, C., \& Kapeller, J. (2017). The Micro-Macro Link in Heterodox Economics. In Tae-Hee Jo, Lynne Chester Carlo and D'Ippoliti (Hrsg.), The Routledge Handbook of Heterodox Economics (S. 145-159). Routledge. https://doi.org/10.4324/9781315707587.ch10
	\item Gräbner-Radkowitsch, C. (2017). How to relate models to reality? An epistemological framework for the validation and verification of computational models.
	\item Gräbner-Radkowitsch, C. (2017). The Complexity of Economies and Pluralism in Economics.
	\item Gräbner-Radkowitsch, C., Elsner, W., \& Lascaux, A. (2017). To trust or to control: Informal value transfer systems and computational analysis in institutional economics.
	\item Gräbner-Radkowitsch, C., Elsner, W., \& Lascaux, A. (2017). Trust and Social Control. Sources of cooperation, performance, and stability in informal value transfer systems.
	\item Gräbner-Radkowitsch, C., Heimberger, P., Kapeller, J., \& Schütz, B. (2017). Is Europe disintegrating? Macroeconomic divergence, structural polarization, trade and fragility.
	\item Heimberger, P. (2017). Did Fiscal Consolidation Cause the Double-Dip Recession in the Euro Area? Review of Keynesian Economics, 5(3), 439-458. https://doi.org/10.4337/roke.2017.03.06
	\item Heimberger, P. (2017). Österreichs Staatsausgabenstrukturen im europäischen Vergleich.
	\item Heimberger, P. (2017, Aug). Österreichs Bildungs-, Gesundheits- und Sozialausgaben im europäischen Vergleich: Wenn der Staat spart, kann das für private Haushalte teuer werden.
	\item Heimberger, P. (2017, Okt). Soll der Staat bei Bildung, Gesundheit und Sozialem kürzen? Austeritätspolitik seit der Finanzkrise im Vergleich.
	\item Heimberger, P. (2017, Okt). Vorsicht bei Ländervergleichen – insbesondere bei Staatsausgaben!
	\item Heimberger, P. (2017, Sep). Weniger Staatsausgaben: Abbau des Sozialstaats und Vertiefung von Wirtschaftskrisen.
	\item Heimberger, P., \& Kapeller, J. (2017). The performativity of potential output: Pro-cyclicality and path dependency in coordinating European fiscal policies. Review of International Political Economy, 24(5), 904-928. https://doi.org/10.1080/09692290.2017.1363797
	\item Heimberger, P., \& Kapeller, J. (2017, Sep). Wie ein makroökonomisches Modell die Spaltung der Eurozone befördert.
	\item Heimberger, P., Kapeller, J., \& Schütz, B. (2017). The NAIRU determinants: what’s structural about unemployment in Europe? Journal of Policy Modeling, 39(5), 883-908. https://doi.org/10.1016/j.jpolmod.2017.04.003
	\item Hirte, K. (2017). Agrarpolitik und Agrarökonomie. Zur Ambivalenz zweier wissenschaftlicher Disziplinen. Universität Jena.
	\item Hirte, K. (2017). Zur Performativität in den Wirtschaftswissenschaften. Kernaussagen, Anwendungspotentiale und Grenzen eines Konzepts. In Pfriem, Reinhard; Schneidewind, Uwe (Hrsg.), Transformative Wirtschaftswissenschaft im Kontext nachhaltiger Entwicklung Metropolis Verlag. 
	\item Hirte, K., \& Pühringer, S. (2017). Zur Performativität ökonomischen Wissens und aktuellen ÖkonomInnen-Netzwerken in Deutschland. In Maeße, Jens; Pahl, Hanno; Sparsam, Jan (Hrsg.), Die Innenwelt der Ökonomie. Wissen, Macht und Performativität in der Wirtschaftswissenschaft (S. 363-390). Springer VS Verlag. 
	\item Kapeller, J. (2017). Delayed by outsourcing? Zur Stabilität des Kapitalismus im 21. Jahrhundert (Doppelrezension). Soziologische Revue, 40(4), 547–555. https://doi.org/10.1515/srsr-2017-0072
	\item Kapeller, J. (2017). Delayed by outsourcing? Zur Stabilität des Kapitalismus im 21. Jahrhundert.
	\item Kapeller, J., \& Steinerberger, S. (2017). Stability , fairness and random walks in the bargaining problem. Physica A: Statistical Mechanics and its Applications, 488, 60-71. https://doi.org/10.1016/j.physa.2017.07.008
	\item Kapeller, J., \& Steinerberger, S. (2017). Stability, Fairness and Random Walks in the Bargaining Problem.
	\item Kapeller, J., Ferschli, B., Schütz, B., \& Wildauer, R. (2017). Bestände und Konzentration privater Vermögen in Österreich. Wirtschaft und Gesellschaft, 43(4), 499-534.
	\item Kapeller, J., Pühringer, S., \& Grimm, C. (2017). Zum Profil der deutschsprachigen Volkwirtschaftslehre. Paradigmatische Ausrichtung und politische Orientierung deutschsprachiger Ökonom\_innen. FGW-Studien.
	\item Kapeller, J., Schütz, B., \& Springholz, F. (2017). Internationale Tendenzen und Potentiale der Vermögensbesteuerung. In Dimmel, Nikolaus; Hofmann, Julia; Schenk, Martin; Schürz, Martin (Hrsg.), Handbuch Reichtum – Neue Erkenntnisse aus der Ungleichheitsforschung (S. 477-492). StudienVerlag. 
	\item Pühringer, S. (2017). The success story of ordoliberalism as guiding principle of German economic policy. In Hien, Josef; Joerges, Christian (Hrsg.), Ordoliberalism. Law and the rule of economics (S. 134-158). Hart Publishing. 
	\item Pühringer, S., Bäuerle, L., \& Engarntner, T. (2017). Was denken (zukünftige) ÖkonomInnen?: Einblicke in die politische und gesellschaftliche Wirkmächtigkeit ökonomischen Denkens. GWP -- Gesellschaft, Wirtschaft, Politik, 66(4), 547-556.
	\item Schütz, B. (2017). Zunehmende Ungleichheit der Vermögensverteilung: Rezension von Michael Schneider, Mike Pottenger und John E. King „The Distribution of Wealth – Growing Inequality?“. Wirtschaft und Gesellschaft, 43(3), 449-451.
	\item Ötsch, W., \& Pühringer, S. (2017). Right-wing populism and market-fundamentalism. Two mutually reinforcing threats to democracy in the 21st century. Journal of Language and Politics, 16(4), 497-509. https://doi.org/10.1075/jlp.17027.ots
	\item Kapeller, J., Ferschli, B., Schütz, B., \& Wildauer, R. (2017). Bestände und Konzentration privater Vermögen in Österreich. (167 Aufl.) (Materialien zu Wirtschaft und Gesellschaft -- Working Paper-Reihe der AK Wien).
	\item Pühringer, S. (2017). The “eternal character” of austerity measures in European crisis policies. Evidences from the Fiscal Compact discourse in Austria. (Working Paper Serie ök; Nr. 32).
	\item Pühringer, S. (2017). Think tank networks of German neoliberalism power structures in economics and economic policies in post-war Germany. (Working Paper Serie ök; Nr. 24).
	\item Pühringer, S., \& Griesser, M. (2017). From the \glqq planning euphoria" to the "bitter economic truth\grqq{}: The transmission of economic ideas into German labour market policies in the 1960s and 2000s. (Working Paper Serie ök; Nr. 30).
	\item Pühringer, S., \& Liedl, B. (2017). Argumentationsstrategien einer neoliberalen Reformagenda: Zum Diskursprofil der Agenda Austria in medialen Debatten. (Working Paper Serie ök; Band 27).
	\item Pühringer, S., \& Ötsch, W. (2017). Neoliberalism and Right-wing Populism: conceptual analogies. (Working Paper Serie ök; Band 36).
\end{enumerate}
\subsection*{2016}
\begin{enumerate}
	\item Aistleitner, M. (2016). Perspektiven für eine nachhaltige Automobilindustrie. In Momentum Kongress Paper (S. 1-27)
	\item Beyer, K., \& Bräutigam, L. (2016). Das europäische Schattenbankensystem Typologisierung und die Bewertung regulatorischer Initiativen auf europäischer Ebene.
	\item Beyer, K., \& Bräutigam, L. (2016, Nov). Das europäische Schattenbankensystem: Bestandsaufnahme und gegenwärtige Entwicklungen.
	\item Derks, L. A. C., Ötsch, W., \& Walker, W. (2016). Relationships are Constructed from Generalized Unconscious Social Images Kept in Steady Locations in Mental Space. Journal of Experiential Psychotherapy, 19(1), 3-16.
	\item Eckerstorfer, P., Halak, J., Kapeller, J., \& Schütz, B. et al. (2016). Correcting for the Missing Rich: An Application to Wealth Survey Data. Review of Income and Wealth, 62(4), 605-627. https://doi.org/10.1111/roiw.12188
	\item Graupe, S., \& Ötsch, W. (2016). Dialog nicht erwünscht. Forschung \& Lehre, 23(11), 1000.
	\item Griesser, M., \& Brand, U. (2016, Dez). Verankerung wohlstandorientierter Politik. Working Paper der Kammer für Arbeiter und Angestellte für Wien, Reihe „Materialien zu Wirtschaft und Gesellschaft“, Nr. 165.
	\item Grimm, C. (2016). Postdemokratie, Machtverhältnisse und Ökonomie. In Momentum Kongress Paper (S. 1-22)
	\item Grimm, C., \& Kapeller, J. (2016). Wahrheit und Ökonomie. Kurswechsel, (1), 18-29.
	\item Gräbner, C. (2016). Agent-based computational models -- a formal heuristic for institutionalist pattern modelling? Journal of Institutional Economics, 12(1), 241-261. https://doi.org/10.1017/S1744137415000193
	\item Gräbner, C., Heinrich, T., \& Schwardt, H. (2016). Introduction. In Gräbner, Claudius; Heinrich, Torsten; Schwardt, Henning (Hrsg.), Policy Implications of Recent Advances in Evolutionary and Institutional Economics (S. xxi-xxx). Routledge. 
	\item Gräbner, C., Heinrich, T., \& Schwardt, H. (Hrsg.) (2016). Policy Implications of Recent Advances in Evolutionary and Institutional Economics. Routledge.
	\item Heimberger, P. (2016). Das \glqq strukturelle Defizit\grqq{} in der österreichischen Budgetpolitik: Berechnungsprobleme, Revisionen und wirtschaftspolitische Relevanz. Wirtschaft und Gesellschaft, 42(3), 451-464.
	\item Heimberger, P. (2016). Die Macht ökonomischer Modelle am Beispiel des »Potential Output«-Modells der Europäischen Kommission. In Momentum Kongress Paper (S. 1-9)
	\item Heimberger, P. (2016). Die aktuelle Krise im wirtschaftshistorischen Vergleich mit der Großen Depression der 1930er-Jahre. Wirtschaft und Gesellschaft, 42(1), 161-173.
	\item Heimberger, P. (2016). Helikoptergeld zur Überwindung der Wachstumsprobleme in Europa? Wirtschaft und Gesellschaft, 42(4), 690-695.
	\item Heimberger, P. (2016). Minsky, die globle Finanzkrise und der nächste Finanz-Crash. Wirtschaft und Gesellschaft, 42(3), 515-520.
	\item Heimberger, P. (2016). Warum die Volkswirtschaften der Eurozone den USA und Großbritannien seit der Finanzkrise hinterherhinken: Zur Rolle von Unterschieden in der Geld– und Fiskalpolitik. Studies Research Report, (5).
	\item Heimberger, P. (2016). Wirtschaftliche Stagnation als \glqq neue Normalsituation\grqq{}? Wirtschaft und Gesellschaft, 42(2), 356-361.
	\item Heimberger, P. (2016, Sep). Mehr öffentliche Investitionen sind sinnvoll und erforderlich.
	\item Heimberger, P., \& Kapeller, J. (2016). How economic policy drives European (dis)integration. Institute for New Economic Thinking (INET).
	\item Heimberger, P., Kapeller, J., \& Schütz, B. (2016). What’s ‘structural’ about unemployment in Europe: On the Determinants of the European Commission’s NAIRU Estimates.
	\item Hirte, K. (2016). Agrarische Regelungspolitik und die drei agrarpolitischen „Syndrome“. Wege für eine bäuerliche Zukunft – Zeitschrift der ÖBV/ Via Campesina Austria, 39(3 (343)), 10-11.
	\item Hirte, K. (2016). Die „Landnahme“-These von Rosa Luxemburg – empirisch beobachtbar, aber theoretisch falsifiziert? In Kapeller, Jakob; Pühringer, Stephan; Hirte, Katrin; Ötsch, Walter (Hrsg.), Ökonomie! Welche Ökonomie? (S. 273-313). Metropolis Verlag. 
	\item Hirte, K. (2016). Netzwerke im Internet – eine neue kritische Öffentlichkeit? Das Beispiel Guttenberg. In Imhof, Kurt; Welz, Frank; Fleck, Christian; Vobruba, Georg (Hrsg.), Neuer Strukturwandel der Öffentlichkeit. Verhandlungen des dritten gemeinsamen Kongresses der Deutschen, Österreichischen und Schweizerischen Gesellschaft für Soziologie (S. 1-16). Springer. 
	\item Hirte, K., \& Kuschel, S. (2016). Agrarpolitik und Arbeit – der Einfluss europäischer Agrarpolitikmaßnahmen auf die Arbeit im Agrarsektor. In Ahlert, Maximilian; Fiederer, Franca; Varelmann, Katharina; Kuschel, Sarah; Ewers, Sylvia; Politor, Merlin; Stamp, Katharina (Hrsg.), Frohes Schaffen!? Arbeit in der Landwirtschaft (S. 9-15). Universtity Press. 
	\item Hirte, K., \& Ötsch, W. (Hrsg.) (2016). Ökonomie! Welche Ökonomie? Stand und Status der Wirtschaftswissenschaften. Metropolis Verlag.
	\item Hirte, K., Kapeller, J., Pühringer, S., \& Ötsch, W. (Hrsg.) (2016). Vorwort im Tagungsband Ökonomie! Welche Ökonomie? In Ökonomie! Welche Ökonomie? (S. 2-10). Metropolis Verlag. 
	\item Kapeller, J. (2016). Ein philosophischer Blick auf die Grundlagen internationaler Ökonomie. In Till van Treeck, Janina Urban (Hrsg.), Wirtschaft neu denken – blinde Flecken der Lehrbuchökonomie (S. 108-116)
	\item Kapeller, J. (2016). Internationaler Freihandel: Theoretische Ausgangspunkte und empirische Folgen.
	\item Kapeller, J. (2016). Internationaler Freihandel: Theoretische Ausgangspunkte und empirische Folgen. Wirtschafts- und Sozialwissenschaftliche Zeitschrift, 39(1), 99-122.
	\item Kapeller, J., \& Heimberger, P. (2016). Spezialisierung, Stratifikation und internationale Wirtschaft: Verteilung, Arbeitsteilung und Klassenlagen aus globaler Perspektive. In Momentum Kongress Paper (S. 1-20)
	\item Kapeller, J., \& Scholz-Wäckerle, M. (2016). Evolutionary Political Economy and the Complexity of Economic Policy. In Gräbner,  Claudius / Heinrich, Torsten / Schwardt, Henning (Hrsg.), Policy Implications of Recent Advances in Evolutionary and Institutional Economics (S. 99-122). Routledge. 
	\item Kapeller, J., \& Schütz, B. (2016). Verteilungstendenzen im Kapitalismus. Globale Perspektiven. In Bundesarbeitskammer (Hrsg.), Die Verteilungsfrage. Von Reichtum, Krisen und Ablenkungsmanövern (S. 49-54). ÖGB Verlag. 
	\item Kapeller, J., \& Steinerberger, S. (2016). Emergent Phenomena in Scientific Publishing: A Simulation Exercise. Research Policy, 45(10), 1945–1952. https://doi.org/10.1016/j.respol.2016.08.004
	\item Kapeller, J., Pühringer, S., Hirte, K., \& Ötsch, W. (Hrsg.) (2016). Entwicklung, Zustand und Leerstellen der Ökonomik. In Pühringer, Stephan; Hirte, Katrin; Ötsch, Walter O. (Hrsg.), Ökonomie! Welche Ökonomie? Stand und Status der Wirtschaftswissenschaften. (S. 2-10). Metropolis Verlag. 
	\item Kapeller, J., Pühringer, S., Hirte, K., \& Ötsch, W. (Hrsg.) (2016). Ökonomie! Welche Ökonomie? Stand und Status der Wirtschaftswissenschaften. Metropolis.
	\item Kapeller, J., Schütz, B., \& Springholz, F. (2016). Internationale Tendenzen und Potentiale der Vermögensbesteuerung.
	\item Kapeller, J., Schütz, B., \& Tamesberger, D. (2016). From free to civilized trade: a European perspective. Review of Social Economy, 74(3), 320-328. https://doi.org/10.1080/00346764.2016.1168033
	\item Pühringer, S. (2016). Agenda Austria: Diskursstrategien einer neoliberalen Reformagenda. In Momentum Kongress Paper (S. 1-21)
	\item Pühringer, S. (2016). Still the queens of social sciences? Economists as “public intellectuals” in/after the crisis. In International Conference in Contemporary Social Sciences (Conference Proceedings) (Hrsg.), Crisis and the social sciences: New challenges and perspectives (S. 507-528)
	\item Pühringer, S. (2016). Ökonomisches Denken in der Krise.
	\item Pühringer, S., \& Egger, J. (2016). Wie krank ist unser Wirtschaftssystem? Krisen als Krankheiten im ökonomischen Diskurs. Kuckuck. Notizen zur Alltagskultur, 31(1), 32-37.
	\item Pühringer, S., \& Stelzer-Orthofer, C. (2016). Neoliberale Think Tanks als (neue) Akteure in österreichischen gesellschafts- und sozialpolitischen Diskursen. Die Beispiele des Hayek-Instituts und der Agenda Austria. SWS-Rundschau, 56(1), 75-96.
	\item Pühringer, S., \& Stelzer-Orthofer, C. (2016). Replik zur Replik: Von Vorwürfen der Unwissenschaftlichkeit. SWS-Rundschau -- Sozialwissenschaftliche Studiengesellschaft Rundschau, 3, 447-449.
	\item Schwardt, H., Gräbner, C., Heinrich, T., \& Cordes, C. et al. (2016). Economic Complexity and Trade-Offs in Policy Decisions. In Gräbner, Claudius; Heinrich, Torsten; Schwardt, Henning (Hrsg.), Policy Implications of Evolutionary and Institutional Economics (S. 3-19). Routledge. https://doi.org/10.4324/9781315714257-3
	\item Álvarez Pereira, B., Henderson, H., Lipari, F., \& Furtado, B. A. et al. (2016). Errata in 'The Political Economy of the Kuznets Curve'. Review of Development Economics, 20(4), 817-819. https://doi.org/10.1111/rode.12285
	\item Ötsch, W. (2016). Die Politische Ökonomie „des“ Marktes. Eine Zusammenfassung zur Wirkungsgeschichte von Friedrich A. Hayek. In Kapeller, Jakob; Pühringer, Stephan; Hirte, Katrin; Ötsch, Walter O. (Hrsg.), Ökonomie! Welche Ökonomie? Stand und Status der Wirtschaftswissenschaften (S. 19-50). Metropolis Verlag. 
	\item Ötsch, W. (2016). Die Widersprüche des Mister Perfect: Berechnung – Beherrschung – Perfektion. Agora42 (Philosophisches Wirtschaftsmagazin), (01/2017), 18-23.
	\item Ötsch, W. (2016). Die neoliberale Utopie als Ende aller Utopien. In Pittl, Sebastian; Prüller-Jagenteufel; Gunter (Hrsg.), Unterwegs zu einer neuen ‚Zivilisation geteilter Genügsamkeit‘. Perspektiven utopischen Denkens 25 Jahre nach dem Tod Ignacio Ellacurías (S. 105-119). Vandenhoeck \& Ruprecht uni press. 
	\item Ötsch, W. (2016). Geld und Raum. Anmerkungen zum Homogenisierungsprogramm der beginnenden Neuzeit. In Brodbeck, Karl-Heinz; Graupe, Silja (Hrsg.), Geld! Welches Geld? Geld als Denkform (S. 71-101). Metropolis Verlag. 
	\item Ötsch, W. (2016). Imaginative Grundlagen bei Adam Smith. Aspekte von Bildlichkeit und ihrem Verlust in der Geschichte der Ökonomik. Allgemeine Zeitschrift für Philosophie, 41(3), 315-340.
	\item Ötsch, W. (2016). Populismus und Demagogie. Mit Beispielen von Jörg Haider, Heinz–Christian Strache und Frank Stronach. Foreign Theoretical Trends, (10), 39-46.
	\item Heimberger, P. (2016). Did Fiscal Consolidation Cause the Double Dip Recession in the Euro Area? (wiiw Working Papers; Nr. 130).
	\item Ötsch, W. (2016). Imaginierte Grundlagen bei Adam Smith. (Working Paper Serie ök; Nr. 19).
	\item Heimberger, P., \& Kapeller, J. (2016). The performativity of potential output: pro-cyclicality and path dependency in coordinating European fiscal policies. (Institute for New Economic Thinking Working Paper Series; Nr. 50).
	\item Beyer, K., \& Bräutigam, L. (2016). Das europäische Schattenbankensystem – Typologisierung und die Bewertung regulatorischer Initiativen auf europäischer Ebene. (Materialien zu Wirtschaft und Gesellschaft. Working Paper-Reihe der AK Wien). Arbeiterkammer Wien.
\end{enumerate}
\subsection*{2015}
\begin{enumerate}
	\item Aistleitner, M., Fölker, M., \& Kapeller, J. (2015). Die Macht der Wissenschaftsstatistik und die Entwicklung der Ökonomie.
	\item Aistleitner, M., Fölker, M., \& Kapeller, J. (2015). Die Macht der Wissenschaftsstatistik und die Entwicklung der Ökonomie. In Momentum Kongress Paper (S. 1-21)
	\item Aistleitner, M., Fölker, M., \& Kapeller, J. (2015). Die Macht der Wissenschaftsstatistik und die Entwicklung der Ökonomie. Journal of Contextual Economics – Schmollers Jahrbuch, 135(2), 111–132.
	\item Aistleitner, M., Fölker, M., Kapeller, J., \& Mohr, F. X. et al. (2015). Verteilung und Gerechtigkeit: Philosophische Perspektiven. Wirtschaft und Gesellschaft, 41(1), 71-106.
	\item Beyer, K. (2015, Jul). Nachfrageseitige Ursachen der Expansion des Schattenbankensystems.
	\item Griesser, M. (2015). Der Staat als Wissensapparat. Konzeptionelle Überlegungen zu einer nicht-funktionalistischen Funktionsanalyse des Sozialstaats. Zeitschrift für Sozialreform, 61, 103-124. https://doi.org/10.1515/zsr-2015-0105
	\item Griesser, M. (2015). Rezension von „Martina Benz: Zwischen Migration und Arbeit. Worker Centers und die Organisierung prekär und informell Beschäftigter in den USA“. Forum Wissenschaft.
	\item Griesser, M., \& Sauer, B. (2015). Zwischen Konjunkturpuffer und Tauschobjekt. Gewerkschaftliche Perspektiven auf Migration im Österreich der Zweiten Republik. Kurswechsel, (Heft 4), 58-66.
	\item Griesser, M., Hirte, K., \& Pühringer, S. (2015). ÖkonomInnen und Politik – Analyse zur politischen Einflussnahme deutschsprachiger ÖkonomInnen. Forschungsbericht: Förderer: Jubiläumsfonds.
	\item Grimm, C. (2015). Wirtschaftspolitische Ausrichtung österreichischer Parteien im historischen Verlauf. Die Ausgestaltung österreichischer Parteiprogrammatiken unter dem Einfluss neoliberalen Gedankenguts.
	\item Gräbner, C., \& Kapeller, J. (2015). New Perspectives on Institutionalist Pattern Modeling: Systemism, Complexity and Agent-Based modeling. Journal of Economic Issues, 49(2), 433-440. https://doi.org/10.1080/00213624.2015.1042765
	\item Heimberger, P. (2015). 'Strukturreformen' und Lohnkürzungen in Griechenland: Erwartungen, Ergebnisse und Folgen. WISO -- Wirtschafts- und sozialpolitische Zeitschrift, 38(3), 104-121.
	\item Heimberger, P. (2015). Die griechische Schuldendebatte und das Mantra von den \glqq notwendigen Strukturreformen\grqq{}. WISO direkt, (05).
	\item Heimberger, P. (2015). Eine fiskalpolitische Lösung für die Eurozone. Wirtschaft und Gesellschaft, 41(3), 449-458.
	\item Heimberger, P. (2015). Griechenland: Das Scheitern der europäischen Krisenpolitik. EU-Infobrief, (3), 11-15.
	\item Heimberger, P. (2015). Raus aus dem Euro? Wirtschaft und Gesellschaft, 41(4), 603-614.
	\item Heise, A., Hirte, K., Ötsch, W., \& Pühringer, S. et al. (2015). ÖkonomInnen und Ökonomie. Eine wissenschaftssoziologische Entwicklungsanalyse zum Verhältnis von ÖkonomInnen und Ökonomie im deutschsprachigen Raum ab 1945. Hans Böckler Stiftung.
	\item Hirte, K. (2015). Bezeichnende Konstellation. Zum Eröffnungstag „Agrarpolitik“ in Österreich auf dem Podium: REWE, RWA Raiffeisen und Landwirtschaftskammer. Wege für eine bäuerliche Zukunft – Zeitschrift der ÖBV/ Via Campesina Austria, 38(2 (337)), 15.
	\item Hirte, K. (2015). Das Ökonomie-Monopol an den Agrarfakultäten. Wege für eine bäuerliche Zukunft – Zeitschrift der ÖBV/ Via Campesina Austria, 38(3 (338)), 22-23.
	\item Hirte, K. (2015). Politik und ihre Ad-hoc- Gremien in Krisenzeiten. In Momentum-Kongress (Hrsg.), Momentum Kongress Paper (S. 1-22)
	\item Hirte, K., Pühringer, S., \& Bräutigam, L. (Hrsg.) (2015). Markt! Welcher Markt? Der interdisziplinäre Diskurs um Märkte und Marktwirtschaft. Metropolis Verlag.
	\item Kapeller, J. (2015). Allgemeine Modelltheorie und ökonomische Modelle. Erwägen, Wissen, Ethik, 26(3), 387-389.
	\item Kapeller, J. (2015). Beyond Foundations: Systemism in Economic Thinking. In Jo, Tae-Hee / Todorovka, Zdravka (Hrsg.), Advancing the Frontiers of Heterodox Economics: Essays in Honor of Frederic S. Lee (S. 115-134). Routledge. 
	\item Kapeller, J. (2015). Wirtschaftspolitik, Verteilungsgerechtigkeit und Demokratie. In AK Burgenland (Hrsg.), Gerechtigkeit muss sein (S. 150-165)
	\item Kapeller, J., \& Gräbner-Radkowitsch, C. (2015). The Micro‐Macro Link in Heterodox Economics.
	\item Kapeller, J., \& Pühringer, S. (2015). Demokratie in Liberalismus und Neoliberalismus. In Seckauer, Hansjörg/Stelzer-Orthofer, Christine/Kepplinger, Brigitte (Hrsg.), Das Vorgefundene und das Mögliche. Beiträge zur Gesellschafts- und Sozialpolitik zwischen Ökonomie und Moral (S. 111-127). Mandelbaum. 
	\item Kapeller, J., \& Schütz, B. (2015). Conspicuous Consumption, Inequality and Debt: The Nature of Consumption-driven Profit-led Regimes. Metroeconomica, 66(1), 51-70. https://doi.org/10.1111/meca.12061
	\item Kapeller, J., \& Schütz, B. (2015). Verteilungstendenzen im Kapitalismus -- Nationale und globale Perspektiven. Kurswechsel, (2/2015), 54-68.
	\item Kapeller, J., \& Schütz, B. (2015). Verteilungstendenzen im Kapitalismus. In Momentum Kongress Paper (S. 1-21)
	\item Kapeller, J., \& Schütz, B. (2015). Verteilungstendenzen im Kapitalismus: Nationale und Globale Perspektiven.
	\item Kapeller, J., \& Schütz, B. (2015, Nov). Verteilungstendenzen im Kapitalismus: Globale Perspektiven.
	\item Kapeller, J., Schütz, B., \& Tamesberger, D. (2015). Moralität, Wettbewerb und internationaler Handel: Eine europäische Perspektive. In Hansjörg Seckauer, Christine Stelzer-Orthofer, Brigitte Kepplinger (Hrsg.), Das Vorgefundene und das Mögliche. Beiträge zur Gesellschafts- und Sozialpolitik zwischen Ökonomie und Moral (S. 213-227). Mandelbaum Verlag. 
	\item Kapeller, J., Schütz, B., \& Tamesberger, D. (2015). Moralität, Wettbewerb und internationaler Handel: Eine europäische Perspektive. In Seckauer, Hansjörg; Stelzer-Orthofer, Christine; Kepplinger, Brigitte (Hrsg.), Das Vorgefundene und das Mögliche. Beiträge zur Gesellschafts- und Sozialpolitik zwischen Ökonomie und Moral. Festschrift für Josef Weidenholzer (S. 213-227). Mandelbaum Verlag. 
	\item Kapeller, J., Schütz, B., \& Tamesberger, D. (2015). Von freien zu zivilisierten Märkten. Ein New Deal für die europäische Handelspolitik.
	\item Pühringer, S. (2015). Kontinuitäten neoliberaler Wirtschaftspolitik. Die Austeritätsdebatte als Spiegelbild diskursiver Machtverwerfungen innerhalb der Ökonomik. In Marterbauer, Markus/Mesch, Michael/Rehm, Miriam/Reiterlechner, Christine (Hrsg.), Das Scheitern des neoklassischen Paradigmas – Wirtschaftspolitik in der EU (S. 159-174). ÖGB Verlag. 
	\item Pühringer, S. (2015). Markets as “ultimate judges” of economic policies -- Angela Merkel´s discourse profile during the economic crisis and the European crisis policies. On the Horizon, 23(3), 246-259.
	\item Pühringer, S. (2015). Marktmetaphoriken in Krisennarrativen von Angela Merkel. In Ötsch, Walter/Hirte, Katrin/Pühringer, Stephan/Bräutigam, Lars (Hrsg.), Markt! Welcher Markt? Der interdisziplinäre Diskurs um Märkte und Marktwirtschaft. (S. 229-252). Metropolis. 
	\item Pühringer, S. (2015). The strange non-crisis of economics. Economic crisis and the crisis policies in economic and political discourses. Universität Linz.
	\item Pühringer, S. (2015). “Harte” Sanktionen für “budgetpolitische Sünder”. Kritische Diskursanalyse der Debatte zum Fiskalpakt in meinungsbildenden österreichischen Qualitätsmedien. Momentum Quarterly, 4(1), 23-41.
	\item Pühringer, S., \& Hirte, K. (2015). The financial crisis as a heart attack: Discourse profiles of economists in the financial crisis. Journal of Language and Politics, 14(4), 599-625. https://doi.org/10.1075/jlp.14.4.06puh
	\item Ötsch, W. (2015). Markt und Markttheorie. Vorwort und Überblick. In Ötsch, Walter O.; Hirte, Katrin; Pühringer, Stephan; Bräutigam, Lars (Hrsg.), Markt! Welcher Markt? Der interdisziplinäre Diskurs um Märkte und Marktwirtschaft (S. 7-24). Metropolis Verlag. 
	\item Ötsch, W. (2015). Schönheit und Macht. Drei Beispiele aus der Kulturgeschichte. In Ridler, Gerda (Hrsg.), Mythos Schönheit. Facetten des Schönen in Natur, Kunst und Gesellschaft (S. 259-263). Hatje Cantz Verlag. 
	\item Ötsch, W. (2015). Ökonomie und Moral. Eine kurze Theoriegeschichte. In Seckauer, Hansjörg; Stelzer-Orthofer, Christine; Kepplinger, Brigitte (Hrsg.), Das Vorgefundene und das Mögliche. Beiträge zur Gesellschafts- und Sozialpolitik zwischen Ökonomie und Moral (S. 100-110). Mandelbaum Verlag. 
	\item Ötsch, W., Hirte, K., Pühringer, S., \& Bräutigam, L. (Hrsg.) (2015). Markt! Welcher Markt? Der interdisziplinäre Diskurs um Märkte und Marktwirtschaft. Metropolis.
	\item Hirte, K. (2015). Märkte und die Anerkennung von Arbeit. Zum Zusammenhang schlecht bezahlter Arbeiten und der Struktur der Arbeitsergebnisse. In Ötsch, Walter; Hirte, Katrin; Pühringer, Stephan; Bräutigam, Lars (Hrsg.), Markt! Welcher Markt? (S. 281-322). (Kritische Studien zu Markt und Gesellschaft). Metropolis. 
\end{enumerate}
\subsection*{2014}
\begin{enumerate}
	\item Beyer, K. (2014). Emanzipation bei Marx und seine Kritik an Proudhon und dessen ideengeschichtlichen Nachfahren. In Momentum Kongress Paper (S. 1-18)
	\item Beyer, K. (2014, Mär). Die Risiken im Schatten des Systems.
	\item Beyer, K., \& Bräutigam, L. (2014). Offshore Aspects of Shadow Banking. With Considerations on the Recent Financial Crisis. In Ötsch, Walter O.; Grözinger, Gert; Beyer, Karl M. (Hrsg.), The Political Economy of Offshore Jurisdictions Metropolis Verlag. 
	\item Eckerstorfer, P., Halak, J., Kapeller, J., \& Schütz, B. et al. (2014). Die Vermögensverteilung in Österreich und das Aufkommenspotenzial einer Vermögenssteuer. Wirtschaft und Gesellschaft, 40(1), 63-81.
	\item Grimm, C., Kapeller, J., \& Springholz, F. (2014). Führt Pluralismus in der ökonomischen Theorie zu mehr Wahrheit? In Wissen! Welches Wissen? Zu Wahrheit, Theorien und Glauben sowie ökonomischen Theorien Metropolis Verlag. 
	\item Hirte, K. (2014). Agrargiganten im Osten. Zur neuerlichen Transformation der transformierten deutschen Agrarstrukturen. In Brähler, Elmar; Wagner, Wolf (Hrsg.), 25 Jahre Mauerfall – kein Ende mit der Wende? (S. 277-­290). Psychosozial-Verlag. 
	\item Hirte, K. (2014). Landwirtschaft, Ideologien und \glqq ...ismen\grqq{}. Wege für eine bäuerliche Zukunft – Zeitschrift der ÖBV/ Via Campesina Austria, 37(2 (332)), 14-15.
	\item Hirte, K., \& Pühringer, S. (2014). Performative Wissenschaft: Ökonomiekritik, Ökonomietheorien und die Verantwortung von ÖkonomInnen. In Hirte Katrin, Thieme Sebastian, Ötsch Walter Otto (Hrsg.), Wissen! Welches Wissen? Zu Wahrheit, Theorien und Glauben sowie ökonomischen Theorien (S. 267-302). Metropolis Verlag. 
	\item Hirte, K., \& Pühringer, S. (2014). ÖkonomInnen und Ökonomie in der Krise? Eine diskurs- und netzwerkanalytische Sicht. WISO -- Wirtschafts- und sozialpolitische Zeitschrift, 1, 159-178.
	\item Hirte, K., \& Ötsch, W. (2014). Vorwort im Sammelband \glqq Wissen! Welches Wissen?\grqq{}. In Hirte Katrin, Thieme Sebastian, Ötsch Walter Otto (Hrsg.), Wissen! Welches Wissen? Zu Wahrheit, Theorien und Glauben sowie ökonomischen Theorien (S. 7-16). Metropolis Verlag. 
	\item Hirte, K., Thieme, S., \& Ötsch, W. (2014). Vorwort Wissen! Welches Wissen? Zu Wahrheit, Theorien und Glauben sowie ökonomischen Theorien. In Wissen! Welches Wissen? Zu Wahrheit, Theorien und Glauben sowie ökonomischen Theorien Metropolis Verlag. 
	\item Hirte, K., Thieme, S., \& Ötsch, W. (Hrsg.) (2014). Wissen! Welches Wissen? Zu Wahrheit, Theorien und Glauben sowie ökonomischen Theorien. Metropolis Verlag.
	\item Hirte, K., Thieme, S., \& Ötsch, W. (Hrsg.) (2014). Wissen! Welches Wissen? Zu Wahrheit, Theorien und Glauben sowie ökonomischen Theorien. Metropolis Verlag.
	\item Kapeller, J. (2014). Die Rückkehr des Rentiers. Rezension zu: Piketty, Thomas (2014): Capital in the 21st century. Cambridge: Harvard University Press. Wirtschaft und Gesellschaft, 40(2), 329-346.
	\item Kapeller, J. (2014). Economic Change and Change in Economics. Universität Linz.
	\item Kapeller, J. (2014). The return of the rentier. Review of: Piketty, Thomas (2014): Capital in the 21st century. Harvard University Press, 685 pages.
	\item Kapeller, J., \& Hubmann, G. (2014). Fortschrittsidee und Politische Vision [Progress and Politics]. Momentum Quarterly, 3(4), 235-245.
	\item Kapeller, J., \& Schütz, B. (2014). Debt, Boom, Bust: A Theory of Minsky-Veblen Cycles. Journal of Post Keynesian Economics, 36(4).
	\item Kapeller, J., \& Steinerberger, S. (2014). Modeling the Evolution of Preferences: An Answer to Schubert and Cordes. Journal of Institutional Economics, 10(2), 337-347. https://doi.org/10.1017/S1744137413000362
	\item Kapeller, J., Grimm, C., \& Springholz, F. (2014). Führt Pluralismus in der ökonomischen Theorie zu mehr Wahrheit? In Hirte, Katrin, Thieme, Sebastian, Ötsch, Walter (Hrsg.), Wissen! Welches Wissen? (S. 147-163). Metropolis. 
	\item Kapeller, J., Schütz, B., \& Tamesberger, D. (2014). From Free to Civilized Markets. In Momentum Kongress Paper (S. 1-29)
	\item Kapeller, J., Schütz, B., \& Tamesberger, D. (2014). From Free to Civilized Markets: First steps towards Eutopia. Semantic Scholar.
	\item Kapeller, J., Schütz, B., \& Tamesberger, D. (2014). Making Morality Matter: Civilized Markets and European Values. Journal for a Progressive Economy.
	\item Pühringer, S. (2014). Mythen über Reichtum und Macht: Demokratie ist nicht käuflich. In Beigewum/ATTAC/Armutskonferenz (Hrsg.), Mythen des Reichtums (S. 149-158). VSA Verlag. 
	\item Pühringer, S. (2014, Mär). Ökonomische Krisen als Krankheiten und Katastrophen?
	\item Stelzer-Orthofer, C., \& Pühringer, S. (2014). Subventionierung von Lohnkosten als Mittel zur Armutsvermeidung. In Dimmel, N./Schenk, M./Stelzer-Orthofer, C. (Hrsg.), Handbuch Armut in Österreich (S. 817-831). StudienVerlag. 
	\item Ötsch, W. (2014). How to Hide Secrecy Jurisdictions. In Ötsch, Walter O.; Grözinger, Gerd; Beyer, Karl M.; Bräutigam, Lars (Hrsg.), The Political Economy of Offshore Jurisdictions (S. 61-75). Metropolis Verlag. 
	\item Ötsch, W. (2014). Populismus und Demagogie. Mit Beispielen von Jörg Haider, Hans-­‐Christian Strache und Frank Stronach sowie der Tea Party”. In Gressl, Martin, Klemenjak, Martin, Klepp. Cornelia, Pichler, Heinz, Rottermann, Doris und Scherling, Josefine (Hrsg.), Populismus und Rassismus im Vormarsch?, Schriftenreihe „Arbeit und Bildung“ (S. 12-26). 
	\item Ötsch, W., \& Schmidt, M. (2014). The Political Economy of Offshore Jurisdictions. An Introduction. In Ötsch, Walter O.; Grözinger, Gerd; Beyer, Karl M.; Bräutigam, Lars (Hrsg.), The Political Economy of Offshore Jurisdictions (S. 7-23). Metropolis Verlag. 
	\item Ötsch, W., Grözinger, G., Beyer, K., \& Bräutigam, L. (Hrsg.) (2014). The Political Economy of Offshore Jurisdictions. Metropolis Verlag.
	\item Eckerstorfer, P., Halak, H., Kapeller, J., \& Schütz, B. et al. (2014). Vermögen in Österreich. (Materialien zu Wirtschaft und Gesellschaft; Nr. 126). AK Wien.
	\item Griesser, M., \& Sauer, B. (2014). MigrantInnen als Zielgruppe. Solidarische Beratungs- und Unterstützungsangebote von ArbeitnehmerInnenorganisationen in Österreich. (Studie -- Abschlussbericht). Universität, Institut für Politikwissenschaften.
\end{enumerate}
\subsection*{2013}
\begin{enumerate}
	\item Dobusch, L., \& Kapeller, J. (2013). Breaking New Paths: Theory and Method in Path Dependence Research. Schmalenbach Business Review, 65(2), 288-311. https://doi.org/10.1007/bf03396859
	\item Dobusch, L., \& Kapeller, J. (2013). Diskutieren statt Ignorieren: Eckpfeiler für interessierten Pluralismus in der Ökonomie. Der öffentliche Sektor -- The Public Sector, 39(3).
	\item Dobusch, L., \& Kapeller, J. (2013). Practicing Pluralism: A Rejoinder to W. Robert Brazelton. Journal of Economic Issues, 47(4), 1035-1037. https://doi.org/10.2753/JEI0021-3624470412
	\item Eckerstorfer, P., Halak, J., Kapeller, J., \& Schütz, B. et al. (2013). Reichtumsverteilung in Österreich. WISO -- Wirtschafts- und sozialpolitische Zeitschrift, 36(4).
	\item Hirte, K. (2013). Deutsche Agrarpolitikprofessoren vor und nach 1945. Wege für eine bäuerliche Zukunft – Zeitschrift der ÖBV/ Via Campesina Austria, 36(2 (327)), 18-20.
	\item Hirte, K. (2013). Deutsche Europapolitik vor und nach 1945. Wege für eine bäuerliche Zukunft – Zeitschrift der ÖBV/ Via Campesina Austria, 36(3 (328)), 22-23.
	\item Hirte, K. (2013). ÖkonomInnen in der Finanzkrise. Diskurse. Netzwerke. Initiativen. Metropolis Verlag.
	\item Hirte, K. (2013). „Persilschein“ – Netzwerke: Für Bruchlosigkeit in Umbruchzeiten. In Schönhuth, Michael; Gamper, Markus; Kronenwett, Michael; Stark, Martin (Hrsg.), Visuelle Netzwerkforschung. Qualitative, quantitative und partizipative Zugänge. (S. 331-353). transcript Verlag. 
	\item Kapeller, J. (2013). Model-Platonism in Economics: On a classical epistemological critique. Journal of Institutional Economics, 9(2), 199-221. https://doi.org/10.1017/S1744137413000052
	\item Kapeller, J., \& Schütz, B. (2013). Exploring Pluralist Economics: The Case of the Minsky-Veblen Cycles. Journal of Economic Issues, 47(2), 515-524. https://doi.org/10.2753/JEI0021-3624470225
	\item Kapeller, J., \& Steinerberger, S. (2013). How Formalism shapes Perception: An Experiment on Mathematics as a Language. International Journal of Pluralism and Economics Education, 4(2), 138-156. https://doi.org/10.1504/ijpee.2013.055441
	\item Kapeller, J., \& Wolkenstein, F. (2013). The grounds of solidarity: From liberty to loyalty. European Journal of Social Theory, 16(4), 476-491. https://doi.org/10.1177/1368431013479689
	\item Kapeller, J., Schütz, B., \& Steinerberger, S. (2013). The impossibility of rational consumer choice -- A problem and its solution. Journal of Evolutionary Economics, 23(1), 39-60. https://doi.org/10.1007/s00191-012-0268-2
	\item Kapeller, J., Schütz, B., \& Tamesberger, D. (2013). Die Regulation der Routine: Über die regulatorischen Spielräume zur Etablierung nachhaltigen Konsums. Wirtschaft und Gesellschaft, 39(2), 207-231.
	\item Nordmann, J. (2013). Grenzen aktueller Krisendebatten Über Konstruktionen der öffentlichen Meinung und das Verhältnis von Sach‐ und Grundsatzdiskussionen in (neo)liberalen Demokratien.
	\item Nordmann, J. (2013). Grenzen aktueller Krisendebatten. Über Konstruktionen der öffentlichen Meinung und das Verhältnis von Sach- und Grundsatzdiskussionen in (neo)liberalen Demokratien. In Wengeler Martin; Ziem, Alexander (Hrsg.), Sprachliche Konstruktionen von Krisen : interdisziplinäre Perspektiven auf ein fortwährend aktuelles Phänomen (S. 53-66). Hempen. 
	\item Plaimer, W., \& Pühringer, S. (2013). Der Fiskalpakt und seine Implementation in Österreich. In Momentum Kongress Paper (S. 1-20)
	\item Pühringer, S. (2013). „Arbeitsmarktferne“ Personen – wer sind die? Zu veränderten Exklusionsdynamiken in neokapitalistischen Gesellschaften. SWS-Rundschau -- Sozialwissenschaftliche Studiengesellschaft Rundschau, 53(4), 361-381.
	\item Pühringer, S. (2013, Sep). Ahnungslos, aber nicht tatenlos – Wie ÖkonomInnen seit der Finanzkrise Politik mach(t)en.
	\item Schütz, B. (2013). Marx, Keynes und die Idee des gesellschaftlichen Fortschritts: Suche nach neuen politischen Visionen. In Momentum Kongress Paper (S. 1-21)
	\item Thieme, S., \& Hirte, K. (2013). Mainstream, Orthodoxie und Heterodoxie – Zur Klassifizierung der Wirtschaftswissenschaften.
	\item Ötsch, W. (2013). Die Macht der Ratingagenturen: Governance in der Ideologie 'des Marktes'. In Brodbeck, Karl-Heinz (Hrsg.), Alternative Länder-Ratings, Schriftenreihe der Finance \& Ethics Academy, Band 5 (S. 58-98). Shaker Verlag. 
	\item Ötsch, W. (2013). Marktradikalität. Der Diskurs von „dem Markt“. In Günther, Lea-Simone; Hertlein, Saskia;  Klüsener, Vea und Raasch, Markus (Hrsg.), Radikalität. Religiöse, politische und künstlerische Radikalismen in Geschichte und Gegenwart. Band 2: Frühe Neuzeit und Moderne (S. 254-279). Königshausen \& Neumann. 
	\item Ötsch, W. (2013). The Deep Meening of ‘Market’: Understanding Neoliberal-Market-Radical Reasoning. Human Geography, 6(2), 11-25.
	\item Ötsch, W., Beyer, K., \& Mader, L. (2013). Die Finanzkrise 2007-2009 als Krise von Schattenbanken. Eine einführende institutionelle Analyse. In Jahrestagung des Ausschusses Evolutorische Ökonomik im Verein für Sozialpolitik 
	\item Eckerstorfer, P., Halak, H., Kapeller, J., \& Schütz, B. et al. (2013). Bestände und Verteilung der Vermögen in Österreich. (Materialien zu Wirtschaft und Gesellschaft; Band 122). Abteilung Wirtschaftswissenschaft und Statstik der Kammer für Arbeiter und Angestellte.
\end{enumerate}
\subsection*{2012}
\begin{enumerate}
	\item Beyer, K. (2012). Illusionen eines Zirkulationskünstlers? Pierre-Joseph Proudhon auf dem ökonomiekritischen Prüfstand. Universität Wien.
	\item Dobusch, L., \& Kapeller, J. (2012). A Guide to Paradigmatic Self-Marginalization -- Lessons for Post-Keynesian Economists. Review of Political Economy, 24(3), 469-487. https://doi.org/10.1080/09538259.2012.701928
	\item Dobusch, L., \& Kapeller, J. (2012). Heterodox United vs. Mainstream City? Sketching a framework for interested pluralism in economics. Journal of Economic Issues, 46(4), 1035-1058. https://doi.org/10.2753/JEI0021-3624460410
	\item Hirte, K. (2012). Die Umstrukturierung der LPGen in Thüringen ab 1990. Druckerei Sömmerda GmbH.
	\item Hirte, K. (2012). Die ersten Professoren für Agrarpolitik und Agrarökonomie ab 1945 an den westdeutschen Universitäten und ihre Vergangenheit. In Buchsteiner Martin, Strahl Antje (Hrsg.), Thünen-Jahrbuch (S. 87-114)
	\item Hirte, K. (2012). Würdigungs-Netzwerke, gewolltes Nichtwissen und Geschichtsschreibung. Österreichische Zeitschrift für Geschichtswissenschaften, 23(1), 155-185.
	\item Hirte, K., \& Pühringer, S. (2012). ÖkonomInnen in der Finanzkrise. Analyse zur Positionierung deutschsprachiger Ökonomen im Kontext ihrer strukturellen Verankerung.
	\item Huber, J., Kaindlstorfer, L., \& Kapeller, J. (2012). Bridges to Past Polls: Die oberösterreichische Erfahrung mit Vorwahlen als demokratisches Instrument. In Momentum Kongress Paper 
	\item Kapeller, J., \& Dobusch, L. (2012). A guide to paradigmatic Self-marginalization -- Lessons for Post-Keynesian Economists. In Lavoie M. und Lee F.S. (Hrsg.), In Defense of Post-Keynesian and Heterodox Economics. (S. 62-86). Routledge. 
	\item Kapeller, J., \& Hubmann, G. (2012). Solidarisch Handeln: Konzeptionen, Ursachen und Implikationen. Momentum Quarterly, 1(3), 139-152.
	\item Kapeller, J., \& Schütz, B. (2012). Conspicuous consumption, inequality and debt: The nature of consumption-driven profit-led regimes.
	\item Kapeller, J., Schütz, B., \& Tamesberger, D. (2012). Die Regulation der Routine. In Momentum Kongress Paper (S. 1-31)
	\item Kapeller, J., Schütz, B., \& Tamesberger, D. (2012). Konsum demokratisch gestalten: Spielräume zur Etablierung nachhaltigen Konsums. WISO -- Wirtschafts- und sozialpolitische Zeitschrift, 167-183.
	\item Nordmann, J. (2012). Die intellektuelle Geschichte des Neoliberalismus im Spiegel des alten Liberalismus. ksoe-Dossier der katholischen Erwachsenenbildung Österreich.
	\item Nordmann, J. (2012). Vorwort in \glqq Demokratie! Welche Demokratie? Postdemokratie kritisch hinterfragt\grqq{}. In Demokratie! Welche Demokratie? Postdemokratie kritisch hinterfragt (S. 7-14)
	\item Nordmann, J., Hirte, K., \& Ötsch, W. (Hrsg.) (2012). Demokratie! Welche Demokratie? Postdemokratie kritisch hinterfragt. Metropolis Verlag.
	\item Plaimer, W. (2012). Postdemokratie in Österreich? In Momentum-Kongress (Hrsg.), Momentum Kongress Paper (S. 1-17)
	\item Plaimer, W. (2012). Postdemokratie in Österreich? In Nordmann, Jürgen; Hirte, Katrin; Ötsch, Walter O. (Hrsg.), Demokratie! Welche Demokratie? Postdemokratie kritisch hinterfragt (S. 159-174). Metropolis Verlag. 
	\item Pühringer, S. (2012). Soziale Frage im Wandel. Probleme und Perspektiven des Sozialstaates und der Arbeitsgesellschaft. Kontraste -- Presse- und Informationsdienst für Sozialpolitik, (2/2012), 22-23.
	\item Pühringer, S. (2012). Öffentlicher Vernunftgebrauch -- ein probantes Mittel zur Bekämpfung von Ungerechtigkeit? Studentisches Soziologiemagazin.
	\item Pühringer, S., \& Hirte, K. (2012). Economists and Economics. Discourse profiles of economists in the financial crisis. In Association française d'économie politique (Hrsg.), Joint conference of AHE, IIPPE, FAPE. Kongress Political economy and the outlook for capitalism 05.-07.07.2012 (S. 1-16). 
	\item Pühringer, S., \& Kapeller, J. (2012). How Liberalism lost its concept of democracy. In Momentum Kongress Paper (S. 1-22)
	\item Ötsch, W. (2012). Der freie Markt. In Czejkowska, Agnieszka (Hrsg.), Imagine Economy. Neoliberale Metaphern im wirtschaftlichen Diskurs (S. 39-45). Erhard Löcker Verlag. 
	\item Ötsch, W. (2012). Krise des Euroraums. In Starke, Herbert; Horvath, Patrick; Weinzierl, Rupert (Hrsg.), Die Vision Zentraleuropa im 21. Jahrhundert (S. 68-71). Arbeitsgemeinschaft für wissenschaftliche Wirtschaftspolitik. 
	\item Ötsch, W. (2012). Politische Ökonomie und Gesellschaft. Eine theoriegeschichtliche Skizze. In Grözinger, Gerd; Reich, Utz-Peter (Hrsg.), Entfremdung – Ausbeutung – Revolte (S. 145-165). Metropolis Verlag. 
	\item Ötsch, W. (2012). Staatsschuldenkrise und ökonomisches Denken – im Euroraum und in Zentraleuropa. In Horvath, Patrick, Skarke, Herbert  und Weinzierl, Rupert (Hrsg.), Die “Vision Zentraleuropa” im 21. Jahrhundert. Festschrift zum 90. Geburtstag von Heinz Kienzl (S. 68-71). Arbeitsgemeinschaft für wissenschaftliche Wirtschaftspolitik (WIWIPOL). 
	\item Dobusch, L., \& Kapeller, J. (2012). Regulatorische Unsicherheit und Private Standardisierung: Koordination durch Ambiguität. In Steuerung durch Regeln (Band 22, S. 43-81). (Managementforschung). https://doi.org/10.1007/978-3-8349-4349-1\_2
\end{enumerate}
\subsection*{2011}
\begin{enumerate}
	\item Blaha, B., Kapeller, J., \& Weidenholzer, J. (Hrsg.) (2011). Solidarität -- Beiträge für eine gerechte Gesellschaft. Braumüller.
	\item Bäuerle, L., Behr, M., \& Hütz-Adams, F. (2011). Im Boden der Tatsachen: Metallische Rohstoffe und ihre Nebenwirkungen. EB-Verl. Brandt.
	\item Dobusch, L., \& Kapeller, J. (2011). Wirtschaft, Wissenschaft und Politik: Die sozialwissenschaftliche Bedingtheit linker Reformpolitik. Prokla, 41(3), 389-404.
	\item Hirte, K. (2011). Gleichheit und Vielfalt – normative Konzeptionen? Die philosophischen Implikationen zum Problem Anerkennung bei Simone de Beauvoir und Hannah Arendt. In Momentum Kongress Paper (S. 1-15)
	\item Hirte, K., \& Ötsch, W. (2011). Vorwort. In Ötsch, Walter O.; Hirte, Katrin; Nordmann, Jürgen (Hrsg.), Gesellschaft! Welche Gesellschaft? (S. 7-15). Metropolis. 
	\item Hirte, K., \& Ötsch, W. (2011). Ökonomische Ausrichtung und Netzwerke -- das Beispiel des Sachverständigenrates. Prokla, (3), 423-447.
	\item Kapeller, J. (2011). Modell-Platonismus in der Ökonomie. Zur Aktualität einer klassisch-epistemologischen Kritik. Universität Linz.
	\item Kapeller, J. (2011). Modell-Platonismus in der Ökonomie: Zur Aktualität einer klassischen epistemologischen Kritik. Peter Lang.
	\item Kapeller, J. (2011). Was sind ökonomische Modelle? In Volker Gadenne und Reinhard Neck (Hrsg.), Philosophie und Wirtschaftswissenschaft (S. 29-50). Mohr Siebeck. 
	\item Knierim, A., \& Hirte, K. (2011). Aktionsforschung -- ein Weg zum Design institutioneller Neuerungen zur regionalen Anpassung an den Klimawandel. In Anpassung an den Klimawandel -- regional umsetzen! (S. 156-174). oecom Verlag. 
	\item Nordmann, J. (2011). Braucht die aktuelle Gesellschaft einen Gesellschaftsvertrag? Der politische Neoliberalismus im Spiegel von John Locke und John Rawls. In Gesellschaft! Welche Gesellschaft? (S. 33-60). Metropolis Verlag. 
	\item Nordmann, J. (Hrsg.) (2011). Braucht die aktuelle Gesellschaft einen Gesellschaftsvertrag? Der politische Neoliberalismus im Spiegel von John Locke und John Rawls. In Gesellschaft! Welche Gesellschaft? Nachdenken über eine sich wandelnde Gesellschaft Metropolis Verlag. 
	\item Plaimer, W., \& Nordmann, J. (2011). Veränderung von Machtverhältnissen in politischen Entscheidungsprozessen. In Momentum-Kongress (Hrsg.), Momentum Kongress Paper (S. 1-21)
	\item Pühringer, S. (2011). Aktivierung und Mindestsicherung. Rezension zum gleichnamigen Buch von Christine Stelzer-Orthofer und Josef Weidenholzer. WISO -- Wirtschafts- und sozialpolitische Zeitschrift, 34(02), 169-171.
	\item Pühringer, S. (2011). Frei handeln? Liberales und neoliberales Freiheitskonzept und ihre Auswirkungen auf die Verteilung von Macht und Eigentum. Peter Lang.
	\item Pühringer, S. (2011). Gleichheit versus Vielfalt. Ein konstruierter Widerspruch? In Momentum Kongress Paper (S. 1-23)
	\item Ötsch, W. (2011). Markt. Sichtweisen auf die Wirtschaft. Praxis Politik, (2/2011), 4-8.
	\item Ötsch, W., Hirte, K., \& Nordmann, J. (Hrsg.) (2011). Gesellschaft! Welche Gesellschaft? Nachdenken über eine sich wandelnde Gesellschaft. Metropolis Verlag.
	\item Ötsch, W. (2011). Der Markt und die großen Ratingagenturen. In Momentum Kongress Paper (S. 1-11). (Momentum quarterly).
	\item Hirte, K. (2011). Crowdsourcing und Regelbezüge -- der Fall GuttenPlag. In Heiss, Hans-Ulrich (Hrsg.), INFORMATIK 2011 -- Informatik schafft Communities. 41. Jahrestagung der Gesellschaft für Informatik , 4.-7.10.2011, Berlin (Band P-192). (Lecture Notes in Informatics (LNI)). Springer. 
	\item Ötsch, W. (2011). Kurt Rothschild als politischer Ökonom. In Bartel, Rainer; Bürger, Hans; Klug, Friedrich (Hrsg.), In Memoriam Univ. Prof. Kurt W. Rothschild (S. 73-84). (Schriftenreihe des Instituts für Kommunalwissenschaft (IKW)). Institut für Kommunalwirtschaften. 
\end{enumerate}
\subsection*{2010}
\begin{enumerate}
	\item Dobusch, L., \& Kapeller, J. (2010). Institutionalisierung zivilgesellschaftlicher Partizipation -- Zwischen Ignoranz, Integration und Invasion. In Blaha, Barbara; Weidenholzer, Josef (Hrsg.), Freiheit -- Beiträge für eine demokratische Gesellschaft (S. 201-217). Wilhelm Braumüller Universitäts-Verlagsbuchhandlung. 
	\item Hirte, K. (2010). Das neoklassische Freihandelsmodell. In Momentum Kongress Paper (S. 1-16)
	\item Hirte, K. (2010). Die Rolle der Agrarpolitik und Agrarökonomie in agrarpolitischen Diskursverläufen. TRANS Internet-Zeitschrift für Kulturwissenschaften, 17(2).
	\item Hirte, K. (2010). Performativity of Economics -- ein tragfähiger Ansatz zur Analyse der Rolle der Ökonomen in der Ökonomie? In Walter Ötsch, Katrin Hirte, Jürgen Nordmann (Hrsg.), Krise. Welche Krise? (S. 49-75). Metropolis Verlag. 
	\item Hirte, K. (2010). Ökonomisierung an den Hochschulen. In Johanna Besier, Hannah Fritsch, Arne Rost, Sven Schmidt, Hannes Schulz, Maggie Selle, Katharina Wenzel (Hrsg.), Agrarpolitik in der Lehre? (S. 17-25). ABL Bauernblatt Verlags-GmbH. 
	\item Hubmann, G., \& Kapeller, J. (2010). Solidarisch Handeln: Konzeptionen, Ursachen und Implikationen. In Momentum Kongress Paper (S. 1-30)
	\item Kapeller, J. (2010). Citation Metrics: Serious drawbacks, perverse incentives and strategic options for heterodox economics. American Journal of Economics and Sociology, 69(5), 1376-1408. https://doi.org/10.1111/j.1536-7150.2010.00750.x
	\item Kapeller, J. (2010). Some critical notes on citation metrics and heterodox economics. Review of Radical Political Economics, 42(3), 330-337. https://doi.org/10.1177/0486613410377855
	\item Kapeller, J., \& Huber, J. (2010). Neoklassische Sozialdemokratie und Sozialdemokratie am Beispiel des Hamburger Programms der SPD. TRANS Internet-Zeitschrift für Kulturwissenschaften, 17(2).
	\item Kapeller, J., \& Pühringer, S. (2010). The internal consistency of perfect competition. Journal of Philosophical Economics, 3(2), 134-152. https://doi.org/10.46298/jpe.10598
	\item Nordmann, J. (2010). Protektionismus. Die Grenzen der Staatsintervention in den 1930er Jahren. European Journal of Economics and Economic Policies: Intervention, 7(1), 42-50. https://doi.org/10.4337/ejeep.2010.01.05
	\item Nordmann, J. (2010). Trash, Skandale und Ratschläge statt Aufklärung und politische Bildung. Über das Zusammenspiel von kommerzialisierten Medien und gemachter Meinung in der neoliberalen Gesellschaft. In Momentum Kongress Paper (S. 1-8)
	\item Nordmann, J. (Hrsg.) (2010). Was ist eine Krise? In Ötsch, Walter O.; Hirte, Katrin; Nordmann, Jürgen (Hrsg.), Krise! Welche Krise? Zur Problematik aktueller Krisendebatten (S. 7-20). Metropolis Verlag. 
	\item Pühringer, M., \& Pühringer, S. (2010). Solidarität im Kapitalismus. In Momentum Kongress Paper (S. 1-32)
	\item Ötsch, W. (2010). Das Bewusstsein des Homo Oeconomicus. In Bauer, Renate (Hrsg.), Bewusstsein und Ich (S. 105–117). Angelika Lenz Verlag. 
	\item Ötsch, W. (2010). Die Tiefenbedeutung von ‘Markt’. In Momentum Kongress Paper (S. 1-27)
	\item Ötsch, W., \& Kapeller, J. (2010). Perpetuing the failure: Economic Education and the Current Crisis. Journal of Social Science Education, 9(2), 16-25.
	\item Ötsch, W., Hirte, K., \& Nordmann, J. (2010). Die Evolution des ökonomischen Wissens und des Wissens über den Kapitalismus. Performativity als Analyseinstrument: das Beispiel der Fabian Society, der Mont Pèlerin Society und der Chicagoer Schule.
	\item Ötsch, W., Hirte, K., \& Nordmann, J. (Hrsg.) (2010). Krise! Welche Krise? Zur Problematik aktueller Krisendebatten. Metropolis.
\end{enumerate}
\subsection*{2009}
\begin{enumerate}
	\item Dobusch, L., \& Kapeller, J. (2009). Diskutieren und Zitieren: Zur paradigmatischen Konstellation aktueller ökonomischer Theorie. European Journal of Economics and Economic Policies: Intervention, 6(2), 145-152.
	\item Dobusch, L., \& Kapeller, J. (2009). Why is Economics not an Evolutionary Science? New Answers to Veblen's old Question. Journal of Economic Issues, 43(4), 867-898. https://doi.org/10.2753/jei0021-3624430403
	\item Hirte, K. (2009). Diskursverläufe in der universitären Agrarpolitik als neoliberales Hegemonialprojekt – Struktur, Ursache und Wirkungen. In Ötsch, Walter; Thomasberger, Claus (Hrsg.), Der neoliberale Marktdiskurs (S. 187-212). Metropolis. 
	\item Hirte, K. (2009). Markt als soziale Struktur -- Zum Diskursszenario zur \glqq Märkte-Störung\grqq{} durch den Milchstreik. arbeitsergebnisse, (62), 14-26.
	\item Kapeller, J., \& Huber, J. (2009). Politische Paradigmata und neoliberale Einflüsse am Beispiel von vier sozialdemokratischen Parteien in Europa. Österreichische Zeitschrift für Politikwissenschaft, (2), 163-192.
	\item Pühringer, S., \& Wolfmayer, G. (2009). Frei Handeln. Überlegungen zur Überwindung des neoliberalen Freiheitsbegriffs. In Momentum Kongress Paper (S. 1-21)
	\item Ötsch, W. (2009). Bilder der Wirtschaft. Metaphern, Diskurse und Hayeks neoliberales Hegemonialprojekt. In Hubert Hieke (Hrsg.), Kapitalismus. Kritische Betrachtungen und Reformansätze. Metropolis Verlag. 
	\item Ötsch, W. (2009). Computer-Welten und Markt-Diskurs. Die neoklassische Propaganda 'des Marktes'. In Ötsch, Walter O. Thomasberger, Claus (Hrsg.), Der neoliberale Markt-Diskurs. Ursprünge, Geschichte, Wirkungen. Metropolis Verlag. 
	\item Ötsch, W. (2009). Kognitive Grundlagen menschlichen Verhaltens. Kognitionswissenschaften und neoklassische Standardtheorie. In Goldschmidt, Niels; Nutzinger, Hans G. (Hrsg.), Vom homo oeconomicus zum homo culturalis. Handlung und Verhalten in der Ökonomie Lit-Verlag. 
	\item Ötsch, W. (2009). Mythos Markt. Marktradikale Propaganda und ökonomische Theorie. Metropolis Verlag.
	\item Ötsch, W., \& Kapeller, J. (2009). Neokonservativer Markt-Radikalismus. Das Fallbeispiel des Iraks. Internationale Politik und Gesellschaft, (2), 40-55.
	\item Ötsch, W., \& Thomasberger, C. (Hrsg.) (2009). Der neoliberale Markt–Diskurs. Ursprünge, Geschichte, Wirkungen. Metropolis Verlag.
	\item Ötsch, W., \& Thomasberger, C. (Hrsg.) (2009). Neoliberale Denkformen, neoliberale Diskurse, neoliberale Hegemonie. In Der neoliberale Markt-Diskurs. Ursprünge, Geschichte, Wirkungen. (S. 7-19). Metropolis Verlag. 
\end{enumerate}
