\subsection*{2025}
\begin{enumerate}
    	 \item Pühringer S., Aistleitner M., Cserjan L., Hieselmayr S.: Idiosyncrasies of the oligarchic elite: On the political economy of wealth concentration in Austria, Serie Working Paper Series, Nr. 157, 2025
	 \item Porak L., Reinke R.: The Charm of Emission Trading, in Journal of Economic Issues, 2025
\end{enumerate}
\subsection*{2024}
\begin{enumerate}
    	 \item Ötsch W., Hilt A.: Das Imaginative der Politischen Ökonomie, Serie Kritische Studien zu Markt und Gesellschaft, Vol. 15, Metropolis Verlag, Marburg, 2024
	 \item Ötsch W.: Theoriegeschichte der Ökonomie als Imaginationsgeschichte, in Ötsch Walter O., Priddat Birger P., Groß Steffen W.: Das Imaginative der Politischen Ökonomie., Metropolis Verlag, Marburg, 2024
	 \item Theine H., Sevignani S.: Introduction to the special issue: Media transformation and the challenge of property, in European Journal of Communication, Vol. 39, Nr. 5, Seite(n) 407–411, 2024
	 \item Theine H., Sevignani S.: Media property: Mapping the field and future trajectories in the digital age, in European Journal of Communication, Vol. 39, Nr. 5, Seite(n) 412–425, 2024
	 \item Solveign D., Müller H., Porak L., Winkin M.: Vergesellschaftung zukunftsweisend gestalten: Vergesellschaftung und die sozial- ökologische Frage Wie wir unsere Gesellschaft gerechter, zukunftsfähiger und resilienter machen können, 2024
	 \item Pühringer S., Wolfmayr G.: Competitive Performativity of Academic Social Networks. The Subjectivation of Competition on ResearchGate?, in Research Evaluation, Vol. online first, 2024
	 \item Pühringer S., Aistleitner M., Cserjan L., Hieselmayr S., Weber J.: Idiosyncrasies of the superrich: On the political economy of wealth concentration in Austria, in Materialien für Wirtschaft und Gesellschaft, Nr. 254, 2024
	 \item Porak L., Reinke R.: The contribution of qualitative methods to economic research in an era of polycrisis, in Review of Evolutionary Political Economy, Vol. 5, Nr. 1, Seite(n) 31, 2024
	 \item Porak L., Premrov T., Witzani-Haim D.: Nach der Wahl: Ist mehr Wohl­stand trotz dro­hen­der Aus­teri­tät möglich?, in Der Kurswechsel, 2024
	 \item Porak L., Gräbner-Radkowitsch C., Kapeller J., Rath J.: Pluralismus in der Volkswirtschaftslehre und seine Relevanz für die Debatte um Armut: Beiträge für eine zukunftsfähige Wirtschafts- und Finanzbildung, Seite(n) 75-86, 2024
	 \item Porak L.: Globale geo-ökonomische Unordnung: Europa braucht industriepolitische Antworten, 2024
	 \item Kapeller J., Pühringer S., Aistleitner M., Rath J.: Proxy failures in practice: Examples from the sociology of science (Commentary), in Behavioral and Brain Sciences, Vol. 47, 2024
	 \item Kapeller J., Hornykewycz A., Weber J., Cserjan L.: Dekarbonisierung des Gebäudesektors als Teil einer sozial-ökologischen Transformation. Ein Gestaltungsvorschlag, in ifso expertise, Vol. 25, Seite(n) 1-23, 2024
	 \item Kapeller J., Gräbner-Radkowitsch C., Hornykewycz A.: Corporate power and global value chains: current approaches for conceptualizing the power of multinationals, in Review of Evolutionary Political Economy, Vol. 5, Seite(n) 371-397, 2024
	 \item Kapeller J., Gräbner-Radkowitsch C.: Development models in the EU:  opportunities and challenges, in Ökonomie und Gesellschaft, Nr. 35, Metropolis, Seite(n) 49-80, 2024
	 \item Kapeller J.: Vermögensriesen und ein Heer von Zwergen, in Kammer für Arbeiter und Angestellte Oberösterreich, 2024
	 \item Hornykewycz A., Burnazoglu M.: Power in Economics without Power in Economics?, in Review of Evolutionary Political Economy, Vol. 5, Seite(n) 301-328, 2024
	 \item Hirte K.: Performativität: Geplante Landwirtschaftsstrukturen – das Beispiel Böckenhoff-Plan, in Gruber, Holle; Henkel, Anna; Scheel, Laura: Land: Digitalisierung, Agrarwandel, Energiewende – soziologische Perspektiven zu ländlichen Räumen im Umbruch, Serie 10 Minuten Soziologie, trancript, Bielefeld, Seite(n) 87-100, 2024
	 \item Heck I., Hornykewycz A., Kapeller J., Wildauer R.: Vermögensverteilung  in Österreich: Eine Analyse auf Basis des HFCS 2021/22, in Materialien zu Wirtschaft und Gesellschaft, Nr. 255, Seite(n) 1-34, 2024
	 \item Hager T.: Lobbying and Macroeconomic Development, in Mause, Karsten; Polk, Andreas: The Political Economy of Lobbying, Serie Studies in Public Choice, Vol. 43, Springer, Cham, Seite(n) 77-99, 2024
	 \item Gräbner-Radkowitsch C., Kapeller J.: The micro-macro link in heterodox economics, Serie Working Paper Series, Nr. 154, 2024
	 \item Gräbner-Radkowitsch C., Kapeller J.: Systemism, Serie Working Paper Series, Nr. 155, 2024
	 \item Gräbner-Radkowitsch C., Kapeller J.: Path Dependence, Serie Working Paper Series, Nr. 154, 2024
	 \item Griesser M., Beyer K., Pühringer S.: Für die "Leistungsträger“ und „uns Österreicher“: Eine Mediendiskursanalyse zu Sozialpolitikreformen der ÖVP/FPÖ-Regierung 2017–2019 in Österreich, in Momentum Quarterly, Vol. 13, Nr. 1, Seite(n) 33-55, 2024
	 \item Gampe A., Hubmann G., Kapeller J.: Sozialer Fortschritt in offenen Gesellschaften des 21. Jahrhunderts: Unrealistische Utopie oder notwendige Möglichkeit?, in Gampe, Anja; Söylemez, Seçkin: Weltoffenheit, Toleranz und Gemeinsinn. Chancen und Herausforderungen in der Gegenwartsgesellschaft, transcript, Bielefeld, Seite(n) 13-38, 2024
	 \item Disslbacher F., Hornykewycz A., Kapeller J.: Vermögen in Österreich: stark konzen­triert, gering be­steuert, 2024
	 \item Deindl R., Hager T., Schäfer K.: Zerrissen, überlastet, prekär – Mittelbau zwischen Doktoratsstudium und universitärer Lehre, 2024
	 \item Bäuerle L., Graupe S.: Enacting Economic Transformation. The Transformative Economic Capabilities (TEC) Approach, in World Futures, Vol. online first, Seite(n) 1-30, 2024
	 \item Buyinza F., Kapeller J., Senono V., Anber M.: Differential impacts of electricity access on educational outcomes: Evidence from Uganda, in Electricity Journal, Vol. 37, Nr. 1, Seite(n) 1-10, 2024
	 \item Azevedo S., Hager T., Porak L.: Kritische Utopien als Methode und Praxis: Der Forschungsstandort Österreich: Momentum Kongress Paper, 2024
	 \item Altreiter C., Pühringer S., Völkl Y.: Politiken der Verwettbewerblichung: Einblicke ins im österreichischen Wissenschaftssystem und darüber hinaus., in BdWi Studienheft, Vol. 14, Seite(n) 29-32, 2024
\end{enumerate}
\subsection*{2023}
\begin{enumerate}
    	 \item Ötsch W.: Wert und Werte in der Ökonomik, in Agora42 (Philosophisches Wirtschaftsmagazin), Nr. 02/2023, Seite(n) 9-13, 2023
	 \item Steffestun T., Ötsch W.: Economization: The (re-)organization of knowledge and ignorance according to ‘the market’, in ephemera, Vol. 23, Nr. 1, Seite(n) 133-159, 2023
	 \item Pühringer S., Altreiter C.: Woran scheitert transformative Wissensproduktion?, 2023
	 \item Pühringer S.: Man ist akademische Einzelunternehmer*in, 2023
	 \item Pühringer S.: Soziale und ökologische Probleme müssen zusammen betrachtet werden, in Renner Institut: Ein aktiver Staat, der die Menschen stärkt und schützt, Seite(n) 24-27, 2023
	 \item Porak L., Schamberger K.: Islands in the Privately Dominated Sea of Capitalist Media, in Güney, Selma; Hille, Lina; Pfeiffer, Juliane; Porak, Laura; Theine, Hendrik: Eigentum, Medien, Öffentlichkeit, Westend Verlag, Frankfurt am Main, Seite(n) 443-449, 2023
	 \item Porak L.: Wettbewerbsfähige Nachhaltigkeit: eine Historisch-Materialistische Analyse der Ideen, Institutionen und Machtverhältnisse in der europäischen grünen Transformation, in Momentum Quarterly, Vol. 12, Nr. 1, Seite(n) 65-83, 2023
	 \item Porak L.: Political sovereignty in tension with global capitalist accumulation: the case of the European socio-economic strategy, in Critical Policy Studies, 2023
	 \item Kapeller J., Wildauer R., Leitch S.: Can a European wealth tax close the green investment gap?, in Ecological Economics, Nr. 209, 2023
	 \item Kapeller J., Hubmann G.: Dilemmata marktliberaler Globalisierung, in Sturn, Richard; Klüh, Ulrich: Wachstums- und Globalisierungsgrenzen, Serie Jahrbuch Normative und institutionelle Grundfragen der Ökonomik, Vol. 20, Metropolis Verlag, Marburg, 2023
	 \item Hubmann G., Kapeller J.: Rezension zu Markus Marterbauer/Martin Schürz: Angst und Angstmacherei, in WISO - Wirtschafts- und sozialpolitische Zeitschrift, Vol. 46, Nr. 1, Seite(n) 102-109, 2023
	 \item Hager T., Pühringer S.: Im Netz der Einfluss-Reichen, 2023
	 \item Hager T., Mellacher P., Rath M.: Endogenous Heterogeneous Gender Norms and the Distribution of Paid and Unpaid Work in an Intra-Household Bargaining Model, 2023
	 \item Hager T.: Lobbyismus und gesamtwirtschaftliche Entwicklung, in Andreas Polk, Karsten Mause: Handbuch Lobbyismus, Springer VS, Wiesbaden, Seite(n) 817–842, 2023
	 \item Gräbner-Radkowitsch C., Strunk B.: Degrowth and the Global South? How Institutionalism can Complement a Timely Discourse on Ecologically Sustainable Development in an Unequal World, in Journal of Economic Issues, Vol. 57, Nr. 2, Seite(n) 476-483, 2023
	 \item Gräbner-Radkowitsch C., Strunk B.: Degrowth and the Global South: The Twin Problem of Global Dependencies, in Ecological Economics, Vol. 213, 2023
	 \item Gräbner-Radkowitsch C., Hager T., Hornykewycz A.: Competing for Sustainability? An Institutionalist Analysis of the New Development Model of the European Union, in Journal of Economic Issues, Vol. 57, Nr. 2, Seite(n) 676-683, 2023
	 \item Gräbner-Radkowitsch C.: Elements of an evolutionary approach to comparative economic studies, in Casagrande, Sara; Dallago, Bruno: The Routledge Handbook of Comparative Economic Systems, Routledge, London, forthcoming, Seite(n) 81-102, 2023
	 \item Benz P., Maesse J., Pühringer S., Rossier T.: Il potere e l'economics, in Nicoletta, Gerardo C.; di Carlo, Michele S.; Ventrone, Oreste: Economisti e Società. Nuove sociologie dell'expertise economica, Liguori Editore, Napoli, Seite(n) 17-24, 2023
	 \item Benz P., Maesse J., Pühringer S., Rossier T.: Economics and Power, in Macknight, Viski; Medvecky, Fabien: Making Economics Public, Routledge, London, Seite(n) 18-25, 2023
	 \item Altreiter C., Gräbner-Radkowitsch C., Pühringer S., Rogojanu A., Wolfmayr G.: The three faces of competitization: From marketization to a multiplicity of competition, 2023
	 \item Altreiter C., Azevedo S., Porak L., Pühringer S., Wolfmayr G.: Winning city competition with a social agenda. The competition imaginary in Viennese urban development plans, in Urban Research \& Practice, Seite(n) 10.1080/17535069.2022.2161834, 2023
	 \item Aistleitner M., Pühringer S.: L´expertise parziale dell´economics: il caso della ricerca (delle politiche) sul commercio, in Nicoletta, Gerardo C.; di Carlo, Michele S.; Ventrone, Oreste: Economisti e Società. Nuove sociologie dell'expertise economica, Liguori Editore, Napoli, Seite(n) 25-40, 2023
	 \item Aistleitner M., Pühringer S.: The Social Field of Elite Trade Economists: A Quantitative Social Studies of Economics Perspective, in Oeconomia, Vol. 13, Nr. 2, Seite(n) 475-515, 2023
	 \item Aistleitner M., Pühringer S.: Biased trade narratives and its impact on development studies: a multi-level mixed-method approach, in European Journal of Development Research, Vol. 35, Seite(n) 1322-1346, 2023
	 \item Aistleitner M., Kapeller J., Kronberger D.: The authors of economics journals revisited: evidence from a large-scale replication of Hodgson and Rothman (1999), in Journal of Institutional Economics, Vol. 19, Nr. 1, Seite(n) 86–101, 2023
\end{enumerate}
\subsection*{2022}
\begin{enumerate}
    	 \item Ötsch W.: Klima, Markt und Zukunftsbilder, 2022
	 \item Theine H., Porak L.: Pluralism and Economics Education, Serie International Journal of Pluralism and Economics Education, Vol. 13 (1), 2022
	 \item Reiner C., Bellak C.: Hat die ökonomische Macht von Unternehmen in Österreich zugenommen?, 2022
	 \item Reiner C.: It’s the End of Globalization as We Know It! Zeitgemäße Betrachtungen zur politischen Ökonomik der Globalisierungskrise, 2022
	 \item Pühringer S., Beyer K.: Who are the economists Germany listens to? What it needs to have academic, public or political impact, in Maesse, Jens; Pühringer, Stephan; Rossier, Thierry; Benz, Pierre: Power and Influence of Economists: Contributions to the Social Studies of Economics., Routledge, London, Seite(n) 147-169, 2022
	 \item Pühringer S., Beyer K.: Divided We Stand? Professional Consensus and Political Conflict in Academic Economics, in Journal of Economic Issues, forthcoming, 2022
	 \item Pühringer S., Aistleitner M., Griesebner T.: Networks of the super-rich in Austria. Evidence from an explorative case study, in Materialien zu Wirtschaft und Gesellschaft, Nr. 238, 2022
	 \item Porak L.: Wettbewerbsfähige Nachhaltigkeit – Die Lösung unserer Probleme?: Momentum Kongress Paper, Seite(n) 1-14, 2022
	 \item Maesse J., Pühringer S., Rossier T., Benz P.: Power and Influence of Economists: Contributions to the Social Studies of Economic, Routledge, London, 2022
	 \item Maesse J., Pühringer S., Rossier T., Benz P.: The role of power in the Social Studies of Economics: an introduction, in Jens Maesse, Stephan Pühringer, Thierry Rossier,  Pierre Benz: Power and Influence of Economists: Contributions to the Social Studies of Economics., Routledge, London, 2022
	 \item Koch L., Ötsch W., Graupe S.: Wissenschaftstheoretische Grundlagen, in Lehmann-Waffenschmidt, Marco; Peneder, Michael: Evolutorische Ökonomik. Konzepte, Wegbereiter und Anwendungsfelder, Metropolis Verlag, Marburg, Seite(n) 349-359, 2022
	 \item Kapeller J., Wildauer R.: Tracing the invisible rich: a new approach to modelling Pareto tails in survey data, in Labour Economics, Vol. 75, Nr. 102145, 2022
	 \item Kapeller J., Pühringer S., Grimm C.: Paradigms and Policies: The state of economics in the German-speaking countries, in Review of International Political Economy, Vol. 29, Nr. 4, Seite(n) 1183-1210, 2022
	 \item Kapeller J., Huwe V.: Critical junctures of hope: how to bridge the gap between the necessary and the feasible?, in GAIA - Ecological Perspectives for Science and Society, Vol. 31, Nr. 1, Seite(n) 10-13, 2022
	 \item Kapeller J., Hubmann G.: Notizen zum ökonomischen Element in der politischen Doktrinbildung, in Zeitschrift für Wirtschafts- und Unternehmensethik, Vol. 23, Seite(n) 34-37, 2022
	 \item Hirte K., Ötsch W., Pühringer S.: Die Netzwerkanalyse und der Umgang mit ihren Forschungsergebnissen, in Berliner Journal für Soziologie, Nr. 32, Seite(n) 153-163, 2022
	 \item Heimberger P., Schütz B.: Evaluierung des Zusammenhangs von Produktionspotenzial und Budgetsemielastizität im Rahmen der deutschen Schuldenbremse, 2022
	 \item Heimberger P., Schütz B.: Die Budgetsemielastizität und ihre Auswirkungen auf Verschuldungsspielräume im Rahmen der Schuldenbremse, in Wirtschaftsdienst, Vol. 102, Nr. 11, Seite(n) 834-837, 2022
	 \item Hager T., Heck I., Rath J.: Polanyi and Schumpeter: Transitional processes via societal spheres, in The European Journal for the History of Economic Thought, Vol. 29, Nr. 6, Seite(n) 1089–1110, 2022
	 \item Gräbner-Radkowitsch C., Tamesberger D., Heimberger P., Kapelari T., Kapeller J.: Trade Models in the European Union, in Economic Annals, Vol. 67, Nr. 235, Seite(n) 7-36, 2022
	 \item Gräbner-Radkowitsch C., Hornykewycz A., Schütz B.: The emergence of debt and secular stagnation in an unequal society: a stockflow consistent agent-based approach, 2022
	 \item Gräbner-Radkowitsch C., Hornykewycz A., Schütz B.: The emergence of debt and secular stagnation in an unequal society: a stock- flow consistent agent-based approach, 2022
	 \item Gräbner-Radkowitsch C., Heimberger P., Kapeller J., Landesmann M., Schütz B.: The evolution of debtor-creditor relationships within a monetary union: Trade imbalances, excess reserves and economic policy, in Structural Change and Economic Dynamics, Vol. 62, Seite(n) 262-289, 2022
	 \item Gräbner-Radkowitsch C., Hager T., Hornykewycz A.: Competing for sustainability? An institutionalist analysis of the new development model of the European Union, 2022
	 \item Gräbner-Radkowitsch C., Hafele J.: Why Fostering Socio-economic Convergence in the EU Is Necessary for Successful Climate Change Mitigation, in Heinrich Böll Foundation, ZOE-Institute for Future-Fit Economies and Finanzwende Recherche: Making the great turnaround work: Economic policy for a green and just transition, Seite(n) 104-114, 2022
	 \item Gräbner-Radkowitsch C.: Elements of an evolutionary approach to comparative economic studies: complexity, systemism, and path dependent development, 2022
	 \item Gräbner C., Hornykewycz A.: Capability accumulation and product innovation: an agent-based perspective, in Journal of Evolutionary Economics, Vol. 32, Seite(n) 87-121, 2022
	 \item Altreiter C., Azevedo S., Porak L., Pühringer S., Wolfmayr G.: Winning urban competition with a social agenda. The competition imaginary in Viennese urban development plans, 2022
	 \item Aistleitner M., Kapeller J., Kronberger D.: The Authors of Economics Journals Revisited: Evidence from a Large-Scale Replication of Hodgson \& Rothman (1999), 2022
	 \item Aistleitner M.: Development and Interdisciplinarity: re-examining the “economics silo”, 2022
\end{enumerate}
\subsection*{2021}
\begin{enumerate}
    	 \item Ötsch W., Wodak R.: Populismus, in Ferstl, Michael G.: Handbuch Liberalismus, J.B. Metzler, Stuttgart, Seite(n) 535-541, 2021
	 \item Ötsch W., Steffestun T.: Wissen und Nichtwissen der ökonomisierten Gesellschaft. Aufgaben einer neuen Politischen Ökonomie, Metropolis Verlag, Marburg, 2021
	 \item Ötsch W., Pühringer S.: Ordoliberalismus, in Michael G. Festl: Handbuch Liberalismus, J.B. Metzler, Stuttgart, Seite(n) 372-378, 2021
	 \item Ötsch W., Graupe S.: Walter Lippmann: Die Illusion von Wahrheit oder die Erfindung der Fake News, Fivty-fivty Verlag, Edition Buchkomplizen, Frankfurt am Main, 2021
	 \item Ötsch W., Graupe S.: Vorwort, in Ötsch, Walter O.; Graupe, Silja, Fifty-fifty Verlag, Frankfurt am Main, 2021
	 \item Ötsch W.: Narration und Imagination. Die Rolle von imaginierten Bildern in der Geschichte der Wirtschaftstheorie, in Künzel, Christine; Priddat, Birger: Fiktion und Narration in der Ökonomie. Interdisziplinäre Perspektiven auf den Umgang mit ungewisser Zukunft, Metropolis, Marburg, Seite(n) 241-267, 2021
	 \item Tamesberger D., Theurl S.: Design and Take Up of Austria’s Coronavirus Short Time Work Model, 2021
	 \item Strohmaier R.: Agent of Sustainable Change - Der unternehmerische Staat und sozial-ökologische Transformation, in Klüh, Ulrich; Sturn, Richard: Der Staat in der großen Transformation. Jahrbuch Normative und institutionelle Grundfragen der Ökonomik, Serie Jahrbuch Normative und institutionelle Grundfragen der Ökonomik, Metropolis, Weimar, Seite(n) 169-192, 2021
	 \item Schütz B.: \glqq Koste es, was es wolle\grqq{}. Eine neue Ära der Ökonomie?, in Economy, 2021
	 \item Schütz B.: Creating a pluralist paradigm: An application to the minimum wage debate, in Journal of Economic Issues, Vol. 55, Nr. 1, Seite(n) 103-124, 2021
	 \item Pühringer S., Rath J., Griesebner T.: The political economy of academic publishing: On the commodification of a public good, in PLOS One, Vol. 16, Nr. 6, 2021
	 \item Pühringer S., Rath J.: Monopolies in Science Publishing. A Black Hole for Public Spending?, in Journal of Management Information and Decision Sciences, Vol. 24, Nr. 6, Seite(n) 1-5, 2021
	 \item Pühringer S., Porak L., Rath J.: Talking about competition? Discursive shifts in the economic imaginary of competition in public debates, 2021
	 \item Pühringer S., Beyer K., Kronberger D.: Soziale Rhetorik, neoliberale Praxis: Eine Analyse der Wirtschafts- und Sozialpolitik der AfD, Serie OBS Arbeitspapier, Nr. 52, Otto Brenner Stiftung, 2021
	 \item Pühringer S., Beyer K., Kronberger D.: Soziale Rhetorik, neoliberale Praxis, in Beuteler Extradienst, 2021
	 \item Pühringer S.: Zur Pluralität der ökonomischen Politikberatung in Deutschland, 2021
	 \item Pühringer S.: Zur Pluralität in der ökonomischen Politikberatung in Deutschland. Eine empirische Untersuchung, in Leviathan - Berliner Zeitschrift für Sozialwissenschaft, Vol. 49, Seite(n) 243-265, 2021
	 \item Porak L., Schröter G.: Strategien für einen Wandel der ökonomischen Lehre, in Forschungsjournal Soziale Bewegungen, Vol. 34, Nr. 4, Seite(n) 718-729, 2021
	 \item Porak L., Pühringer S., Rath J.: So denken Ökonom*innen über Wettbewerb – eine Kritische Analyse des österreichischen Expert*innendiskurses, 2021
	 \item Porak L.: Governing the Ungovernable - Recontextualizations of ‘Competition’ in European Policy Discourse, 2021
	 \item Porak L.: Warum müssen wir (noch immer) arbeiten? Eine hegemonietheoretische Analyse der Bedeutung und des Wertes von Lohnarbeit für den modernen Staat: Momentum Kongress Paper, 2021
	 \item Kapeller J., Wildauer R.: Eine europäische Vermögenssteuer für das Klima, 2021
	 \item Kapeller J., Wildauer R.: A Fitting Pareto tails to wealth survey data: A practitioners’ guide, in Journal of Income Distribution, 2021
	 \item Kapeller J., Rehm M.: Vom empiristischen Humanismus zum partizipativen Sozialismus – Review von Thomas Piketty ‚Kapital und Ideologie‘, in Soziologische Revue, Vol. 44, Nr. 1, Seite(n) 25-33, 2021
	 \item Kapeller J., Leitch S., Wildauer R.: A European wealth tax, 2021
	 \item Kapeller J., Leitch S., Wildauer R.: Is a 10 trillion euro European climate investment initiative fiscally sustainable?, 2021
	 \item Kapeller J., Leitch S., Wildauer R.: A European Wealth Tax for a Fair and Green Recovery, 2021
	 \item Kapeller J., Leitch S., Wildauer R.: Is a € 10 Trillion European climate investment initiative fiscally sustainable?, in Renner Institut \& Foundation for European Progressive Studies (FEPS), in Policy Study, 2021
	 \item Kapeller J., Leitch S., Wildauer R.: A European wealth tax for a fair and green recovery, in Renner Institut \& Foundation for European Progressive Studies (FEPS), in Policy Study, 2021
	 \item Kapeller J., Gräbner-Radkowitsch C.: Standortwettbewerb und Deindustrialisierung: Das Beispiel MAN als Lehrbuchfall, 2021
	 \item Kapeller J., Gräbner-Radkowitsch C.: Konzernmacht in globalen Güterketten: Globale Warenketten und ungleiche Entwicklung. Arbeit, Kapital, Konsum, Natur., Mandelbaum Verlag, Wien, 2021
	 \item Kapeller J., Gräbner C.: Standortwettbewerb und Deindustrialisierung: Das Beispiel MAN als Lehrbuchfall, in WISO - Wirtschafts- und sozialpolitische Zeitschrift, Vol. 44, Nr. 4, Seite(n) 34-52, 2021
	 \item Kapeller J.: Ökonomische Polarisierung in Europa, in Zeitschrift Bürger und Staat, Vol. 71, Nr. 4, Seite(n) 246-251, 2021
	 \item Kapeller J.: Polarisierung oder Konvergenz? Zur ökonomischen Zukunft des vereinten Europas, in WISO direkt - Analysen und Konzepte zur Wirtschafts- und Sozialpolitik, 2021
	 \item Kapeller J.: Intangible Flow Theory: A New Way for Conceptualizing Embeddedness?, in Accounting, Economics and Law, Vol. 14, Nr. 1, Seite(n) 159-164, 2021
	 \item Hornykewycz A., Rath J.: Shaping sustainable employment relationships in the age of Digitalisation: analysing policy measures in an agent-based framework: Momentum Kongress Paper, 2021
	 \item Hirte K.: Unternehmenskonzentrationen in der Fleischbranche und die performative Rolle der Agrarökonomik, in ÖZS - Österreichische Zeitschrift für Soziologie, Vol. 46, Nr. 2, Seite(n) 187-206, 2021
	 \item Heimberger P., Kowall N.: Il governo Draghi: sette fatti sorprendenti sull’Italia, 2021
	 \item Heimberger P., Gechert S.: Corporate tax cuts do not boost growth, 2021
	 \item Heimberger P., Gechert S.: Do Corporate Tax Cuts Boost Economic Growth?, Serie wiiw Working Papers, Nr. 201, forthcoming, doi/full/10.1080/09692290.2021.1904269, 2021
	 \item Heimberger P.: Verschwenderisches, reformfaules Italien? Warum gängige Mythen falsch und gefährlich sind, in Marie Jahoda – Otto Bauer Institut, 2021
	 \item Heimberger P.: The push for a global minimum corporate tax rate, in Vienna Institute for International Economic (wiiw), 2021
	 \item Heimberger P.: Keynes, the output gap and the EU’s fiscal rules, in Vienna Institute for International Economic (wiiw), 2021
	 \item Heimberger P.: Keynes, output gap nonsense and the EU’s fiscal rules, 2021
	 \item Heimberger P.: Fiscal austerity and the rise of the Nazis, 2021
	 \item Heimberger P.: Financial globalisation has increased income inequality, 2021
	 \item Heimberger P.: European fiscal rules: reform urgently needed, 2021
	 \item Heimberger P.: EU bonds are a model for the future of Europe, in Vienna Institute for International Economic (wiiw), 2021
	 \item Heimberger P.: Draghi government: Seven ‘surprising’ facts about Italy, in Vienna Institute for International Economic (wiiw), 2021
	 \item Heimberger P.: Budgetkürzungen durch „Outputlücken-Nonsens“, 2021
	 \item Heimberger P.: Beeld over Italiaanse economie klopt niet, 2021
	 \item Heimberger P.: Do Higher Public Debt Levels Reduce Economic Growth?, Serie wiiw Working Papers, Nr. 211, 2021
	 \item Heimberger P.: Keynes, die Outputlücke und Probleme mit den Fiskalregeln, in Blog der Keynes-Gesellschaft, 2021
	 \item Heimberger P.: What is structural about unemployment in OECD countries?, in Review of Social Economy, Vol. 79, Nr. 2, Seite(n) 380-412, 2021
	 \item Heimberger P.: Does employment protection affect unemployment? A meta-analysis, in Oxford Economic Papers, Vol. 73, Nr. 3, Seite(n) 982-1007, 2021
	 \item Heimberger P.: Does economic globalization affect government spending? A meta-analysis, in Public Choice, Vol. 187, Seite(n) 349-374, 2021
	 \item Heimberger P.: Does economic globalisation promote economic growth?, in The World Economy, forthcoming, doi.org/10.1111/twec.13235, 2021
	 \item Heimberger P.: Corporate tax competition: A meta-analysis, in European Journal of Political Economy, Vol. 69, Nr. 1020002, 2021
	 \item Hager T., Hornykewycz A., Jonjic M., Porak L., Rath J.: „Hinter jeder erfolgreichen Frau steht ein Mann, der ihr den Rücken stärkt.“: Momentum Kongress Paper, 2021
	 \item Gräbner-Radkowitsch C., Pühringer S.: Competition universalism: Its historical origins and timely alternatives, 2021
	 \item Gräbner-Radkowitsch C., Heimberger P., Kapeller J., Landesmann M., Schütz B.: The evolution of debtor-creditor relationships within a monetary union: Trade imbalances, excess reserves and economic policy, in Universität Duisburg, Serie Ifso working paper, Nr. 10, 2021
	 \item Gräbner-Radkowitsch C., Heimberger P., Kapeller J., Landesmann M., Schütz B.: The evolution of debtor-creditor relationships within a monetary union: Trade imbalances, excess reserves and economic policy, 2021
	 \item Gräbner C., Strunk B.: Pluralism in economics – its critiques and their lessons, in Developing Economics, 2021
	 \item Gräbner C., Pühringer S.: Competition Universalism: Its Historical Origins and Timely Alternatives, 2021
	 \item Gräbner C., Kapeller J.: Konzernmacht in globalen Güterketten, in Karin Fischer, Christian Reiner und Cornelia Staritz: Globale Güterketten und ungleiche Entwicklung. Arbeit, Kapital, Natur und Konsum, Mandelbaum, Seite(n) 195-217, 2021
	 \item Gräbner C., Hornykewycz A., Schütz B.: The emergence of debt and secular stagnation in an unequal society: A stock-flow consistent agent-based approach: Momentum Kongress Paper, 2021
	 \item Gräbner C., Heinrich T.: Introduction to the symposium: The Complexity of Institutions: Theory and Computational Models, in Forum for Social Economics, Vol. 50, Nr. 2, Seite(n) 153-156, 2021
	 \item Gräbner C., Heimberger P., Kapeller J., Springholz F.: Understanding economic openness: A review of existing measures, in Review of World Economics, Vol. 157, Nr. 1, Seite(n) 87-120, 2021
	 \item Gräbner C., Hager T.: (Mis)Measuring Competitiveness:  The Quantification of a Malleable Concept in the European Semester, 2021
	 \item Gräbner C., Elsner W., Lascaux A.: Trust and Social Control. The Sources of Stability in Informal Value Transfer Systems, in Computational Economics, Nr. 58, Seite(n) 1077-1102, 2021
	 \item Griesser M., Beyer K., Pühringer S.: Für die „Leistungsträger“ und „uns Österreicher“: Eine Mediendiskursanalyse zu Sozialreformen der ÖVP/FPÖ-Regierung 2017-2019 in Österreich: Momentum Kongress Paper, 2021
	 \item Eder J.: Decreasing Dependency through Self-Reliance: Strengthening Local Economies through Community Wealth Building, 2021
	 \item Cordes C., Elsner W., Gräbner C., Heinrich T., Henkel J., Schwardt H., Schwesinger G., Su T.: The collapse of cooperation: The endogeneity of institutional break-up and its asymmetry with emergence, in Journal of Evolutionary Economics, Vol. 31, Nr. 4, Seite(n) 1291-1315, 2021
	 \item Aistleitner M., Pühringer S.: The Trade (Policy) Discourse in Top Economic Journals, in New Political Economy, Vol. 26, Nr. 5, Seite(n) 748-764, 2021
	 \item Aistleitner M., Gräbner C., Hornykewycz A.: Theory and Empirics of Capability Accumulation: Implications for Macroeconomic Modelling, in Research Policy, Vol. 50, Nr. 6, e-no. 104258, 2021
\end{enumerate}
\subsection*{2020}
\begin{enumerate}
    	 \item Ötsch W., Steffestun T.: Zur Einführung: Wissen und Nichtwissen in einer ökonomisierten Gesellschaft. Konturen einer neuen Politischen Ökonomie: Wissen und Nichtwissen der ökonomisierten Gesellschaft., Metropolis Verlag, Marburg, Seite(n) 7–35, 2020
	 \item Ötsch W., Graupe S.: Imagination und Bildlichkeit in der Ökonomie – eine Einführung: Imagination und Bildlichkeit der Wirtschaft. Zur Geschichte und Aktualität imaginativer Fähigkeiten in der Ökonomie, Springer, Wiesbaden, Seite(n) 1-33, 2020
	 \item Ötsch W.: Wissen, Selbstwissen und Nichtwissen der marktfundamentalen Ökonomie, in Ötsch, Walter O.; Steffestun, Theresa: Wissen und Nichtwissen der ökonomisierten Gesellschaft, Metropolis Verlag, Marburg, Seite(n) 85-131, 2020
	 \item Ötsch W.: Ist der Neoliberalismus am Ende?, in Schmidinger, Thomas; Weidenholzer, Josef: Virenregime. Wie die Coronakrise unsere Welt verändert. Befunde, Analyse, Anregungen., Bahoe Books, Wien, 2020
	 \item Wildauer R., Leitch S., Kapeller J.: How to boost the European Green Deal’s scale and ambition, 2020
	 \item Vogel L., Jühlke R., Porak L., Quinz H.: Allbetroffenheit in der Pandemie? Ein soziologischer Blick auf das Erleben der Auswirkungen der Corona-Krise: Momentum Kongress Paper, Seite(n) 1-19, 2020
	 \item Steffestun T.: The Constitution of Ignorance - zur Bedeutung von Nichtwissen in der Verhaltensökonomie, in Ötsch, Walter O.; Steffestun, Theresa: Wissen und Nichtwissen der ökonomisierten Gesellschaft., Metropolis Verlag, Marburg, Seite(n) 85-132, 2020
	 \item Schütz B.: Die Auswirkung von Mindestlöhnen auf die Arbeitslosigkeit: Ein Paradigmenvergleich, in Stephan Pühringer, Silja Graupe, Katrin Hirte, Jakob Kapeller, Stephan Panther: Jenseits der Konventionen: Alternatives Denken zu Wirtschaft, Gesellschaft und Politik, Metropolis, Marburg, Seite(n) 157-173, 2020
	 \item Schulmeister S.: Fixing long-term price paths for fossil energy – the optimal incentive for limiting global warming, 2020
	 \item Pühringer S., Rath J., Griesebner T.: The political economy of academic publishing: On the commodification of a public good, 2020
	 \item Pühringer S., Griesser M.: From the ʻplanning euphoriaʼ to the ʻbitter economic truthʼ: The Transmission of economic ideas into German Labour Market Policies in the 1960s and 2000s, in Critical Discourse Studies, Vol. 17, Nr. 5, Seite(n) 476-493, 2020
	 \item Pühringer S., Graupe S., Hirte K., Kapeller J., Panther S.: Jenseits der Konventionen. Alternatives Denken zu Wirtschaft, Gesellschaft und Politik. Festschrift für Walter Ötsch, Metropolis, Marburg, 2020
	 \item Pühringer S., Graupe S., Hirte K., Kapeller J., Panther S.: Vorwort. Jenseits der Konventionen, in Pühringer, Stephan; Graupe, Silja; Hirte, Katrin; Kapeller, Jakob; Panther, Stephan: Jenseits der Konventionen. Eine Festschrift für Walter Ötsch, Metropolis, Marburg, Seite(n) 9-16, 2020
	 \item Pühringer S.: Think Tank Networks of German Neoliberalism. Power Structures in Economics and Economic Policies in Post-War Germany, in Mirowski, Philip; Plehwe, Dieter; Slobodian, Quinn: Nines Lives of Neoliberalism, Verso Books, New York, Seite(n) 283-306, 2020
	 \item Porak L., Neuffer S.: Die moderne Lehrbuchwissenschaft als Zombiewissenschaft, in Agora42 (Philosophisches Wirtschaftsmagazin), 2020
	 \item Porak L.: Der größte ‘Trumpf’ Europas: Eine Analyse des ‘economic imaginary’ der Europäischen Kommission, 2020
	 \item Porak L.: Miteinander und voneinander lernen. Vielfalt in der ökonomischen Lehre, in Hochmann, Lars: Economists4future, Murmann Verlag, Hamburg, Seite(n) 127-142, 2020
	 \item Porak L.: Wohin steuert die Europäische Union? Ein Klärungsversuch der strategischen Ausrichtung der EU seit Lissabon: Momentum Kongress Paper, Seite(n) 1-22, 2020
	 \item Piétron D., Porak L., Thieme S.: Plurale Ökonomik - Eine kurze Einführung, in Thielscher, Christian: Wirtschaftswissenschaften verstehen, Springer Gabler, Wiesbaden, Seite(n) 189-205, 2020
	 \item Kapeller J., Wildauer R., Leitch S.: How to boost the European Green Deal’s scale and ambition, Serie FEPS Policy Paper, 2020
	 \item Kapeller J., Wildauer R., Heck I.: Vermögenskonzentration in Österreich: Ein Update auf Basis des HFCS 2017, in Wirtschaft und Gesellschaft, Nr. 206, Seite(n) 1-38, 2020
	 \item Kapeller J., Pühringer S.: Paradigmen und Politik. Der derzeitige Stand der Ökonomie, in Jakob Kapeller, Stephan Pühringer, Silja Graupe, Kathrin Hirte, Stephan Panther: Jenseits der Konventionen: Alternatives Denken zu Wirtschaft, Gesellschaft und Politik, Metropolis, Marburg, Seite(n) 221-252, 2020
	 \item Kapeller J., Gräbner-Radkowitsch C.: Konzernmacht in globalen Güterketten, 2020
	 \item Kapeller J.: Polarisierung oder Konvergenz? Zur ökonomischen Zukunft des vereinten Europa, 2020
	 \item Hornykewycz A., Gräbner-Radkowitsch C.: Capability Accumulation and Product Innovation: Agent-Based Perspective, Serie Rebuilding Macroeconomics Working Paper Series, Nr. 9, Seite(n) 1-22, 2020
	 \item Hirte K.: Friedman’s Instrumentalismus und das Problem von Kopernikus. Zur zentralen Rolle von Ausgangsannahmen in Theorien, in Pühringer, Stephan; Graupe, Silja; Hirte, Katrin; Kapeller, Jakob; Panther, Stephan: Jenseits der Konventionen: Alternatives Denken zu Wirtschaft, Gesellschaft und Politik., Metropolis Verlag, Marburg, Seite(n) 97-122, 2020
	 \item Hirte K.: Friedman’s Instrumentalismus und das Problem von Kopernikus, in Pühringer, Stephan; Graupe, Silja; Hirte, Katrin; Kapeller, Jakob; Panther, Stephan: Jenseits der Konventionen, Metropolis Verlag, Marburg, Seite(n) 97-122, 2020
	 \item Hirte K.: Das doppelte Reflektionsproblem, in Hochmann, Lars: Economists4future, Murmann Verlag, Hamburg, Seite(n) 43-58, 2020
	 \item Heimberger P., Krahé M., Ponattu D., van't Klooster J.: Keeping the promise of eurozone convergence, 2020
	 \item Heimberger P., Kowall N.: Seven ’surprising’ facts about the Italian economy, 2020
	 \item Heimberger P., Kapeller J.: ‘Output gap nonsense’ and the EU’s fiscal rules: A response to the European Commission’s economists, in Blog Institute for New Economic Thinking, 2020
	 \item Heimberger P., Huber J., Kapeller J.: The power of economic models: The case of the EU's fiscal regulation framework, in Socio-Economic Review, Vol. 18, Nr. 2, Seite(n) 337-366, 2020
	 \item Heimberger P.: Wie die ökonomische Globalisierung die Einkommensungleichheit beeinflusst, 2020
	 \item Heimberger P.: Hyperinflation and the Rise of the Nazis, 2020
	 \item Heimberger P.: EU-Wiederaufbaufonds als Kernstück europäischer Krisenbekämpfung: Progressiver Durchbruch oder Enttäuschung?, 2020
	 \item Heimberger P.: Budgetpolitik in der Corona-Krise: Reform der Budgetregeln erforderlich, 2020
	 \item Heimberger P.: Budgetpolitik im Wirtschaftsabschwung: erhebliche Spielräume vorhanden, 2020
	 \item Heimberger P.: Structural polarisation and path dependent development models in the EU, in Blog Developing Economics, 2020
	 \item Heimberger P.: The dynamic effects of fiscal consolidation episodes on income inequality: Evidence for 17 OECD countries over 1978-2013, in Empirica, Vol. 47, Seite(n) 53-81, 2020
	 \item Heimberger P.: Does economic globalisation affect income inequality? A meta-analysis, in The World Economy, Seite(n) 1-23, 2020
	 \item Hager T., Heck I., Rath J.: Competition in Transformational Processes: Polanyi \& Schumpeter, in Momentum: Momentum Kongress Paper, Seite(n) 1-25, 2020
	 \item Gräbner-Radkowitsch C., Hornykewycz A.: Capability accumulation and product innovation: an agent-based perspective, 2020
	 \item Gräbner-Radkowitsch C., Heimberger P., Kapeller J.: Pandemic pushes polarisation: The Corona crisis and macroeconomic divergence in the Eurozone, 2020
	 \item Gräbner-Radkowitsch C., Heimberger P., Kapeller J.: Do the “smart kids” catch up? Technological capabilities, globalisation and economic growth, 2020
	 \item Gräbner-Radkowitsch C., Hafele J.: The emergence of core-periphery structures in the European Union: a complexity perspective, 2020
	 \item Gräbner-Radkowitsch C., Hafele J.: Emergence of Core-Periphery Structures in the European Union: A Complexity Perspective, Serie Rebuilding Macroeconomics Working Paper Series, Nr. 17, Seite(n) 1-21, 2020
	 \item Gräbner C., Strunk B.: Pluralism in economics: its critiques and their lessons, in Journal of Economic Methodology, Vol. 27, Nr. 4, Seite(n) 311-329, 2020
	 \item Gräbner C., Heimberger P., Kapeller J., Schütz B.: Structural change in times of increasing openness: assessing path dependency in European economic integration, in Journal of Evolutionary Economics, Vol. 30, Nr. 5, Seite(n) 1467-1495, 2020
	 \item Gräbner C., Heimberger P., Kapeller J., Schütz B.: Is the Eurozone disintegrating? Macroeconomic divergence, structural polarization, trade and fragility, in Cambridge Journal of Economics, Vol. 44, Nr. 3, Seite(n) 647-669, 2020
	 \item Gräbner C., Heimberger P., Kapeller J.: Pandemic pushes polarisation: the Corona crisis and macroeconomic divergence in the Eurozone, in Journal of Industrial and Business Economics, Vol. 47, Nr. 3, Seite(n) 425-438, 2020
	 \item Bäuerle L., Ötsch W., Pühringer S.: Wirtschaft(lich) studieren. Erlebniswelten von Studierenden der Volkswirtschaftslehre, Springer, Wiesbaden, 2020
	 \item Beyer K., Griesser M., Pühringer S.: Türkis-blaue Arbeitsmarkt- und Sozialpolitik revisited: zwischen Meritokratie und Wohlfahrtschauvinismus, 2020
	 \item Altreiter C., Gräbner-Radkowitsch C., Pühringer S., Rogojanu A., Wolfmayr G.: Theorizing competition: an interdisciplinary framework, 2020
	 \item Altreiter C., Gräbner-Radkowitsch C., Pühringer S., Rogojanu A., Wolfmayr G.: Theorizing Competition: an interdisciplinary approach, 2020
	 \item Aistleitner M., Pühringer S.: Exploring the trade (policy) narratives in economic elite discourse, 2020
	 \item Aistleitner M., Pühringer S.: Exploring the trade (policy) narratives in economic elite discourse, Serie SSRN, 2020
	 \item Aistleitner M., Gräbner-Radkowitsch C., Hornykewycz A.: Theory and empirics of capability accumulation: implications for macroeconomics modelling, Serie Rebuilding Macroeconomics Working Paper Series, Nr. 6, Seite(n) 1-29, 2020
	 \item Aistleitner M., Gräbner-Radkowitsch C., Hornykewycz A.: Theory and Empirics of Capability Accumulation: Implications for Macroeconomic Modelling, 2020
\end{enumerate}
\subsection*{2019}
\begin{enumerate}
    	 \item Ötsch W., Pühringer S.: The anti-democratic logic of right-wing Populism and neoliberal market-fundamentalism, in Institut für Ökonomie, Cusanus Hochschule, Serie Working Paper Serie ök, Nr. 48, 2019
	 \item Ötsch W., Pühringer S.: Marktfundamentalismus als Kollektivgedanke. Mises und die Ordoliberalen, in Richard Sturn, Nenad Pantelic: Dem Markt vertrauen? Beiträge zur Tiefenstruktur neoliberaler Regulierung., Metropolis, Marburg, Seite(n) 185-210, 2019
	 \item Ötsch W., Pühringer S.: Was ist eine Krise? Ein Rückblick auf die Wirtschafts- und Finanzkrisen 2008 und 2010, in Blickpunkt WISO, 2019
	 \item Ötsch W., Graupe S., Loske R.: ’Erkühne Dich, weise zu sein!‘ Grundlegung einer Gemeinsinn-Ökonomie, in GWP - Gesellschaft, Wirtschaft, Politik, Vol. 68, Nr. 2, Seite(n) 243-250, 2019
	 \item Ötsch W.: Überwachungskapitalismus: Das Internet als totalitärer Markt, 2019
	 \item Ötsch W.: Ökonomie als Lachnummer, in Krauß, Dietrich: Die Rache des Mainstreams an sich selbst. 5 Jahre »Die Anstalt«, Westend Verlag, Frankfurt am Main, Seite(n) 260-271, 2019
	 \item Ötsch W.: Wissen und Nichtwissen angesichts ‚des Marktes‘. Das Konzept von Hayek, in Graupe, Silja; Ötsch, Walter O.; Rommel, Florian: Spielräume des Denkens, Metropolis, Marburg, Seite(n) 311-339, 2019
	 \item Wildauer R., Kapeller J.: Rank Correction: A New Approach to Differential Non-Response in Wealth Survey Data, 2019
	 \item Schütz B.: Creating a pluralist paradigm: An application to the minimum wage debate: Momentum Kongress Paper, Seite(n) 1-41, 2019
	 \item Pühringer S., Ötsch W.: Die Wirkmacht der „Liebe zum Markt”: Zum anhaltenden Einfluss ordoliberaler ÖkonomInnen-Netzwerke in Politik und Gesellschaft, 2019
	 \item Pühringer S., Ötsch W.: Die Wirkmacht der \glqq Liebe zum Markt\grqq{}: Zum anhaltenden Einfluss ordoliberaler ÖkonomInnenNetzwerke in Politik und Gesellschaft., in Institut für Ökonomie, Cusanus Hochschule, Serie Working Paper Serie ök, Nr. 51, 2019
	 \item Pühringer S., Bäuerle L.: What economics education is missing: the real world, in International Journal of Social Economics, Vol. 46, Nr. 8, Seite(n) 977-991, 2019
	 \item Pühringer S.: The “eternal character” of austerity measures in European crisis policies. Evidences from the Fiscal Compact discourse in Austria., in Power, Kate; Ali, Tanweer; Lebduskova, Eva: Discourse Analysis and Austerity: Critical Studies from Economics and Linguistics, Routledge, London, Seite(n) 134-158, 2019
	 \item Porak L.: Der Wert des Widerspruchs für die demokratische Praxis: Momentum Kongress Paper, Track 10, Seite(n) 1-15, 2019
	 \item Kapeller J., Schütz B., Ferschli B.: Finanzialisierung und Globale Ungleichheit: Globale Ungleichheit: Über Zusammenhänge von Kolonialismus, Arbeitsverhältnissen und Naturverbrauch., Mandelbaum Verlag, Wien, 2019
	 \item Kapeller J., Schütz B., Ferschli B.: Exkurs: Die Macht der internationalen Vermögensverwalter: Das Beispiel BlackRock: Globale Ungleichheit: Über Zusammenhänge von Kolonialismus, Arbeitsverhältnissen und Naturverbrauch, Mandelbaum Verlag, Wien, 2019
	 \item Kapeller J., Meyer D.: Change and persistence in contemporary economics, Serie Special Issue Science in Context, Vol. 32, 2019
	 \item Kapeller J., Meyer D.: Introduction: change and persistence in contemporary economics, in Science in Context, Vol. 32, Nr. 4, Seite(n) 357-360, 2019
	 \item Kapeller J., Gräbner-Radkowitsch C., Heimberger P.: Economic Polarisation in Europe: Causes and Policy Options, 2019
	 \item Kapeller J., Gräbner C., Heimberger P.: Holding Together what Belongs Together: A Strategy to Counteract Economic Polarisation in Europe, 2019
	 \item Kapeller J., Gräbner C., Heimberger P.: Wirtschaftliche Polarisierung in Europa: Ursachen und Handlungsoptionen, Friedrich-Ebert Stiftung, Bonn, 2019
	 \item Kapeller J., Ferschli B.: Hans Albert und die Kritik am Modell-Platonismus in den Wirtschaftswissenschaften, in Franco, Giuseppe: Handbuch Karl Popper, Springer Fachbuch, Heidelberg, Seite(n) 733-749, 2019
	 \item Kapeller J.: Pluralism in Economics: Epistemological Rationales and Pedagogical Implementation, in Decker, Samuel; Elsner, Wolfram; Flechtner, Svenja: Advancing Pluralism in Teaching Economics, Routledge, London, Seite(n) 55-77, 2019
	 \item Kapeller J.: Humankapital, in von Braunmühl, Claudia; Gerstenberger, Heide; Ptak, Ralf; Wichterich, Christa: ABC der globalen Unordnung. Von »Anthropozän« bis »Zivilgesellschaft«, VSA-Verlag, Hamburg, Seite(n) 120-121, 2019
	 \item Hirte K., Poppinga O.: Das Gemeine an der Gemeinwohldebatte, in Wege für eine bäuerliche Zukunft – Zeitschrift der ÖBV/ Via Campesina Austria, Vol. 42, Nr. 3 (358), Seite(n) 7-9, 2019
	 \item Hirte K.: Die deutsche Agrarpolitik und Agrarökonomik. Entstehung und Wandel zweier ambivalenter Disziplinen, Springer, Wiesbaden, 2019
	 \item Hirte K.: Das dritte gossensche Gesetz, in Hochmann, Lars; Graupe, Silja; Korbun, Thomas; Panther, Stephan; Schneidewind, Uwe: Möglichkeits¬wissen¬schaften. Ökonomie mit Möglichkeitssinn, Metropolis Verlag, Marburg, Seite(n) 133-176, 2019
	 \item Heinrich T., Gräbner C.: Beyond equilibrium: revisiting two-sided markets from an agent-based modelling perspective, in International Journal of Computational Economics and Econometrics, Vol. 9, Nr. 3, Seite(n) 153-180, 2019
	 \item Heimberger P., Pekanov A.: Dem wirtschaftlichen Abschwung entgegenwirken: Zur wichtigen Rolle der Fiskalpolitik, 2019
	 \item Heimberger P., Kapeller J.: What to do about divergence between EU countries? The problem of structural polarization, 2019
	 \item Heimberger P.: ‘Output gap nonsense': Understanding the budget conflict between the EC and Italy’s government, 2019
	 \item Heimberger P.: Unemployment in Europe: What should be done?, 2019
	 \item Heimberger P.: The current economic downturn in Europe must be seen in the context of a wider problem of economic polarisation, in Vienna Institute for International Economic, 2019
	 \item Heimberger P.: Italien vs. EU-Kommission: Warum ein Defizitverfahren kontraproduktiv wäre, 2019
	 \item Heimberger P.: How much space for fiscal expansion? Germany falls victim to 'output gap nonsense’, in Vienna Institute for International Economic (wiiw), 2019
	 \item Heimberger P.: Arbeitslosigkeit in Europa: Was man tun könnte, 2019
	 \item Heimberger P.: The Impact of Labour Market Institutions and Capital Accumulation on Unemployment: Evidence for the OECD, 1985-2013, Serie wiiw Working Papers, Nr. 164, 2019
	 \item Heimberger P.: Does economic globalisation affect income inequality? A meta-analysis, Serie wiiw Working Papers, Nr. 165, 2019
	 \item Heimberger P.: Arbeitsmarktinstitutionen, Kapitalakkumulation und Arbeitslosigkeit in OECD-Ländern, Wirtschaft und Gesellschaft, in Wirtschaft und Gesellschaft, Vol. 45, Nr. 1, 2019
	 \item Gräbner-Radkowitsch C., Tamesberger D., Heimberger P., Kapelari T., Kapeller J.: Trade Models in the European Union, 2019
	 \item Gräbner C., Kapeller J., Heimberger P.: Economic Polarisation in Europe: Causes and Policy Options, 2019
	 \item Gräbner C., Bale C., Furtado B., Alvarez-Pereira B., Gentile J., Henderson H., Lipari F.: Getting the Best of Both Worlds? Developing Complementary Equation-Based and Agent-Based Models, in Computational Economics, Vol. 53, Nr. 2, Seite(n) 763-782, 2019
	 \item Grimm C.: Der Einfluss des Neoliberalismus auf österreichische Parteiprogramme, 2019
	 \item Grimm C.: Ideas have Consequences: Eine vergleichende Analyse zur transformativen Rolle von Ideen., in Momentum Quarterly, Vol. 8, Nr. 4, Seite(n) 183-247, 2019
	 \item Griesser M.: Deutungsrahmen der aktiven Arbeitsmarktpolitik: ein deutsch-österreichischer Vergleich von diskursiven Frames aus Anlass von 50 Jahren Arbeits(markt)förderungsgesetz, in Momentum Quarterly, Vol. 8, Nr. 3, Seite(n) 166-182, 2019
	 \item Graupe S., Ötsch W., Rommel F.: Spielräume des Denkens, Metropolis, Marburg, 2019
	 \item Graupe S., Ötsch W., Rommel F.: Spielräume des Denkens. Zur Einführung, in Graupe, Silja; Ötsch, Walter O.; Rommel, Florian: Spielräume des Denkens, Metropolis, Marburg, Seite(n) 9-32, 2019
	 \item Flechtner S., Gräbner-Radkowitsch C.: The heterogeneous relationship between income and inequality: a panel co-integration approach, 2019
	 \item Flechtner S., Gräbner C.: The heterogeneous relationship between income and inequality: a panel co-integration approach, in Economics Bulletin, Vol. 39, Nr. 4, Seite(n) 2540–2549, 2019
	 \item Bäuerle L., Pühringer S., Ötsch W.: \glqq Ohne Effizienz geht es nicht\grqq{}. Ergebnisse einer qualitativ-empirischen Erhebung unter Studierenden der Volkswirtschaftslehre, Serie FGW-Studien, Nr. 13, FGW, Düsseldorf, 2019
	 \item Beyer K., Pühringer S.: Divided we stand? Professional consensus and political conflict in academic economics, 2019
	 \item Beyer K., Pühringer S.: Divided we stand? Professional consensus and political conflict in academic economics, in Institut für Ökonomie, Cusanus Hochschule, Serie Working Paper Serie ök, Nr. 51, 2019
	 \item Beyer K.: Antonino Palumbo and Alan Scott (2018): Remaking Market Society: A Critique of Social Theory and Political Economy in Neoliberal Times, in ÖZS - Österreichische Zeitschrift für Soziologie, Vol. 44, Nr. 2, Seite(n) 249-252, 2019
	 \item Aistleitner M., Kapeller J., Steinerberger S.: Citation Patterns in Economics and Beyond: Assessing the Peculiarities of Economics from Two Scientometric Perspectives, in Science in Context, Vol. 32, Nr. 4, Seite(n) 361-380, 2019
	 \item Aistleitner M., Grimm C., Kapeller J.: Auftragsvergabe, Leistungsqualität und Kostenintensität im Schienenpersonenverkehr: Momentum Kongress Paper, Seite(n) 1-56, 2019
\end{enumerate}
\subsection*{2018}
\begin{enumerate}
    	 \item Ötsch W., Pühringer S., Hirte K.: Netzwerke des Marktes : Ordoliberalismus als Politische Ökonomie, Springer VS, Wiesbaden, 2018
	 \item Ötsch W., Pühringer S.: Marktfundamentalismus als Kollektivgedanke. Mises und die Ordoliberalen, in Institut für Ökonomie, Cusanus Hochschule, Bernkastel-Kues, Serie Working Paper Serie ök, Nr. 41, 2018
	 \item Ötsch W., Pühringer S.: Was ist eine Krise? Wie ökonomische Theorien Wahrnehmung formen, in Kurswechsel, Nr. 4/2018, Seite(n) 7-17, 2018
	 \item Ötsch W., Graupe S.: Der vergessene Lippmann: Politik, Propaganda und Markt, in Institut für Ökonomie, Cusanus Hochschule, Bernkastel-Kues, Serie Working Paper Serie ök, Nr. 39, 2018
	 \item Ötsch W., Graupe S.: Vorwort zu Walter Lippmann: Die öffentliche Meinung. Wie sie entsteht und manipuliert wird: Die öffentliche Meinung. Wie sie entsteht und manipuliert wird, Westend Verlag, Frankfurt am Main, 2018
	 \item Ötsch W., Graupe S.: Einführung: Die Bedeutung von Bildern für das Denken, in Ötsch, Walter O.; Graupe, Silja: Macht der Bilder, Macht der Sprache, Angelika Lenz Verlag, Isenburg, Seite(n) 9-19, 2018
	 \item Ötsch W., Graupe S.: Einführung: Die Bedeutung von Bildern für Denken und Sprechen: Macht der Bilder, Macht der Sprache. Band 37 der Schriftenreihe der Freien Akademie Falkensee., Angelika Lenz Verlag, Isenburg, Seite(n) 75–86, 2018
	 \item Ötsch W.: Wissen und Nicht-Wissen angesichts \glqq des Marktes\grqq{}: Das Konzept von Hayek, in Institut für Ökonomie, Cusanus Hochschule, Bernkastel-Kues, Serie Working Paper Serie ök, Vol. 43, 2018
	 \item Ötsch W.: Bilder in der Geschichte der Ökonomie: Das Beispiel der Metapher von der Wirtschaft als Maschine, in Institut für Ökonomie, Cusanus Hochschule, Bernkastel-Kues., Serie Working Paper Serie ök, Vol. 42, 2018
	 \item Ötsch W.: Mythos Markt. Mythos Neoklassik. Das Elend des Marktfundamentalismus, Serie Kritische Studien zu Markt und Gesellschaft, Vol. 11, Metropolis, Marburg, 2018
	 \item Ötsch W.: Bilder des Rechtspopulismus, in Ötsch, Walter O.; Graupe, Silja: Macht der Bilder, Macht der Sprache, Angelika Lenz Verlag, Isenburg, Seite(n) 113-128, 2018
	 \item Ötsch W.: Es wächst das Bewusstsein einer multiplen Krise, in Agora42 (Philosophisches Wirtschaftsmagazin), 2018
	 \item Ötsch W.: Rechtspopulismus: Ein Gesellschaftsbild mit eskalierender Wirkung, in Salzburger Theologische Zeitschrift, Vol. 21, Nr. 1, Seite(n) 7-22, 2018
	 \item Pühringer S., Ötsch W.: Neoliberalism and Right-wing Populism: conceptual analogies, in Forum for Social Economics, Vol. 47, Nr. 2, Seite(n) 192-203, 2018
	 \item Pühringer S., Liedl B.: Ökonomische Expertise und polit-ökonomische Machtstrukturen, in AK Kärnten: Welt aus den Fugen. Wie der Neoliberalismus unser Leben verändert., ÖGB-Verlag, Wien, Seite(n) 41-56, 2018
	 \item Pühringer S., Grimm C.: Die deutschsprachige Volkswirtschaftslehre, in beigewum.at, 2018
	 \item Pühringer S., Egger J.: Krisenbilder von ÖkonomInnen in der Presse, in Walter Ötsch; Silja Graupe: Macht der Bilder, Macht der Sprache, Seite(n) 75-86, 2018
	 \item Pühringer S., Bäuerle L.: What economics education is missing: The real world, in Institut für Ökonomie, Cusanus Hochschule, Serie Working Paper Serie ök, Nr. 37, 2018
	 \item Pühringer S.: Die Krise als Katalysator für den Aufschwung des Rechtspopulismus, Vol. 47, 2018
	 \item Pühringer S.: The “eternal character” of austerity measures in European crisis policies., 2018
	 \item Pühringer S.: Politische und gesellschaftliche Wirkmächtigkeit von ÖkonomInnen-Netzwerken: Momentum Kongress Paper, Seite(n) 1-10, 2018
	 \item Porak L.: Die Positionierung von Studierenden an öffentlichen Hochschulen: Momentum Kongress Paper, Seite(n) 1-18, 2018
	 \item Landesmann M., Kapeller J., Mohr F., Schütz B.: Government policies and financial crises: mitigation, postponement or prevention?, in Cambridge Journal of Economics, Vol. 42, Nr. 2, Seite(n) 309-330, 2018
	 \item Kapeller J., Ferschli B., Schütz B., Wildauer R.: Wie viel bringt die Vermögenssteuer? Neue Aufkommensschätzungen für Österreich., in ISW Institut für Sozial- und Wirtschaftswissenschaften, in Wirtschafts- und Sozialwissenschaftliche Zeitschrift, Vol. 40, Nr. 1, Seite(n) 146-160, 2018
	 \item Kapeller J., Dobusch L.: Open strategy-making with crowds and communities: Comparing Wikimedia and Creative Commons, in Long Range Planning, Vol. 51, Nr. 4, Seite(n) 561-579, 2018
	 \item Kapeller J., Böck M., Schütz B., Zens G.: Ökonomische Effekte der Verkehrsreform des Landes Tirol, Johannes Kepler Universität, Linz, 2018
	 \item Kapeller J.: The Top Journals Club in Economics, in Institute for New Economic Thinking (INET), Commentaries, 2018
	 \item Hirte K., Thieme S.: Heterodoxie in der Ökonomik, in Schetsche, Michael; Schmied-Knittel, Ina: Heterodoxie. Konzepte, Traditionen, Figuren der Abweichung, Halem Verlag, Köln, 2018
	 \item Hirte K.: Zeitlichkeit und Tauschfähigkeit bei Rosa Luxemburg und Joseph A. Schumpeter, in Bies, Michael; Giacovelli, Sebastian; Langenohl, Andreas: Ästhetische Eigenzeiten von Tausch und Gabe, Wehrhahn Verlag, Hannover, 2018
	 \item Heimberger P.: The dynamic effects of fiscal consolidation episodes on income inequality: Evidence for 17 OECD Countries over 1978-2013, 2018
	 \item Heimberger P.: Fiscal multipliers, unemployment and debt, Wirtschaftsuniversität Wien, 2018
	 \item Hager T., Rath J., Wimmler L.: Work or Die? How Wage dependency determines the production process: Momentum Kongress Paper, Seite(n) 1-37, 2018
	 \item Gräbner-Radkowitsch C., Strunk B.: Pluralism in economics: its critiques and their lessons, 2018
	 \item Gräbner-Radkowitsch C., Heimberger P., Kapeller J., Springholz F.: Measuring Economic Openness: A review of existing measures and empirical practices, 2018
	 \item Gräbner-Radkowitsch C., Heimberger P., Kapeller J., Schütz B.: Structural change in times of increasing openness: assessing path dependency in European economic integration, 2018
	 \item Gräbner C., Heinrich T., Kudic M., Vermeulen B.: The dynamics of and on networks, in International Journal of Computational Economics and Econometrics, Vol. 8, 2018
	 \item Gräbner C., Heinrich T., Kudic M., Vermeulen B.: The dynamics of and on networks: an introduction, in International Journal of Computational Economics and Econometrics, Vol. 8, Nr. 3/4, Seite(n) 229-241, 2018
	 \item Gräbner C., Heimberger P., Kapeller J., Schütz B.: Desintegration in Europa? Makroökonomische Divergenz und strukturelle Polarisierung: Momentum Kongress Paper, Serie Momentum quarterly, Seite(n) 1-32, 2018
	 \item Gräbner C., Elsner W., Lascaux A.: To trust or to control: Informal value transfer systems and computational analysis in institutional economics, in Journal of Economic Issues, Vol. 52, Nr. 2, Seite(n) 559-569, 2018
	 \item Gräbner C.: Formal Approaches to Socio-economic Analysis - Past and Perspectives, in Forum for Social Economics, Vol. 47, Nr. 1, Seite(n) 32-63, 2018
	 \item Gräbner C.: How to Relate Models to Reality? An Epistemological Framework for the Validation and Verification of Computational Models, in Journal of Artificial Societies and Social Simulation, Vol. 21, Nr. 3, 2018
	 \item Grimm C., Kapeller J., Pühringer S.: Paradigms and Policies: The state of economics in the german-speaking countries, 2018
	 \item Grimm C.: Wirtschaftspolitische Positionen österreichischer Parteien im historischen Verlauf. Die Ausgestaltung österreichischer Parteiprogrammatiken hinsichtlich neoliberalen Gedankenguts, in Momentum Quarterly, Vol. 7, Nr. 3, Seite(n) 136-154, 2018
	 \item Griesser M., Hofmann J.: Editorial: Freie Fahrt für reiche Burschen? Schwarz-Blau ist zurück!, in Kurswechsel, Nr. 3, 2018
	 \item Griesser M., Brand U.: Wachstum? Wohlstand und Lebensqualität!, in Momentum Quarterly, Vol. 7, Nr. 2, Seite(n) 53-72, 2018
	 \item Graupe S., Ötsch W.: Macht der Bilder, Macht der Sprache. Band 37 der Schriftenreihe der Freien Akademie Falkensee., Angelika Lenz Verlag, Isenburg, 2018
	 \item Ferschli B., Kapeller J., Wildauer R.: Zur Verteilung und Klassenstruktur der Österreichischen Vermögen: Momentum Kongress Paper, Seite(n) 1-55, 2018
	 \item Beyer K., Pühringer S.: Freiheitliche Flügelkämpfe? (Historische) Konfliktlinien in der FPÖ, in BEIGEWUM, in Kurswechsel, Nr. 3, Seite(n) 19-27, 2018
	 \item Beyer K., Grimm C., Kapeller J., Pühringer S.: Netzwerke, Paradigmen, Attitüden. Der deutsche Sonderweg im Fokus. Paradigmatische Ausrichtung und politische Orientierung von deutschen und US-amerikanischen Ökonomi\_nnen im Vergleich, Nr. 7, FGW, Düsseldorf, 2018
	 \item Aistleitner M., Kapeller J., Steinerberger S.: The Power of Scientometrics and the Development of Economics, in Journal of Economic Issues, Vol. 52, Nr. 3, Routledge, Seite(n) 816-834, 2018
	 \item Aigner E., Aistleitner M., Glötzl F., Kapeller J.: The Focus of Academic Economics: Before and After the Crisis, in Institute for New Economic Thinking (INET), Serie Institute for New Economic Thinking  Working Paper Series, 2018
\end{enumerate}
\subsection*{2017}
\begin{enumerate}
    	 \item Ötsch W., Pühringer S.: Right-wing populism and market-fundamentalism. Two mutually reinforcing threats to democracy in the 21st century, in Journal of Language and Politics, Vol. 16, Nr. 4, Seite(n) 497-509, 2017
	 \item Schütz B.: Zunehmende Ungleichheit der Vermögensverteilung: Rezension von Michael Schneider, Mike Pottenger und John E. King „The Distribution of Wealth – Growing Inequality?“, in Wirtschaft und Gesellschaft, Vol. 43, Nr. 3, Seite(n) 449-451, 2017
	 \item Pühringer S., Ötsch W.: Neoliberalism and Right-wing Populism: conceptual analogies, in Institut für Ökonomie, Cusanus Hochschule, Bernkastel-Kues, Serie Working Paper Serie ök, Vol. 36, 2017
	 \item Pühringer S., Liedl B.: Argumentationsstrategien einer neoliberalen Reformagenda: Zum Diskursprofil der Agenda Austria in medialen Debatten, in Institut für Ökonomie, Cusanus Hochschule, Bernkastel-Kues, Serie Working Paper Serie ök, Vol. 27, 2017
	 \item Pühringer S., Griesser M.: From the \glqq planning euphoria" to the "bitter economic truth\grqq{}: The transmission of economic ideas into German labour market policies in the 1960s and 2000s, in Institut für Ökonomie, Cusanus Hochschule, Bernkastel-Kues, Serie Working Paper Serie ök, Nr. 30, 2017
	 \item Pühringer S., Bäuerle L., Engarntner T.: Was denken (zukünftige) ÖkonomInnen?: Einblicke in die politische und gesellschaftliche Wirkmächtigkeit ökonomischen Denkens., in GWP - Gesellschaft, Wirtschaft, Politik, Vol. 66, Nr. 4, Seite(n) 547-556, 2017
	 \item Pühringer S.: Think tank networks of German neoliberalism power structures in economics and economic policies in post-war Germany, in Institut für Ökonomie, Cusanus Hochschule, Bernkastel-Kues, Serie Working Paper Serie ök, Nr. 24, 2017
	 \item Pühringer S.: The “eternal character” of austerity measures in European crisis policies. Evidences from the Fiscal Compact discourse in Austria., in Institut für Ökonomie, Cusanus Hochschule, Bernkastel-Kues, Serie Working Paper Serie ök, Nr. 32, 2017
	 \item Pühringer S.: The success story of ordoliberalism as guiding principle of German economic policy, in Hien, Josef; Joerges, Christian: Ordoliberalism. Law and the rule of economics, Hart Publishing, Oxford, Portland, Seite(n) 134-158, 2017
	 \item Kapeller J., Steinerberger S.: Stability, Fairness and Random Walks in the Bargaining Problem, 2017
	 \item Kapeller J., Steinerberger S.: Stability , fairness and random walks in the bargaining problem, in Physica A: Statistical Mechanics and its Applications, Vol. 488, Elsevier, Seite(n) 60-71, 2017
	 \item Kapeller J., Schütz B., Springholz F.: Internationale Tendenzen und Potentiale der Vermögensbesteuerung, in Dimmel, Nikolaus; Hofmann, Julia; Schenk, Martin; Schürz, Martin: Handbuch Reichtum – Neue Erkenntnisse aus der Ungleichheitsforschung, Studienverlag, Wien, Seite(n) 477-492, 2017
	 \item Kapeller J., Pühringer S., Grimm C.: Zum Profil der deutschsprachigen Volkwirtschaftslehre. Paradigmatische Ausrichtung und politische Orientierung deutschsprachiger Ökonom\_innen, Nr. 2, FGW-Studien, Düsseldorf, 2017
	 \item Kapeller J., Ferschli B., Schütz B., Wildauer R.: Bestände und Konzentration privater Vermögen in Österreich, in Abteilung Wirtschaftswissenschaft der AK Wien, in Materialien zu Wirtschaft und Gesellschaft, Nr. 167, 2017
	 \item Kapeller J., Ferschli B., Schütz B., Wildauer R.: Bestände und Konzentration privater Vermögen in Österreich, in Arbeiterkammer Wien, in Wirtschaft und Gesellschaft, Vol. 43, Nr. 4, Seite(n) 499-534, 2017
	 \item Kapeller J.: Delayed by outsourcing? Zur Stabilität des Kapitalismus im 21. Jahrhundert, 2017
	 \item Kapeller J.: Delayed by outsourcing? Zur Stabilität des Kapitalismus im 21. Jahrhundert (Doppelrezension), in Hrsg. v. Hollstein, Betina / Schimank, Uwe / Struck, Olaf / Weiß, Anja, in Soziologische Revue, Vol. 40, Nr. 4, DeGruyter, Seite(n) 547–555, 2017
	 \item Hirte K., Pühringer S.: Zur Performativität ökonomischen Wissens und aktuellen ÖkonomInnen-Netzwerken in Deutschland, in Maeße, Jens; Pahl, Hanno; Sparsam, Jan: Die Innenwelt der Ökonomie. Wissen, Macht und Performativität in der Wirtschaftswissenschaft, Springer VS Verlag, Wiesbaden, Seite(n) 363-390, 2017
	 \item Hirte K.: Agrarpolitik und Agrarökonomie. Zur Ambivalenz zweier wissenschaftlicher Disziplinen, Universität Jena, 2017
	 \item Hirte K.: Zur Performativität in den Wirtschaftswissenschaften. Kernaussagen, Anwendungspotentiale und Grenzen eines Konzepts, in Pfriem, Reinhard; Schneidewind, Uwe: Transformative Wirtschaftswissenschaft im Kontext nachhaltiger Entwicklung, Metropolis Verlag, Marburg, 2017
	 \item Heimberger P., Kapeller J., Schütz B.: The NAIRU determinants: what’s structural about unemployment in Europe?, in Journal of Policy Modeling, Vol. 39, Nr. 5, Seite(n) 883-908, 2017
	 \item Heimberger P., Kapeller J.: Wie ein makroökonomisches Modell die Spaltung der Eurozone befördert, 2017
	 \item Heimberger P., Kapeller J.: The performativity of potential output: Pro-cyclicality and path dependency in coordinating European fiscal policies, in Review of International Political Economy, Vol. 24, Nr. 5, Seite(n) 904-928, 2017
	 \item Heimberger P.: Österreichs Staatsausgabenstrukturen im europäischen Vergleich, 2017
	 \item Heimberger P.: Österreichs Bildungs-, Gesundheits- und Sozialausgaben im europäischen Vergleich: Wenn der Staat spart, kann das für private Haushalte teuer werden, 2017
	 \item Heimberger P.: Weniger Staatsausgaben: Abbau des Sozialstaats und Vertiefung von Wirtschaftskrisen, 2017
	 \item Heimberger P.: Vorsicht bei Ländervergleichen – insbesondere bei Staatsausgaben!, 2017
	 \item Heimberger P.: Soll der Staat bei Bildung, Gesundheit und Sozialem kürzen? Austeritätspolitik seit der Finanzkrise im Vergleich, 2017
	 \item Heimberger P.: Did Fiscal Consolidation Cause the Double-Dip Recession in the Euro Area?, in Review of Keynesian Economics, Vol. 5, Nr. 3, Seite(n) 439-458, 2017
	 \item Gräbner-Radkowitsch C., Heimberger P., Kapeller J., Schütz B.: Is Europe disintegrating? Macroeconomic divergence, structural polarization, trade and fragility, 2017
	 \item Gräbner-Radkowitsch C., Elsner W., Lascaux A.: Trust and Social Control. Sources of cooperation, performance, and stability in informal value transfer systems, 2017
	 \item Gräbner-Radkowitsch C., Elsner W., Lascaux A.: To trust or to control: Informal value transfer systems and computational analysis in institutional economics, 2017
	 \item Gräbner-Radkowitsch C.: The Complexity of Economies and Pluralism in Economics, 2017
	 \item Gräbner-Radkowitsch C.: How to relate models to reality? An epistemological framework for the validation and verification of computational models, 2017
	 \item Gräbner C., Kapeller J.: The Micro-Macro Link in Heterodox Economics, in Tae-Hee Jo, Lynne Chester Carlo and D'Ippoliti: The Routledge Handbook of Heterodox Economics, Routledge, Seite(n) 145-159, 2017
	 \item Gräbner C., Heinrich T.: Von Onlineplattformen und mittelalterlichen Märkten - Gleichgewichtsmodelle und agentenbasierte Modellierung zweiseitiger Märkte, in TATuP - Zeitschrift für Technikfolgenabschätzung in Theorie und Praxis, Vol. 26, Nr. 3, Seite(n) 23-29, 2017
	 \item Gräbner C.: Die Rolle des Gleichgewichtskonzepts in der mikroökonomischen Ausbildung, in Till van Treek, Janina Urban: Wirtschaft neu denken, iRIGHTS media, Berlin, Seite(n) 60-73, 2017
	 \item Gräbner C.: Dealing adequately with the political element in formal modelling, in Katsikides, Savas and Hanappi, Hardy and Scholz-Wäckerle, Manuel: Theory and Method of Evolutionary Political Economy, Routledge, New York, Seite(n) 236-254, 2017
	 \item Gräbner C.: The Complexity of Economies and Pluralism in Economics, in Journal of Contextual Economics, Vol. 137, Nr. 3, Seite(n) 193-225, 2017
	 \item Gräbner C.: The Complementary Relationship Between Institutional and Complexity Economics: The Example of Deep Mechanismic Explanations, in Journal of Economic Issues, Vol. 51, Nr. 2, Seite(n) 392-400, 2017
	 \item Grimm C.: Paradigmatische Homogenität? Aktueller Status und Zukunftsperspektiven der Ökonomik in Deutschland und den USA: Momentum Kongress Paper, Seite(n) 1-16, 2017
	 \item Griesser M., Sauer B.: Von der sozialen Neuzusammensetzung zur gewerkschaftlichen Erneuerung? MigrantInnen als Zielgruppe der österreichischen Gewerkschaftsbewegung, in ÖZS - Österreichische Zeitschrift für Soziologie, Seite(n) 147-166, 2017
	 \item Griesser M.: Rezension von „Monika Burmester, Emma Dowling \& Norbert Wohlfahrt (Hg.) (2017): Privates Kapital für soziale Dienste? Wirkungsorientiertes Investment und seine Folgen für die Soziale Arbeit“, in Soziales Kapital, Seite(n) 264-267, 2017
	 \item Griesser M.: sezonieri.at: Kollektive Handlungsfähigkeit von ErntearbeiterInnen in Österreich, in Schmidjell, Cornelia; Sedmak, Clemens; Koch, Andreas; Kapferer, Elisabeth; Gaisbauer, Helmut P.; Bogner, Stefan; Wimmer, Bernd: Lesebuch Soziale Ausgrenzung III, Mandelbaum Verlag, Wien, Seite(n) 89-92, 2017
	 \item Griesser M.: Images and imaginaries of unemployed people. Discursive shifts in the transition from active to activating labour market policies in Germany, in Critical Social Policy, Seite(n) [online first], 2017
	 \item Gerhartinger P., Haunschmid P., Tamesberger D.: How to explain Wage Growth Slowdown in Austria?, 2017
	 \item Ferschli B., Kapeller J., Schütz B., Wildauer R.: Bestände und Konzentration privater Vermögen in Österreich, 2017
	 \item Ferschli B., Kapeller J., Schütz B., Wildauer R.: Bestände und Konzentration privater Vermögen in Österreich: Materialien zu Wirtschaft und Gesellschaft, 2017
	 \item Aistleitner M., Kapeller J., Steinerberger S.: Citation Patterns in Economics and Beyond: Assessing the Peculiarities of Economics from Two Scientometric Perspectives, 2017
	 \item Aistleitner M., Kapeller J., Steinerberger S.: Citation Patterns in Economics and Beyond: Assessing the Peculiarities of Economics from Two Scientometric Perspectives: Momentum Kongress Paper, Seite(n) 1-22, 2017
\end{enumerate}
\subsection*{2016}
\begin{enumerate}
    	 \item Ötsch W.: Imaginierte Grundlagen bei Adam Smith, in Institut für Ökonomie, Cusanus Hochschule, Bernkastel-Kues, Serie Working Paper Serie ök, Nr. 19, 2016
	 \item Ötsch W.: Geld und Raum. Anmerkungen zum Homogenisierungsprogramm der beginnenden Neuzeit, in Brodbeck, Karl-Heinz; Graupe, Silja: Geld! Welches Geld? Geld als Denkform, Metropolis Verlag, Marburg, Seite(n) 71-101, 2016
	 \item Ötsch W.: Die neoliberale Utopie als Ende aller Utopien, in Pittl, Sebastian; Prüller-Jagenteufel; Gunter: Unterwegs zu einer neuen ‚Zivilisation geteilter Genügsamkeit‘. Perspektiven utopischen Denkens 25 Jahre nach dem Tod Ignacio Ellacurías, Vandenhoeck \& Ruprecht uni press, Wien, Seite(n) 105-119, 2016
	 \item Ötsch W.: Die Politische Ökonomie „des“ Marktes. Eine Zusammenfassung zur Wirkungsgeschichte von Friedrich A. Hayek, in Kapeller, Jakob; Pühringer, Stephan; Hirte, Katrin; Ötsch, Walter O.: Ökonomie! Welche Ökonomie? Stand und Status der Wirtschaftswissenschaften, Metropolis Verlag, Marburg, Seite(n) 19-50, 2016
	 \item Ötsch W.: Die Widersprüche des Mister Perfect: Berechnung – Beherrschung – Perfektion, in Agora42 (Philosophisches Wirtschaftsmagazin), Nr. 01/2017, Seite(n) 18-23, 2016
	 \item Ötsch W.: Populismus und Demagogie. Mit Beispielen von Jörg Haider, Heinz–Christian Strache und Frank Stronach, in Foreign Theoretical Trends, Nr. 10, Seite(n) 39-46, 2016
	 \item Ötsch W.: Imaginative Grundlagen bei Adam Smith. Aspekte von Bildlichkeit und ihrem Verlust in der Geschichte der Ökonomik, in Allgemeine Zeitschrift für Philosophie, Vol. 41, Nr. 3, Seite(n) 315-340, 2016
	 \item Álvarez Pereira B., Henderson H., Lipari F., Furtado B., Bale C., Gräbner C., Gentile J.: Errata in 'The Political Economy of the Kuznets Curve', in Review of Development Economics, Vol. 20, Nr. 4, Seite(n) 817-819, 2016
	 \item Schwardt H., Gräbner C., Heinrich T., Cordes C., Schwesinger G.: Economic Complexity and Trade-Offs in Policy Decisions, in Gräbner, Claudius; Heinrich, Torsten; Schwardt, Henning: Policy Implications of Evolutionary and Institutional Economics, Routledge, London, New York, Seite(n) 3-19, 2016
	 \item Pühringer S., Stelzer-Orthofer C.: Replik zur Replik: Von Vorwürfen der Unwissenschaftlichkeit, in SWS-Rundschau - Sozialwissenschaftliche Studiengesellschaft Rundschau, Vol. 3, Seite(n) 447-449, 2016
	 \item Pühringer S., Stelzer-Orthofer C.: Neoliberale Think Tanks als (neue) Akteure in österreichischen gesellschafts- und sozialpolitischen Diskursen. Das Beispiel des Hayek Institut und der Agenda Austria, in SWS-Rundschau - Sozialwissenschaftliche Studiengesellschaft Rundschau, Vol. 56, Nr. 1, Seite(n) 75-96, 2016
	 \item Pühringer S., Egger J.: Wie krank ist unser Wirtschaftssystem? Krisen als Krankheiten im ökonomischen Diskurs, in Kuckuck. Notizen zur Alltagskultur, Vol. 31, Nr. 1, Seite(n) 32-37, 2016
	 \item Pühringer S.: Ökonomisches Denken in der Krise, 2016
	 \item Pühringer S.: Still the queens of social sciences? Economists as “public intellectuals” in/after the crisis., in International Conference in Contemporary Social Sciences (Conference Proceedings): Crisis and the social sciences: New challenges and perspectives, Seite(n) 507-528, 2016
	 \item Pühringer S.: Agenda Austria: Diskursstrategien einer neoliberalen Reformagenda: Momentum Kongress Paper, Seite(n) 1-21, 2016
	 \item Kapeller J., Steinerberger S.: Emergent Phenomena in Scientific Publishing: A Simulation Exercise, in Research Policy, Vol. 45, Nr. 10, Seite(n) 1945–1952, 2016
	 \item Kapeller J., Schütz B., Tamesberger D.: From free to civilized trade: a European perspective, in Review of Social Economy, Vol. 74, Nr. 3, Seite(n) 320-328, 2016
	 \item Kapeller J., Schütz B., Springholz F.: Internationale Tendenzen und Potentiale der Vermögensbesteuerung, 2016
	 \item Kapeller J., Schütz B.: Verteilungstendenzen im Kapitalismus. Globale Perspektiven, in Bundesarbeitskammer: Die Verteilungsfrage. Von Reichtum, Krisen und Ablenkungsmanövern, ÖGB-Verlag, Wien, Seite(n) 49-54, 2016
	 \item Kapeller J., Scholz-Wäckerle M.: Evolutionary Political Economy and the Complexity of Economic Policy, in Gräbner,  Claudius / Heinrich, Torsten / Schwardt, Henning: Policy  Implications of Recent Advances in Evolutionary and Institutional  Economics, Routledge, London, Seite(n) 99-122, 2016
	 \item Kapeller J., Pühringer S., Hirte K., Ötsch W.: Ökonomie! Welche Ökonomie? Stand und Status der Wirtschaftswissenschaften, Metropolis, Marburg, 2016
	 \item Kapeller J., Pühringer S., Hirte K., Ötsch W.: Entwicklung, Zustand und Leerstellen der Ökonomik, in Pühringer, Stephan; Hirte, Katrin; Ötsch, Walter O.: Ökonomie! Welche Ökonomie? Stand und Status der Wirtschaftswissenschaften., Metropolis Verlag, Marburg, Seite(n) 2-10, 2016
	 \item Kapeller J., Heimberger P.: Spezialisierung, Stratifikation und internationale Wirtschaft: Verteilung, Arbeitsteilung und Klassenlagen aus globaler Perspektive: Momentum Kongress Paper, Seite(n) 1-20, 2016
	 \item Kapeller J.: Internationaler Freihandel: Theoretische Ausgangspunkte und empirische Folgen, 2016
	 \item Kapeller J.: Ein philosophischer Blick auf die Grundlagen internationaler Ökonomie., in Till van Treeck, Janina Urban: Wirtschaft neu denken – blinde Flecken der Lehrbuchökonomie, Seite(n) 108-116, 2016
	 \item Kapeller J.: Internationaler Freihandel: Theoretische Ausgangspunkte und empirische Folgen, in Wirtschafts- und Sozialwissenschaftliche Zeitschrift, Vol. 39, Nr. 1, Seite(n) 99-122, 2016
	 \item Hirte K., Ötsch W.: Ökonomie! Welche Ökonomie? Stand und Status der Wirtschaftswissenschaften, Metropolis Verlag, Marburg, 2016
	 \item Hirte K., Kuschel S.: Agrarpolitik und Arbeit – der Einfluss europäischer Agrarpolitikmaßnahmen auf die Arbeit im Agrarsektor, in Ahlert, Maximilian; Fiederer, Franca; Varelmann, Katharina; Kuschel, Sarah; Ewers, Sylvia; Politor, Merlin; Stamp, Katharina: Frohes Schaffen!? Arbeit in der Landwirtschaft, Universtity Press, Kassel, Seite(n) 9-15, 2016
	 \item Hirte K., Kapeller J., Pühringer S., Ötsch W.: Vorwort im Tagungsband Ökonomie! Welche Ökonomie?: Ökonomie! Welche Ökonomie?, Metropolis Verlag, Marburg, Seite(n) 2-10, 2016
	 \item Hirte K.: Netzwerke im Internet – eine neue kritische Öffentlichkeit? Das Beispiel Guttenberg, in Imhof, Kurt; Welz, Frank; Fleck, Christian; Vobruba, Georg: Neuer Strukturwandel der Öffentlichkeit. Verhandlungen des dritten gemeinsamen Kongresses der Deutschen, Österreichischen und Schweizerischen Gesellschaft für Soziologie, Springer, Wiesbaden, Seite(n) 1-16, 2016
	 \item Hirte K.: Die „Landnahme“-These von Rosa Luxemburg – empirisch beobachtbar, aber theoretisch falsifiziert?, in Kapeller, Jakob; Pühringer, Stephan; Hirte, Katrin; Ötsch, Walter: Ökonomie! Welche Ökonomie?, Metropolis Verlag, Marburg, Seite(n) 273-313, 2016
	 \item Hirte K.: Agrarische Regelungspolitik und die drei agrarpolitischen „Syndrome“, in Via Campesina Austria, in Wege für eine bäuerliche Zukunft – Zeitschrift der ÖBV/ Via Campesina Austria, Vol. 39, Nr. 3 (343), Seite(n) 10-11, 2016
	 \item Heimberger P., Kapeller J., Schütz B.: What’s ‘structural’ about unemployment in Europe: On the Determinants of the European Commission’s NAIRU Estimates, 2016
	 \item Heimberger P., Kapeller J.: The performativity of potential output: pro-cyclicality and path dependency in coordinating European fiscal policies, in Institute for New Economic Thinking, Serie Institute for New Economic Thinking  Working Paper Series, Nr. 50, 2016
	 \item Heimberger P., Kapeller J.: How economic policy drives European (dis)integration, in Institute for New Economic Thinking (INET), 2016
	 \item Heimberger P.: Mehr öffentliche Investitionen sind sinnvoll und erforderlich, 2016
	 \item Heimberger P.: Did Fiscal Consolidation Cause the Double Dip Recession in the Euro Area?, Serie wiiw Working Papers, Nr. 130, 2016
	 \item Heimberger P.: Wirtschaftliche Stagnation als \glqq neue Normalsituation\grqq{}?, in Wirtschaft und Gesellschaft, Vol. 42, Nr. 2, Seite(n) 356-361, 2016
	 \item Heimberger P.: Minsky, die globle Finanzkrise und der nächste Finanz-Crash, in Wirtschaft und Gesellschaft, Vol. 42, Nr. 3, Seite(n) 515-520, 2016
	 \item Heimberger P.: Helikoptergeld zur Überwindung der Wachstumsprobleme in Europa?, in AK Wien, in Wirtschaft und Gesellschaft, Vol. 42, Nr. 4, Wien, Seite(n) 690-695, 2016
	 \item Heimberger P.: Die aktuelle Krise im wirtschaftshistorischen Vergleich mit der Großen Depression der 1930er-Jahre, in Wirtschaft und Gesellschaft, Vol. 42, Nr. 1, Seite(n) 161-173, 2016
	 \item Heimberger P.: Die Macht ökonomischer Modelle am Beispiel des »Potential Output«-Modells der Europäischen Kommission: Momentum Kongress Paper, Seite(n) 1-9, 2016
	 \item Heimberger P.: Warum die Volkswirtschaften der Eurozone den USA und Großbritannien seit der Finanzkrise hinterherhinken: Zur Rolle von Unterschieden in der Geld– und Fiskalpolitik, in Vienna Institute for International Economic (wiiw), in Studies Research Report, Nr. 5, Wien, 2016
	 \item Heimberger P.: Das \glqq strukturelle Defizit\grqq{} in der österreichischen Budgetpolitik: Berechnungsprobleme, Revisionen und wirtschaftspolitische Relevanz, in Wirtschaft und Gesellschaft, Vol. 42, Nr. 3, Seite(n) 451-464, 2016
	 \item Gräbner C., Heinrich T., Schwardt H.: Policy Implications of Recent Advances in Evolutionary and Institutional Economics, Routledge, London, New York, 2016
	 \item Gräbner C., Heinrich T., Schwardt H.: Introduction, in Gräbner, Claudius; Heinrich, Torsten; Schwardt, Henning: Policy Implications of Recent Advances in Evolutionary and Institutional Economics, Routledge, London, New York, Seite(n) xxi-xxx, 2016
	 \item Gräbner C.: Agent-based computational models - a formal heuristic for institutionalist pattern modelling?, in Journal of Institutional Economics, Vol. 12, Nr. 1, Seite(n) 241-261, 2016
	 \item Grimm C., Kapeller J.: Wahrheit und Ökonomie, in Kurswechsel, Nr. 1, Seite(n) 18-29, 2016
	 \item Grimm C.: Postdemokratie, Machtverhältnisse und Ökonomie: Momentum Kongress Paper, Seite(n) 1-22, 2016
	 \item Griesser M., Brand U.: Verankerung wohlstandorientierter Politik. Working Paper der Kammer für Arbeiter und Angestellte für Wien, Reihe „Materialien zu Wirtschaft und Gesellschaft“, Nr. 165, 2016
	 \item Graupe S., Ötsch W.: Dialog nicht erwünscht, in Forschung \& Lehre, Vol. 23, Nr. 11, Seite(n) 1000, 2016
	 \item Eckerstorfer P., Halak J., Kapeller J., Schütz B., Springholz F., Wildauer R.: Correcting for the Missing Rich: An Application to Wealth Survey Data, in Review of Income and Wealth, Vol. 62, Nr. 4, Seite(n) 605-627, 2016
	 \item Derks L., Ötsch W., Walker W.: Relationships are Constructed from Generalized Unconscious Social Images Kept in Steady Locations in Mental Space, in Journal of Experiential Psychotherapy, Vol. 19, Nr. 1, Seite(n) 3-16, 2016
	 \item Beyer K., Bräutigam L.: Das europäische Schattenbankensystem: Bestandsaufnahme und gegenwärtige Entwicklungen, 2016
	 \item Beyer K., Bräutigam L.: Das europäische Schattenbankensystem Typologisierung und die Bewertung regulatorischer Initiativen auf europäischer Ebene, 2016
	 \item Beyer K., Bräutigam L.: Das europäische Schattenbankensystem – Typologisierung und die Bewertung regulatorischer Initiativen auf europäischer Ebene, Serie Materialien zu Wirtschaft und Gesellschaft. Working Paper-Reihe der AK Wien, Nr. 154, Arbeiterkammer Wien, 2016
	 \item Aistleitner M.: Perspektiven für eine nachhaltige Automobilindustrie: Momentum Kongress Paper, Seite(n) 1-27, 2016
\end{enumerate}
\subsection*{2015}
\begin{enumerate}
    	 \item Ötsch W., Hirte K., Pühringer S., Bräutigam L.: Markt! Welcher Markt? Der interdisziplinäre Diskurs um Märkte und Marktwirtschaft, Metropolis, Marburg, 2015
	 \item Ötsch W.: Schönheit und Macht. Drei Beispiele aus der Kulturgeschichte, in Ridler, Gerda: Mythos Schönheit. Facetten des Schönen in Natur, Kunst und Gesellschaft, Hatje Cantz Verlag, Stuttgart, Seite(n) 259-263, 2015
	 \item Ötsch W.: Ökonomie und Moral. Eine kurze Theoriegeschichte, in Seckauer, Hansjörg; Stelzer-Orthofer, Christine; Kepplinger, Brigitte: Das Vorgefundene und das Mögliche. Beiträge zur Gesellschafts- und Sozialpolitik zwischen Ökonomie und Moral, Mandelbaum Verlag, Wien, Seite(n) 100-110, 2015
	 \item Ötsch W.: Markt und Markttheorie. Vorwort und Überblick, in Ötsch, Walter O.; Hirte, Katrin; Pühringer, Stephan; Bräutigam, Lars: Markt! Welcher Markt? Der interdisziplinäre Diskurs um Märkte und Marktwirtschaft, Metropolis Verlag, Marburg, Seite(n) 7-24, 2015
	 \item Pühringer S., Hirte K.: The financial crisis as a heart attack: Discourse profiles of economists in the financial crisis, in Journal of Language and Politics, Vol. 14, Nr. 4, Seite(n) 599-626, 2015
	 \item Pühringer S.: The strange non-crisis of economics. Economic crisis and the crisis policies in economic and political discourses., Universität Linz, 2015
	 \item Pühringer S.: Marktmetaphoriken in Krisennarrativen von Angela Merkel., in Ötsch, Walter/Hirte, Katrin/Pühringer, Stephan/Bräutigam, Lars: Markt! Welcher Markt? Der interdisziplinäre Diskurs um Märkte und Marktwirtschaft., Metropolis, Marburg, Seite(n) 229-252, 2015
	 \item Pühringer S.: Kontinuitäten neoliberaler Wirtschaftspolitik. Die Austeritätsdebatte als Spiegelbild diskursiver Machtverwerfungen innerhalb der Ökonomik, in Marterbauer, Markus/Mesch, Michael/Rehm, Miriam/Reiterlechner, Christine: Das Scheitern des neoklassischen Paradigmas – Wirtschaftspolitik in der EU, ÖGB Verlag, Wien, Seite(n) 159-174, 2015
	 \item Pühringer S.: “Harte” Sanktionen für “budgetpolitische Sünder”. Kritische Diskursanalyse der Debatte zum Fiskalpakt in meinungsbildenden österreichischen Qualitätsmedien., in Momentum Quarterly, Vol. 4, Nr. 1, Seite(n) 23-41, 2015
	 \item Pühringer S.: Markets as “ultimate judges” of economic policies - Angela Merkel´s discourse profile during the economic crisis and the European crisis policies., in On the Horizon, Vol. 23, Nr. 3, Seite(n) 246-259, 2015
	 \item Kapeller J., Schütz B., Tamesberger D.: Von freien zu zivilisierten Märkten. Ein New Deal für die europäische Handelspolitik, 2015
	 \item Kapeller J., Schütz B., Tamesberger D.: Moralität, Wettbewerb und internationaler Handel: Eine europäische Perspektive, in Hansjörg Seckauer, Christine Stelzer-Orthofer, Brigitte Kepplinger: Das Vorgefundene und das Mögliche. Beiträge zur Gesellschafts- und Sozialpolitik zwischen Ökonomie und Moral, Mandelbaum Verlag, Wien, Seite(n) 213-227, 2015
	 \item Kapeller J., Schütz B., Tamesberger D.: Moralität, Wettbewerb und internationaler Handel: Eine europäische Perspektive, in Seckauer, Hansjörg; Stelzer-Orthofer, Christine; Kepplinger, Brigitte: Das Vorgefundene und das Mögliche. Beiträge zur Gesellschafts- und Sozialpolitik zwischen Ökonomie und Moral. Festschrift für Josef Weidenholzer, Mandelbaum Verlag, Wien, Seite(n) 213-227, 2015
	 \item Kapeller J., Schütz B.: Verteilungstendenzen im Kapitalismus: Globale Perspektiven, 2015
	 \item Kapeller J., Schütz B.: Verteilungstendenzen im Kapitalismus: Nationale und Globale Perspektiven, 2015
	 \item Kapeller J., Schütz B.: Verteilungstendenzen im Kapitalismus: Momentum Kongress Paper, Seite(n) 1-21, 2015
	 \item Kapeller J., Schütz B.: Verteilungstendenzen im Kapitalismus - Nationale und globale Perspektiven, in Kurswechsel, Nr. 2/2015, Seite(n) 54-68, 2015
	 \item Kapeller J., Schütz B.: Conspicuous Consumption, Inequality and Debt: The Nature of Consumption-driven Profit-led Regimes, in Metroeconomica, Vol. 66, Nr. 1, Seite(n) 51-70, 2015
	 \item Kapeller J., Pühringer S.: Demokratie in Liberalismus und Neoliberalismus, in Seckauer, Hansjörg/Stelzer-Orthofer, Christine/Kepplinger, Brigitte: Das Vorgefundene und das Mögliche. Beiträge zur Gesellschafts- und Sozialpolitik zwischen Ökonomie und Moral, Mandelbaum, Wien, Seite(n) 111-127, 2015
	 \item Kapeller J., Gräbner-Radkowitsch C.: The Micro‐Macro Link in Heterodox Economics, 2015
	 \item Kapeller J.: Wirtschaftspolitik, Verteilungsgerechtigkeit und Demokratie, in AK Burgenland: Gerechtigkeit muss sein, Seite(n) 150-165, 2015
	 \item Kapeller J.: Beyond Foundations: Systemism in Economic  Thinking, in Jo, Tae-Hee / Todorovka, Zdravka: Advancing the Frontiers of Heterodox Economics: Essays in Honor of Frederic S. Lee, Routledge, London, Seite(n) 115-134, 2015
	 \item Kapeller J.: Allgemeine Modelltheorie und ökonomische Modelle, in EWE - Erwägen, Wissen, Ethik, Vol. 26, Nr. 3, Seite(n) 387-389, 2015
	 \item Hirte K., Pühringer S., Bräutigam L.: Markt! Welcher Markt? Der interdisziplinäre Diskurs um Märkte und Marktwirtschaft, Metropolis Verlag, Marburg, 2015
	 \item Hirte K.: Politik und ihre Ad-hoc- Gremien in Krisenzeiten, in Momentum-Kongress: Momentum Kongress Paper, Seite(n) 1-22, 2015
	 \item Hirte K.: Märkte und die Anerkennung von Arbeit. Zum Zusammenhang schlecht bezahlter Arbeiten und der Struktur der Arbeitsergebnisse, in Ötsch, Walter; Hirte, Katrin; Pühringer, Stephan; Bräutigam, Lars: Markt! Welcher Markt?, Serie Kritische Studien zu Markt und Gesellschaft, Metropolis, Marburg, Seite(n) 281-322, 2015
	 \item Hirte K.: Das Ökonomie-Monopol an den Agrarfakultäten, in Wege für eine bäuerliche Zukunft – Zeitschrift der ÖBV/ Via Campesina Austria, Vol. 38, Nr. 3 (338), Seite(n) 22-23, 2015
	 \item Hirte K.: Bezeichnende Konstellation. Zum Eröffnungstag „Agrarpolitik“ in Österreich auf dem Podium: REWE, RWA Raiffeisen und Landwirtschaftskammer, in Via Campesina Austria, in Wege für eine bäuerliche Zukunft – Zeitschrift der ÖBV/ Via Campesina Austria, Vol. 38, Nr. 2 (337), Seite(n) 15, 2015
	 \item Heise A., Hirte K., Ötsch W., Pühringer S., Reichl A., Sander H., Thieme S.: ÖkonomInnen und Ökonomie. Eine wissenschaftssoziologische Entwicklungsanalyse zum Verhältnis von ÖkonomInnen und Ökonomie im deutschsprachigen Raum ab 1945, Hans-Böckler-Stiftung, Düsseldorf, 2015
	 \item Heimberger P.: Raus aus dem Euro?, in Wirtschaft und Gesellschaft, Vol. 41, Nr. 4, Seite(n) 603-614, 2015
	 \item Heimberger P.: Eine fiskalpolitische Lösung für die Eurozone, in Kammer für Arbeiter und Angestellte Wien, in Wirtschaft und Gesellschaft, Vol. 41, Nr. 3, Lexis Nexis, Seite(n) 449-458, 2015
	 \item Heimberger P.: Griechenland: Das Scheitern der europäischen Krisenpolitik, in Arbeiterkammer Wien, in EU-Infobrief, Nr. 3, Seite(n) 11-15, 2015
	 \item Heimberger P.: Die griechische Schuldendebatte und das Mantra von den \glqq notwendigen Strukturreformen\grqq{}, in WISO direkt, Nr. 05, 2015
	 \item Heimberger P.: 'Strukturreformen' und Lohnkürzungen in Griechenland: Erwartungen, Ergebnisse und Folgen, in ISW, in WISO - Wirtschafts- und sozialpolitische Zeitschrift, Vol. 38, Nr. 3, Seite(n) 104-121, 2015
	 \item Gräbner C., Kapeller J.: New Perspectives on  Institutionalist Pattern Modeling: Systemism, Complexity and  Agent-Based modeling, in Journal of Economic Issues, Vol. 49, Nr. 2, Seite(n) 433-440, 2015
	 \item Grimm C.: Wirtschaftspolitische Ausrichtung österreichischer Parteien im historischen Verlauf. Die Ausgestaltung österreichischer Parteiprogrammatiken unter dem Einfluss neoliberalen Gedankenguts, 2015
	 \item Griesser M., Sauer B.: Zwischen Konjunkturpuffer und Tauschobjekt. Gewerkschaftliche Perspektiven auf Migration im Österreich der Zweiten Republik, in Kurswechsel, Nr. Heft 4, Seite(n) 58-66, 2015
	 \item Griesser M., Hirte K., Pühringer S.: ÖkonomInnen und Politik – Analyse zur politischen Einflussnahme deutschsprachiger ÖkonomInnen, Forschungsbericht: Förderer: Jubiläumsfonds, Universität Linz, 2015
	 \item Griesser M.: Rezension von „Martina Benz: Zwischen Migration und Arbeit. Worker Centers und die Organisierung prekär und informell Beschäftigter in den USA“, in Bund demokratischer Wissenschaftlerinnen und Wissenschaftler, in Forum Wissenschaft, BdWi-Verlag, 2015
	 \item Griesser M.: Der Staat als Wissensapparat. Konzeptionelle Überlegungen zu einer nicht-funktionalistischen Funktionsanalyse des Sozialstaats, in Zeitschrift für Sozialreform, Vol. 61, Seite(n) 103-124, 2015
	 \item Beyer K.: Nachfrageseitige Ursachen der Expansion des Schattenbankensystems, 2015
	 \item Aistleitner M., Fölker M., Kapeller J., Mohr F., Pühringer S.: Verteilung und Gerechtigkeit: Philosophische  Perspektiven, in Wirtschaft und Gesellschaft, Vol. 41, Nr. 1, Seite(n) 71-106, 2015
	 \item Aistleitner M., Fölker M., Kapeller J.: Die Macht der Wissenschaftsstatistik und die Entwicklung der Ökonomie, 2015
	 \item Aistleitner M., Fölker M., Kapeller J.: Die Macht der Wissenschaftsstatistik und die Entwicklung der Ökonomie: Momentum Kongress Paper, Seite(n) 1-21, 2015
	 \item Aistleitner M., Fölker M., Kapeller J.: Die Macht der Wissenschaftsstatistik und die Entwicklung der Ökonomie, in Journal of Contextual Economics – Schmollers Jahrbuch, Vol. 135, Nr. 2, Seite(n) 111–132, 2015
\end{enumerate}
\subsection*{2014}
\begin{enumerate}
    	 \item Ötsch W., Schmidt M.: The Political Economy of Offshore Jurisdictions. An Introduction, in Ötsch, Walter O.; Grözinger, Gerd; Beyer, Karl M.; Bräutigam, Lars: The Political Economy of Offshore Jurisdictions, Metropolis Verlag, Marburg, Seite(n) 7-23, 2014
	 \item Ötsch W., Grözinger G., Beyer K., Bräutigam L.: The Political Economy of Offshore Jurisdictions, Metropolis Verlag, Marburg, 2014
	 \item Ötsch W.: Populismus und Demagogie. Mit Beispielen von Jörg Haider, Hans-­‐Christian Strache und Frank Stronach sowie der Tea Party”., in Gressl, Martin, Klemenjak, Martin, Klepp. Cornelia, Pichler, Heinz, Rottermann, Doris und Scherling, Josefine: Populismus und Rassismus im Vormarsch?, Schriftenreihe „Arbeit und Bildung“, Klagenfurt, Seite(n) 12-26, 2014
	 \item Ötsch W.: How to Hide Secrecy Jurisdictions, in Ötsch, Walter O.; Grözinger, Gerd; Beyer, Karl M.; Bräutigam, Lars: The Political Economy of Offshore Jurisdictions, : Metropolis Verlag, : Metropolis Verlagrburg, Seite(n) 61-75, 2014
	 \item Stelzer-Orthofer C., Pühringer S.: Subventionierung von Lohnkosten als Mittel zur Armutsvermeidung, in Dimmel, N./Schenk, M./Stelzer-Orthofer, C.: Handbuch Armut in Österreich, Studienverlag, Innbruck, Seite(n) 817-831, 2014
	 \item Pühringer S.: Ökonomische Krisen als Krankheiten und Katastrophen?, 2014
	 \item Pühringer S.: Mythen über Reichtum und Macht: Demokratie ist nicht käuflich, in Beigewum/ATTAC/Armutskonferenz: Mythen des Reichtums, VSA Verlag, Hamburg, Seite(n) 149-158, 2014
	 \item Kapeller J., Steinerberger S.: Modeling the Evolution of Preferences: An Answer to Schubert and Cordes, in Journal of Institutional Economics, Vol. 10, Nr. 2, Seite(n) 337-347, 2014
	 \item Kapeller J., Schütz B., Tamesberger D.: From Free to Civilized Markets: Momentum Kongress Paper, Seite(n) 1-29, 2014
	 \item Kapeller J., Schütz B., Tamesberger D.: Making Morality Matter: Civilized Markets and European Values., in Journal for a Progressive Economy, 2014
	 \item Kapeller J., Schütz B., Tamesberger D.: From Free to Civilized Markets: First steps towards Eutopia, in Semantic Scholar, 2014
	 \item Kapeller J., Schütz B.: Debt, Boom, Bust: A Theory of Minsky-Veblen Cycles, in Journal of Post Keynesian Economics, Vol. 36, Nr. 4, 2014
	 \item Kapeller J., Hubmann G.: Fortschrittsidee und Politische Vision [Progress and Politics], in Momentum Quarterly, Vol. 3, Nr. 4, Seite(n) 235-245, 2014
	 \item Kapeller J., Grimm C., Springholz F.: Führt Pluralismus in der ökonomischen Theorie zu mehr Wahrheit?, in Hirte, Katrin, Thieme, Sebastian, Ötsch, Walter: Wissen! Welches Wissen?, Metropolis, Marburg, Seite(n) 147-163, 2014
	 \item Kapeller J.: The return of the rentier. Review of: Piketty, Thomas (2014): Capital in the 21st century. Harvard University Press, 685 pages, 2014
	 \item Kapeller J.: Die Rückkehr des Rentiers. Rezension zu: Piketty, Thomas (2014): Capital in the 21st century. Cambridge: Harvard University Press, in Wirtschaft und Gesellschaft, Vol. 40, Nr. 2, Seite(n) 329-346, 2014
	 \item Kapeller J.: Economic Change and Change in Economics, Universität Linz, 2014
	 \item Hirte K., Ötsch W.: Vorwort im Sammelband \glqq Wissen! Welches Wissen?\grqq{}, in Hirte Katrin, Thieme Sebastian, Ötsch Walter Otto: Wissen! Welches Wissen? Zu Wahrheit, Theorien und Glauben sowie ökonomischen Theorien, Metropolis Verlag, Marburg, Seite(n) 7-16, 2014
	 \item Hirte K., Thieme S., Ötsch W.: Wissen! Welches Wissen? Zu Wahrheit, Theorien und Glauben sowie ökonomischen Theorien, Metropolis Verlag, Marburg, 2014
	 \item Hirte K., Thieme S., Ötsch W.: Wissen! Welches Wissen? Zu Wahrheit, Theorien und Glauben sowie ökonomischen Theorien, Metropolis Verlag, Marburg, 2014
	 \item Hirte K., Thieme S., Ötsch W.: Vorwort Wissen! Welches Wissen? Zu Wahrheit, Theorien und Glauben sowie ökonomischen Theorien: Wissen! Welches Wissen? Zu Wahrheit, Theorien und Glauben sowie ökonomischen Theorien, Metropolis Verlag, Marburg, 2014
	 \item Hirte K., Pühringer S.: Performative Wissenschaft: Ökonomiekritik, Ökonomietheorien und die Verantwortung von ÖkonomInnen., in Hirte Katrin, Thieme Sebastian, Ötsch Walter Otto: Wissen! Welches Wissen? Zu Wahrheit, Theorien und Glauben sowie ökonomischen Theorien, Metropolis Verlag, Marburg, Seite(n) 267-302, 2014
	 \item Hirte K., Pühringer S.: ÖkonomInnen und Ökonomie in der Krise? Eine diskurs- und netzwerkanalytische Sicht., in WISO - Wirtschafts- und sozialpolitische Zeitschrift, Vol. 1, Seite(n) 159-178, 2014
	 \item Hirte K.: Agrargiganten im Osten. Zur neuerlichen Transformation der transformierten deutschen Agrarstrukturen., in Brähler, Elmar; Wagner, Wolf: 25 Jahre Mauerfall – kein Ende mit der Wende?, Psychosozial-Verlag, Gießen, Seite(n) 277-­290, 2014
	 \item Hirte K.: Landwirtschaft, Ideologien und \glqq ...ismen\grqq{}, in Via Campesina Austria, in Wege für eine bäuerliche Zukunft – Zeitschrift der ÖBV/ Via Campesina Austria, Vol. 37, Nr. 2 (332), Seite(n) 14-15, 2014
	 \item Grimm C., Kapeller J., Springholz F.: Führt Pluralismus in der ökonomischen Theorie zu mehr Wahrheit?: Wissen! Welches Wissen? Zu Wahrheit, Theorien und Glauben sowie ökonomischen Theorien, Metropolis Verlag, Marburg, 2014
	 \item Griesser M., Sauer B.: MigrantInnen als Zielgruppe. Solidarische Beratungs- und Unterstützungsangebote von ArbeitnehmerInnenorganisationen in Österreich, Serie Studie - Abschlussbericht, Universität, Institut für Politikwissenschaften, Wien, 2014
	 \item Eckerstorfer P., Halak J., Kapeller J., Schütz B., Springholz F., Wildauer R.: Die Vermögensverteilung in Österreich und das Aufkommenspotenzial einer Vermögenssteuer, in Wirtschaft und Gesellschaft, Vol. 40, Nr. 1, Seite(n) 63-81, 2014
	 \item Eckerstorfer P., Halak H., Kapeller J., Schütz B., Springholz F., Wildauer R.: Vermögen in Österreich, Serie Materialien zu Wirtschaft und Gesellschaft, Nr. 126, AK Wien, WIen, 2014
	 \item Beyer K., Bräutigam L.: Offshore Aspects of Shadow Banking. With Considerations on the Recent Financial Crisis, in Ötsch, Walter O.; Grözinger, Gert; Beyer, Karl M.: The Political Economy of Offshore Jurisdictions, Metropolis Verlag, Marburg, 2014
	 \item Beyer K.: Die Risiken im Schatten des Systems, 2014
	 \item Beyer K.: Emanzipation bei Marx und seine Kritik an Proudhon und dessen ideengeschichtlichen Nachfahren: Momentum Kongress Paper, Seite(n) 1-18, 2014
\end{enumerate}
\subsection*{2013}
\begin{enumerate}
    	 \item Ötsch W., Beyer K., Mader L.: Die Finanzkrise 2007-2009 als Krise von Schattenbanken. Eine einführende institutionelle Analyse: Jahrestagung des Ausschusses Evolutorische Ökonomik im Verein für Sozialpolitik, Delft, 2013
	 \item Ötsch W.: Marktradikalität. Der Diskurs von „dem Markt“, in Günther, Lea-Simone; Hertlein, Saskia;  Klüsener, Vea und Raasch, Markus: Radikalität. Religiöse, politische und künstlerische Radikalismen in Geschichte und Gegenwart.  Band 2: Frühe Neuzeit und Moderne, Königshausen \& Neumann, Würzburg, Seite(n) 254-279, 2013
	 \item Ötsch W.: Die Macht der Ratingagenturen: Governance in der Ideologie 'des Marktes', in Brodbeck, Karl-Heinz: Alternative Länder-Ratings, Schriftenreihe der Finance \& Ethics Academy, Band 5, Shaker Verlag, Aachen, Seite(n) 58-98, 2013
	 \item Ötsch W.: The Deep Meening of ‘Market’: Understanding Neoliberal-Market-Radical Reasoning, in Human Geography, Vol. 6, Nr. 2, Seite(n) 11-25, 2013
	 \item Thieme S., Hirte K.: Mainstream, Orthodoxie und Heterodoxie – Zur Klassifizierung der Wirtschaftswissenschaften, 2013
	 \item Schütz B.: Marx, Keynes und die Idee des gesellschaftlichen Fortschritts: Suche nach neuen politischen Visionen: Momentum Kongress Paper, Seite(n) 1-21, 2013
	 \item Pühringer S.: Ahnungslos, aber nicht tatenlos – Wie ÖkonomInnen seit der Finanzkrise Politik mach(t)en, 2013
	 \item Pühringer S.: „Arbeitsmarktferne“ Personen – wer sind die? Zu veränderten Exklusionsdynamiken in neokapitalistischen Gesellschaften, in SWS-Rundschau - Sozialwissenschaftliche Studiengesellschaft Rundschau, Vol. 53, Nr. 4, Seite(n) 361-381, 2013
	 \item Plaimer W., Pühringer S.: Der Fiskalpakt und seine Implementation in Österreich: Momentum Kongress Paper, Seite(n) 1-20, 2013
	 \item Nordmann J.: Grenzen aktueller Krisendebatten Über Konstruktionen der öffentlichen Meinung und das Verhältnis von Sach‐ und Grundsatzdiskussionen in (neo)liberalen Demokratien, 2013
	 \item Nordmann J.: Grenzen aktueller Krisendebatten. Über Konstruktionen der öffentlichen Meinung und das Verhältnis von Sach- und Grundsatzdiskussionen in (neo)liberalen Demokratien, in Wengeler Martin; Ziem, Alexander: Sprachliche Konstruktionen von Krisen : interdisziplinäre Perspektiven auf ein fortwährend aktuelles Phänomen, Hempen, Bremen, Seite(n) 53-66, 2013
	 \item Kapeller J., Wolkenstein F.: The grounds of solidarity: From liberty to loyalty, in European Journal of Social Theory, Vol. 16, Nr. 4, Seite(n) 476-491, 2013
	 \item Kapeller J., Steinerberger S.: How Formalism shapes Perception: An Experiment on Mathematics as a Language, in International Journal of Pluralism and Economics Education, Vol. 4, Nr. 2, Seite(n) 138-156, 2013
	 \item Kapeller J., Schütz B., Tamesberger D.: Die Regulation der Routine: Über die regulatorischen Spielräume zur Etablierung nachhaltigen Konsums, in Wirtschaft und Gesellschaft, Vol. 39, Nr. 2, Seite(n) 207-231, 2013
	 \item Kapeller J., Schütz B., Steinerberger S.: The impossibility of rational consumer choice - A problem and its solution, in Journal of Evolutionary Economics, Vol. 23, Nr. 1, Seite(n) 39-60, 2013
	 \item Kapeller J., Schütz B.: Exploring Pluralist Economics: The Case of the Minsky-Veblen Cycles, in Journal of Economic Issues, Vol. 47, Nr. 2, Seite(n) 515-524, 2013
	 \item Kapeller J.: Model-Platonism in Economics: On a classical epistemological critique, in Journal of Institutional Economics, Vol. 9, Nr. 2, Seite(n) 199-221, 2013
	 \item Hirte K.: ÖkonomInnen in der Finanzkrise. Diskurse. Netzwerke. Initiativen., Metropolis Verlag, Marburg, 2013
	 \item Hirte K.: „Persilschein“ – Netzwerke: Für Bruchlosigkeit in Umbruchzeiten., in Schönhuth, Michael; Gamper, Markus; Kronenwett, Michael; Stark, Martin: Visuelle Netzwerkforschung. Qualitative, quantitative und partizipative Zugänge., Transcript Verlag, Seite(n) 331-353, 2013
	 \item Hirte K.: Deutsche Europapolitik vor und nach 1945, in Wege für eine bäuerliche Zukunft – Zeitschrift der ÖBV/ Via Campesina Austria, Vol. 36, Nr. 3 (328), Seite(n) 22-23, 2013
	 \item Hirte K.: Deutsche Agrarpolitikprofessoren vor und nach 1945, in Via Campesina Austria, in Wege für eine bäuerliche Zukunft – Zeitschrift der ÖBV/ Via Campesina Austria, Vol. 36, Nr. 2 (327), Seite(n) 18-20, 2013
	 \item Eckerstorfer P., Halak J., Kapeller J., Schütz B., Springholz F., Wildauer R.: Reichtumsverteilung in Österreich, in WISO - Wirtschafts- und sozialpolitische Zeitschrift, Vol. 36, Nr. 4, 2013
	 \item Eckerstorfer P., Halak H., Kapeller J., Schütz B., Springholz F., Wildauer R.: Bestände und Verteilung der Vermögen in Österreich, Serie Materialien zu Wirtschaft und Gesellschaft, Vol. 122, Abteilung Wirtschaftswissenschaft und Statstik der Kammer für Arbeiter und Angestellte, Wien, 2013
	 \item Dobusch L., Kapeller J.: Diskutieren statt Ignorieren: Eckpfeiler für interessierten Pluralismus in der Ökonomie, in Der öffentliche Sektor - The Public Sector, Vol. 39, Nr. 3, 2013
	 \item Dobusch L., Kapeller J.: Breaking New Paths: Theory and Method in Path Dependence Research, in Schmalenbach Business Review, Vol. 65, Nr. 2, Seite(n) 288-311, 2013
	 \item Dobusch L., Kapeller J.: Practicing Pluralism: A Rejoinder to W. Robert Brazelton, in Journal of Economic Issues, Vol. 47, Nr. 4, Seite(n) 1035-1037, 2013
\end{enumerate}
\subsection*{2012}
\begin{enumerate}
    	 \item Ötsch W.: Staatsschuldenkrise und ökonomisches Denken – im Euroraum und in Zentraleuropa, in Horvath, Patrick, Skarke, Herbert  und Weinzierl, Rupert: Die “Vision Zentraleuropa” im 21. Jahrhundert. Festschrift zum 90. Geburtstag von Heinz Kienzl, Arbeitsgemeinschaft für wissenschaftliche Wirtschaftspolitik (WIWIPOL), Wien, Seite(n) 68-71, 2012
	 \item Ötsch W.: Politische Ökonomie und Gesellschaft. Eine theoriegeschichtliche Skizze, in Grözinger, Gerd; Reich, Utz-Peter: Entfremdung – Ausbeutung – Revolte, Metropolis Verlag, Marburg, Seite(n) 145-165, 2012
	 \item Ötsch W.: Krise des Euroraums, in Starke, Herbert; Horvath, Patrick; Weinzierl, Rupert: Die Vision Zentraleuropa im 21. Jahrhundert, Arbeitsgemeinschaft für wissenschaftliche Wirtschaftspolitik, Wien, Seite(n) 68-71, 2012
	 \item Ötsch W.: Der freie Markt, in Czejkowska, Agnieszka: Imagine Economy. Neoliberale Metaphern im wirtschaftlichen Diskurs, Erhard Löcker Verlag, Wien, Seite(n) 39-45, 2012
	 \item Pühringer S., Kapeller J.: How Liberalism lost its concept of democracy: Momentum Kongress Paper, Seite(n) 1-22, 2012
	 \item Pühringer S., Hirte K.: Economists and Economics. Discourse profiles of economists in the financial crisis, in Association française d'économie politique: Joint conference of AHE, IIPPE, FAPE. Kongress Political economy and the outlook for capitalism 05.-07.07.2012, Paris, Seite(n) 1-16, 2012
	 \item Pühringer S.: Öffentlicher Vernunftgebrauch - ein probantes Mittel zur Bekämpfung von Ungerechtigkeit?, in Studentisches Soziologiemagazin, 2012
	 \item Pühringer S.: Soziale Frage im Wandel. Probleme und Perspektiven des Sozialstaates und der Arbeitsgesellschaft, in Kontraste - Presse- und Informationsdienst für Sozialpolitik, Nr. 2/2012, Seite(n) 22-23, 2012
	 \item Plaimer W.: Postdemokratie in Österreich?, in Nordmann, Jürgen; Hirte, Katrin; Ötsch, Walter O.: Demokratie! Welche Demokratie? Postdemokratie kritisch hinterfragt, Metropolis Verlag, Marburg, Seite(n) 159-174, 2012
	 \item Plaimer W.: Postdemokratie in Österreich?, in Momentum-Kongress: Momentum Kongress Paper, Seite(n) 1-17, 2012
	 \item Nordmann J., Hirte K., Ötsch W.: Demokratie! Welche Demokratie? Postdemokratie kritisch hinterfragt, Metropolis Verlag, Marburg, 2012
	 \item Nordmann J.: Vorwort in \glqq Demokratie! Welche Demokratie? Postdemokratie kritisch hinterfragt\grqq{}: Demokratie! Welche Demokratie? Postdemokratie kritisch hinterfragt, Seite(n) 7-14, 2012
	 \item Nordmann J.: Die intellektuelle Geschichte des Neoliberalismus im Spiegel des alten Liberalismus, in ksoe-Dossier der katholischen Erwachsenenbildung Österreich, 2012
	 \item Kapeller J., Schütz B., Tamesberger D.: Die Regulation der Routine: Momentum Kongress Paper, Seite(n) 1-31, 2012
	 \item Kapeller J., Schütz B., Tamesberger D.: Konsum demokratisch gestalten: Spielräume zur Etablierung nachhaltigen Konsums., in WISO - Wirtschafts- und sozialpolitische Zeitschrift, Seite(n) 167-183, 2012
	 \item Kapeller J., Schütz B.: Conspicuous consumption, inequality and debt: The nature of consumption-driven profit-led regimes, 2012
	 \item Kapeller J., Hubmann G.: Solidarisch Handeln: Konzeptionen, Ursachen und Implikationen, in Momentum Quarterly, Vol. 1, Nr. 3, Seite(n) 139-152, 2012
	 \item Kapeller J., Dobusch L.: A guide to paradigmatic Self-marginalization - Lessons for Post-Keynesian Economists, in Lavoie M. und Lee F.S.: In Defense of Post-Keynesian and Heterodox Economics., Routledge, London, Seite(n) 62-86, 2012
	 \item Huber J., Kaindlstorfer L., Kapeller J.: Bridges to Past Polls: Die oberösterreichische Erfahrung mit Vorwahlen als demokratisches Instrument: Momentum Kongress Paper, 2012
	 \item Hirte K., Pühringer S.: ÖkonomInnen in der Finanzkrise. Analyse zur Positionierung deutschsprachiger Ökonomen im Kontext ihrer strukturellen Verankerung, 2012
	 \item Hirte K.: Die Umstrukturierung der LPGen in Thüringen ab 1990., in Landeszentrale für politische Bildung Thüringen., Druckerei Sömmerda GmbH, Erfurt, 2012
	 \item Hirte K.: Die ersten Professoren für Agrarpolitik und Agrarökonomie ab 1945 an den westdeutschen Universitäten und ihre Vergangenheit., in Buchsteiner Martin, Strahl Antje: Thünen-Jahrbuch, Nr. 7, Seite(n) 87-114, 2012
	 \item Hirte K.: Würdigungs-Netzwerke, gewolltes Nichtwissen und Geschichtsschreibung., in Österreichische Zeitschrift für Geschichtswissenschaften, Vol. 23, Nr. 1, Studienverlag Innsbruck, Seite(n) 155-185, 2012
	 \item Dobusch L., Kapeller J.: Regulatorische Unsicherheit und Private Standardisierung: Koordination durch Ambiguität: Steuerung durch Regeln, Serie Managementforschung, Vol. 22, Seite(n) 43-81, 2012
	 \item Dobusch L., Kapeller J.: A Guide to Paradigmatic Self-Marginalization - Lessons for Post-Keynesian Economists, in Review of Political Economy, Seite(n) 469-487, 2012
	 \item Dobusch L., Kapeller J.: Heterodox United vs. Mainstream City? Sketching a framework for interested pluralism in economics, in Journal of Economic Issues, Vol. 46, Nr. 4, Seite(n) 1035-1057, 2012
	 \item Beyer K.: Illusionen eines Zirkulationskünstlers? Pierre-Joseph Proudhon auf dem ökonomiekritischen Prüfstand, Universität Wien, 2012
\end{enumerate}
\subsection*{2011}
\begin{enumerate}
    	 \item Ötsch W., Hirte K., Nordmann J.: Gesellschaft! Welche Gesellschaft? Nachdenken über eine sich wandelnde Gesellschaft, Metropolis-Verlag, Marburg, 2011
	 \item Ötsch W.: Kurt Rothschild als politischer Ökonom, in Bartel, Rainer; Bürger, Hans; Klug, Friedrich: In Memoriam Univ. Prof. Kurt W. Rothschild, Serie Schriftenreihe des Instituts für Kommunalwissenschaft (IKW), Nr. Band 122, Institut für Kommunalwirtschaften, Wien, Seite(n) 73-84, 2011
	 \item Ötsch W.: Der Markt und die großen Ratingagenturen: Momentum Kongress Paper, Serie Momentum quarterly, Seite(n) 1-11, 2011
	 \item Ötsch W.: Markt. Sichtweisen auf die Wirtschaft, in Praxis Politik, Nr. 2/2011, Seite(n) 4-8, 2011
	 \item Pühringer S.: Aktivierung und Mindestsicherung. Rezension zum gleichnamigen Buch von Christine Stelzer-Orthofer und Josef Weidenholzer, in WISO - Wirtschafts- und sozialpolitische Zeitschrift, Vol. 34, Nr. 02, Seite(n) 169-171, 2011
	 \item Pühringer S.: Frei handeln? Liberales und neoliberales Freiheitskonzept und ihre Auswirkungen auf die Verteilung von Macht und Eigentum, Peter Lang, Frankfurt am Main, 2011
	 \item Pühringer S.: Gleichheit versus Vielfalt. Ein konstruierter Widerspruch?: Momentum Kongress Paper, Seite(n) 1-23, 2011
	 \item Plaimer W., Nordmann J.: Veränderung von Machtverhältnissen in politischen Entscheidungsprozessen, in Momentum-Kongress: Momentum Kongress Paper, Seite(n) 1-21, 2011
	 \item Nordmann J.: Braucht die aktuelle Gesellschaft einen Gesellschaftsvertrag? Der politische Neoliberalismus im Spiegel von John Locke und John Rawls: Gesellschaft! Welche Gesellschaft? Nachdenken über eine sich wandelnde Gesellschaft, Metropolis Verlag, Marburg, 2011
	 \item Nordmann J.: Braucht die aktuelle Gesellschaft einen Gesellschaftsvertrag? Der politische Neoliberalismus im Spiegel von John Locke und John Rawls: Gesellschaft! Welche Gesellschaft?, Metropolis-Verlag, Marburg, Seite(n) 33-60, 2011
	 \item Knierim A., Hirte K.: Aktionsforschung - ein Weg zum Design institutioneller Neuerungen zur regionalen Anpassung an den Klimawandel: Anpassung an den Klimawandel - regional umsetzen!, oecom Verlag, München, Seite(n) 156-174, 2011
	 \item Kapeller J.: Modell-Platonismus in der Ökonomie. Zur Aktualität einer klassisch-epistemologischen Kritik, Universität Linz, 2011
	 \item Kapeller J.: Modell-Platonismus in der Ökonomie: Zur Aktualität einer klassischen epistemologischen Kritik, Peter Lang, Frankfurt/Main, 2011
	 \item Kapeller J.: Was sind ökonomische Modelle?, in Volker Gadenne und Reinhard Neck: Philosophie und Wirtschaftswissenschaft, Mohr Siebeck, Seite(n) 29-50, 2011
	 \item Hirte K., Ötsch W.: Vorwort, in Ötsch, Walter O.; Hirte, Katrin; Nordmann, Jürgen: Gesellschaft! Welche Gesellschaft?, Metropolis, Marburg, Seite(n) 7-15, 2011
	 \item Hirte K., Ötsch W.: Ökonomische Ausrichtung und Netzwerke - das Beispiel des Sachverständigenrates., in Prokla, Nr. 3, Seite(n) 423-447, 2011
	 \item Hirte K.: Gleichheit und Vielfalt – normative Konzeptionen? Die philosophischen Implikationen zum Problem Anerkennung bei Simone de Beauvoir und Hannah Arendt: Momentum Kongress Paper, Seite(n) 1-15, 2011
	 \item Hirte K.: Crowdsourcing und Regelbezüge - der Fall GuttenPlag., in Heiss, Hans-Ulrich: INFORMATIK 2011 - Informatik schafft Communities. 41. Jahrestagung der Gesellschaft für Informatik , 4.-7.10.2011, Berlin, Serie Lecture Notes in Informatics (LNI), Vol. P-192, Springer, 2011
	 \item Dobusch L., Kapeller J.: Wirtschaft, Wissenschaft und Politik: Die sozialwissenschaftliche Bedingtheit linker Reformpolitik, in Prokla, Vol. 41, Nr. 3, Seite(n) 389-404, 2011
	 \item Blaha B., Kapeller J., Weidenholzer J.: Solidarität - Beiträge für eine gerechte Gesellschaft, Braumüller, Wien, 2011
\end{enumerate}
\subsection*{2010}
\begin{enumerate}
    	 \item Ötsch W., Kapeller J.: Perpetuing the failure: Economic Education and the Current Crisis, in Journal of Social Science Education, Vol. 9, Nr. 2, Seite(n) 16-25, 2010
	 \item Ötsch W., Hirte K., Nordmann J.: Die Evolution des ökonomischen Wissens und des Wissens über den Kapitalismus. Performativity als Analyseinstrument: das Beispiel der Fabian Society, der Mont Pèlerin Society und der Chicagoer Schule, 2010
	 \item Ötsch W., Hirte K., Nordmann J.: Krise! Welche Krise? Zur Problematik aktueller Krisendebatten, Metropolis, Marburg, 2010
	 \item Ötsch W.: Das Bewusstsein des Homo Oeconomicus, in Bauer, Renate: Bewusstsein und Ich, Angelika Lenz Verlag, Neu-Isenberg, Seite(n) 105–117, 2010
	 \item Ötsch W.: Die Tiefenbedeutung von ‘Markt’: Momentum Kongress Paper, Seite(n) 1-27, 2010
	 \item Pühringer M., Pühringer S.: Solidarität im Kapitalismus: Momentum Kongress Paper, Seite(n) 1-32, 2010
	 \item Nordmann J.: Was ist eine Krise?, in Ötsch, Walter O.; Hirte, Katrin; Nordmann, Jürgen: Krise! Welche Krise? Zur Problematik aktueller Krisendebatten, Metropolis Verlag, Marburg, Seite(n) 7-20, 2010
	 \item Nordmann J.: Trash, Skandale und Ratschläge statt Aufklärung und politische Bildung. Über das Zusammenspiel von kommerzialisierten Medien und gemachter Meinung in der neoliberalen Gesellschaft: Momentum Kongress Paper, Seite(n) 1-8, 2010
	 \item Nordmann J.: Protektionismus. Die Grenzen der Staatsintervention in den 1930er Jahren, in European Journal of Economics and Economic Policies: Intervention, Vol. 7, Nr. 1, Seite(n) 42-49, 2010
	 \item Kapeller J., Pühringer S.: The internal consistency of perfect competition, in Journal of Philosophical Economics, Vol. 3, Nr. 2, Seite(n) 134-152, 2010
	 \item Kapeller J., Huber J.: Neoklassische Sozialdemokratie und Sozialdemokratie am Beispiel des Hamburger Programms der SPD, in TRANS Internet-Zeitschrift für Kulturwissenschaften , Vol. 17, Nr. 2, 2010
	 \item Kapeller J.: Some critical notes on citation metrics and heterodox economics, in Review of Radical Political Economics, Vol. 42, Nr. 3, Seite(n) 330-337, 2010
	 \item Kapeller J.: Citation Metrics: Serious drawbacks, perverse incentives and strategic options for heterodox economics, in American Journal of Economics and Sociology, Vol. 69, Nr. 5, Seite(n) 1376-1408, 2010
	 \item Hubmann G., Kapeller J.: Solidarisch Handeln: Konzeptionen, Ursachen und Implikationen: Momentum Kongress Paper, Seite(n) 1-30, 2010
	 \item Hirte K.: Ökonomisierung an den Hochschulen., in Johanna Besier, Hannah Fritsch, Arne Rost, Sven Schmidt, Hannes Schulz, Maggie Selle, Katharina Wenzel: Agrarpolitik in der Lehre?, ABL Bauernblatt Verlags-GmbH, Seite(n) 17-25, 2010
	 \item Hirte K.: Performativity of Economics - ein tragfähiger Ansatz zur Analyse der Rolle der Ökonomen in der Ökonomie?, in Walter Ötsch, Katrin Hirte, Jürgen Nordmann: Krise. Welche Krise?, Metropolis Verlag, Marburg, Seite(n) 49-75, 2010
	 \item Hirte K.: Das neoklassische Freihandelsmodell: Momentum Kongress Paper, Seite(n) 1-16, 2010
	 \item Hirte K.: Die Rolle der Agrarpolitik und Agrarökonomie in agrarpolitischen Diskursverläufen, in TRANS Internet-Zeitschrift für Kulturwissenschaften , Vol. 17, Nr. 2, 2010
	 \item Dobusch L., Kapeller J.: Institutionalisierung zivilgesellschaftlicher Partizipation - Zwischen Ignoranz, Integration und Invasion, in Blaha, Barbara; Weidenholzer, Josef: Freiheit - Beiträge für eine demokratische Gesellschaft, Wilhelm Braumüller Universitäts-Verlagsbuchhandlung, Wien, Seite(n) 201-217, 2010
\end{enumerate}
\subsection*{2009}
\begin{enumerate}
    	 \item Ötsch W., Thomasberger C.: Der neoliberale Markt–Diskurs. Ursprünge, Geschichte, Wirkungen., Metropolis Verlag, Marburg, 2009
	 \item Ötsch W., Thomasberger C.: Neoliberale Denkformen, neoliberale Diskurse, neoliberale Hegemonie: Der neoliberale Markt-Diskurs. Ursprünge, Geschichte, Wirkungen., Metropolis Verlag, Marburg, Seite(n) 7-19, 2009
	 \item Ötsch W., Kapeller J.: Neokonservativer Markt-Radikalismus. Das Fallbeispiel des Iraks, in Internationale Politik und Gesellschaft, Nr. 2, Seite(n) 40-55, 2009
	 \item Ötsch W.: Mythos Markt. Marktradikale Propaganda und ökonomische Theorie, Metropolis Verlag, Marburg, 2009
	 \item Ötsch W.: Kognitive Grundlagen menschlichen Verhaltens. Kognitionswissenschaften und neoklassische Standardtheorie, in Goldschmidt, Niels; Nutzinger, Hans G.: Vom homo oeconomicus zum homo culturalis. Handlung und Verhalten in der Ökonomie, LIT-Verlag, Münster, 2009
	 \item Ötsch W.: Computer-Welten und Markt-Diskurs. Die neoklassische Propaganda 'des Marktes'., in Ötsch, Walter O. Thomasberger, Claus: Der neoliberale Markt-Diskurs. Ursprünge, Geschichte, Wirkungen., Metropolis Verlag, Marburg, 2009
	 \item Ötsch W.: Bilder der Wirtschaft. Metaphern, Diskurse und Hayeks neoliberales Hegemonialprojekt., in Hubert Hieke: Kapitalismus. Kritische Betrachtungen und Reformansätze., Metropolis Verlag, Marburg, 2009
	 \item Pühringer S., Wolfmayer G.: Frei Handeln. Überlegungen zur Überwindung des neoliberalen Freiheitsbegriffs: Momentum Kongress Paper, Seite(n) 1-21, 2009
	 \item Kapeller J., Huber J.: Politische Paradigmata und neoliberale Einflüsse am Beispiel von vier sozialdemokratischen Parteien in Europa., in ÖZP - Österreichische Zeitschrift für Politikwissenschaft, Nr. 2, Seite(n) 163-192, 2009
	 \item Hirte K.: Diskursverläufe in der universitären Agrarpolitik als neoliberales Hegemonialprojekt – Struktur, Ursache und Wirkungen, in Ötsch, Walter; Thomasberger, Claus: Der neoliberale Marktdiskurs, Metropolis, Marburg, Seite(n) 187-212, 2009
	 \item Hirte K.: Markt als soziale Struktur - Zum Diskursszenario zur \glqq Märkte-Störung\grqq{} durch den Milchstreik, in arbeitsergebnisse, Nr. 62, University Press, Seite(n) 14-26, 2009
	 \item Dobusch L., Kapeller J.: Diskutieren und Zitieren: Zur paradigmatischen Konstellation aktueller ökonomischer Theorie, in European Journal of Economics and Economic Policies: Intervention, Vol. 6, Nr. 2, Seite(n) 145-152, 2009
	 \item Dobusch L., Kapeller J.: Why is Economics not an Evolutionary Science? New Answers to Veblen's old Question., in Journal of Economic Issues, Vol. 43, Nr. 4, Seite(n) 867-898, 2009
\end{enumerate}
