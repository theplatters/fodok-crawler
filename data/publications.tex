\begin{itemize}
\end{itemize} 
 \subsection{2024} 
 \begin{itemize} 
	 \item Das Imaginative der Politischen Ökonomie: Ötsch W., Hilt A. , Serie Kritische Studien zu Markt und Gesellschaft, Vol. 15, Metropolis Verlag, Marburg, 2024 - Wissenschaftliches Sammelwerk Herausgeberschaft (Erstauflage)
	 \item Globale geo-ökonomische Unordnung: Europa braucht industriepolitische Antworten: Porak L. , 2024 - Sonstige
	 \item Dekarbonisierung des Gebäudesektors als Teil einer sozial-ökologischen Transformation. Ein Gestaltungsvorschlag: Kapeller J., Hornykewycz A., Weber J., Cserjan L. , in ifso expertise, Vol. 25, Seite(n) 1-23, 2024 - Aufsatz / Paper in nicht-referierter Fachzeitschrift
	 \item Vermögensriesen und ein Heer von Zwergen: Kapeller J. , in Kammer für Arbeiter und Angestellte Oberösterreich, 2024 - Sonstige
	 \item Performativität: Geplante Landwirtschaftsstrukturen – das Beispiel Böckenhoff-Plan: Hirte K. , in Gruber, Holle; Henkel, Anna; Scheel, Laura: Land: Digitalisierung, Agrarwandel, Energiewende – soziologische Perspektiven zu ländlichen Räumen im Umbruch, Serie 10 Minuten Soziologie, trancript, Bielefeld, Seite(n) 87-100, 2024 - Aufsatz / Paper in Sammelwerk (referiert)
	 \item Lobbying and Macroeconomic Development: Hager T. , in Mause, Karsten; Polk, Andreas: The Political Economy of Lobbying, Serie Studies in Public Choice, Vol. 43, Springer, Cham, Seite(n) 77-99, 2024 - Aufsatz / Paper in Sammelwerk (referiert)
	 \item Quelle place pour le Sud global dans la décroissance?: Gräbner-Radkowitsch C., Strunk B. , in The Conversation, 2024 - Presseartikel / Medienberichte
	 \item Differential impacts of electricity access on educational outcomes: Evidence from Uganda: Buyinza F., Kapeller J., Senono V., Anber M. , in Electricity Journal, Vol. 37, Nr. 1, Seite(n) 1-10, 2024 - Aufsatz / Paper in sonstiger referierter Fachzeitschrift
\end{itemize} 
 \subsection{2023} 
 \begin{itemize} 
	 \item Wert und Werte in der Ökonomik: Ötsch W. , in Agora42 (Philosophisches Wirtschaftsmagazin), Nr. 02/2023, Seite(n) 9-13, 2023 - Aufsatz / Paper in nicht-referierter Fachzeitschrift
	 \item Economization: The (re-)organization of knowledge and ignorance according to ‘the market’: Steffestun T., Ötsch W. , in ephemera, Vol. 23, Nr. 1, Seite(n) 133-159, 2023 - Aufsatz / Paper in sonstiger referierter Fachzeitschrift
	 \item Prekäre Arbeit an Universitäten kann man nicht wegrechnen: Pühringer S., Partheymüller J. , in Der Standard, 2023 - Presseartikel / Medienberichte
	 \item Woran scheitert transformative Wissensproduktion?: Pühringer S., Altreiter C. , 2023 - Sonstige
	 \item Die (selbstauferlegten) Grenzen der Wissenschaft: Pühringer S., Altreiter C. , in Makronom, 2023 - Presseartikel / Medienberichte
	 \item Man ist akademische Einzelunternehmer*in: Pühringer S. , 2023 - Sonstige
	 \item Soziale und ökologische Probleme müssen zusammen betrachtet werden: Pühringer S. , in Renner Institut: Ein aktiver Staat, der die Menschen stärkt und schützt, Seite(n) 24-27, 2023 - Aufsatz / Paper in Sammelwerk (referiert)
	 \item Islands in the Privately Dominated Sea of Capitalist Media: Porak L., Schamberger K. , in Güney, Selma; Hille, Lina; Pfeiffer, Juliane; Porak, Laura; Theine, Hendrik: Eigentum, Medien, Öffentlichkeit, Westend Verlag, Frankfurt am Main, Seite(n) 443-449, 2023 - Aufsatz / Paper in Sammelwerk (referiert)
	 \item Wettbewerbsfähige Nachhaltigkeit: eine Historisch-Materialistische Analyse der Ideen, Institutionen und Machtverhältnisse in der europäischen grünen Transformation: Porak L. , in Momentum Quarterly, Vol. 12, Nr. 1, Seite(n) 65-83, 2023 - Aufsatz / Paper in sonstiger referierter Fachzeitschrift
	 \item Political sovereignty in tension with global capitalist accumulation: the case of the European socio-economic strategy: Porak L. , in Critical Policy Studies, 2023 - Aufsatz / Paper in SSCI-Zeitschrift
	 \item Can a European wealth tax close the green investment gap?: Kapeller J., Wildauer R., Leitch S. , in Ecological Economics, Nr. 209, 2023 - Aufsatz / Paper in SSCI-Zeitschrift
	 \item Dilemmata marktliberaler Globalisierung: Kapeller J., Hubmann G. , in Sturn, Richard; Klüh, Ulrich: Wachstums- und Globalisierungsgrenzen, Serie Jahrbuch Normative und institutionelle Grundfragen der Ökonomik, Vol. 20, Metropolis Verlag, Marburg, 2023 - Aufsatz / Paper in Sammelwerk (referiert)
	 \item Rezension zu Markus Marterbauer/Martin Schürz: Angst und Angstmacherei: Hubmann G., Kapeller J. , in WISO – Wirtschafts- und Sozialpolitische Zeitschrift des ISW, Vol. 46, Nr. 1, Seite(n) 102-109, 2023 - Aufsatz / Paper in sonstiger referierter Fachzeitschrift
	 \item Euphemistische Ökonomik: Hirte K. , in Blog Postwachstum, 2023 - Presseartikel / Medienberichte
	 \item Im Netz der Einfluss-Reichen: Hager T., Pühringer S. , 2023 - Sonstige
	 \item Endogenous Heterogeneous Gender Norms and the Distribution of Paid and Unpaid Work in an Intra-Household Bargaining Model: Hager T., Mellacher P., Rath M. , 2023 - Sonstige
	 \item Lobbyismus und gesamtwirtschaftliche Entwicklung: Hager T. , in Andreas Polk, Karsten Mause: Handbuch Lobbyismus, Springer VS, Wiesbaden, Seite(n) 817–842, 2023 - Aufsatz / Paper in Sammelwerk (nicht-referiert)
	 \item Degrowth und der globale Süden: Gräbner-Radkowitsch C., Strunk B. , in Blog Postwachstum, 2023 - Presseartikel / Medienberichte
	 \item Degrowth and the Global South: remarks on the twin problem of structural interdependencies: Gräbner-Radkowitsch C., Strunk B. , in Developing economics blog, 2023 - Presseartikel / Medienberichte
	 \item Degrowth and the Global South? How Institutionalism can Complement a Timely Discourse on Ecologically Sustainable Development in an Unequal World: Gräbner-Radkowitsch C., Strunk B. , in Journal of Economic Issues, Vol. 57, Nr. 2, Seite(n) 476-483, 2023 - Aufsatz / Paper in SSCI-Zeitschrift
	 \item Degrowth and the Global South: The Twin Problem of Global Dependencies: Gräbner-Radkowitsch C., Strunk B. , in Ecological Economics, Vol. 213, 2023 - Aufsatz / Paper in SSCI-Zeitschrift
	 \item Competing for Sustainability? An Institutionalist Analysis of the New Development Model of the European Union: Gräbner-Radkowitsch C., Hager T., Hornykewycz A. , in Journal of Economic Issues, Vol. 57, Nr. 2, Seite(n) 676-683, 2023 - Aufsatz / Paper in SSCI-Zeitschrift
	 \item Elements of an evolutionary approach to comparative economic studies: Gräbner-Radkowitsch C. , in Casagrande, Sara; Dallago, Bruno: The Routledge Handbook of Comparative Economic Systems, Routledge, London, forthcoming, Seite(n) 81-102, 2023 - Aufsatz / Paper in Sammelwerk (referiert)
	 \item Il potere e l'economics: Benz P., Maesse J., Pühringer S., Rossier T. , in Nicoletta, Gerardo C.; di Carlo, Michele S.; Ventrone, Oreste: Economisti e Società. Nuove sociologie dell'expertise economica, Liguori Editore, Napoli, Seite(n) 17-24, 2023 - Aufsatz / Paper in Sammelwerk (referiert)
	 \item Economics and Power: Benz P., Maesse J., Pühringer S., Rossier T. , in Macknight, Viski; Medvecky, Fabien: Making Economics Public, Routledge, London, Seite(n) 18-25, 2023 - Aufsatz / Paper in Sammelwerk (referiert)
	 \item Winning city competition with a social agenda. The competition imaginary in Viennese urban development plans: Altreiter C., Azevedo S., Porak L., Pühringer S., Wolfmayr G. , in Urban Research & Practice, Seite(n) 10.1080/17535069.2022.2161834, 2023 - Aufsatz / Paper in SSCI-Zeitschrift
	 \item L´expertise parziale dell´economics: il caso della ricerca (delle politiche) sul commercio: Aistleitner M., Pühringer S. , in Nicoletta, Gerardo C.; di Carlo, Michele S.; Ventrone, Oreste: Economisti e Società. Nuove sociologie dell'expertise economica, Liguori Editore, Napoli, Seite(n) 25-40, 2023 - Aufsatz / Paper in Sammelwerk (referiert)
	 \item The Social Field of Elite Trade Economists: A Quantitative Social Studies of Economics Perspective: Aistleitner M., Pühringer S. , in Oeconomia, Vol. 13, Nr. 2, Seite(n) 475-515, 2023 - Aufsatz / Paper in sonstiger referierter Fachzeitschrift
	 \item Biased trade narratives and its impact on development studies: a multi-level mixed-method approach: Aistleitner M., Pühringer S. , in European Journal of Development Research, Vol. 35, Seite(n) 1322-1346, 2023 - Aufsatz / Paper in SSCI-Zeitschrift
	 \item The authors of economics journals revisited: evidence from a large-scale replication of Hodgson and Rothman (1999): Aistleitner M., Kapeller J., Kronberger D. , in Journal of Institutional Economics, Vol. 19, Nr. 1, Seite(n) 86–101, 2023 - Aufsatz / Paper in SSCI-Zeitschrift
\end{itemize} 
 \subsection{2022} 
 \begin{itemize} 
	 \item Klima, Markt und Zukunftsbilder: Ötsch W. , 2022 - Sonstige
	 \item Pluralism and Economics Education: Theine H., Porak L. , Serie International Journal of Pluralism and Economics Education, Vol. 13 (1), 2022 - Wissenschaftliche Fachzeitschrift (Herausgeberschaft)
	 \item Who are the economists Germany listens to? What it needs to have academic, public or political impact: Pühringer S., Beyer K. , in Maesse, Jens; Pühringer, Stephan; Rossier, Thierry; Benz, Pierre: Power and Influence of Economists: Contributions to the Social Studies of Economics., Routledge, London, Seite(n) 147-169, 2022 - Aufsatz / Paper in Sammelwerk (referiert)
	 \item Divided We Stand? Professional Consensus and Political Conflict in Academic Economics: Pühringer S., Beyer K. , in Journal of Economic Issues, forthcoming, 2022 - Aufsatz / Paper in SSCI-Zeitschrift
	 \item Networks of the super-rich in Austria. Evidence from an explorative case study: Pühringer S., Aistleitner M., Griesebner T. , in Materialien zu Wirtschaft und Gesellschaft, Nr. 238, 2022 - Aufsatz / Paper in nicht-referierter Fachzeitschrift
	 \item Wettbewerbsfähige Nachhaltigkeit – Die Lösung unserer Probleme?: Porak L. : Momentum Kongress Paper, Seite(n) 1-14, 2022 - Aufsatz / Paper in Tagungsband (referiert)
	 \item Power and Influence of Economists: Contributions to the Social Studies of Economic: Maesse J., Pühringer S., Rossier T., Benz P. , Routledge, London, 2022 - Wissenschaftliches Sammelwerk Mitherausgeberschaft (Erstauflage)
	 \item The role of power in the Social Studies of Economics: an introduction: Maesse J., Pühringer S., Rossier T., Benz P. , in Jens Maesse, Stephan Pühringer, Thierry Rossier,  Pierre Benz: Power and Influence of Economists: Contributions to the Social Studies of Economics., Routledge, London, 2022 - Aufsatz / Paper in Sammelwerk (referiert)
	 \item Wissenschaftstheoretische Grundlagen: Koch L., Ötsch W., Graupe S. , in Lehmann-Waffenschmidt, Marco; Peneder, Michael: Evolutorische Ökonomik. Konzepte, Wegbereiter und Anwendungsfelder, Metropolis Verlag, Marburg, Seite(n) 349-359, 2022 - Aufsatz / Paper in Sammelwerk (referiert)
	 \item Tracing the invisible rich: a new approach to modelling Pareto tails in survey data: Kapeller J., Wildauer R. , in Labour Economics, Vol. 75, Nr. 102145, 2022 - Aufsatz / Paper in SSCI-Zeitschrift
	 \item Paradigms and Policies: The state of economics in the German-speaking countries: Kapeller J., Pühringer S., Grimm C. , in Review of International Political Economy, Vol. 29, Nr. 4, Seite(n) 1183-1210, 2022 - Aufsatz / Paper in SSCI-Zeitschrift
	 \item Critical junctures of hope: how to bridge the gap between the necessary and the feasible?: Kapeller J., Huwe V. , in GAIA - Ecological Perspectives for Science and Society, Vol. 31, Nr. 1, Seite(n) 10-13, 2022 - Aufsatz / Paper in SSCI-Zeitschrift
	 \item Notizen zum ökonomischen Element in der politischen Doktrinbildung: Kapeller J., Hubmann G. , in Zeitschrift für Wirtschafts- und Unternehmensethik, Vol. 23, Seite(n) 34-37, 2022 - Aufsatz / Paper in sonstiger referierter Fachzeitschrift
	 \item Wechselseitige Kritik ist nur möglich mit nachvollziehbarer Forschung: Kapeller J. , in ZBW Leibniz Informationszentrum Wirtschaft, Seite(n) 1-2, 2022 - Presseartikel / Medienberichte
	 \item Die Netzwerkanalyse und der Umgang mit ihren Forschungsergebnissen: Hirte K., Ötsch W., Pühringer S. , in Berliner Journal für Soziologie, Nr. 32, Seite(n) 153-163, 2022 - Aufsatz / Paper in SSCI-Zeitschrift
	 \item Strukturkonzentrationen in der Schlachthofbranche und die Rolle von Ökonomen: Hirte K. , in Oxi - Wirtschaft anders denken, Nr. 01/22, Seite(n) 8, 2022 - Presseartikel / Medienberichte
	 \item Evaluierung des Zusammenhangs von Produktionspotenzial und Budgetsemielastizität im Rahmen der deutschen Schuldenbremse: Heimberger P., Schütz B. , 2022 - Sonstige
	 \item Das wackelige Fundament der Schuldenbremse: Heimberger P., Schütz B. , in Makronom, 2022 - Presseartikel / Medienberichte
	 \item Die Budgetsemielastizität und ihre Auswirkungen auf Verschuldungsspielräume im Rahmen der Schuldenbremse: Heimberger P., Schütz B. , in Wirtschaftsdienst, Vol. 102, Nr. 11, Seite(n) 834-837, 2022 - Aufsatz / Paper in sonstiger referierter Fachzeitschrift
	 \item Polanyi and Schumpeter: Transitional processes via societal spheres: Hager T., Heck I., Rath J. , in The European Journal for the History of Economic Thought, Vol. 29, Nr. 6, Seite(n) 1089–1110, 2022 - Aufsatz / Paper in SSCI-Zeitschrift
	 \item Trade Models in the European Union: Gräbner-Radkowitsch C., Tamesberger D., Heimberger P., Kapelari T., Kapeller J. , in Economic Annals, Vol. 67, Nr. 235, Seite(n) 7-36, 2022 - Aufsatz / Paper in sonstiger referierter Fachzeitschrift
	 \item Zur ökonomischen Bedeutung von Suffizienz: Gräbner-Radkowitsch C., Lage J., Wiese F. , in Makronom, 2022 - Presseartikel / Medienberichte
	 \item The evolution of debtor-creditor relationships within a monetary union: Trade imbalances, excess reserves and economic policy: Gräbner-Radkowitsch C., Heimberger P., Kapeller J., Landesmann M., Schütz B. , in Structural Change and Economic Dynamics, Vol. 62, Seite(n) 262-289, 2022 - Aufsatz / Paper in SSCI-Zeitschrift
	 \item Why Fostering Socio-economic Convergence in the EU Is Necessary for Successful Climate Change Mitigation: Gräbner-Radkowitsch C., Hafele J. , in Heinrich Böll Foundation, ZOE-Institute for Future-Fit Economies and Finanzwende Recherche: Making the great turnaround work: Economic policy for a green and just transition, Seite(n) 104-114, 2022 - Aufsatz / Paper in Sammelwerk (referiert)
	 \item Capability accumulation and product innovation: an agent-based perspective: Gräbner C., Hornykewycz A. , in Journal of Evolutionary Economics, Vol. 32, Seite(n) 87-121, 2022 - Aufsatz / Paper in SSCI-Zeitschrift
	 \item Arbeitsplatz Universität: Gefangen in der Teilzeit: Breth L., Part F., Pühringer S., Schlitz N., Sperner P., Völkl Y. , in Der Standard, 2022 - Presseartikel / Medienberichte
	 \item Verzerrter Wettbewerb in der Forschung: Aistleitner M. , in science.orf.at, 2022 - Presseartikel / Medienberichte
\end{itemize} 
 \subsection{2021} 
 \begin{itemize} 
	 \item Populismus: Ötsch W., Wodak R. , in Ferstl, Michael G.: Handbuch Liberalismus, J.B. Metzler, Stuttgart, Seite(n) 535-541, 2021 - Aufsatz / Paper in Sammelwerk (referiert)
	 \item Wissen und Nichtwissen der ökonomisierten Gesellschaft. Aufgaben einer neuen Politischen Ökonomie: Ötsch W., Steffestun T. , Metropolis Verlag, Marburg, 2021 - Wissenschaftliches Sammelwerk Herausgeberschaft (Erstauflage)
	 \item Ordoliberalismus: Ötsch W., Pühringer S. , in Michael G. Festl: Handbuch Liberalismus, J.B. Metzler, Stuttgart, Seite(n) 372-378, 2021 - Aufsatz / Paper in Sammelwerk (referiert)
	 \item Walter Lippmann: Die Illusion von Wahrheit oder die Erfindung der Fake News: Ötsch W., Graupe S. , Fivty-fivty Verlag, Edition Buchkomplizen, Frankfurt am Main, 2021 - Sonstige
	 \item Vorwort: Ötsch W., Graupe S. , in Ötsch, Walter O.; Graupe, Silja, Fifty-fifty Verlag, Frankfurt am Main, 2021 - Sonstige
	 \item Narration und Imagination. Die Rolle von imaginierten Bildern in der Geschichte der Wirtschaftstheorie: Ötsch W. , in Künzel, Christine; Priddat, Birger: Fiktion und Narration in der Ökonomie. Interdisziplinäre Perspektiven auf den Umgang mit ungewisser Zukunft, Metropolis, Marburg, Seite(n) 241-267, 2021 - Aufsatz / Paper in Sammelwerk (referiert)
	 \item Agent of Sustainable Change - Der unternehmerische Staat und sozial-ökologische Transformation: Strohmaier R. , in Klüh, Ulrich; Sturn, Richard: Der Staat in der großen Transformation. Jahrbuch Normative und institutionelle Grundfragen der Ökonomik, Serie Jahrbuch Normative und institutionelle Grundfragen der Ökonomik, Metropolis, Weimar, Seite(n) 169-192, 2021 - Aufsatz / Paper in Sammelwerk (referiert)
	 \item "Koste es, was es wolle". Eine neue Ära der Ökonomie?: Schütz B. , in economy, 2021 - Aufsatz / Paper in nicht-referierter Fachzeitschrift
	 \item Creating a pluralist paradigm: An application to the minimum wage debate: Schütz B. , in Journal of Economic Issues, Vol. 55, Nr. 1, Seite(n) 103-124, 2021 - Aufsatz / Paper in SSCI-Zeitschrift
	 \item The political economy of academic publishing: On the commodification of a public good: Pühringer S., Rath J., Griesebner T. , in PLoS One, Vol. 16, Nr. 6, 2021 - Aufsatz / Paper in SCI-Expanded-Zeitschrift
	 \item Monopolies in Science Publishing. A Black Hole for Public Spending?: Pühringer S., Rath J. , in Journal of Management Information and Decision Sciences, Vol. 24, Nr. 6, Seite(n) 1-5, 2021 - Aufsatz / Paper in sonstiger referierter Fachzeitschrift
	 \item Soziale Rhetorik, neoliberale Praxis: Eine Analyse der Wirtschafts- und Sozialpolitik der AfD: Pühringer S., Beyer K., Kronberger D. , Serie OBS Arbeitspapier, Nr. 52, Otto Brenner Stiftung, 2021 - Forschungsbericht: geförderte Forschung (andere Geldgeber)
	 \item Soziale Rhetorik, neoliberale Praxis: Pühringer S., Beyer K., Kronberger D. , in Beuteler Extradienst, 2021 - Aufsatz / Paper in nicht-referierter Fachzeitschrift
	 \item Zur Pluralität in der ökonomischen Politikberatung in Deutschland. Eine empirische Untersuchung: Pühringer S. , in Leviathan - Berliner Zeitschrift für Sozialwissenschaft, Vol. 49, Seite(n) 243-265, 2021 - Aufsatz / Paper in sonstiger referierter Fachzeitschrift
	 \item Strategien für einen Wandel der ökonomischen Lehre: Porak L., Schröter G. , in Forschungsjournal Soziale Bewegungen, Vol. 34, Nr. 4, Seite(n) 718-729, 2021 - Aufsatz / Paper in sonstiger referierter Fachzeitschrift
	 \item So denken Ökonom*innen über Wettbewerb – eine Kritische Analyse des österreichischen Expert*innendiskurses: Porak L., Pühringer S., Rath J. , 2021 - Sonstige
	 \item Und ewig lockt der Wettbewerb: Porak L. , in Makronom, 2021 - Presseartikel / Medienberichte
	 \item Warum müssen wir (noch immer) arbeiten? Eine hegemonietheoretische Analyse der Bedeutung und des Wertes von Lohnarbeit für den modernen Staat: Porak L. : Momentum Kongress Paper, 2021 - Aufsatz / Paper in Tagungsband (referiert)
	 \item Eine europäische Vermögenssteuer für das Klima: Kapeller J., Wildauer R. , 2021 - Sonstige
	 \item A Fitting Pareto tails to wealth survey data: A practitioners’ guide: Kapeller J., Wildauer R. , in Journal of Income Distribution, 2021 - Aufsatz / Paper in sonstiger referierter Fachzeitschrift
	 \item Vom empiristischen Humanismus zum partizipativen Sozialismus – Review von Thomas Piketty ‚Kapital und Ideologie‘: Kapeller J., Rehm M. , in Soziologische Revue, Vol. 44, Nr. 1, Seite(n) 25-33, 2021 - Aufsatz / Paper in sonstiger referierter Fachzeitschrift
	 \item A European wealth tax: Kapeller J., Leitch S., Wildauer R. , 2021 - Sonstige
	 \item Is a € 10 Trillion European climate investment initiative fiscally sustainable?: Kapeller J., Leitch S., Wildauer R. , in Renner Institut & Foundation for European Progressive Studies (FEPS), in Policy Study, 2021 - Aufsatz / Paper in nicht-referierter Fachzeitschrift
	 \item A European wealth tax for a fair and green recovery: Kapeller J., Leitch S., Wildauer R. , in Renner Institut & Foundation for European Progressive Studies (FEPS), in Policy Study, 2021 - Aufsatz / Paper in nicht-referierter Fachzeitschrift
	 \item Standortwettbewerb und Deindustrialisierung: Das Beispiel MAN als Lehrbuchfall: Kapeller J., Gräbner C. , in WISO - Wirtschafts- und sozialpolitische Zeitschrift, Vol. 44, Nr. 4, Seite(n) 34-52, 2021 - Aufsatz / Paper in sonstiger referierter Fachzeitschrift
	 \item Ökonomische Polarisierung in Europa: Kapeller J. , in Zeitschrift Bürger und Staat, Vol. 71, Nr. 4, Seite(n) 246-251, 2021 - Aufsatz / Paper in nicht-referierter Fachzeitschrift
	 \item Polarisierung oder Konvergenz? Zur ökonomischen Zukunft des vereinten Europas: Kapeller J. , in WISO direkt (Analysen und Konzepte zur Wirtschafts- und Sozialpolitik), 2021 - Aufsatz / Paper in sonstiger referierter Fachzeitschrift
	 \item Intangible Flow Theory: A New Way for Conceptualizing Embeddedness?: Kapeller J. , in Accounting, Economics and Law, Vol. 14, Nr. 1, Seite(n) 159-164, 2021 - Aufsatz / Paper in sonstiger referierter Fachzeitschrift
	 \item Shaping sustainable employment relationships in the age of Digitalisation: analysing policy measures in an agent-based framework: Hornykewycz A., Rath J. : Momentum Kongress Paper, 2021 - Aufsatz / Paper in Tagungsband (referiert)
	 \item Ökonomie und Ideologie: Hirte K. , in OXI - Wirtschaft anders denken, Seite(n) 16-17, 2021 - Presseartikel / Medienberichte
	 \item Unternehmenskonzentrationen in der Fleischbranche und die performative Rolle der Agrarökonomik: Hirte K. , in ÖZS - Österreichische Zeitschrift für Soziologie, Vol. 46, Nr. 2, Seite(n) 187-206, 2021 - Aufsatz / Paper in sonstiger referierter Fachzeitschrift
	 \item Il governo Draghi: sette fatti sorprendenti sull’Italia: Heimberger P., Kowall N. , 2021 - Sonstige
	 \item Sieben „überraschende“ Fakten zu Italien: Heimberger P., Kowall N. , in Makronom, 2021 - Presseartikel / Medienberichte
	 \item Corporate tax cuts do not boost growth: Heimberger P., Gechert S. , 2021 - Sonstige
	 \item Erhöhen Unternehmenssteuersenkungen das Wirtschaftswachstum?: Heimberger P., Gechert S. , in Ökonomenstimme, 2021 - Presseartikel / Medienberichte
	 \item Verschwenderisches, reformfaules Italien? Warum gängige Mythen falsch und gefährlich sind: Heimberger P. , in Marie Jahoda – Otto Bauer Institut, 2021 - Sonstige
	 \item The push for a global minimum corporate tax rate: Heimberger P. , in Vienna Institute for International Economic (wiiw), 2021 - Sonstige
	 \item Keynes, the output gap and the EU’s fiscal rules: Heimberger P. , in Vienna Institute for International Economic (wiiw), 2021 - Sonstige
	 \item Keynes, output gap nonsense and the EU’s fiscal rules: Heimberger P. , 2021 - Sonstige
	 \item Fiscal austerity and the rise of the Nazis: Heimberger P. , 2021 - Sonstige
	 \item Financial globalisation has increased income inequality: Heimberger P. , 2021 - Sonstige
	 \item European fiscal rules: reform urgently needed: Heimberger P. , 2021 - Sonstige
	 \item EU bonds are a model for the future of Europe: Heimberger P. , in Vienna Institute for International Economic (wiiw), 2021 - Sonstige
	 \item Draghi government: Seven ‘surprising’ facts about Italy: Heimberger P. , in Vienna Institute for International Economic (wiiw), 2021 - Sonstige
	 \item Budgetkürzungen durch „Outputlücken-Nonsens“: Heimberger P. , 2021 - Sonstige
	 \item Beeld over Italiaanse economie klopt niet: Heimberger P. , 2021 - Sonstige
	 \item Wie stark der globale Steuerwettbewerb tatsächlich ist: Heimberger P. , in Makronom, 2021 - Presseartikel / Medienberichte
	 \item Wachstum durch Unternehmensteuersenkungen? Die FDP weckt übertriebene Hoffnungen: Heimberger P. , in Handelsblatt, 2021 - Presseartikel / Medienberichte
	 \item Höhere Staatsschulden = weniger Wachstum?: Heimberger P. , in Makronom, 2021 - Presseartikel / Medienberichte
	 \item Fakten-Mensch: Ökonom Philipp Heimberger kritisiert die aus seiner Sicht verzerrte Debatte über die EU: Heimberger P. , in Süddeutsche Zeitung, 2021 - Presseartikel / Medienberichte
	 \item Eine globale Mindeststeuer stoppt die Steuerflucht der Konzerne: Heimberger P. , in Handelsblatt, 2021 - Presseartikel / Medienberichte
	 \item Drei Gründe, warum Staatsschulden nicht zwingend problematisch sind: Heimberger P. , in Handelsblatt, 2021 - Presseartikel / Medienberichte
	 \item Draghi darf das Sparen nicht übertreiben: Heimberger P. , in Handelsblatt, 2021 - Presseartikel / Medienberichte
	 \item Die deutschen Inflationssorgen speisen sich aus einem verzerrten Geschichtsbild: Heimberger P. , in Handelsblatt, 2021 - Presseartikel / Medienberichte
	 \item Die Finanzglobalisierung verschärft die Einkommensungleichheit: Heimberger P. , in Handelsblatt, 2021 - Presseartikel / Medienberichte
	 \item Die EU-Anleihen sind ein Zukunftsmodell für Europa: Heimberger P. , in Handelsblatt, 2021 - Presseartikel / Medienberichte
	 \item Der Nobelpreis für David Card hat die Mindestlohn-Befürworter in Deutschland gestärkt: Heimberger P. , in Handelsblatt, 2021 - Presseartikel / Medienberichte
	 \item Beschäftigungsschutz erhöht die Arbeitslosigkeit nicht: Heimberger P. , in Ökonomenstimme, 2021 - Presseartikel / Medienberichte
	 \item Arbeitnehmerrechte sind kein Jobkiller: Heimberger P. , in Handelsblatt, 2021 - Presseartikel / Medienberichte
	 \item Keynes, die Outputlücke und Probleme mit den Fiskalregeln: Heimberger P. , in Blog der Keynes-Gesellschaft, 2021 - Aufsatz / Paper in nicht-referierter Fachzeitschrift
	 \item What is structural about unemployment in OECD countries?: Heimberger P. , in Review of Social Economy, Vol. 79, Nr. 2, Seite(n) 380-412, 2021 - Aufsatz / Paper in sonstiger referierter Fachzeitschrift
	 \item Does employment protection affect unemployment? A meta-analysis: Heimberger P. , in Oxford Economic Papers, Vol. 73, Nr. 3, Seite(n) 982-1007, 2021 - Aufsatz / Paper in SSCI-Zeitschrift
	 \item Does economic globalization affect government spending? A meta-analysis: Heimberger P. , in Public Choice, Vol. 187, Seite(n) 349-374, 2021 - Aufsatz / Paper in SSCI-Zeitschrift
	 \item Does economic globalisation promote economic growth?: Heimberger P. , in The World Economy, forthcoming, doi.org/10.1111/twec.13235, 2021 - Aufsatz / Paper in SSCI-Zeitschrift
	 \item Corporate tax competition: A meta-analysis: Heimberger P. , in European Journal of Political Economy, Vol. 69, Nr. 1020002, 2021 - Aufsatz / Paper in SSCI-Zeitschrift
	 \item „Hinter jeder erfolgreichen Frau steht ein Mann, der ihr den Rücken stärkt.“: Hager T., Hornykewycz A., Jonjic M., Porak L., Rath J. : Momentum Kongress Paper, 2021 - Aufsatz / Paper in Tagungsband (referiert)
	 \item Pluralism in economics – its critiques and their lessons: Gräbner C., Strunk B. , in Developing Economics, 2021 - Aufsatz / Paper in nicht-referierter Fachzeitschrift
	 \item Competition Universalism: Its Historical Origins and Timely Alternatives: Gräbner C., Pühringer S. , 2021 - Sonstige
	 \item Konzernmacht in globalen Güterketten: Gräbner C., Kapeller J. , in Karin Fischer, Christian Reiner und Cornelia Staritz: Globale Güterketten und ungleiche Entwicklung. Arbeit, Kapital, Natur und Konsum, Mandelbaum, Seite(n) 195-217, 2021 - Aufsatz / Paper in Sammelwerk (referiert)
	 \item The emergence of debt and secular stagnation in an unequal society: A stock-flow consistent agent-based approach: Gräbner C., Hornykewycz A., Schütz B. : Momentum Kongress Paper, 2021 - Aufsatz / Paper in Tagungsband (referiert)
	 \item Introduction to the symposium: The Complexity of Institutions: Theory and Computational Models: Gräbner C., Heinrich T. , in Forum for Social Economics, Vol. 50, Nr. 2, Seite(n) 153-156, 2021 - Aufsatz / Paper in sonstiger referierter Fachzeitschrift
	 \item Understanding economic openness: A review of existing measures: Gräbner C., Heimberger P., Kapeller J., Springholz F. , in Review of World Economics, Vol. 157, Nr. 1, Seite(n) 87-120, 2021 - Aufsatz / Paper in SSCI-Zeitschrift
	 \item Ökonomische Offenheit: Die Vermessung der Globalisierung?: Gräbner C., Heimberger P., Kapeller J. , in Ökonomenstimme, 2021 - Presseartikel / Medienberichte
	 \item (Mis)Measuring Competitiveness:  The Quantification of a Malleable Concept in the European Semester: Gräbner C., Hager T. , 2021 - Sonstige
	 \item Trust and Social Control. The Sources of Stability in Informal Value Transfer Systems: Gräbner C., Elsner W., Lascaux A. , in Computational Economics, Nr. 58, Seite(n) 1077-1102, 2021 - Aufsatz / Paper in SSCI-Zeitschrift
	 \item Für die „Leistungsträger“ und „uns Österreicher“: Eine Mediendiskursanalyse zu Sozialreformen der ÖVP/FPÖ-Regierung 2017-2019 in Österreich: Griesser M., Beyer K., Pühringer S. : Momentum Kongress Paper, 2021 - Aufsatz / Paper in Tagungsband (referiert)
	 \item The collapse of cooperation: The endogeneity of institutional break-up and its asymmetry with emergence: Cordes C., Elsner W., Gräbner C., Heinrich T., Henkel J., Schwardt H., Schwesinger G., Su T. , in Journal of Evolutionary Economics, Vol. 31, Nr. 4, Seite(n) 1291-1315, 2021 - Aufsatz / Paper in SSCI-Zeitschrift
	 \item 20 Jahre gute Absichten in der Wissenschaftspolitik: Altreiter C., Rogojanu A., Gräbner C., Pühringer S., Wolfmayr G. , in Die Presse, 2021 - Presseartikel / Medienberichte
	 \item Wie befristete Uni-Stellen Innovation verhindern: Altreiter C., Gräbner C., Pühringer S., Rogojanu A., Wolfmayr G. , in Der Standard, 2021 - Presseartikel / Medienberichte
	 \item The Trade (Policy) Discourse in Top Economic Journals: Aistleitner M., Pühringer S. , in New Political Economy, Vol. 26, Nr. 5, Seite(n) 748-764, 2021 - Aufsatz / Paper in SSCI-Zeitschrift
	 \item Theory and Empirics of Capability Accumulation: Implications for Macroeconomic Modelling: Aistleitner M., Gräbner C., Hornykewycz A. , in Research Policy, Vol. 50, Nr. 6, e-no. 104258, 2021 - Aufsatz / Paper in SSCI-Zeitschrift
\end{itemize} 
 \subsection{2020} 
 \begin{itemize} 
	 \item Wissen, Selbstwissen und Nichtwissen der marktfundamentalen Ökonomie: Ötsch W. , in Ötsch, Walter O.; Steffestun, Theresa: Wissen und Nichtwissen der ökonomisierten Gesellschaft, Metropolis Verlag, Marburg, Seite(n) 85-131, 2020 - Aufsatz / Paper in Sammelwerk (nicht-referiert)
	 \item Allbetroffenheit in der Pandemie? Ein soziologischer Blick auf das Erleben der Auswirkungen der Corona-Krise: Vogel L., Jühlke R., Porak L., Quinz H. : Momentum Kongress Paper, Seite(n) 1-19, 2020 - Aufsatz / Paper in Tagungsband (referiert)
	 \item Gewohnter Umsatz wird auf sich warten lassen: Schütz B. , in Kronen Zeitung, Interview (von Elisabeth Rathenböck) mit Bernhard Schütz, 2020 - Presseartikel / Medienberichte
	 \item Die Auswirkung von Mindestlöhnen auf die Arbeitslosigkeit: Ein Paradigmenvergleich: Schütz B. , in Stephan Pühringer, Silja Graupe, Katrin Hirte, Jakob Kapeller, Stephan Panther: Jenseits der Konventionen: Alternatives Denken zu Wirtschaft, Gesellschaft und Politik, Metropolis, Marburg, Seite(n) 157-173, 2020 - Aufsatz / Paper in Sammelwerk (referiert)
	 \item From the ʻplanning euphoriaʼ to the ʻbitter economic truthʼ: The Transmission of economic ideas into German Labour Market Policies in the 1960s and 2000s: Pühringer S., Griesser M. , in Critical Discourse Studies, Vol. 17, Nr. 5, Seite(n) 476-493, 2020 - Aufsatz / Paper in SSCI-Zeitschrift
	 \item Jenseits der Konventionen. Alternatives Denken zu Wirtschaft, Gesellschaft und Politik. Festschrift für Walter Ötsch: Pühringer S., Graupe S., Hirte K., Kapeller J., Panther S. , Metropolis, Marburg, 2020 - Wissenschaftliches Sammelwerk Herausgeberschaft (Erstauflage)
	 \item Vorwort. Jenseits der Konventionen: Pühringer S., Graupe S., Hirte K., Kapeller J., Panther S. , in Pühringer, Stephan; Graupe, Silja; Hirte, Katrin; Kapeller, Jakob; Panther, Stephan: Jenseits der Konventionen. Eine Festschrift für Walter Ötsch, Metropolis, Marburg, Seite(n) 9-16, 2020 - Aufsatz / Paper in Sammelwerk (nicht-referiert)
	 \item Think Tank Networks of German Neoliberalism. Power Structures in Economics and Economic Policies in Post-War Germany: Pühringer S. , in Mirowski, Philip; Plehwe, Dieter; Slobodian, Quinn: Nines Lives of Neoliberalism, Verso Books, New York, Seite(n) 283-306, 2020 - Aufsatz / Paper in Sammelwerk (referiert)
	 \item Die moderne Lehrbuchwissenschaft als Zombiewissenschaft: Porak L., Neuffer S. , in Agora42 (Philosophisches Wirtschaftsmagazin), 2020 - Aufsatz / Paper in nicht-referierter Fachzeitschrift
	 \item Miteinander und voneinander lernen. Vielfalt in der ökonomischen Lehre: Porak L. , in Hochmann, Lars: Economists4future, Murmann Verlag, Hamburg, Seite(n) 127-142, 2020 - Aufsatz / Paper in Sammelwerk (referiert)
	 \item Wohin steuert die Europäische Union? Ein Klärungsversuch der strategischen Ausrichtung der EU seit Lissabon: Porak L. : Momentum Kongress Paper, Seite(n) 1-22, 2020 - Aufsatz / Paper in Tagungsband (referiert)
	 \item Plurale Ökonomik - Eine kurze Einführung: Piétron D., Porak L., Thieme S. , in Thielscher, Christian: Wirtschaftswissenschaften verstehen, Springer Gabler, Wiesbaden, Seite(n) 189-205, 2020 - Aufsatz / Paper in Sammelwerk (referiert)
	 \item How to boost the European Green Deal’s scale and ambition: Kapeller J., Wildauer R., Leitch S. , Serie FEPS Policy Paper, 2020 - Forschungsbericht: geförderte Forschung (EU)
	 \item Vermögenskonzentration in Österreich: Ein Update auf Basis des HFCS 2017: Kapeller J., Wildauer R., Heck I. , in Wirtschaft und Gesellschaft, Nr. 206, Seite(n) 1-38, 2020 - Aufsatz / Paper in sonstiger referierter Fachzeitschrift
	 \item Paradigmen und Politik. Der derzeitige Stand der Ökonomie: Kapeller J., Pühringer S. , in Jakob Kapeller, Stephan Pühringer, Silja Graupe, Kathrin Hirte, Stephan Panther: Jenseits der Konventionen: Alternatives Denken zu Wirtschaft, Gesellschaft und Politik, Metropolis, Marburg, Seite(n) 221-252, 2020 - Aufsatz / Paper in Sammelwerk (referiert)
	 \item Polarisierung oder Konvergenz? Zur ökonomischen Zukunft des vereinten Europa: Kapeller J. , 2020 - Sonstige
	 \item Wahrheiten über Wertfreiheit – eine Replik auf den Gastkommentar ‚Werte, Wahrheit, Wirtschaftsforschung‘ von Harald Oberhofer: Kapeller J. , in Der Standard, 2020 - Presseartikel / Medienberichte
	 \item „Um welches Ziel es in der Ökonomie geht, ist also bestimmbar“: Hirte K. , in Agora42 (Philosophisches Wirtschaftsmagazin), 2020 - Presseartikel / Medienberichte
	 \item Agrarsubventionen als Preis der Marktwirtschaft: Hirte K. , in Wege für eine Bäuerliche Zukunft – Zeitschrift der ÖBV/ Via Campesina Austria, Vol. 43, Nr. 1 (361), Wien, Seite(n) 10-11, 2020 - Presseartikel / Medienberichte
	 \item Friedman’s Instrumentalismus und das Problem von Kopernikus: Hirte K. , in Pühringer, Stephan; Graupe, Silja; Hirte, Katrin; Kapeller, Jakob; Panther, Stephan: Jenseits der Konventionen, Metropolis Verlag, Marburg, Seite(n) 97-122, 2020 - Aufsatz / Paper in Sammelwerk (referiert)
	 \item Das doppelte Reflektionsproblem: Hirte K. , in Hochmann, Lars: Economists4future, Murmann Verlag, Hamburg, Seite(n) 43-58, 2020 - Aufsatz / Paper in Sammelwerk (referiert)
	 \item Der Outputlücken-Nonsense gefährdet Deutschlands Erholung von der Corona-Krise: Heimberger P., Truger A. , in Makronom, 2020 - Presseartikel / Medienberichte
	 \item Gleichwertige Lebensverhältnisse im Euroraum: Heimberger P., Krahé M., van't Klooster J., Ponattu D. , in Frankfurter Allgemeine Zeitung, 2020 - Presseartikel / Medienberichte
	 \item Keeping the promise of eurozone convergence: Heimberger P., Krahé M., Ponattu D., van't Klooster J. , 2020 - Sonstige
	 \item Seven ’surprising’ facts about the Italian economy: Heimberger P., Kowall N. , 2020 - Sonstige
	 \item Wie die ‚Herren der Modelle‘ versuchen, den Outputlücken-Nonsense zu rechtfertigen: Heimberger P., Kapeller J. , in Makronom, 2020 - Presseartikel / Medienberichte
	 \item ‘Output gap nonsense’ and the EU’s fiscal rules: A response to the European Commission’s economists: Heimberger P., Kapeller J. , in Blog Institute for New Economic Thinking, 2020 - Aufsatz / Paper in nicht-referierter Fachzeitschrift
	 \item The power of economic models: The case of the EU's fiscal regulation framework: Heimberger P., Huber J., Kapeller J. , in Socio-Economic Review, Vol. 18, Nr. 2, Seite(n) 337-366, 2020 - Aufsatz / Paper in SSCI-Zeitschrift
	 \item Wie die ökonomische Globalisierung die Einkommensungleichheit beeinflusst: Heimberger P. , 2020 - Sonstige
	 \item Hyperinflation and the Rise of the Nazis: Heimberger P. , 2020 - Sonstige
	 \item EU-Wiederaufbaufonds als Kernstück europäischer Krisenbekämpfung: Progressiver Durchbruch oder Enttäuschung?: Heimberger P. , 2020 - Sonstige
	 \item Budgetpolitik in der Corona-Krise: Reform der Budgetregeln erforderlich: Heimberger P. , 2020 - Sonstige
	 \item Budgetpolitik im Wirtschaftsabschwung: erhebliche Spielräume vorhanden: Heimberger P. , 2020 - Sonstige
	 \item Zu niets doen voor Z-Europa is gevaarlijk (Nichts für Südeuropa zu tun ist gefährlich): Heimberger P. , in NRC Handelsblad, 2020 - Presseartikel / Medienberichte
	 \item Wie ein Rechenfehler zu fatalen Konsequenzen für die Haushalte der EU-Staaten führen kann: Heimberger P. , in Handelsblatt, 2020 - Presseartikel / Medienberichte
	 \item Wie die EU-Kommission Deutschlands Budgetsituation schlechtrechnet: Heimberger P. , in Handelsblatt, 2020 - Presseartikel / Medienberichte
	 \item Kein Weg in die „Schuldenunion“: Heimberger P. , in Der Standard, 2020 - Presseartikel / Medienberichte
	 \item Enttäuschender EU-Gipfel: Zeit für eine „Neue Südpolitik”: Heimberger P. , in Makronom, 2020 - Presseartikel / Medienberichte
	 \item Einigung der Eurogruppe: Bestenfalls ein erster Schritt: Heimberger P. , in Makronom, 2020 - Presseartikel / Medienberichte
	 \item Die gefährliche Geschichtsverdrehung des Hans-Werner Sinn: Heimberger P. , in Handelsblatt, 2020 - Presseartikel / Medienberichte
	 \item Beschäftigungsschutz ist kein Jobkiller: Heimberger P. , in Makronom, 2020 - Presseartikel / Medienberichte
	 \item Structural polarisation and path dependent development models in the EU: Heimberger P. , in Blog Developing Economics, 2020 - Aufsatz / Paper in nicht-referierter Fachzeitschrift
	 \item The dynamic effects of fiscal consolidation episodes on income inequality: Evidence for 17 OECD countries over 1978-2013: Heimberger P. , in Empirica, Vol. 47, Seite(n) 53-81, 2020 - Aufsatz / Paper in SSCI-Zeitschrift
	 \item Does economic globalisation affect income inequality? A meta-analysis: Heimberger P. , in The World Economy, Seite(n) 1-23, 2020 - Aufsatz / Paper in SSCI-Zeitschrift
	 \item Competition in Transformational Processes: Polanyi & Schumpeter: Hager T., Heck I., Rath J. , in Momentum: Momentum Kongress Paper, Seite(n) 1-25, 2020 - Aufsatz / Paper in Tagungsband (referiert)
	 \item Kritik an der Pluralen Ökonomik – Was ist dran und warum ist das wichtig?: Gräbner C., Strunk B. , in Ökonomenstimme, 2020 - Presseartikel / Medienberichte
	 \item Pluralism in economics: its critiques and their lessons: Gräbner C., Strunk B. , in Journal of Economic Methodology, Vol. 27, Nr. 4, Seite(n) 311-329, 2020 - Aufsatz / Paper in SSCI-Zeitschrift
	 \item Structural change in times of increasing openness: assessing path dependency in European economic integration: Gräbner C., Heimberger P., Kapeller J., Schütz B. , in Journal of Evolutionary Economics, Vol. 30, Nr. 5, Seite(n) 1467-1495, 2020 - Aufsatz / Paper in SSCI-Zeitschrift
	 \item Is the Eurozone disintegrating? Macroeconomic divergence, structural polarization, trade and fragility: Gräbner C., Heimberger P., Kapeller J., Schütz B. , in Cambridge Journal of Economics, Vol. 44, Nr. 3, Seite(n) 647-669, 2020 - Aufsatz / Paper in SSCI-Zeitschrift
	 \item Ökonomische Offenheit: Die Vermessung der Globalisierung: Gräbner C., Heimberger P., Kapeller J. , in Makronom, 2020 - Presseartikel / Medienberichte
	 \item Pandemic pushes polarisation: the Corona crisis and macroeconomic divergence in the Eurozone: Gräbner C., Heimberger P., Kapeller J. , in Journal of Industrial and Business Economics, Vol. 47, Nr. 3, Seite(n) 425-438, 2020 - Aufsatz / Paper in sonstiger referierter Fachzeitschrift
	 \item Wirtschaft(lich) studieren. Erlebniswelten von Studierenden der Volkswirtschaftslehre: Bäuerle L., Ötsch W., Pühringer S. , Springer, Wiesbaden, 2020 - Monographie (Erstauflage)
	 \item Türkis-blaue Arbeitsmarkt- und Sozialpolitik revisited: zwischen Meritokratie und Wohlfahrtschauvinismus: Beyer K., Griesser M., Pühringer S. , 2020 - Sonstige
	 \item Exploring the trade (policy) narratives in economic elite discourse: Aistleitner M., Pühringer S. , Serie SSRN, 2020 - Preprint
\end{itemize} 
 \subsection{2019} 
 \begin{itemize} 
	 \item Marktfundamentalismus als Kollektivgedanke. Mises und die Ordoliberalen: Ötsch W., Pühringer S. , in Richard Sturn, Nenad Pantelic: Dem Markt vertrauen? Beiträge zur Tiefenstruktur neoliberaler Regulierung., Metropolis, Marburg, Seite(n) 185-210, 2019 - Aufsatz / Paper in Sammelwerk (referiert)
	 \item Was ist eine Krise? Ein Rückblick auf die Wirtschafts- und Finanzkrisen 2008 und 2010: Ötsch W., Pühringer S. , in Blickpunkt WISO, 2019 - Aufsatz / Paper in nicht-referierter Fachzeitschrift
	 \item ’Erkühne Dich, weise zu sein!‘ Grundlegung einer Gemeinsinn-Ökonomie: Ötsch W., Graupe S., Loske R. , in GWP - Gesellschaft, Wirtschaft, Politik, Vol. 68, Nr. 2, Seite(n) 243-250, 2019 - Aufsatz / Paper in sonstiger referierter Fachzeitschrift
	 \item Überwachungskapitalismus: Das Internet als totalitärer Markt: Ötsch W. , 2019 - Sonstige
	 \item Ökonomie als Lachnummer: Ötsch W. , in Krauß, Dietrich: Die Rache des Mainstreams an sich selbst. 5 Jahre »Die Anstalt«, Westend Verlag, Frankfurt am Main, Seite(n) 260-271, 2019 - Aufsatz / Paper in Sammelwerk (nicht-referiert)
	 \item Wissen und Nichtwissen angesichts ‚des Marktes‘. Das Konzept von Hayek: Ötsch W. , in Graupe, Silja; Ötsch, Walter O.; Rommel, Florian: Spielräume des Denkens, Metropolis, Marburg, Seite(n) 311-339, 2019 - Aufsatz / Paper in Sammelwerk (referiert)
	 \item Creating a pluralist paradigm: An application to the minimum wage debate: Schütz B. : Momentum Kongress Paper, Seite(n) 1-41, 2019 - Aufsatz / Paper in Tagungsband (referiert)
	 \item Männlich, mikroökonomisch, Mainstream? Eine Untersuchung der VWL-Lehrstühle in Deutschland: Pühringer S., Grimm C. , in Makronom, 2019 - Presseartikel / Medienberichte
	 \item What economics education is missing: the real world: Pühringer S., Bäuerle L. , in International Journal of Social Economics, Vol. 46, Nr. 8, Seite(n) 977-991, 2019 - Aufsatz / Paper in sonstiger referierter Fachzeitschrift
	 \item The “eternal character” of austerity measures in European crisis policies. Evidences from the Fiscal Compact discourse in Austria.: Pühringer S. , in Power, Kate; Ali, Tanweer; Lebduskova, Eva: Discourse Analysis and Austerity: Critical Studies from Economics and Linguistics, Routledge, London, Seite(n) 134-158, 2019 - Aufsatz / Paper in Sammelwerk (referiert)
	 \item Change and persistence in contemporary economics: Kapeller J., Meyer D. , Serie Special Issue Science in Context, Vol. 32, 2019 - Wissenschaftliche Fachzeitschrift (Mitherausgeberschaft)
	 \item Introduction: change and persistence in contemporary economics: Kapeller J., Meyer D. , in Science in Context, Vol. 32, Nr. 4, Seite(n) 357-360, 2019 - Aufsatz / Paper in SSCI-Zeitschrift
	 \item Holding Together what Belongs Together: A Strategy to Counteract Economic Polarisation in Europe: Kapeller J., Gräbner C., Heimberger P. , 2019 - Sonstige
	 \item Wirtschaftliche Polarisierung in Europa: Ursachen und Handlungsoptionen: Kapeller J., Gräbner C., Heimberger P. , Friedrich-Ebert Stiftung, Bonn, 2019 - Forschungsbericht: geförderte Forschung (sonstige überwieg. aus öff. Hand)
	 \item Hans Albert und die Kritik am Modell-Platonismus in den Wirtschaftswissenschaften: Kapeller J., Ferschli B. , in Franco, Giuseppe: Handbuch Karl Popper, Springer Fachbuch, Heidelberg, Seite(n) 733-749, 2019 - Aufsatz / Paper in Tagungsband (referiert)
	 \item Pluralism in Economics: Epistemological Rationales and Pedagogical Implementation: Kapeller J. , in Decker, Samuel; Elsner, Wolfram; Flechtner, Svenja: Advancing Pluralism in Teaching Economics, Routledge, London, Seite(n) 55-77, 2019 - Aufsatz / Paper in Sammelwerk (referiert)
	 \item Humankapital: Kapeller J. , in von Braunmühl, Claudia; Gerstenberger, Heide; Ptak, Ralf; Wichterich, Christa: ABC der globalen Unordnung. Von »Anthropozän« bis »Zivilgesellschaft«, VSA-Verlag, Hamburg, Seite(n) 120-121, 2019 - Aufsatz / Paper in Sammelwerk (referiert)
	 \item Das Gemeine an der Gemeinwohldebatte: Hirte K., Poppinga O. , in Wege für eine bäuerliche Zukunft – Zeitschrift der ÖBV/ Via Campesina Austria, Vol. 42, Nr. 3 (358), Seite(n) 7-9, 2019 - Aufsatz / Paper in nicht-referierter Fachzeitschrift
	 \item Die deutsche Agrarpolitik und Agrarökonomik. Entstehung und Wandel zweier ambivalenter Disziplinen: Hirte K. , Springer, Wiesbaden, 2019 - Monographie (Erstauflage)
	 \item Das dritte gossensche Gesetz: Hirte K. , in Hochmann, Lars; Graupe, Silja; Korbun, Thomas; Panther, Stephan; Schneidewind, Uwe: Möglichkeits¬wissen¬schaften. Ökonomie mit Möglichkeitssinn, Metropolis Verlag, Marburg, Seite(n) 133-176, 2019 - Aufsatz / Paper in Sammelwerk (referiert)
	 \item Beyond equilibrium: revisiting two-sided markets from an agent-based modelling perspective: Heinrich T., Gräbner C. , in International Journal of Computational Economics and Econometrics, Vol. 9, Nr. 3, Seite(n) 153-180, 2019 - Aufsatz / Paper in sonstiger referierter Fachzeitschrift
	 \item Dem wirtschaftlichen Abschwung entgegenwirken: Zur wichtigen Rolle der Fiskalpolitik: Heimberger P., Pekanov A. , 2019 - Sonstige
	 \item Den wirtschaftlichen Abschwung bekämpfen!: Heimberger P., Pekanov A. , in Der Standard, 2019 - Presseartikel / Medienberichte
	 \item What to do about divergence between EU countries? The problem of structural polarization: Heimberger P., Kapeller J. , 2019 - Sonstige
	 \item Auch Deutschland wird zum Opfer des "Outputlücken-Nonsens": Heimberger P., Kapeller J. , in Makronom, 2019 - Presseartikel / Medienberichte
	 \item Eine Strategie gegen die ökonomische Polarisierung Europas: Heimberger P., Gräbner C., Kapeller J. , in Makronom, 2019 - Presseartikel / Medienberichte
	 \item ‘Output gap nonsense': Understanding the budget conflict between the EC and Italy’s government: Heimberger P. , 2019 - Sonstige
	 \item Unemployment in Europe: What should be done?: Heimberger P. , 2019 - Sonstige
	 \item The current economic downturn in Europe must be seen in the context of a wider problem of economic polarisation: Heimberger P. , in Vienna Institute for International Economic, 2019 - Sonstige
	 \item Italien vs. EU-Kommission: Warum ein Defizitverfahren kontraproduktiv wäre: Heimberger P. , 2019 - Sonstige
	 \item How much space for fiscal expansion? Germany falls victim to 'output gap nonsense’: Heimberger P. , in Vienna Institute for International Economic (wiiw), 2019 - Sonstige
	 \item Arbeitslosigkeit in Europa: Was man tun könnte: Heimberger P. , 2019 - Sonstige
	 \item Wirtschaftlicher Abschwung: Relevanz und Gegenmaßnahmen: Heimberger P. , in Momentum Newsletter, 2019 - Presseartikel / Medienberichte
	 \item Welche Rolle spielt der „Outputlücken-Nonsense“?: Heimberger P. , in Makronom, 2019 - Presseartikel / Medienberichte
	 \item The danger of  'nonsense Output gaps': Heimberger P. , in Financial Times, 2019 - Presseartikel / Medienberichte
	 \item Arbeitsmarktinstitutionen, Kapitalakkumulation und Arbeitslosigkeit in OECD-Ländern, Wirtschaft und Gesellschaft: Heimberger P. , in Wirtschaft und Gesellschaft, Vol. 45, Nr. 1, 2019 - Aufsatz / Paper in sonstiger referierter Fachzeitschrift
	 \item Economic Polarisation in Europe: Causes and Policy Options: Gräbner C., Kapeller J., Heimberger P. , 2019 - Sonstige
	 \item Getting the Best of Both Worlds? Developing Complementary Equation-Based and Agent-Based Models: Gräbner C., Bale C., Furtado B., Alvarez-Pereira B., Gentile J., Henderson H., Lipari F. , in Computational Economics, Vol. 53, Nr. 2, Seite(n) 763-782, 2019 - Aufsatz / Paper in SSCI-Zeitschrift
	 \item Der Einfluss des Neoliberalismus auf österreichische Parteiprogramme: Grimm C. , 2019 - Sonstige
	 \item Ideas have Consequences: Eine vergleichende Analyse zur transformativen Rolle von Ideen.: Grimm C. , in Momentum Quarterly, Vol. 8, Nr. 4, Seite(n) 183-247, 2019 - Aufsatz / Paper in sonstiger referierter Fachzeitschrift
	 \item Deutungsrahmen der aktiven Arbeitsmarktpolitik: ein deutsch-österreichischer Vergleich von diskursiven Frames aus Anlass von 50 Jahren Arbeits(markt)förderungsgesetz: Griesser M. , in Momentum Quarterly, Vol. 8, Nr. 3, Seite(n) 166-182, 2019 - Aufsatz / Paper in sonstiger referierter Fachzeitschrift
	 \item Spielräume des Denkens: Graupe S., Ötsch W., Rommel F. , Metropolis, Marburg, 2019 - Wissenschaftliches Sammelwerk Mitherausgeberschaft (Erstauflage)
	 \item Spielräume des Denkens. Zur Einführung: Graupe S., Ötsch W., Rommel F. , in Graupe, Silja; Ötsch, Walter O.; Rommel, Florian: Spielräume des Denkens, Metropolis, Marburg, Seite(n) 9-32, 2019 - Aufsatz / Paper in Sammelwerk (nicht-referiert)
	 \item The heterogeneous relationship between income and inequality: a panel co-integration approach: Flechtner S., Gräbner C. , in Economics Bulletin, Vol. 39, Nr. 4, Seite(n) 2540–2549, 2019 - Aufsatz / Paper in sonstiger referierter Fachzeitschrift
	 \item "Ohne Effizienz geht es nicht". Ergebnisse einer qualitativ-empirischen Erhebung unter Studierenden der Volkswirtschaftslehre: Bäuerle L., Pühringer S., Ötsch W. , Serie FGW-Studien, Nr. 13, FGW, Düsseldorf, 2019 - Forschungsbericht: geförderte Forschung (Bund/Land/Gemeinden)
	 \item Antonino Palumbo and Alan Scott (2018): Remaking Market Society: A Critique of Social Theory and Political Economy in Neoliberal Times: Beyer K. , in ÖZS - Österreichische Zeitschrift für Soziologie, Vol. 44, Nr. 2, Seite(n) 249-252, 2019 - Rezension in sonstiger referierter Fachzeitschrift
	 \item Citation Patterns in Economics and Beyond: Assessing the Peculiarities of Economics from Two Scientometric Perspectives: Aistleitner M., Kapeller J., Steinerberger S. , in Science in Context, Vol. 32, Nr. 4, Seite(n) 361-380, 2019 - Aufsatz / Paper in SSCI-Zeitschrift
	 \item Auftragsvergabe, Leistungsqualität und Kostenintensität im Schienenpersonenverkehr: Aistleitner M., Grimm C., Kapeller J. : Momentum Kongress Paper, Seite(n) 1-56, 2019 - Aufsatz / Paper in Tagungsband (referiert)
\end{itemize} 
 \subsection{2018} 
 \begin{itemize} 
	 \item Netzwerke des Marktes : Ordoliberalismus als Politische Ökonomie: Ötsch W., Pühringer S., Hirte K. , Springer VS, Wiesbaden, 2018 - Monographie (Erstauflage)
	 \item Was ist eine Krise? Wie ökonomische Theorien Wahrnehmung formen: Ötsch W., Pühringer S. , in Kurswechsel, Nr. 4/2018, Seite(n) 7-17, 2018 - Aufsatz / Paper in nicht-referierter Fachzeitschrift
	 \item Einführung: Die Bedeutung von Bildern für das Denken: Ötsch W., Graupe S. , in Ötsch, Walter O.; Graupe, Silja: Macht der Bilder, Macht der Sprache, Angelika Lenz Verlag, Isenburg, Seite(n) 9-19, 2018 - Aufsatz / Paper in Sammelwerk (nicht-referiert)
	 \item Mythos Markt. Mythos Neoklassik. Das Elend des Marktfundamentalismus: Ötsch W. , Serie Kritische Studien zu Markt und Gesellschaft, Vol. 11, Metropolis, Marburg, 2018 - Monographie (Erstauflage)
	 \item Bilder des Rechtspopulismus: Ötsch W. , in Ötsch, Walter O.; Graupe, Silja: Macht der Bilder, Macht der Sprache, Angelika Lenz Verlag, Isenburg, Seite(n) 113-128, 2018 - Aufsatz / Paper in Sammelwerk (referiert)
	 \item Es wächst das Bewusstsein einer multiplen Krise: Ötsch W. , in Agora42 (Philosophisches Wirtschaftsmagazin), 2018 - Aufsatz / Paper in nicht-referierter Fachzeitschrift
	 \item Rechtspopulismus: Ein Gesellschaftsbild mit eskalierender Wirkung: Ötsch W. , in Salzburger Theologische Zeitschrift, Vol. 21, Nr. 1, Seite(n) 7-22, 2018 - Aufsatz / Paper in AHCI-Zeitschrift
	 \item Neoliberalism and Right-wing Populism: conceptual analogies: Pühringer S., Ötsch W. , in Forum for Social Economics, Vol. 47, Nr. 2, Seite(n) 192-203, 2018 - Aufsatz / Paper in sonstiger referierter Fachzeitschrift
	 \item Ökonomische Expertise und polit-ökonomische Machtstrukturen: Pühringer S., Liedl B. , in AK Kärnten: Welt aus den Fugen. Wie der Neoliberalismus unser Leben verändert., ÖGB-Verlag, Wien, Seite(n) 41-56, 2018 - Aufsatz / Paper in Tagungsband (nicht-referiert)
	 \item Die deutschsprachige Volkswirtschaftslehre: Pühringer S., Grimm C. , in beigewum.at, 2018 - Aufsatz / Paper in nicht-referierter Fachzeitschrift
	 \item Krisenbilder von ÖkonomInnen in der Presse: Pühringer S., Egger J. , in Walter Ötsch; Silja Graupe: Macht der Bilder, Macht der Sprache, Seite(n) 75-86, 2018 - Aufsatz / Paper in Tagungsband (nicht-referiert)
	 \item Die Krise als Katalysator für den Aufschwung des Rechtspopulismus: Pühringer S. , Vol. 47, 2018 - Sonstige
	 \item Politische und gesellschaftliche Wirkmächtigkeit von ÖkonomInnen-Netzwerken: Pühringer S. : Momentum Kongress Paper, Seite(n) 1-10, 2018 - Aufsatz / Paper in Tagungsband (referiert)
	 \item Die Positionierung von Studierenden an öffentlichen Hochschulen: Porak L. : Momentum Kongress Paper, Seite(n) 1-18, 2018 - Aufsatz / Paper in Tagungsband (referiert)
	 \item Government policies and financial crises: mitigation, postponement or prevention?: Landesmann M., Kapeller J., Mohr F., Schütz B. , in Cambridge Journal of Economics, Vol. 42, Nr. 2, Seite(n) 309-330, 2018 - Aufsatz / Paper in SSCI-Zeitschrift
	 \item Wie viel bringt die Vermögenssteuer? Neue Aufkommensschätzungen für Österreich.: Kapeller J., Ferschli B., Schütz B., Wildauer R. , in ISW Institut für Sozial- und Wirtschaftswissenschaften, in Wirtschafts- und Sozialwissenschaftliche Zeitschrift, Vol. 40, Nr. 1, Seite(n) 146-160, 2018 - Aufsatz / Paper in nicht-referierter Fachzeitschrift
	 \item Open strategy-making with crowds and communities: Comparing Wikimedia and Creative Commons: Kapeller J., Dobusch L. , in Long Range Planning, Vol. 51, Nr. 4, Seite(n) 561-579, 2018 - Aufsatz / Paper in SSCI-Zeitschrift
	 \item Ökonomische Effekte der Verkehrsreform des Landes Tirol: Kapeller J., Böck M., Schütz B., Zens G. , Johannes Kepler Universität, Linz, 2018 - Forschungsbericht über Auftragsforschung
	 \item The Top Journals Club in Economics: Kapeller J. , in Institute for New Economic Thinking (INET), Commentaries, 2018 - Aufsatz / Paper in nicht-referierter Fachzeitschrift
	 \item Heterodoxie in der Ökonomik: Hirte K., Thieme S. , in Schetsche, Michael; Schmied-Knittel, Ina: Heterodoxie. Konzepte, Traditionen, Figuren der Abweichung, Halem Verlag, Köln, 2018 - Aufsatz / Paper in Sammelwerk (referiert)
	 \item Was Märkte (nicht) mit Demokratie zu tun haben: Hirte K., Poppinga O. , in Wege für eine Bäuerliche Zukunft – Zeitschrift der ÖBV/ Via Campesina Austria, Vol. 41, Nr. 3 (353), Wien, Seite(n) 4-6, 2018 - Presseartikel / Medienberichte
	 \item Die Diskrepanz zwischen Reden und Handeln ist im ökonomischen Bereich besonders groß: Hirte K. , in Agora42 (Philosophisches Wirtschaftsmagazin), 2018 - Presseartikel / Medienberichte
	 \item Zeitlichkeit und Tauschfähigkeit bei Rosa Luxemburg und Joseph A. Schumpeter: Hirte K. , in Bies, Michael; Giacovelli, Sebastian; Langenohl, Andreas: Ästhetische Eigenzeiten von Tausch und Gabe, Wehrhahn Verlag, Hannover, 2018 - Aufsatz / Paper in Sammelwerk (referiert)
	 \item Warum Europa trotz Aufschwung ökonomisch weiter auseinander driftet: Heimberger P., Gräbner C., Kapeller J. , in Makronom, 2018 - Presseartikel / Medienberichte
	 \item Vier europäische Lehren aus den Turbulenzen in Italien: Heimberger P. , in Makronom, 2018 - Presseartikel / Medienberichte
	 \item Hilf dir selbst, dann hilft dir Deutschland: Fortschritt bei der Reform der Eurozone: Heimberger P. , in Der Standard, 2018 - Presseartikel / Medienberichte
	 \item Fiscal multipliers, unemployment and debt: Heimberger P. , Wirtschaftsuniversität Wien, 2018 - Dissertation
	 \item Work or Die? How Wage dependency determines the production process: Hager T., Rath J., Wimmler L. : Momentum Kongress Paper, Seite(n) 1-37, 2018 - Aufsatz / Paper in Tagungsband (referiert)
	 \item The dynamics of and on networks: Gräbner C., Heinrich T., Kudic M., Vermeulen B. , in wiiw Opinion Piece, Serie International Journal of Computational Economics and Econometrics, Vol. 8, 2018 - Wissenschaftliche Fachzeitschrift (Mitherausgeberschaft)
	 \item The dynamics of and on networks: an introduction: Gräbner C., Heinrich T., Kudic M., Vermeulen B. , in International Journal of Computational Economics and Econometrics, Vol. 8, Nr. 3/4, Seite(n) 229-241, 2018 - Aufsatz / Paper in sonstiger referierter Fachzeitschrift
	 \item Desintegration in Europa? Makroökonomische Divergenz und strukturelle Polarisierung: Gräbner C., Heimberger P., Kapeller J., Schütz B. : Momentum Kongress Paper, Serie Momentum quarterly, Seite(n) 1-32, 2018 - Aufsatz / Paper in Tagungsband (referiert)
	 \item To trust or to control: Informal value transfer systems and computational analysis in institutional economics: Gräbner C., Elsner W., Lascaux A. , in Journal of Economic Issues, Vol. 52, Nr. 2, Seite(n) 559-569, 2018 - Aufsatz / Paper in SSCI-Zeitschrift
	 \item Formal Approaches to Socio-economic Analysis - Past and Perspectives: Gräbner C. , in Forum for Social Economics, Vol. 47, Nr. 1, Seite(n) 32-63, 2018 - Aufsatz / Paper in sonstiger referierter Fachzeitschrift
	 \item How to Relate Models to Reality? An Epistemological Framework for the Validation and Verification of Computational Models: Gräbner C. , in Journal of Artificial Societies and Social Simulation, Vol. 21, Nr. 3, 2018 - Aufsatz / Paper in SSCI-Zeitschrift
	 \item Wirtschaftspolitische Positionen österreichischer Parteien im historischen Verlauf. Die Ausgestaltung österreichischer Parteiprogrammatiken hinsichtlich neoliberalen Gedankenguts: Grimm C. , in Momentum Quarterly, Vol. 7, Nr. 3, Seite(n) 136-154, 2018 - Aufsatz / Paper in sonstiger referierter Fachzeitschrift
	 \item Editorial: Freie Fahrt für reiche Burschen? Schwarz-Blau ist zurück!: Griesser M., Hofmann J. , in Kurswechsel, Nr. 3, 2018 - Editorial in Fachzeitschrift
	 \item Wachstum? Wohlstand und Lebensqualität!: Griesser M., Brand U. , in Momentum Quarterly, Vol. 7, Nr. 2, Seite(n) 53-72, 2018 - Aufsatz / Paper in sonstiger referierter Fachzeitschrift
	 \item Macht der Bilder, Macht der Sprache: Graupe S., Ötsch W. , Angelika Lenz Verlag, Isenburg, 2018 - Wissenschaftliches Sammelwerk Herausgeberschaft (Erstauflage)
	 \item Zur Verteilung und Klassenstruktur der Österreichischen Vermögen: Ferschli B., Kapeller J., Wildauer R. : Momentum Kongress Paper, Seite(n) 1-55, 2018 - Aufsatz / Paper in Tagungsband (referiert)
	 \item Freiheitliche Flügelkämpfe? (Historische) Konfliktlinien in der FPÖ: Beyer K., Pühringer S. , in BEIGEWUM, in Kurswechsel, Nr. 3, Seite(n) 19-27, 2018 - Aufsatz / Paper in sonstiger referierter Fachzeitschrift
	 \item Netzwerke, Paradigmen, Attitüden. Der deutsche Sonderweg im Fokus. Paradigmatische Ausrichtung und politische Orientierung von deutschen und US-amerikanischen Ökonomi_nnen im Vergleich: Beyer K., Grimm C., Kapeller J., Pühringer S. , Nr. 7, FGW, Düsseldorf, 2018 - Forschungsbericht: geförderte Forschung (sonstige überwieg. aus öff. Hand)
	 \item The Power of Scientometrics and the Development of Economics: Aistleitner M., Kapeller J., Steinerberger S. , in Journal of Economic Issues, Vol. 52, Nr. 3, Routledge, Seite(n) 816-834, 2018 - Aufsatz / Paper in SSCI-Zeitschrift
\end{itemize} 
 \subsection{2017} 
 \begin{itemize} 
	 \item Right-wing populism and market-fundamentalism. Two mutually reinforcing threats to democracy in the 21st century: Ötsch W., Pühringer S. , in Journal of Language and Politics, Vol. 16, Nr. 4, Seite(n) 497-509, 2017 - Aufsatz / Paper in SSCI-Zeitschrift
	 \item Zunehmende Ungleichheit der Vermögensverteilung: Rezension von Michael Schneider, Mike Pottenger und John E. King „The Distribution of Wealth – Growing Inequality?“: Schütz B. , in Wirtschaft und Gesellschaft, Vol. 43, Nr. 3, Seite(n) 449-451, 2017 - Rezension in nicht-referierter Fachzeitschrift
	 \item Was denken (zukünftige) ÖkonomInnen?: Einblicke in die politische und gesellschaftliche Wirkmächtigkeit ökonomischen Denkens.: Pühringer S., Bäuerle L., Engarntner T. , in GWP - Gesellschaft, Wirtschaft, Politik, Vol. 66, Nr. 4, Seite(n) 547-556, 2017 - Aufsatz / Paper in sonstiger referierter Fachzeitschrift
	 \item Migration: It's still the economy, stupid!: Pühringer S. , in Der Standard, 2017 - Presseartikel / Medienberichte
	 \item The success story of ordoliberalism as guiding principle of German economic policy: Pühringer S. , in Hien, Josef; Joerges, Christian: Ordoliberalism. Law and the rule of economics, Hart Publishing, Oxford, Portland, Seite(n) 134-158, 2017 - Aufsatz / Paper in Sammelwerk (referiert)
	 \item Stability , fairness and random walks in the bargaining problem: Kapeller J., Steinerberger S. , in Physica A: Statistical Mechanics and its Applications, Vol. 488, Elsevier, Seite(n) 60-71, 2017 - Aufsatz / Paper in SCI-Expanded-Zeitschrift
	 \item Internationale Tendenzen und Potentiale der Vermögensbesteuerung: Kapeller J., Schütz B., Springholz F. , in Dimmel, Nikolaus; Hofmann, Julia; Schenk, Martin; Schürz, Martin: Handbuch Reichtum – Neue Erkenntnisse aus der Ungleichheitsforschung, Studienverlag, Wien, Seite(n) 477-492, 2017 - Aufsatz / Paper in Sammelwerk (referiert)
	 \item Zum Profil der deutschsprachigen Volkwirtschaftslehre. Paradigmatische Ausrichtung und politische Orientierung deutschsprachiger Ökonom_innen: Kapeller J., Pühringer S., Grimm C. , Nr. 2, FGW-Studien, Düsseldorf, 2017 - Forschungsbericht: geförderte Forschung (sonstige überwieg. aus öff. Hand)
	 \item Bestände und Konzentration privater Vermögen in Österreich: Kapeller J., Ferschli B., Schütz B., Wildauer R. , in Abteilung Wirtschaftswissenschaft der AK Wien, in Materialen zu Wirtschaft und Gesellschaft, Nr. 167, 2017 - Aufsatz / Paper in nicht-referierter Fachzeitschrift
	 \item Bestände und Konzentration privater Vermögen in Österreich: Kapeller J., Ferschli B., Schütz B., Wildauer R. , in Arbeiterkammer Wien, in Wirtschaft und Gesellschaft, Vol. 43, Nr. 4, Seite(n) 499-534, 2017 - Aufsatz / Paper in sonstiger referierter Fachzeitschrift
	 \item Delayed by outsourcing? Zur Stabilität des Kapitalismus im 21. Jahrhundert (Doppelrezension): Kapeller J. , in Hrsg. v. Hollstein, Betina / Schimank, Uwe / Struck, Olaf / Weiß, Anja, in Soziologische Revue, Vol. 40, Nr. 4, DeGruyter, Seite(n) 547–555, 2017 - Rezension (AHCI-Zeitschrift)
	 \item Zur Performativität ökonomischen Wissens und aktuellen ÖkonomInnen-Netzwerken in Deutschland: Hirte K., Pühringer S. , in Maeße, Jens; Pahl, Hanno; Sparsam, Jan: Die Innenwelt der Ökonomie. Wissen, Macht und Performativität in der Wirtschaftswissenschaft, Springer VS Verlag, Wiesbaden, Seite(n) 363-390, 2017 - Aufsatz / Paper in Tagungsband (referiert)
	 \item Agrarpolitik und Agrarökonomie. Zur Ambivalenz zweier wissenschaftlicher Disziplinen: Hirte K. , Universität Jena, 2017 - Habilitationsschrift
	 \item Zur Performativität in den Wirtschaftswissenschaften. Kernaussagen, Anwendungspotentiale und Grenzen eines Konzepts: Hirte K. , in Pfriem, Reinhard; Schneidewind, Uwe: Transformative Wirtschaftswissenschaft im Kontext nachhaltiger Entwicklung, Metropolis Verlag, Marburg, 2017 - Aufsatz / Paper in Sammelwerk (referiert)
	 \item The NAIRU determinants: what’s structural about unemployment in Europe?: Heimberger P., Kapeller J., Schütz B. , in Journal of Policy Modeling, Vol. 39, Nr. 5, Seite(n) 883-908, 2017 - Aufsatz / Paper in SSCI-Zeitschrift
	 \item Wie ein makroökonomisches Modell die Spaltung der Eurozone befördert: Heimberger P., Kapeller J. , 2017 - Sonstige
	 \item The performativity of potential output: Pro-cyclicality and path dependency in coordinating European fiscal policies: Heimberger P., Kapeller J. , in Review of International Political Economy, Vol. 24, Nr. 5, Seite(n) 904-928, 2017 - Aufsatz / Paper in SSCI-Zeitschrift
	 \item Österreichs Staatsausgabenstrukturen im europäischen Vergleich: Heimberger P. , 2017 - Sonstige
	 \item Österreichs Bildungs-, Gesundheits- und Sozialausgaben im europäischen Vergleich: Wenn der Staat spart, kann das für private Haushalte teuer werden: Heimberger P. , 2017 - Sonstige
	 \item Weniger Staatsausgaben: Abbau des Sozialstaats und Vertiefung von Wirtschaftskrisen: Heimberger P. , 2017 - Sonstige
	 \item Vorsicht bei Ländervergleichen – insbesondere bei Staatsausgaben!: Heimberger P. , 2017 - Sonstige
	 \item Soll der Staat bei Bildung, Gesundheit und Sozialem kürzen? Austeritätspolitik seit der Finanzkrise im Vergleich: Heimberger P. , 2017 - Sonstige
	 \item Zerfällt Europa? Trump als Entscheidungsbeschleuniger: Heimberger P. , in Der Standard, 2017 - Presseartikel / Medienberichte
	 \item Europa am Scheidepunkt: Österreichs wichtige Rolle: Heimberger P. , in Der Standard, 2017 - Presseartikel / Medienberichte
	 \item Die EU braucht einen wirtschaftspolitischen Kurswechsel: Heimberger P. , in Der Standard, 2017 - Presseartikel / Medienberichte
	 \item Did Fiscal Consolidation Cause the Double-Dip Recession in the Euro Area?: Heimberger P. , in Review of Keynesian Economics, Vol. 5, Nr. 3, Seite(n) 439-458, 2017 - Aufsatz / Paper in SSCI-Zeitschrift
	 \item The Micro-Macro Link in Heterodox Economics: Gräbner C., Kapeller J. , in Tae-Hee Jo, Lynne Chester Carlo and D'Ippoliti: The Routledge Handbook of Heterodox Economics, Routledge, Seite(n) 145-159, 2017 - Aufsatz / Paper in Sammelwerk (referiert)
	 \item Von Onlineplattformen und mittelalterlichen Märkten - Gleichgewichtsmodelle und agentenbasierte Modellierung zweiseitiger Märkte: Gräbner C., Heinrich T. , in TATuP - Zeitschrift für Technikfolgenabschätzung in Theorie und Praxis, Vol. 26, Nr. 3, Seite(n) 23-29, 2017 - Aufsatz / Paper in sonstiger referierter Fachzeitschrift
	 \item Die Rolle des Gleichgewichtskonzepts in der mikroökonomischen Ausbildung: Gräbner C. , in Till van Treek, Janina Urban: Wirtschaft neu denken, iRIGHTS media, Berlin, Seite(n) 60-73, 2017 - Aufsatz / Paper in Sammelwerk (referiert)
	 \item Dealing adequately with the political element in formal modelling: Gräbner C. , in Katsikides, Savas and Hanappi, Hardy and Scholz-Wäckerle, Manuel: Theory and Method of Evolutionary Political Economy, Routledge, New York, Seite(n) 236-254, 2017 - Aufsatz / Paper in Sammelwerk (referiert)
	 \item The Complexity of Economies and Pluralism in Economics: Gräbner C. , in Journal of Contextual Economics, Vol. 137, Nr. 3, Seite(n) 193-225, 2017 - Aufsatz / Paper in sonstiger referierter Fachzeitschrift
	 \item The Complementary Relationship Between Institutional and Complexity Economics: The Example of Deep Mechanismic Explanations: Gräbner C. , in Journal of Economic Issues, Vol. 51, Nr. 2, Seite(n) 392-400, 2017 - Aufsatz / Paper in SSCI-Zeitschrift
	 \item Paradigmatische Homogenität? Aktueller Status und Zukunftsperspektiven der Ökonomik in Deutschland und den USA: Grimm C. : Momentum Kongress Paper, Seite(n) 1-16, 2017 - Aufsatz / Paper in Tagungsband (referiert)
	 \item Von der sozialen Neuzusammensetzung zur gewerkschaftlichen Erneuerung? MigrantInnen als Zielgruppe der österreichischen Gewerkschaftsbewegung: Griesser M., Sauer B. , in ÖZS - Österreichische Zeitschrift für Soziologie, Seite(n) 147-166, 2017 - Aufsatz / Paper in sonstiger referierter Fachzeitschrift
	 \item Rezension von „Monika Burmester, Emma Dowling & Norbert Wohlfahrt (Hg.) (2017): Privates Kapital für soziale Dienste? Wirkungsorientiertes Investment und seine Folgen für die Soziale Arbeit“: Griesser M. , in Soziales Kapital, Seite(n) 264-267, 2017 - Rezension in nicht-referierter Fachzeitschrift
	 \item sezonieri.at: Kollektive Handlungsfähigkeit von ErntearbeiterInnen in Österreich: Griesser M. , in Schmidjell, Cornelia; Sedmak, Clemens; Koch, Andreas; Kapferer, Elisabeth; Gaisbauer, Helmut P.; Bogner, Stefan; Wimmer, Bernd: Lesebuch Soziale Ausgrenzung III, Mandelbaum Verlag, Wien, Seite(n) 89-92, 2017 - Aufsatz / Paper in Sammelwerk (referiert)
	 \item Images and imaginaries of unemployed people. Discursive shifts in the transition from active to activating labour market policies in Germany: Griesser M. , in Critical Social Policy, Seite(n) [online first], 2017 - Aufsatz / Paper in SSCI-Zeitschrift
	 \item Citation Patterns in Economics and Beyond: Assessing the Peculiarities of Economics from Two Scientometric Perspectives: Aistleitner M., Kapeller J., Steinerberger S. : Momentum Kongress Paper, Seite(n) 1-22, 2017 - Aufsatz / Paper in Tagungsband (referiert)
\end{itemize} 
 \subsection{2016} 
 \begin{itemize} 
	 \item Ökonomisches Denken, Rechtspopulismus und Rechtsextremismus: Ötsch W. , in Makroskop, 2016 - Presseartikel / Medienberichte
	 \item Markt-Glauben, Klima-Krise und Katastrophen-Leugnung, Teil 1-4: Ötsch W. , in Makroskop, 2016 - Presseartikel / Medienberichte
	 \item Der Blick von oben und der Blick von unten: Ötsch W. , in Makroskop, 2016 - Presseartikel / Medienberichte
	 \item Geld und Raum. Anmerkungen zum Homogenisierungsprogramm der beginnenden Neuzeit: Ötsch W. , in Brodbeck, Karl-Heinz; Graupe, Silja: Geld! Welches Geld? Geld als Denkform, Metropolis Verlag, Marburg, Seite(n) 71-101, 2016 - Aufsatz / Paper in Sammelwerk (referiert)
	 \item Die neoliberale Utopie als Ende aller Utopien: Ötsch W. , in Pittl, Sebastian; Prüller-Jagenteufel; Gunter: Unterwegs zu einer neuen ‚Zivilisation geteilter Genügsamkeit‘. Perspektiven utopischen Denkens 25 Jahre nach dem Tod Ignacio Ellacurías, Vandenhoeck & Ruprecht uni press, Wien, Seite(n) 105-119, 2016 - Aufsatz / Paper in Sammelwerk (referiert)
	 \item Die Politische Ökonomie „des“ Marktes. Eine Zusammenfassung zur Wirkungsgeschichte von Friedrich A. Hayek: Ötsch W. , in Kapeller, Jakob; Pühringer, Stephan; Hirte, Katrin; Ötsch, Walter O.: Ökonomie! Welche Ökonomie? Stand und Status der Wirtschaftswissenschaften, Metropolis Verlag, Marburg, Seite(n) 19-50, 2016 - Aufsatz / Paper in Tagungsband (nicht-referiert)
	 \item Die Widersprüche des Mister Perfect: Berechnung – Beherrschung – Perfektion: Ötsch W. , in Agora42 (Philosophisches Wirtschaftsmagazin), Nr. 01/2017, Seite(n) 18-23, 2016 - Aufsatz / Paper in nicht-referierter Fachzeitschrift
	 \item Populismus und Demagogie. Mit Beispielen von Jörg Haider, Heinz–Christian Strache und Frank Stronach: Ötsch W. , in Foreign Theoretical Trends, Nr. 10, Seite(n) 39-46, 2016 - Aufsatz / Paper in sonstiger referierter Fachzeitschrift
	 \item Imaginative Grundlagen bei Adam Smith. Aspekte von Bildlichkeit und ihrem Verlust in der Geschichte der Ökonomik: Ötsch W. , in Allgemeine Zeitschrift für Philosophie, Vol. 41, Nr. 3, Seite(n) 315-340, 2016 - Aufsatz / Paper in AHCI-Zeitschrift
	 \item Errata in 'The Political Economy of the Kuznets Curve': Álvarez Pereira B., Henderson H., Lipari F., Furtado B., Bale C., Gräbner C., Gentile J. , in Review of Development Economics, Vol. 20, Nr. 4, Seite(n) 817-819, 2016 - Aufsatz / Paper in SSCI-Zeitschrift
	 \item Economic Complexity and Trade-Offs in Policy Decisions: Schwardt H., Gräbner C., Heinrich T., Cordes C., Schwesinger G. , in Gräbner, Claudius; Heinrich, Torsten; Schwardt, Henning: Policy Implications of Evolutionary and Institutional Economics, Routledge, London, New York, Seite(n) 3-19, 2016 - Aufsatz / Paper in Sammelwerk (referiert)
	 \item Replik zur Replik: Von Vorwürfen der Unwissenschaftlichkeit: Pühringer S., Stelzer-Orthofer C. , in SWS-Rundschau - Sozialwissenschaftliche Studiengesellschaft Rundschau, Vol. 3, Seite(n) 447-449, 2016 - Aufsatz / Paper in nicht-referierter Fachzeitschrift
	 \item Neoliberale Think Tanks als (neue) Akteure in österreichischen gesellschafts- und sozialpolitischen Diskursen. Das Beispiel des Hayek Institut und der Agenda Austria: Pühringer S., Stelzer-Orthofer C. , in SWS-Rundschau - Sozialwissenschaftliche Studiengesellschaft Rundschau, Vol. 56, Nr. 1, Seite(n) 75-96, 2016 - Aufsatz / Paper in nicht-referierter Fachzeitschrift
	 \item Wie krank ist unser Wirtschaftssystem? Krisen als Krankheiten im ökonomischen Diskurs: Pühringer S., Egger J. , in Kuckuck. Notizen zur Alltagskultur, Vol. 31, Nr. 1, Seite(n) 32-37, 2016 - Aufsatz / Paper in sonstiger referierter Fachzeitschrift
	 \item Ökonomisches Denken in der Krise: Pühringer S. , 2016 - Sonstige
	 \item Still the queens of social sciences? Economists as “public intellectuals” in/after the crisis.: Pühringer S. , in International Conference in Contemporary Social Sciences (Conference Proceedings): Crisis and the social sciences: New challenges and perspectives, Seite(n) 507-528, 2016 - Aufsatz / Paper in Tagungsband (referiert)
	 \item Agenda Austria: Diskursstrategien einer neoliberalen Reformagenda: Pühringer S. : Momentum Kongress Paper, Seite(n) 1-21, 2016 - Aufsatz / Paper in Tagungsband (referiert)
	 \item Emergent Phenomena in Scientific Publishing: A Simulation Exercise: Kapeller J., Steinerberger S. , in Research Policy, Vol. 45, Nr. 10, Seite(n) 1945–1952, 2016 - Aufsatz / Paper in SSCI-Zeitschrift
	 \item From free to civilized trade: a European perspective: Kapeller J., Schütz B., Tamesberger D. , in Review of Social Economy, Vol. 74, Nr. 3, Seite(n) 320-328, 2016 - Aufsatz / Paper in sonstiger referierter Fachzeitschrift
	 \item Verteilungstendenzen im Kapitalismus. Globale Perspektiven: Kapeller J., Schütz B. , in Bundesarbeitskammer: Die Verteilungsfrage. Von Reichtum, Krisen und Ablenkungsmanövern, ÖGB-Verlag, Wien, Seite(n) 49-54, 2016 - Aufsatz / Paper in Sammelwerk (referiert)
	 \item Evolutionary Political Economy and the Complexity of Economic Policy: Kapeller J., Scholz-Wäckerle M. , in Gräbner,  Claudius / Heinrich, Torsten / Schwardt, Henning: Policy  Implications of Recent Advances in Evolutionary and Institutional  Economics, Routledge, London, Seite(n) 99-122, 2016 - Aufsatz / Paper in Sammelwerk (referiert)
	 \item Ökonomie! Welche Ökonomie? Stand und Status der Wirtschaftswissenschaften: Kapeller J., Pühringer S., Hirte K., Ötsch W. , Metropolis, Marburg, 2016 - Tagungsband Mitherausgeberschaft (Erstauflage)
	 \item Spezialisierung, Stratifikation und internationale Wirtschaft: Verteilung, Arbeitsteilung und Klassenlagen aus globaler Perspektive: Kapeller J., Heimberger P. : Momentum Kongress Paper, Seite(n) 1-20, 2016 - Aufsatz / Paper in Tagungsband (referiert)
	 \item Ein philosophischer Blick auf die Grundlagen internationaler Ökonomie.: Kapeller J. , in Till van Treeck, Janina Urban: Wirtschaft neu denken – blinde Flecken der Lehrbuchökonomie, Seite(n) 108-116, 2016 - Aufsatz / Paper in Sammelwerk (nicht-referiert)
	 \item Internationaler Freihandel: Theoretische Ausgangspunkte und empirische Folgen: Kapeller J. , in Wirtschafts- und Sozialwissenschaftliche Zeitschrift, Vol. 39, Nr. 1, Seite(n) 99-122, 2016 - Aufsatz / Paper in nicht-referierter Fachzeitschrift
	 \item Agrarpolitik und Arbeit – der Einfluss europäischer Agrarpolitikmaßnahmen auf die Arbeit im Agrarsektor: Hirte K., Kuschel S. , in Ahlert, Maximilian; Fiederer, Franca; Varelmann, Katharina; Kuschel, Sarah; Ewers, Sylvia; Politor, Merlin; Stamp, Katharina: Frohes Schaffen!? Arbeit in der Landwirtschaft, Universtity Press, Kassel, Seite(n) 9-15, 2016 - Aufsatz / Paper in Tagungsband (nicht-referiert)
	 \item Vorwort im Tagungsband Ökonomie! Welche Ökonomie?: Hirte K., Kapeller J., Pühringer S., Ötsch W. : Ökonomie! Welche Ökonomie?, Metropolis Verlag, Marburg, Seite(n) 2-10, 2016 - Aufsatz / Paper in Tagungsband (nicht-referiert)
	 \item Netzwerke im Internet – eine neue kritische Öffentlichkeit? Das Beispiel Guttenberg: Hirte K. , in Imhof, Kurt; Welz, Frank; Fleck, Christian; Vobruba, Georg: Neuer Strukturwandel der Öffentlichkeit. Verhandlungen des dritten gemeinsamen Kongresses der Deutschen, Österreichischen und Schweizerischen Gesellschaft für Soziologie, Springer, Wiesbaden, Seite(n) 1-16, 2016 - Aufsatz / Paper in Sammelwerk (referiert)
	 \item Die „Landnahme“-These von Rosa Luxemburg – empirisch beobachtbar, aber theoretisch falsifiziert?: Hirte K. , in Kapeller, Jakob; Pühringer, Stephan; Hirte, Katrin; Ötsch, Walter: Ökonomie! Welche Ökonomie?, Metropolis Verlag, Marburg, Seite(n) 273-313, 2016 - Aufsatz / Paper in Tagungsband (nicht-referiert)
	 \item Agrarische Regelungspolitik und die drei agrarpolitischen „Syndrome“: Hirte K. , in Via Campesina Austria, in Wege für eine bäuerliche Zukunft – Zeitschrift der ÖBV/ Via Campesina Austria, Vol. 39, Nr. 3 (343), Seite(n) 10-11, 2016 - Aufsatz / Paper in nicht-referierter Fachzeitschrift
	 \item How economic policy drives European (dis)integration: Heimberger P., Kapeller J. , in Institute for New Economic Thinking (INET), 2016 - Aufsatz / Paper in nicht-referierter Fachzeitschrift
	 \item Mehr öffentliche Investitionen sind sinnvoll und erforderlich: Heimberger P. , 2016 - Sonstige
	 \item Trumps Team: Politik von und für Vermögende: Heimberger P. , in Die Presse, 2016 - Presseartikel / Medienberichte
	 \item Investitionen gegen die Dauerkrise im Euroland: Heimberger P. , in Der Standard, 2016 - Presseartikel / Medienberichte
	 \item Austeritätspolitik in der Eurozone: Ein Schuss ins eigene Knie: Heimberger P. , in Makronom, 2016 - Presseartikel / Medienberichte
	 \item Wirtschaftliche Stagnation als "neue Normalsituation"?: Heimberger P. , in Wirtschaft und Gesellschaft, Vol. 42, Nr. 2, Seite(n) 356-361, 2016 - Rezension in nicht-referierter Fachzeitschrift
	 \item Minsky, die globle Finanzkrise und der nächste Finanz-Crash: Heimberger P. , in Wirtschaft und Gesellschaft, Vol. 42, Nr. 3, Seite(n) 515-520, 2016 - Rezension in nicht-referierter Fachzeitschrift
	 \item Helikoptergeld zur Überwindung der Wachstumsprobleme in Europa?: Heimberger P. , in AK Wien, in Wirtschaft und Gesellschaft, Vol. 42, Nr. 4, Wien, Seite(n) 690-695, 2016 - Rezension in nicht-referierter Fachzeitschrift
	 \item Die aktuelle Krise im wirtschaftshistorischen Vergleich mit der Großen Depression der 1930er-Jahre: Heimberger P. , in Wirtschaft und Gesellschaft, Vol. 42, Nr. 1, Seite(n) 161-173, 2016 - Rezension in nicht-referierter Fachzeitschrift
	 \item Die Macht ökonomischer Modelle am Beispiel des »Potential Output«-Modells der Europäischen Kommission: Heimberger P. : Momentum Kongress Paper, Seite(n) 1-9, 2016 - Aufsatz / Paper in Tagungsband (referiert)
	 \item Warum die Volkswirtschaften der Eurozone den USA und Großbritannien seit der Finanzkrise hinterherhinken: Zur Rolle von Unterschieden in der Geld– und Fiskalpolitik: Heimberger P. , in Vienna Institute for International Economic (wiiw), in Studies Research Report, Nr. 5, Wien, 2016 - Aufsatz / Paper in nicht-referierter Fachzeitschrift
	 \item Das "strukturelle Defizit" in der österreichischen Budgetpolitik: Berechnungsprobleme, Revisionen und wirtschaftspolitische Relevanz: Heimberger P. , in Wirtschaft und Gesellschaft, Vol. 42, Nr. 3, Seite(n) 451-464, 2016 - Aufsatz / Paper in nicht-referierter Fachzeitschrift
	 \item Policy Implications of Recent Advances in Evolutionary and Institutional Economics: Gräbner C., Heinrich T., Schwardt H. , Routledge, London, New York, 2016 - Wissenschaftliches Sammelwerk Herausgeberschaft (Erstauflage)
	 \item Introduction: Gräbner C., Heinrich T., Schwardt H. , in Gräbner, Claudius; Heinrich, Torsten; Schwardt, Henning: Policy Implications of Recent Advances in Evolutionary and Institutional Economics, Routledge, London, New York, Seite(n) xxi-xxx, 2016 - Aufsatz / Paper in Sammelwerk (nicht-referiert)
	 \item Agent-based computational models - a formal heuristic for institutionalist pattern modelling?: Gräbner C. , in Journal of Institutional Economics, Vol. 12, Nr. 1, Seite(n) 241-261, 2016 - Aufsatz / Paper in SSCI-Zeitschrift
	 \item Wahrheit und Ökonomie: Grimm C., Kapeller J. , in Kurswechsel, Nr. 1, Seite(n) 18-29, 2016 - Aufsatz / Paper in nicht-referierter Fachzeitschrift
	 \item Postdemokratie, Machtverhältnisse und Ökonomie: Grimm C. : Momentum Kongress Paper, Seite(n) 1-22, 2016 - Aufsatz / Paper in Tagungsband (referiert)
	 \item Verankerung wohlstandorientierter Politik. Working Paper der Kammer für Arbeiter und Angestellte für Wien, Reihe „Materialien zu Wirtschaft und Gesellschaft“, Nr. 165: Griesser M., Brand U. , 2016 - Sonstige
	 \item Dialog nicht erwünscht: Graupe S., Ötsch W. , in Forschung & Lehre, Vol. 23, Nr. 11, Seite(n) 1000, 2016 - Rezension in sonstiger referierter Fachzeitschrift
	 \item Correcting for the Missing Rich: An Application to Wealth Survey Data: Eckerstorfer P., Halak J., Kapeller J., Schütz B., Springholz F., Wildauer R. , in Review of Income and Wealth, Vol. 62, Nr. 4, Seite(n) 605-627, 2016 - Aufsatz / Paper in SSCI-Zeitschrift
	 \item Relationships are Constructed from Generalized Unconscious Social Images Kept in Steady Locations in Mental Space: Derks L., Ötsch W., Walker W. , in Journal of Experiential Psychotherapy, Vol. 19, Nr. 1, Seite(n) 3-16, 2016 - Aufsatz / Paper in sonstiger referierter Fachzeitschrift
	 \item Das europäische Schattenbankensystem: Bestandsaufnahme und gegenwärtige Entwicklungen: Beyer K., Bräutigam L. , 2016 - Sonstige
	 \item Das europäische Schattenbankensystem – Typologisierung und die Bewertung regulatorischer Initiativen auf europäischer Ebene: Beyer K., Bräutigam L. , Serie Materialien zu Wirtschaft und Gesellschaft. Working Paper-Reihe der AK Wien, Nr. 154, Arbeiterkammer Wien, 2016 - Fachbuch als Band einer Schriftenreihe/Serie (Erstauflage)
	 \item Perspektiven für eine nachhaltige Automobilindustrie: Aistleitner M. : Momentum Kongress Paper, Seite(n) 1-27, 2016 - Aufsatz / Paper in Tagungsband (referiert)
\end{itemize} 
 \subsection{2015} 
 \begin{itemize} 
	 \item Markt! Welcher Markt? Der interdisziplinäre Diskurs um Märkte und Marktwirtschaft: Ötsch W., Hirte K., Pühringer S., Bräutigam L. , Metropolis, Marburg, 2015 - Tagungsband Mitherausgeberschaft (Erstauflage)
	 \item Schönheit und Macht. Drei Beispiele aus der Kulturgeschichte: Ötsch W. , in Ridler, Gerda: Mythos Schönheit. Facetten des Schönen in Natur, Kunst und Gesellschaft, Hatje Cantz Verlag, Stuttgart, Seite(n) 259-263, 2015 - Aufsatz / Paper in Sammelwerk (nicht-referiert)
	 \item Ökonomie und Moral. Eine kurze Theoriegeschichte: Ötsch W. , in Seckauer, Hansjörg; Stelzer-Orthofer, Christine; Kepplinger, Brigitte: Das Vorgefundene und das Mögliche. Beiträge zur Gesellschafts- und Sozialpolitik zwischen Ökonomie und Moral, Mandelbaum Verlag, Wien, Seite(n) 100-110, 2015 - Aufsatz / Paper in Sammelwerk (referiert)
	 \item Markt und Markttheorie. Vorwort und Überblick: Ötsch W. , in Ötsch, Walter O.; Hirte, Katrin; Pühringer, Stephan; Bräutigam, Lars: Markt! Welcher Markt? Der interdisziplinäre Diskurs um Märkte und Marktwirtschaft, Metropolis Verlag, Marburg, Seite(n) 7-24, 2015 - Aufsatz / Paper in Tagungsband (nicht-referiert)
	 \item The financial crisis as a heart attack: Discourse profiles of economists in the financial crisis: Pühringer S., Hirte K. , in Journal of Language and Politics, Vol. 14, Nr. 4, Seite(n) 599-626, 2015 - Aufsatz / Paper in SSCI-Zeitschrift
	 \item Märkte als Richter: Zur Dominanz neoliberaler Krisennarrative: Pühringer S. , in Ksoe-Dossier (Ksoe Nachrichten der Katholischen Sozialakademie), Nr. 1, Seite(n) 1-3, 2015 - Presseartikel / Medienberichte
	 \item The strange non-crisis of economics. Economic crisis and the crisis policies in economic and political discourses.: Pühringer S. , Universität Linz, 2015 - Dissertation
	 \item Marktmetaphoriken in Krisennarrativen von Angela Merkel.: Pühringer S. , in Ötsch, Walter/Hirte, Katrin/Pühringer, Stephan/Bräutigam, Lars: Markt! Welcher Markt? Der interdisziplinäre Diskurs um Märkte und Marktwirtschaft., Metropolis, Marburg, Seite(n) 229-252, 2015 - Aufsatz / Paper in Tagungsband (referiert)
	 \item Kontinuitäten neoliberaler Wirtschaftspolitik. Die Austeritätsdebatte als Spiegelbild diskursiver Machtverwerfungen innerhalb der Ökonomik: Pühringer S. , in Marterbauer, Markus/Mesch, Michael/Rehm, Miriam/Reiterlechner, Christine: Das Scheitern des neoklassischen Paradigmas – Wirtschaftspolitik in der EU, ÖGB Verlag, Wien, Seite(n) 159-174, 2015 - Aufsatz / Paper in Tagungsband (referiert)
	 \item “Harte” Sanktionen für “budgetpolitische Sünder”. Kritische Diskursanalyse der Debatte zum Fiskalpakt in meinungsbildenden österreichischen Qualitätsmedien.: Pühringer S. , in Momentum Quarterly, Vol. 4, Nr. 1, Seite(n) 23-41, 2015 - Aufsatz / Paper in sonstiger referierter Fachzeitschrift
	 \item Markets as “ultimate judges” of economic policies - Angela Merkel´s discourse profile during the economic crisis and the European crisis policies.: Pühringer S. , in On the Horizon, Vol. 23, Nr. 3, Seite(n) 246-259, 2015 - Aufsatz / Paper in sonstiger referierter Fachzeitschrift
	 \item Von freien zu zivilisierten Märkten. Ein New Deal für die europäische Handelspolitik: Kapeller J., Schütz B., Tamesberger D. , 2015 - Sonstige
	 \item Moralität, Wettbewerb und internationaler Handel: Eine europäische Perspektive: Kapeller J., Schütz B., Tamesberger D. , in Seckauer, Hansjörg; Stelzer-Orthofer, Christine; Kepplinger, Brigitte: Das Vorgefundene und das Mögliche. Beiträge zur Gesellschafts- und Sozialpolitik zwischen Ökonomie und Moral. Festschrift für Josef Weidenholzer, Mandelbaum Verlag, Wien, Seite(n) 213-227, 2015 - Aufsatz / Paper in Sammelwerk (referiert)
	 \item Verteilungstendenzen im Kapitalismus: Globale Perspektiven: Kapeller J., Schütz B. , 2015 - Sonstige
	 \item Verteilungstendenzen im Kapitalismus: Kapeller J., Schütz B. : Momentum Kongress Paper, Seite(n) 1-21, 2015 - Aufsatz / Paper in Tagungsband (referiert)
	 \item Verteilungstendenzen im Kapitalismus - Nationale und globale Perspektiven: Kapeller J., Schütz B. , in Kurswechsel, Nr. 2/2015, Seite(n) 54-68, 2015 - Aufsatz / Paper in nicht-referierter Fachzeitschrift
	 \item Conspicuous Consumption, Inequality and Debt: The Nature of Consumption-driven Profit-led Regimes: Kapeller J., Schütz B. , in Metroeconomica, Vol. 66, Nr. 1, Seite(n) 51-70, 2015 - Aufsatz / Paper in SSCI-Zeitschrift
	 \item Demokratie in Liberalismus und Neoliberalismus: Kapeller J., Pühringer S. , in Seckauer, Hansjörg/Stelzer-Orthofer, Christine/Kepplinger, Brigitte: Das Vorgefundene und das Mögliche. Beiträge zur Gesellschafts- und Sozialpolitik zwischen Ökonomie und Moral, Mandelbaum, Wien, Seite(n) 111-127, 2015 - Aufsatz / Paper in Sammelwerk (referiert)
	 \item Wirtschaftspolitik, Verteilungsgerechtigkeit und Demokratie: Kapeller J. , in AK Burgenland: Gerechtigkeit muss sein, Seite(n) 150-165, 2015 - Aufsatz / Paper in Sammelwerk (nicht-referiert)
	 \item Beyond Foundations: Systemism in Economic  Thinking: Kapeller J. , in Jo, Tae-Hee / Todorovka, Zdravka: Advancing the Frontiers of Heterodox Economics: Essays in Honor of Frederic S. Lee, Routledge, London, Seite(n) 115-134, 2015 - Aufsatz / Paper in Sammelwerk (referiert)
	 \item Allgemeine Modelltheorie und ökonomische Modelle: Kapeller J. , in EWE - Erwägen, Wissen, Ethik, Vol. 26, Nr. 3, Seite(n) 387-389, 2015 - Aufsatz / Paper in nicht-referierter Fachzeitschrift
	 \item Politik und ihre Ad-hoc- Gremien in Krisenzeiten: Hirte K. , in Momentum-Kongress: Momentum Kongress Paper, Seite(n) 1-22, 2015 - Aufsatz / Paper in Tagungsband (referiert)
	 \item Märkte und die Anerkennung von Arbeit. Zum Zusammenhang schlecht bezahlter Arbeiten und der Struktur der Arbeitsergebnisse: Hirte K. , in Ötsch, Walter; Hirte, Katrin; Pühringer, Stephan; Bräutigam, Lars: Markt! Welcher Markt?, Serie Kritische Studien zu Markt und Gesellschaft, Metropolis, Marburg, Seite(n) 281-322, 2015 - Aufsatz / Paper in Tagungsband (referiert)
	 \item Das Ökonomie-Monopol an den Agrarfakultäten: Hirte K. , in Wege für eine bäuerliche Zukunft – Zeitschrift der ÖBV/ Via Campesina Austria, Vol. 38, Nr. 3 (338), Seite(n) 22-23, 2015 - Aufsatz / Paper in nicht-referierter Fachzeitschrift
	 \item Bezeichnende Konstellation. Zum Eröffnungstag „Agrarpolitik“ in Österreich auf dem Podium: REWE, RWA Raiffeisen und Landwirtschaftskammer: Hirte K. , in Via Campesina Austria, in Wege für eine bäuerliche Zukunft – Zeitschrift der ÖBV/ Via Campesina Austria, Vol. 38, Nr. 2 (337), Seite(n) 15, 2015 - Aufsatz / Paper in nicht-referierter Fachzeitschrift
	 \item ÖkonomInnen und Ökonomie. Eine wissenschaftssoziologische Entwicklungsanalyse zum Verhältnis von ÖkonomInnen und Ökonomie im deutschsprachigen Raum ab 1945: Heise A., Hirte K., Ötsch W., Pühringer S., Reichl A., Sander H., Thieme S. , Hans-Böckler-Stiftung, Düsseldorf, 2015 - Forschungsbericht: geförderte Forschung (sonstige überwieg. aus öff. Hand)
	 \item Raus aus dem Euro?: Heimberger P. , in Wirtschaft und Gesellschaft, Vol. 41, Nr. 4, Seite(n) 603-614, 2015 - Rezension in nicht-referierter Fachzeitschrift
	 \item Eine fiskalpolitische Lösung für die Eurozone: Heimberger P. , in Kammer für Arbeiter und Angestellte Wien, in Wirtschaft und Gesellschaft, Vol. 41, Nr. 3, Lexis Nexis, Seite(n) 449-458, 2015 - Rezension in nicht-referierter Fachzeitschrift
	 \item Griechenland: Das Scheitern der europäischen Krisenpolitik: Heimberger P. , in Arbeiterkammer Wien, in EU-Infobrief, Nr. 3, Seite(n) 11-15, 2015 - Aufsatz / Paper in nicht-referierter Fachzeitschrift
	 \item Die griechische Schuldendebatte und das Mantra von den "notwendigen Strukturreformen": Heimberger P. , in WISO direkt, Nr. 05, 2015 - Aufsatz / Paper in nicht-referierter Fachzeitschrift
	 \item 'Strukturreformen' und Lohnkürzungen in Griechenland: Erwartungen, Ergebnisse und Folgen: Heimberger P. , in ISW, in WISO - Wirtschafts- und sozialpolitische Zeitschrift, Vol. 38, Nr. 3, Seite(n) 104-121, 2015 - Aufsatz / Paper in nicht-referierter Fachzeitschrift
	 \item New Perspectives on  Institutionalist Pattern Modeling: Systemism, Complexity and  Agent-Based modeling: Gräbner C., Kapeller J. , in Journal of Economic Issues, Vol. 49, Nr. 2, Seite(n) 433-440, 2015 - Aufsatz / Paper in SSCI-Zeitschrift
	 \item Wirtschaftspolitische Ausrichtung österreichischer Parteien im historischen Verlauf. Die Ausgestaltung österreichischer Parteiprogrammatiken unter dem Einfluss neoliberalen Gedankenguts: Grimm C. , 2015 - Diplom- oder Masterarbeit
	 \item Zwischen Konjunkturpuffer und Tauschobjekt. Gewerkschaftliche Perspektiven auf Migration im Österreich der Zweiten Republik: Griesser M., Sauer B. , in Kurswechsel, Nr. Heft 4, Seite(n) 58-66, 2015 - Aufsatz / Paper in nicht-referierter Fachzeitschrift
	 \item ÖkonomInnen und Politik – Analyse zur politischen Einflussnahme deutschsprachiger ÖkonomInnen: Griesser M., Hirte K., Pühringer S. , Forschungsbericht: Förderer: Jubiläumsfonds, Universität Linz, 2015 - Forschungsbericht: geförderte Forschung (sonstige überwieg. aus öff. Hand)
	 \item Rezension von „Martina Benz: Zwischen Migration und Arbeit. Worker Centers und die Organisierung prekär und informell Beschäftigter in den USA“: Griesser M. , in Bund demokratischer Wissenschaftlerinnen und Wissenschaftler, in Forum Wissenschaft, BdWi-Verlag, 2015 - Rezension in sonstiger referierter Fachzeitschrift
	 \item Der Staat als Wissensapparat. Konzeptionelle Überlegungen zu einer nicht-funktionalistischen Funktionsanalyse des Sozialstaats: Griesser M. , in Zeitschrift für Sozialreform, Vol. 61, Seite(n) 103-124, 2015 - Aufsatz / Paper in sonstiger referierter Fachzeitschrift
	 \item Nachfrageseitige Ursachen der Expansion des Schattenbankensystems: Beyer K. , 2015 - Sonstige
	 \item Verteilung und Gerechtigkeit: Philosophische  Perspektiven: Aistleitner M., Fölker M., Kapeller J., Mohr F., Pühringer S. , in Wirtschaft und Gesellschaft, Vol. 41, Nr. 1, Seite(n) 71-106, 2015 - Aufsatz / Paper in nicht-referierter Fachzeitschrift
	 \item Die Macht der Wissenschaftsstatistik und die Entwicklung der Ökonomie: Aistleitner M., Fölker M., Kapeller J. : Momentum Kongress Paper, Seite(n) 1-21, 2015 - Aufsatz / Paper in Tagungsband (referiert)
\end{itemize} 
 \subsection{2014} 
 \begin{itemize} 
	 \item The Political Economy of Offshore Jurisdictions. An Introduction: Ötsch W., Schmidt M. , in Ötsch, Walter O.; Grözinger, Gerd; Beyer, Karl M.; Bräutigam, Lars: The Political Economy of Offshore Jurisdictions, Metropolis Verlag, Marburg, Seite(n) 7-23, 2014 - Aufsatz / Paper in Tagungsband (nicht-referiert)
	 \item The Political Economy of Offshore Jurisdictions: Ötsch W., Grözinger G., Beyer K., Bräutigam L. , Metropolis Verlag, Marburg, 2014 - Wissenschaftliches Sammelwerk Herausgeberschaft (Erstauflage)
	 \item Die Finanzkrise 2007-2009 als Krise von Schattenbanken. Eine einführende institutionelle Analyse: Ötsch W., Beyer K., Mader L. , Delft, 2014 - Sonstige
	 \item Populismus und Demagogie. Mit Beispielen von Jörg Haider, Hans-­‐Christian Strache und Frank Stronach sowie der Tea Party”.: Ötsch W. , in Gressl, Martin, Klemenjak, Martin, Klepp. Cornelia, Pichler, Heinz, Rottermann, Doris und Scherling, Josefine: Populismus und Rassismus im Vormarsch?, Schriftenreihe „Arbeit und Bildung“, Klagenfurt, Seite(n) 12-26, 2014 - Aufsatz / Paper in Sammelwerk (nicht-referiert)
	 \item How to Hide Secrecy Jurisdictions: Ötsch W. , in Ötsch, Walter O.; Grözinger, Gerd; Beyer, Karl M.; Bräutigam, Lars: The Political Economy of Offshore Jurisdictions, : Metropolis Verlag, : Metropolis Verlagrburg, Seite(n) 61-75, 2014 - Aufsatz / Paper in Tagungsband (referiert)
	 \item Subventionierung von Lohnkosten als Mittel zur Armutsvermeidung: Stelzer-Orthofer C., Pühringer S. , in Dimmel, N./Schenk, M./Stelzer-Orthofer, C.: Handbuch Armut in Österreich, Studienverlag, Innbruck, Seite(n) 817-831, 2014 - Aufsatz / Paper in Sammelwerk (referiert)
	 \item Ökonomische Krisen als Krankheiten und Katastrophen?: Pühringer S. , 2014 - Sonstige
	 \item Mythen über Reichtum und Macht: Demokratie ist nicht käuflich: Pühringer S. , in Beigewum/ATTAC/Armutskonferenz: Mythen des Reichtums, VSA Verlag, Hamburg, Seite(n) 149-158, 2014 - Aufsatz / Paper in Sammelwerk (nicht-referiert)
	 \item Modeling the Evolution of Preferences: An Answer to Schubert and Cordes: Kapeller J., Steinerberger S. , in Journal of Institutional Economics, Vol. 10, Nr. 2, Seite(n) 337-347, 2014 - Aufsatz / Paper in SSCI-Zeitschrift
	 \item From Free to Civilized Markets: First steps towards Eutopia: Kapeller J., Schütz B., Tamesberger D. , Bremen, 2014 - Sonstige
	 \item From Free to Civilized Markets: Kapeller J., Schütz B., Tamesberger D. : Momentum Kongress Paper, Seite(n) 1-29, 2014 - Aufsatz / Paper in Tagungsband (referiert)
	 \item Making Morality Matter: Civilized Markets and European Values.: Kapeller J., Schütz B., Tamesberger D. , in Journal for a Progressive Economy, 2014 - Aufsatz / Paper in nicht-referierter Fachzeitschrift
	 \item Fortschrittsidee und Politische Vision [Progress and Politics]: Kapeller J., Hubmann G. , in Momentum Quarterly, Vol. 3, Nr. 4, Seite(n) 235-245, 2014 - Aufsatz / Paper in sonstiger referierter Fachzeitschrift
	 \item Führt Pluralismus in der ökonomischen Theorie zu mehr Wahrheit?: Kapeller J., Grimm C., Springholz F. , in Hirte, Katrin, Thieme, Sebastian, Ötsch, Walter: Wissen! Welches Wissen?, Metropolis, Marburg, Seite(n) 147-163, 2014 - Aufsatz / Paper in Sammelwerk (referiert)
	 \item Die Rückkehr des Rentiers. Rezension zu: Piketty, Thomas (2014): Capital in the 21st century. Cambridge: Harvard University Press: Kapeller J. , in Wirtschaft und Gesellschaft, Vol. 40, Nr. 2, Seite(n) 329-346, 2014 - Rezension in nicht-referierter Fachzeitschrift
	 \item Economic Change and Change in Economics: Kapeller J. , Universität Linz, 2014 - Habilitationsschrift
	 \item Vorwort im Sammelband "Wissen! Welches Wissen?": Hirte K., Ötsch W. , in Hirte Katrin, Thieme Sebastian, Ötsch Walter Otto: Wissen! Welches Wissen? Zu Wahrheit, Theorien und Glauben sowie ökonomischen Theorien, Metropolis Verlag, Marburg, Seite(n) 7-16, 2014 - Aufsatz / Paper in Tagungsband (nicht-referiert)
	 \item Wissen! Welches Wissen? Zu Wahrheit, Theorien und Glauben sowie ökonomischen Theorien: Hirte K., Thieme S., Ötsch W. , Metropolis Verlag, Marburg, 2014 - Tagungsband Mitherausgeberschaft (Erstauflage)
	 \item Performative Wissenschaft: Ökonomiekritik, Ökonomietheorien und die Verantwortung von ÖkonomInnen.: Hirte K., Pühringer S. , in Hirte Katrin, Thieme Sebastian, Ötsch Walter Otto: Wissen! Welches Wissen? Zu Wahrheit, Theorien und Glauben sowie ökonomischen Theorien, Metropolis Verlag, Marburg, Seite(n) 267-302, 2014 - Aufsatz / Paper in Tagungsband (referiert)
	 \item ÖkonomInnen und Ökonomie in der Krise? Eine diskurs- und netzwerkanalytische Sicht.: Hirte K., Pühringer S. , in WISO - Wirtschafts- und sozialpolitische Zeitschrift, Vol. 1, Seite(n) 159-178, 2014 - Aufsatz / Paper in sonstiger referierter Fachzeitschrift
	 \item Agrargiganten im Osten. Zur neuerlichen Transformation der transformierten deutschen Agrarstrukturen.: Hirte K. , in Brähler, Elmar; Wagner, Wolf: 25 Jahre Mauerfall – kein Ende mit der Wende?, Psychosozial-Verlag, Gießen, Seite(n) 277-­290, 2014 - Aufsatz / Paper in Sammelwerk (referiert)
	 \item Landwirtschaft, Ideologien und "...ismen": Hirte K. , in Via Campesina Austria, in Wege für eine bäuerliche Zukunft – Zeitschrift der ÖBV/ Via Campesina Austria, Vol. 37, Nr. 2 (332), Seite(n) 14-15, 2014 - Aufsatz / Paper in nicht-referierter Fachzeitschrift
	 \item MigrantInnen als Zielgruppe. Solidarische Beratungs- und Unterstützungsangebote von ArbeitnehmerInnenorganisationen in Österreich: Griesser M., Sauer B. , Serie Studie - Abschlussbericht, Universität, Institut für Politikwissenschaften, Wien, 2014 - Forschungsbericht: geförderte Forschung (sonstige überwieg. aus öff. Hand)
	 \item Die Vermögensverteilung in Österreich und das Aufkommenspotenzial einer Vermögenssteuer: Eckerstorfer P., Halak J., Kapeller J., Schütz B., Springholz F., Wildauer R. , in Wirtschaft und Gesellschaft, Vol. 40, Nr. 1, Seite(n) 63-81, 2014 - Aufsatz / Paper in nicht-referierter Fachzeitschrift
	 \item Vermögen in Österreich: Eckerstorfer P., Halak H., Kapeller J., Schütz B., Springholz F., Wildauer R. , Serie Materialien zu Wirtschaft und Gesellschaft, Nr. 126, AK Wien, WIen, 2014 - Forschungsbericht: geförderte Forschung (sonstige überwieg. aus öff. Hand)
	 \item Offshore Aspects of Shadow Banking. With Considerations on the Recent Financial Crisis: Beyer K., Bräutigam L. , in Ötsch, Walter O.; Grözinger, Gert; Beyer, Karl M.: The Political Economy of Offshore Jurisdictions, Metropolis Verlag, Marburg, 2014 - Aufsatz / Paper in Tagungsband (nicht-referiert)
	 \item Die Risiken im Schatten des Systems: Beyer K. , 2014 - Sonstige
	 \item Emanzipation bei Marx und seine Kritik an Proudhon und dessen ideengeschichtlichen Nachfahren: Beyer K. : Momentum Kongress Paper, Seite(n) 1-18, 2014 - Aufsatz / Paper in Tagungsband (referiert)
\end{itemize} 
 \subsection{2013} 
 \begin{itemize} 
	 \item Das Team Stronach: Die österreichische Tea Party: Ötsch W. , in Die Presse, 2013 - Presseartikel / Medienberichte
	 \item Marktradikalität. Der Diskurs von „dem Markt“: Ötsch W. , in Günther, Lea-Simone; Hertlein, Saskia;  Klüsener, Vea und Raasch, Markus: Radikalität. Religiöse, politische und künstlerische Radikalismen in Geschichte und Gegenwart.  Band 2: Frühe Neuzeit und Moderne, Königshausen & Neumann, Würzburg, Seite(n) 254-279, 2013 - Aufsatz / Paper in Sammelwerk (referiert)
	 \item Die Macht der Ratingagenturen: Governance in der Ideologie 'des Marktes': Ötsch W. , in Brodbeck, Karl-Heinz: Alternative Länder-Ratings, Schriftenreihe der Finance & Ethics Academy, Band 5, Shaker Verlag, Aachen, Seite(n) 58-98, 2013 - Aufsatz / Paper in Sammelwerk (referiert)
	 \item The Deep Meening of ‘Market’: Understanding Neoliberal-Market-Radical Reasoning: Ötsch W. , in Human Geography, Vol. 6, Nr. 2, Seite(n) 11-25, 2013 - Aufsatz / Paper in sonstiger referierter Fachzeitschrift
	 \item Marx, Keynes und die Idee des gesellschaftlichen Fortschritts: Suche nach neuen politischen Visionen: Schütz B. : Momentum Kongress Paper, Seite(n) 1-21, 2013 - Aufsatz / Paper in Tagungsband (referiert)
	 \item Ahnungslos, aber nicht tatenlos – Wie ÖkonomInnen seit der Finanzkrise Politik mach(t)en: Pühringer S. , 2013 - Sonstige
	 \item „Arbeitsmarktferne“ Personen – wer sind die? Zu veränderten Exklusionsdynamiken in neokapitalistischen Gesellschaften: Pühringer S. , in SWS-Rundschau - Sozialwissenschaftliche Studiengesellschaft Rundschau, Vol. 53, Nr. 4, Seite(n) 361-381, 2013 - Aufsatz / Paper in nicht-referierter Fachzeitschrift
	 \item Der Fiskalpakt und seine Implementation in Österreich: Plaimer W., Pühringer S. : Momentum Kongress Paper, Seite(n) 1-20, 2013 - Aufsatz / Paper in Tagungsband (referiert)
	 \item Grenzen aktueller Krisendebatten. Über Konstruktionen der öffentlichen Meinung und das Verhältnis von Sach- und Grundsatzdiskussionen in (neo)liberalen Demokratien: Nordmann J. , in Wengeler Martin; Ziem, Alexander: Sprachliche Konstruktionen von Krisen : interdisziplinäre Perspektiven auf ein fortwährend aktuelles Phänomen, Hempen, Bremen, Seite(n) 53-66, 2013 - Aufsatz / Paper in Tagungsband (referiert)
	 \item The grounds of solidarity: From liberty to loyalty: Kapeller J., Wolkenstein F. , in European Journal of Social Theory, Vol. 16, Nr. 4, Seite(n) 476-491, 2013 - Aufsatz / Paper in SSCI-Zeitschrift
	 \item How Formalism shapes Perception: An Experiment on Mathematics as a Language: Kapeller J., Steinerberger S. , in International Journal of Pluralism and Economics Education, Vol. 4, Nr. 2, Seite(n) 138-156, 2013 - Aufsatz / Paper in sonstiger referierter Fachzeitschrift
	 \item Die Regulation der Routine: Über die regulatorischen Spielräume zur Etablierung nachhaltigen Konsums: Kapeller J., Schütz B., Tamesberger D. , in Wirtschaft und Gesellschaft, Vol. 39, Nr. 2, Seite(n) 207-231, 2013 - Aufsatz / Paper in nicht-referierter Fachzeitschrift
	 \item The impossibility of rational consumer choice - A problem and its solution: Kapeller J., Schütz B., Steinerberger S. , in Journal of Evolutionary Economics, Vol. 23, Nr. 1, Seite(n) 39-60, 2013 - Aufsatz / Paper in SSCI-Zeitschrift
	 \item Exploring Pluralist Economics: The Case of the Minsky-Veblen Cycles: Kapeller J., Schütz B. , in Journal of Economic Issues, Vol. 47, Nr. 2, Seite(n) 515-524, 2013 - Aufsatz / Paper in SSCI-Zeitschrift
	 \item Model-Platonism in Economics: On a classical epistemological critique: Kapeller J. , in Journal of Institutional Economics, Vol. 9, Nr. 2, Seite(n) 199-221, 2013 - Aufsatz / Paper in SSCI-Zeitschrift
	 \item ÖkonomInnen in der Finanzkrise. Diskurse. Netzwerke. Initiativen.: Hirte K. , Metropolis Verlag, Marburg, 2013 - Monographie (Erstauflage)
	 \item „Persilschein“ – Netzwerke: Für Bruchlosigkeit in Umbruchzeiten.: Hirte K. , in Schönhuth, Michael; Gamper, Markus; Kronenwett, Michael; Stark, Martin: Visuelle Netzwerkforschung. Qualitative, quantitative und partizipative Zugänge., Transcript Verlag, Seite(n) 331-353, 2013 - Aufsatz / Paper in Tagungsband (referiert)
	 \item Deutsche Europapolitik vor und nach 1945: Hirte K. , in Wege für eine bäuerliche Zukunft – Zeitschrift der ÖBV/ Via Campesina Austria, Vol. 36, Nr. 3 (328), Seite(n) 22-23, 2013 - Aufsatz / Paper in nicht-referierter Fachzeitschrift
	 \item Deutsche Agrarpolitikprofessoren vor und nach 1945: Hirte K. , in Via Campesina Austria, in Wege für eine bäuerliche Zukunft – Zeitschrift der ÖBV/ Via Campesina Austria, Vol. 36, Nr. 2 (327), Seite(n) 18-20, 2013 - Aufsatz / Paper in nicht-referierter Fachzeitschrift
	 \item Reichtumsverteilung in Österreich: Eckerstorfer P., Halak J., Kapeller J., Schütz B., Springholz F., Wildauer R. , in WISO - Wirtschafts- und sozialpolitische Zeitschrift, Vol. 36, Nr. 4, 2013 - Aufsatz / Paper in nicht-referierter Fachzeitschrift
	 \item Bestände und Verteilung der Vermögen in Österreich: Eckerstorfer P., Halak H., Kapeller J., Schütz B., Springholz F., Wildauer R. , Serie Materialien zu Wirtschaft und Gesellschaft, Vol. 122, Abteilung Wirtschaftswissenschaft und Statstik der Kammer für Arbeiter und Angestellte, Wien, 2013 - Forschungsbericht: geförderte Forschung (sonstige überwieg. aus öff. Hand)
	 \item Diskutieren statt Ignorieren: Eckpfeiler für interessierten Pluralismus in der Ökonomie: Dobusch L., Kapeller J. , in Der öffentliche Sektor - The Public Sector, Vol. 39, Nr. 3, 2013 - Aufsatz / Paper in nicht-referierter Fachzeitschrift
	 \item Breaking New Paths: Theory and Method in Path Dependence Research: Dobusch L., Kapeller J. , in Schmalenbach Business Review, Vol. 65, Nr. 2, Seite(n) 288-311, 2013 - Aufsatz / Paper in sonstiger referierter Fachzeitschrift
	 \item Practicing Pluralism: A Rejoinder to W. Robert Brazelton: Dobusch L., Kapeller J. , in Journal of Economic Issues, Vol. 47, Nr. 4, Seite(n) 1035-1037, 2013 - Aufsatz / Paper in SSCI-Zeitschrift
\end{itemize} 
 \subsection{2012} 
 \begin{itemize} 
	 \item Staatsschuldenkrise und ökonomisches Denken – im Euroraum und in Zentraleuropa: Ötsch W. , in Horvath, Patrick, Skarke, Herbert  und Weinzierl, Rupert: Die “Vision Zentraleuropa” im 21. Jahrhundert. Festschrift zum 90. Geburtstag von Heinz Kienzl, Arbeitsgemeinschaft für wissenschaftliche Wirtschaftspolitik (WIWIPOL), Wien, Seite(n) 68-71, 2012 - Aufsatz / Paper in Sammelwerk (nicht-referiert)
	 \item Politische Ökonomie und Gesellschaft. Eine theoriegeschichtliche Skizze: Ötsch W. , in Grözinger, Gerd; Reich, Utz-Peter: Entfremdung – Ausbeutung – Revolte, Metropolis Verlag, Marburg, Seite(n) 145-165, 2012 - Aufsatz / Paper in Sammelwerk (nicht-referiert)
	 \item Krise des Euroraums: Ötsch W. , in Starke, Herbert; Horvath, Patrick; Weinzierl, Rupert: Die Vision Zentraleuropa im 21. Jahrhundert, Arbeitsgemeinschaft für wissenschaftliche Wirtschaftspolitik, Wien, Seite(n) 68-71, 2012 - Aufsatz / Paper in Sammelwerk (nicht-referiert)
	 \item Der freie Markt: Ötsch W. , in Czejkowska, Agnieszka: Imagine Economy. Neoliberale Metaphern im wirtschaftlichen Diskurs, Erhard Löcker Verlag, Wien, Seite(n) 39-45, 2012 - Aufsatz / Paper in Sammelwerk (nicht-referiert)
	 \item How Liberalism lost its concept of democracy: Pühringer S., Kapeller J. : Momentum Kongress Paper, Seite(n) 1-22, 2012 - Aufsatz / Paper in Tagungsband (referiert)
	 \item Economists and Economics. Discourse profiles of economists in the financial crisis: Pühringer S., Hirte K. , in Association française d'économie politique: Joint conference of AHE, IIPPE, FAPE. Kongress Political economy and the outlook for capitalism 05.-07.07.2012, Paris, Seite(n) 1-16, 2012 - Aufsatz / Paper in Tagungsband (nicht-referiert)
	 \item Öffentlicher Vernunftgebrauch - ein probantes Mittel zur Bekämpfung von Ungerechtigkeit?: Pühringer S. , in Studentisches Soziologiemagazin, 2012 - Rezension in nicht-referierter Fachzeitschrift
	 \item Soziale Frage im Wandel. Probleme und Perspektiven des Sozialstaates und der Arbeitsgesellschaft: Pühringer S. , in Kontraste - Presse- und Informationsdienst für Sozialpolitik, Nr. 2/2012, Seite(n) 22-23, 2012 - Rezension in nicht-referierter Fachzeitschrift
	 \item Postdemokratie in Österreich?: Plaimer W. , in Nordmann, Jürgen; Hirte, Katrin; Ötsch, Walter O.: Demokratie! Welche Demokratie? Postdemokratie kritisch hinterfragt, Metropolis Verlag, Marburg, Seite(n) 159-174, 2012 - Aufsatz / Paper in Sammelwerk (nicht-referiert)
	 \item Postdemokratie in Österreich?: Plaimer W. , in Momentum-Kongress: Momentum Kongress Paper, Seite(n) 1-17, 2012 - Aufsatz / Paper in Tagungsband (referiert)
	 \item Demokratie! Welche Demokratie? Postdemokratie kritisch hinterfragt: Nordmann J., Hirte K., Ötsch W. , Metropolis Verlag, Marburg, 2012 - Tagungsband Mitherausgeberschaft (Erstauflage)
	 \item Vorwort in "Demokratie! Welche Demokratie? Postdemokratie kritisch hinterfragt": Nordmann J. : Demokratie! Welche Demokratie? Postdemokratie kritisch hinterfragt, Seite(n) 7-14, 2012 - Aufsatz / Paper in Sammelwerk (nicht-referiert)
	 \item Die intellektuelle Geschichte des Neoliberalismus im Spiegel des alten Liberalismus: Nordmann J. , in ksoe-Dossier der katholischen Erwachsenenbildung Österreich, 2012 - Aufsatz / Paper in nicht-referierter Fachzeitschrift
	 \item Die Regulation der Routine: Kapeller J., Schütz B., Tamesberger D. : Momentum Kongress Paper, Seite(n) 1-31, 2012 - Aufsatz / Paper in Tagungsband (referiert)
	 \item Konsum demokratisch gestalten: Spielräume zur Etablierung nachhaltigen Konsums.: Kapeller J., Schütz B., Tamesberger D. , in WISO - Wirtschafts- und sozialpolitische Zeitschrift, Seite(n) 167-183, 2012 - Aufsatz / Paper in nicht-referierter Fachzeitschrift
	 \item Solidarisch Handeln: Konzeptionen, Ursachen und Implikationen: Kapeller J., Hubmann G. , in Momentum Quarterly, Vol. 1, Nr. 3, Seite(n) 139-152, 2012 - Aufsatz / Paper in sonstiger referierter Fachzeitschrift
	 \item A guide to paradigmatic Self-marginalization - Lessons for Post-Keynesian Economists: Kapeller J., Dobusch L. , in Lavoie M. und Lee F.S.: In Defense of Post-Keynesian and Heterodox Economics., Routledge, London, Seite(n) 62-86, 2012 - Aufsatz / Paper in Sammelwerk (nicht-referiert)
	 \item Bridges to Past Polls: Die oberösterreichische Erfahrung mit Vorwahlen als demokratisches Instrument: Huber J., Kaindlstorfer L., Kapeller J. : Momentum Kongress Paper, 2012 - Aufsatz / Paper in Tagungsband (referiert)
	 \item ÖkonomInnen in der Finanzkrise. Analyse zur Positionierung deutschsprachiger Ökonomen im Kontext ihrer strukturellen Verankerung: Hirte K., Pühringer S. , 2012 - Forschungsbericht: geförderte Forschung (andere Geldgeber)
	 \item Die Umstrukturierung der LPGen in Thüringen ab 1990.: Hirte K. , in Landeszentrale für politische Bildung Thüringen., Druckerei Sömmerda GmbH, Erfurt, 2012 - Sonstige
	 \item Die ersten Professoren für Agrarpolitik und Agrarökonomie ab 1945 an den westdeutschen Universitäten und ihre Vergangenheit.: Hirte K. , in Buchsteiner Martin, Strahl Antje: Thünen-Jahrbuch, Nr. 7, Seite(n) 87-114, 2012 - Aufsatz / Paper in Sammelwerk (referiert)
	 \item Würdigungs-Netzwerke, gewolltes Nichtwissen und Geschichtsschreibung.: Hirte K. , in Österreichische Zeitschrift für Geschichtswissenschaften, Vol. 23, Nr. 1, Studienverlag Innsbruck, Seite(n) 155-185, 2012 - Aufsatz / Paper in sonstiger referierter Fachzeitschrift
	 \item Regulatorische Unsicherheit und Private Standardisierung: Koordination durch Ambiguität: Dobusch L., Kapeller J. : Steuerung durch Regeln, Serie Managementforschung, Vol. 22, Seite(n) 43-81, 2012 - Aufsatz / Paper in Sammelwerk (referiert)
	 \item A Guide to Paradigmatic Self-Marginalization - Lessons for Post-Keynesian Economists: Dobusch L., Kapeller J. , in Review of Political Economy, Seite(n) 469-487, 2012 - Aufsatz / Paper in sonstiger referierter Fachzeitschrift
	 \item Heterodox United vs. Mainstream City? Sketching a framework for interested pluralism in economics: Dobusch L., Kapeller J. , in Journal of Economic Issues, Vol. 46, Nr. 4, Seite(n) 1035-1057, 2012 - Aufsatz / Paper in SSCI-Zeitschrift
	 \item Illusionen eines Zirkulationskünstlers? Pierre-Joseph Proudhon auf dem ökonomiekritischen Prüfstand: Beyer K. , Universität Wien, 2012 - Diplom- oder Masterarbeit
\end{itemize} 
 \subsection{2011} 
 \begin{itemize} 
	 \item Gesellschaft! Welche Gesellschaft? Nachdenken über eine sich wandelnde Gesellschaft: Ötsch W., Hirte K., Nordmann J. , Metropolis-Verlag, Marburg, 2011 - Tagungsband Mitherausgeberschaft (Erstauflage)
	 \item Kurt Rothschild als politischer Ökonom: Ötsch W. , in Bartel, Rainer; Bürger, Hans; Klug, Friedrich: In Memoriam Univ. Prof. Kurt W. Rothschild, Serie Schriftenreihe des Instituts für Kommunalwissenschaft (IKW), Nr. Band 122, Institut für Kommunalwirtschaften, Wien, Seite(n) 73-84, 2011 - Aufsatz / Paper in Sammelwerk (nicht-referiert)
	 \item Der Markt und die großen Ratingagenturen: Ötsch W. : Momentum Kongress Paper, Serie Momentum quarterly, Seite(n) 1-11, 2011 - Aufsatz / Paper in Tagungsband (referiert)
	 \item Markt. Sichtweisen auf die Wirtschaft: Ötsch W. , in Praxis Politik, Nr. 2/2011, Seite(n) 4-8, 2011 - Aufsatz / Paper in nicht-referierter Fachzeitschrift
	 \item Aktivierung und Mindestsicherung. Rezension zum gleichnamigen Buch von Christine Stelzer-Orthofer und Josef Weidenholzer: Pühringer S. , in WISO - Wirtschafts- und sozialpolitische Zeitschrift, Vol. 34, Nr. 02, Seite(n) 169-171, 2011 - Rezension in nicht-referierter Fachzeitschrift
	 \item Frei handeln? Liberales und neoliberales Freiheitskonzept und ihre Auswirkungen auf die Verteilung von Macht und Eigentum: Pühringer S. , Peter Lang, Frankfurt am Main, 2011 - Monographie (Erstauflage)
	 \item Gleichheit versus Vielfalt. Ein konstruierter Widerspruch?: Pühringer S. : Momentum Kongress Paper, Seite(n) 1-23, 2011 - Aufsatz / Paper in Tagungsband (referiert)
	 \item Veränderung von Machtverhältnissen in politischen Entscheidungsprozessen: Plaimer W., Nordmann J. , in Momentum-Kongress: Momentum Kongress Paper, Seite(n) 1-21, 2011 - Aufsatz / Paper in Tagungsband (referiert)
	 \item Braucht die aktuelle Gesellschaft einen Gesellschaftsvertrag? Der politische Neoliberalismus im Spiegel von John Locke und John Rawls: Nordmann J. : Gesellschaft! Welche Gesellschaft?, Metropolis-Verlag, Marburg, Seite(n) 33-60, 2011 - Aufsatz / Paper in Tagungsband (referiert)
	 \item Aktionsforschung - ein Weg zum Design institutioneller Neuerungen zur regionalen Anpassung an den Klimawandel: Knierim A., Hirte K. : Anpassung an den Klimawandel - regional umsetzen!, oecom Verlag, München, Seite(n) 156-174, 2011 - Aufsatz / Paper in Tagungsband (referiert)
	 \item Modell-Platonismus in der Ökonomie. Zur Aktualität einer klassisch-epistemologischen Kritik: Kapeller J. , Universität Linz, 2011 - Dissertation
	 \item Modell-Platonismus in der Ökonomie: Zur Aktualität einer klassischen epistemologischen Kritik: Kapeller J. , Peter Lang, Frankfurt/Main, 2011 - Monographie (Erstauflage)
	 \item Was sind ökonomische Modelle?: Kapeller J. , in Volker Gadenne und Reinhard Neck: Philosophie und Wirtschaftswissenschaft, Mohr Siebeck, Seite(n) 29-50, 2011 - Aufsatz / Paper in Tagungsband (nicht-referiert)
	 \item Vorwort: Hirte K., Ötsch W. , in Ötsch, Walter O.; Hirte, Katrin; Nordmann, Jürgen: Gesellschaft! Welche Gesellschaft?, Metropolis, Marburg, Seite(n) 7-15, 2011 - Aufsatz / Paper in Tagungsband (nicht-referiert)
	 \item Ökonomische Ausrichtung und Netzwerke - das Beispiel des Sachverständigenrates.: Hirte K., Ötsch W. , in Prokla, Nr. 3, Seite(n) 423-447, 2011 - Aufsatz / Paper in sonstiger referierter Fachzeitschrift
	 \item Gleichheit und Vielfalt – normative Konzeptionen? Die philosophischen Implikationen zum Problem Anerkennung bei Simone de Beauvoir und Hannah Arendt: Hirte K. : Momentum Kongress Paper, Seite(n) 1-15, 2011 - Aufsatz / Paper in Tagungsband (referiert)
	 \item Crowdsourcing und Regelbezüge - der Fall GuttenPlag.: Hirte K. , in Heiss, Hans-Ulrich: INFORMATIK 2011 - Informatik schafft Communities. 41. Jahrestagung der Gesellschaft für Informatik , 4.-7.10.2011, Berlin, Serie Lecture Notes in Informatics (LNI), Vol. P-192, Springer, 2011 - Aufsatz / Paper in Tagungsband (referiert)
	 \item Wirtschaft, Wissenschaft und Politik: Die sozialwissenschaftliche Bedingtheit linker Reformpolitik: Dobusch L., Kapeller J. , in Prokla, Vol. 41, Nr. 3, Seite(n) 389-404, 2011 - Aufsatz / Paper in sonstiger referierter Fachzeitschrift
	 \item Solidarität - Beiträge für eine gerechte Gesellschaft: Blaha B., Kapeller J., Weidenholzer J. , Braumüller, Wien, 2011 - Tagungsband Mitherausgeberschaft (Erstauflage)
\end{itemize} 
 \subsection{2010} 
 \begin{itemize} 
	 \item Perpetuing the failure: Economic Education and the Current Crisis: Ötsch W., Kapeller J. , in Journal of Social Science Education, Vol. 9, Nr. 2, Seite(n) 16-25, 2010 - Aufsatz / Paper in sonstiger referierter Fachzeitschrift
	 \item Die Evolution des ökonomischen Wissens und des Wissens über den Kapitalismus. Performativity als Analyseinstrument: das Beispiel der Fabian Society, der Mont Pèlerin Society und der Chicagoer Schule: Ötsch W., Hirte K., Nordmann J. , 2010 - Sonstige
	 \item Krise! Welche Krise? Zur Problematik aktueller Krisendebatten: Ötsch W., Hirte K., Nordmann J. , Metropolis, Marburg, 2010 - Tagungsband Mitherausgeberschaft (Erstauflage)
	 \item Das Bewusstsein des Homo Oeconomicus: Ötsch W. , in Bauer, Renate: Bewusstsein und Ich, Angelika Lenz Verlag, Neu-Isenberg, Seite(n) 105–117, 2010 - Aufsatz / Paper in Sammelwerk (nicht-referiert)
	 \item Die Tiefenbedeutung von ‘Markt’: Ötsch W. : Momentum Kongress Paper, Seite(n) 1-27, 2010 - Aufsatz / Paper in Tagungsband (referiert)
	 \item Solidarität im Kapitalismus: Pühringer M., Pühringer S. : Momentum Kongress Paper, Seite(n) 1-32, 2010 - Aufsatz / Paper in Tagungsband (referiert)
	 \item Was ist eine Krise?: Nordmann J. , in Ötsch, Walter O.; Hirte, Katrin; Nordmann, Jürgen: Krise! Welche Krise?, Metropolis, Marburg, Seite(n) 7-20, 2010 - Aufsatz / Paper in Tagungsband (nicht-referiert)
	 \item Trash, Skandale und Ratschläge statt Aufklärung und politische Bildung. Über das Zusammenspiel von kommerzialisierten Medien und gemachter Meinung in der neoliberalen Gesellschaft: Nordmann J. : Momentum Kongress Paper, Seite(n) 1-8, 2010 - Aufsatz / Paper in Tagungsband (referiert)
	 \item Protektionismus. Die Grenzen der Staatsintervention in den 1930er Jahren: Nordmann J. , in European Journal of Economics and Economic Policies: Intervention, Vol. 7, Nr. 1, Seite(n) 42-49, 2010 - Aufsatz / Paper in sonstiger referierter Fachzeitschrift
	 \item The internal consistency of perfect competition: Kapeller J., Pühringer S. , in Journal of Philosophical Economics, Vol. 3, Nr. 2, Seite(n) 134-152, 2010 - Aufsatz / Paper in sonstiger referierter Fachzeitschrift
	 \item Neoklassische Sozialdemokratie und Sozialdemokratie am Beispiel des Hamburger Programms der SPD: Kapeller J., Huber J. , in TRANS Internet-Zeitschrift für Kulturwissenschaften , Vol. 17, Nr. 2, 2010 - Aufsatz / Paper in nicht-referierter Fachzeitschrift
	 \item Some critical notes on citation metrics and heterodox economics: Kapeller J. , in Review of Radical Political Economics, Vol. 42, Nr. 3, Seite(n) 330-337, 2010 - Aufsatz / Paper in SSCI-Zeitschrift
	 \item Citation Metrics: Serious drawbacks, perverse incentives and strategic options for heterodox economics: Kapeller J. , in American Journal of Economics and Sociology, Vol. 69, Nr. 5, Seite(n) 1376-1408, 2010 - Aufsatz / Paper in SSCI-Zeitschrift
	 \item Solidarisch Handeln: Konzeptionen, Ursachen und Implikationen: Hubmann G., Kapeller J. : Momentum Kongress Paper, Seite(n) 1-30, 2010 - Aufsatz / Paper in Tagungsband (referiert)
	 \item Ökonomisierung an den Hochschulen.: Hirte K. , in Johanna Besier, Hannah Fritsch, Arne Rost, Sven Schmidt, Hannes Schulz, Maggie Selle, Katharina Wenzel: Agrarpolitik in der Lehre?, ABL Bauernblatt Verlags-GmbH, Seite(n) 17-25, 2010 - Aufsatz / Paper in Tagungsband (nicht-referiert)
	 \item Performativity of Economics - ein tragfähiger Ansatz zur Analyse der Rolle der Ökonomen in der Ökonomie?: Hirte K. , in Walter Ötsch, Katrin Hirte, Jürgen Nordmann: Krise. Welche Krise?, Metropolis Verlag, Marburg, Seite(n) 49-75, 2010 - Aufsatz / Paper in Tagungsband (referiert)
	 \item Das neoklassische Freihandelsmodell: Hirte K. : Momentum Kongress Paper, Seite(n) 1-16, 2010 - Aufsatz / Paper in Tagungsband (referiert)
	 \item Die Rolle der Agrarpolitik und Agrarökonomie in agrarpolitischen Diskursverläufen: Hirte K. , in TRANS Internet-Zeitschrift für Kulturwissenschaften , Vol. 17, Nr. 2, 2010 - Aufsatz / Paper in nicht-referierter Fachzeitschrift
	 \item Institutionalisierung zivilgesellschaftlicher Partizipation - Zwischen Ignoranz, Integration und Invasion: Dobusch L., Kapeller J. , in Blaha, Barbara; Weidenholzer, Josef: Freiheit - Beiträge für eine demokratische Gesellschaft, Wilhelm Braumüller Universitäts-Verlagsbuchhandlung, Wien, Seite(n) 201-217, 2010 - Aufsatz / Paper in Tagungsband (nicht-referiert)
\end{itemize} 
 \subsection{2009} 
 \begin{itemize} 
	 \item Der neoliberale Markt–Diskurs. Ursprünge, Geschichte, Wirkungen.: Ötsch W., Thomasberger C. , Metropolis Verlag, Marburg, 2009 - Tagungsband Herausgeberschaft (Erstauflage)
	 \item Neokonservativer Markt-Radikalismus. Das Fallbeispiel des Iraks: Ötsch W., Kapeller J. , in Internationale Politik und Gesellschaft, Nr. 2, Seite(n) 40-55, 2009 - Aufsatz / Paper in sonstiger referierter Fachzeitschrift
	 \item Mythos Markt. Marktradikale Propaganda und ökonomische Theorie: Ötsch W. , Metropolis Verlag, Marburg, 2009 - Monographie (Erstauflage)
	 \item Frei Handeln. Überlegungen zur Überwindung des neoliberalen Freiheitsbegriffs: Pühringer S., Wolfmayer G. : Momentum Kongress Paper, Seite(n) 1-21, 2009 - Aufsatz / Paper in Tagungsband (referiert)
	 \item Politische Paradigmata und neoliberale Einflüsse am Beispiel von vier sozialdemokratischen Parteien in Europa.: Kapeller J., Huber J. , in ÖZP - Österreichische Zeitschrift für Politikwissenschaft, Nr. 2, Seite(n) 163-192, 2009 - Aufsatz / Paper in sonstiger referierter Fachzeitschrift
	 \item Diskursverläufe in der universitären Agrarpolitik als neoliberales Hegemonialprojekt – Struktur, Ursache und Wirkungen: Hirte K. , in Ötsch, Walter; Thomasberger, Claus: Der neoliberale Marktdiskurs, Metropolis, Marburg, Seite(n) 187-212, 2009 - Aufsatz / Paper in Sammelwerk (nicht-referiert)
	 \item Markt als soziale Struktur - Zum Diskursszenario zur "Märkte-Störung" durch den Milchstreik: Hirte K. , in arbeitsergebnisse, Nr. 62, University Press, Seite(n) 14-26, 2009 - Aufsatz / Paper in nicht-referierter Fachzeitschrift
	 \item Diskutieren und Zitieren: Zur paradigmatischen Konstellation aktueller ökonomischer Theorie: Dobusch L., Kapeller J. , in European Journal of Economics and Economic Policies: Intervention, Vol. 6, Nr. 2, Seite(n) 145-152, 2009 - Aufsatz / Paper in sonstiger referierter Fachzeitschrift
	 \item Why is Economics not an Evolutionary Science? New Answers to Veblen's old Question.: Dobusch L., Kapeller J. , in Journal of Economic Issues, Vol. 43, Nr. 4, Seite(n) 867-898, 2009 - Aufsatz / Paper in SSCI-Zeitschrift
\end{itemize}