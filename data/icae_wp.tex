\subsection*{2025}
\begin{enumerate}
	 \item Altreiter, C., Hager, T., \& Pühringer, S. (2025). Political Economy of Knowledge Production in Capitalist Academia: Challenges and Opportunities for Socio-Ecological Transformation. (ICAE Working Paper Series). https://www.jku.at/fileadmin/gruppen/108/ICAE\_Working\_Papers/wp164.pdf
	 \item Bäuerle, L., \& Groenewald, M. M. (2025). From monoculture to pluricultures: Recent trends in economics education. (ICAE Working Paper Series; Nr. 172). https://www.jku.at/fileadmin/gruppen/108/ICAE\_Working\_Papers/wp172.pdf
	 \item Bäuerle, L., \& Maasen, S. (2025). Transforming Economics from the Side Lane? The New Economy Space and its Thinktanks. (ICAE Working Paper Series; Nr. 170). https://www.jku.at/fileadmin/gruppen/108/ICAE\_Working\_Papers/wp170.pdf
	 \item Cserjan, L., Eder, J. T., Hornykewycz, A., \& Porak, L. et al. (2025). Mobilitätswende produzieren: Produktionsbedingungen der österreichischen Bahnindustrie und industrielle Potenziale durch den Ausbau des öffentlichen Verkehrs. (ICAE Working Paper Series; Nr. 162). https://www.jku.at/fileadmin/gruppen/108/ICAE\_Working\_Papers/wp162.pdf
	 \item Pühringer, S., \& Bäuerle, L. (2025). The Economics of Socio-Ecological Transformations: A conceptual framework. (ICAE Working Paper Series; Nr. 171). https://www.jku.at/fileadmin/gruppen/108/ICAE\_Working\_Papers/wp171.pdf
	 \item Pühringer, S., Aistleitner, M., Cserjan, L., \& Hieselmayr, S. (2025). Idiosyncrasies of the oligarchic elite: On the political economy of wealth concentration in Austria. (ICAE Working Paper Series; Nr. 157).
\end{enumerate}
\subsection*{2024}
\begin{enumerate}
	 \item Gräbner-Radkowitsch, C., \& Kapeller, J. (2024). Path Dependence. (ICAE Working Paper Series; Nr. 154).
	 \item Gräbner-Radkowitsch, C., \& Kapeller, J. (2024). Systemism. (ICAE Working Paper Series; Nr. 155).
	 \item Gräbner-Radkowitsch, C., \& Kapeller, J. (2024). The micro-macro link in heterodox economics. (ICAE Working Paper Series; Nr. 153).
	 \item Kapeller, J., Hornykewycz, A., Cserjan, L., \& Weber, J. D. (2024). Dekarbonisierung des Gebäudesektors als Teil einer sozial-ökologischen Transformation: ein Gestaltungsvorschlag. (ICAE Working Paper Series; Nr. 156).
\end{enumerate}
\subsection*{2023}
\begin{enumerate}
	 \item Altreiter, C., Gräbner-Radkowitsch, C., Pühringer, S., \& Rogojanu, A. et al. (2023). The three faces of competitization: From marketization to a multiplicity of competition. (ICAE Working Paper Series; Nr. 146).
	 \item Hager, T., \& Pühringer, S. (2023). Gendered Competitive Practices in Economics. A Multi-Layer Model of Women’s Underrepresentation. (S. 1-22). (ICAE Working Paper Series; Band 148).
	 \item Hager, T., Mellacher, P., \& Rath, M. (2023). Endogenous Heterogeneous Gender Norms and the Distribution of Paid and Unpaid Work in an Intra-Household Bargaining Model. (S. 1-32). (ICAE Working Paper Series; Nr. 147).
	 \item Kapeller, J., \& Hubmann, G. (2023). Dilemmata marktliberaler Globalisierung – Globale Freiheit durch globalen Wettbewerb? (S. 1-18). (ICAE Working Paper Series; Nr. 151).
	 \item Porak, L., \& Reinke, R. (2023). The charm of emission trading: Ideas of German public economists on economic policy in times of crises. (ICAE Working Paper Series; Nr. 145).
	 \item Pühringer, S. (2023). Wie viel Wettbewerb wollen wir (uns leisten)? Zur Verwettbewerblichung der Universitäten in Österreich und darüber hinaus. (S. 1-19). (ICAE Working Paper Series; Band 149).
	 \item Pühringer, S., \& Wolfmayer, G. (2023). Organizers and promotors of academic competition? The role of (academic) social networks and platforms in the competitization of science. (S. 1-21). (ICAE Working Paper Series; Nr. 152).
	 \item Pühringer, S., \& Wolfmayr, G. (2023). Competitive Performativity of (Academic) Social Networks. The subjectivation of Competition on ResearchGate, Google Scholar und Twitter. (ICAE Working Paper Series; Nr. 150).
\end{enumerate}
\subsection*{2022}
\begin{enumerate}
	 \item Gräbner-Radkowitsch, C., \& Strunk, B. (2022). Degrowth and the Global South? How institutionalism can complement a timely discourse on ecologically sustainable development in an unequal world. (ICAE Working Paper Series; Nr. 144).
	 \item Gräbner-Radkowitsch, C., \& Strunk, B. (2022). Degrowth and the global South: the twin problem of global dependencies. (ICAE Working Paper Series; Nr. 142).
	 \item Schütz, B. (2022). Investment booms, diverging competitiveness and wage growth within a monetary union: An AB-SFC model. (ICAE Working Paper Series; Nr. 138).
\end{enumerate}
\subsection*{2021}
\begin{enumerate}
	 \item Gräbner, C., \& Hager, T. (2021). (Mis)Measuring Competitiveness: The Quantification of a Malleable Concept in the European Semester. (ICAE Working Paper Series; Nr. 130).
	 \item Hager, T., Heck, I., \& Rath, J. (2021). Competition in Transitional Processes: Polanyi \& Schumpeter. (ICAE Working Paper Series; Nr. 128).
\end{enumerate}
\subsection*{2020}
\begin{enumerate}
	 \item Beyer, K., Griesser, M., \& Pühringer, S. (2020). Zwischen Meritokratie und Wohlfahrtschauvinismus. (ICAE Working Paper Series; Nr. 109).
	 \item Hager, T. (2020). Special Interest Groups \& Growth. (ICAE Working Paper Series; Nr. 116).
	 \item Hirte, K. (2020). Das doppelte Relexionsproblem in der Ökonomik. (S. 1-19). (ICAE Working Paper Series; Nr. 115).
	 \item Hirte, K. (2020). Friedman’s Instrumentalismus und das Problem von Kopernikus. (S. 1-25). (ICAE Working Paper Series; Nr. 114).
	 \item Rath, J. (2020). Substituting Trust by Technology: A Comparative Study. (ICAE Working Paper Series; Nr. 107).
\end{enumerate}
\subsection*{2019}
\begin{enumerate}
	 \item Aistleitner, M., \& Pühringer, S. (2019). Exploring the trade narrative in top economics journals. (ICAE Working Paper Series; Nr. 97).
	 \item Gräbner, C. (2019). Unrealistic models and how to identify them: on accounts of model realisticness. (ICAE Working Paper Series; Nr. 90).
	 \item Gräbner, C., \& Ghorbani, A. (2019). Defining institutions -- A review and a synthesis. (ICAE Working Paper Series; Nr. 89).
	 \item Gräbner, C., Heimberger, P., \& Kapeller, J. (2019). Export performance, price comeptitiveness and technology: Revisiting the Kaldor paradox. (ICAE Working Paper Series; Nr. 88).
	 \item Hirte, K. (2019). Das dritte Gossensche Gesetz. Zur Überlieferungspraxis in der ökonomischen Dogmengeschichte. (S. 1-63). (ICAE Working Paper Series; Nr. 93).
	 \item Kapeller, J., Gräbner, C., \& Heimberger, P. (2019). Wirtschaftliche Polarisierung in Europa: Ursachen und Handlungsoptionen. (ICAE Working Paper Series; Nr. 98).
	 \item Raghavendra, S., \& Piiroinen, P. T. (2019). Confict as a closure: A Kaleckian model of growth and distribution under fnancialization. (ICAE Working Paper Series; Nr. 96).
	 \item Tamesberger, D., \& Theurl, S. (2019). Vorschlag für eine Jobgarantie für Langzeitarbeitslose in Österreich. (ICAE Working Paper Series; Nr. 100).
	 \item Wildauer, R., \& Kapeller, J. (2019). A Comment on Fitting Pareto Tails to Complex Survey Data. (ICAE Working Paper Series; Nr. 102).
\end{enumerate}
\subsection*{2018}
\begin{enumerate}
	 \item Aigner, E., Aistleitner, M., Glötzl, F., \& Kapeller, J. (2018). The focus of academic economics: before and after the crisis. (ICAE Working Paper Series; Nr. 75).
	 \item Aistleitner, M., Grimm, C., \& Kapeller, J. (2018). Auftragsvergabe, Leistungsqualität und Kostenintensität im Schienenpersonenverkehr. (ICAE Working Paper Series; Nr. 86).
	 \item Buyinza, F., \& Kapeller, J. (2018). Household Electrification and Education Outcomes: Panel Evidence from Uganda. (ICAE Working Paper Series; Nr. 85).
	 \item Buyinza, F., Tibaingana, A., \& Mutenyo, J. (2018). Factors Affecting Acces to Formal Credit by Micro and Small Enterprises in Ugande. (ICAE Working Paper Series; Nr. 83).
	 \item Foissner, F. (2018). Folgen einer möglichen Abschaffung der Notstandshilfe in Oberösterreich. (ICAE Working Paper Series; Nr. 87).
	 \item Griesser, M. (2018). Arbeitsmarktpolitik als Gesellschaftspolitik. (ICAE Working Paper Series; Nr. 78).
	 \item Schütz, B. (2018). Employment and the minimum wage: A pluralist approach. (ICAE Working Paper Series; Nr. 81).
\end{enumerate}
\subsection*{2017}
\begin{enumerate}
	 \item Beyer, K., Grimm, C., Kapeller, J., \& Pühringer, S. (2017). Der deutsche Sonderweg im Fokus: Eine vergleichende Analyse der paradig¬matischen Struktur und der politischen Orientierung der deutschen und US-amerikanischen Ökonomie. (ICAE Working Paper Series; Nr. 71).
	 \item Ferschli, B., Kapeller, J., Schütz, B., \& Wildauer, R. (2017). Bestände und Konzentration privater Vermögen in Österreich. (ICAE Working Paper Series; Nr. 72).
	 \item Grimm, C., Kapeller, J., \& Pühringer, S. (2017). Zum Profil der deutschsprachigen Volkswirtschaftslehre. (ICAE Working Paper Series; Nr. 70).
	 \item Huber, J., Heimberger, P., \& Kapeller, J. (2017). From paradigms to policies: Economic models in the EU’s fiscal regulation framework. (ICAE Working Paper Series; Nr. 61).
	 \item Humer, S., Moser, M., \& Schnetzer, M. (2017). Inheritances and the Accumulation of Wealth in the Eurozone. (ICAE Working Paper Series; Nr. 73).
	 \item Kapeller, J. (2017). Pluralism in Economics: Epistemological Rationales and Pedagogical Implementation. (ICAE Working Paper Series; Nr. 68).
	 \item Kapeller, J., \& Steinerberger, S. (2017). Emergent Phenomena in Scientific Publishing: A Simulation Exercise. (ICAE Working Paper Series; Nr. 66).
	 \item Ötsch, W., \& Pühringer, S. (2017). Right-wing populism and market-fundamentalism. (ICAE Working Paper Series; Band 59).
\end{enumerate}
\subsection*{2016}
\begin{enumerate}
	 \item Aistleitner, M., Kapeller, J., \& Steinerberger, S. (2016). The Power of Scientometrics and the Development of Economics. (ICAE Working Paper Series; Nr. 46).
	 \item Dobusch, L., Wandl, S., \& Kapeller, J. (2016). Monocular Accounting and its Discontents: The case of the stability and growth pact. (ICAE Working Paper Series; Nr. 43).
	 \item Grimm, C. (2016). Postdemokratie, Machtverhältnisse und Ökonomik. (ICAE Working Paper Series; Nr. 54).
	 \item Grimm, C. (2016). Wirtschaftspolitische Positionen österreichischer Parteien im historischen Verlauf. (ICAE Working Paper Series; Nr. 51).
	 \item Heimberger, P. (2016). Die aktuelle Krise im wirtschaftshistorischen Vergleich mit der Großen Depression der 1930er-Jahre. (ICAE Working Paper Series; Nr. 42).
	 \item Heimberger, P. (2016). Wirtschaftliche Stagnation als \glqq neue Normalsituation\grqq{}? (ICAE Working Paper Series; Nr. 48).
	 \item Heimberger, P., \& Kapeller, J. (2016). A model-based measurement device in European fiscal policy-making: The ontology and epistemology of potential output. (ICAE Working Paper Series; Nr. 55).
	 \item Heimberger, P., \& Kapeller, J. (2016). The performativity of potential output. (ICAE Working Paper Series; Nr. 50).
	 \item Pühringer, S. (2016). Still the queen of the social sciences? (ICAE Working Paper Series; Nr. 52).
	 \item Pühringer, S. (2016). Think Tank Networks of German Neoliberalism. (ICAE Working Paper Series; Nr. 53).
	 \item Pühringer, S., \& Griesser, M. (2016). Has economics returned to being the “dismal science”? The changing role of economic thought in German labour market reforms. (ICAE Working Paper Series; Band 49).
	 \item Pühringer, S., \& Stelzer-Orthofer, C. (2016). Neoliberale Think Tanks als (neue) Akteure in österreichischen gesellschafts- und sozialpolitischen Diskursen. (ICAE Working Paper Series; Nr. 44).
\end{enumerate}
\subsection*{2015}
\begin{enumerate}
	 \item Aistleitner, M., Fölker, M., Kapeller, J., \& Mohr, F. X. et al. (2015). Verteilung und Gerechtigkeit: Philosophische Perspektiven. (ICAE Working Paper Series; Nr. 32).
	 \item Beyer, K. (2015). Emanzipation bei Marx und seine Kritik an Proudhon. (ICAE Working Paper Series; Nr. 34).
	 \item Beyer, K., \& Bräutigam, L. (2015). Shadow Banking and the Offshore Nexus -- Some Considerations on the Systemic Linkages of Two Important Economic Phenomena. (ICAE Working Paper Series; Nr. 40).
	 \item Heimberger, P. (2015). Did Fiscal Consolidation Cause the Double-Dip Recession in the Euro Area? (ICAE Working Paper Series; Nr. 41).
	 \item Pühringer, S. (2015). Markets as “ultimate judges” of economic policies -- Angela Merkel´s discourse profile during the economic crisis and the European crisis policies. (ICAE Working Paper Series; Nr. 31).
	 \item Pühringer, S. (2015). Wie wirken ÖkonomInnen und Ökonomik auf Politik und Gesellschaft? Darstellung des gesellschaftlichen und politischen Einflusspotenzials von ÖkonomInnen anhand eines „Performativen Fußabdrucks“. (ICAE Working Paper Series; Nr. 35).
	 \item Ötsch, W. (2015). Ökonomie und Moral. (ICAE Working Paper Series; Nr. 39).
	 \item Ötsch, W., \& Pühringer, S. (2015). Marktradikalismus als Politische Ökonomie. Wirtschaftswissenschaften und ihre Netzwerke in Deutschland ab 1945. (ICAE Working Paper Series; Nr. 38).
\end{enumerate}
\subsection*{2014}
\begin{enumerate}
	 \item Kapeller, J., Schütz, B., \& Tamesberger, D. (2014). From Free to Civilized Markets: First Steps towards Eutopia. (ICAE Working Paper Series; Nr. 28).
	 \item Pühringer, S. (2014). Kontinuitäten neoliberaler Wirtschaftspolitik in der Krise. (ICAE Working Paper Series; Nr. 30).
	 \item Ötsch, W. (2014). Relationships are Constructed from Generalized Unconscious Social Images Kept in Steady Locations in Mental Space. (ICAE Working Paper Series; Nr. 29).
	 \item Ötsch, W. (2014). Warum war Keynes so erfolgreich? Eine Darstellung anhand der Methode von Bruno Latour. (ICAE Working Paper Series; Nr. 27).
\end{enumerate}
\subsection*{2013}
\begin{enumerate}
	 \item Beyer, K., \& Bräutigam, L. (2013). CDOs – A Critical Phenomenon of the Financial System in Crisis. (ICAE Working Paper Series; Nr. 21).
	 \item Beyer, K., \& Ötsch, W. (2013). Die Finanzkrise 2007-2009 als Krise von Schattenbanken. Eine einführende institutionelle Analyse. (ICAE Working Paper Series; Nr. 17).
	 \item Hirte, K. (2013). Mainstream, Orthodoxie und Heterodoxie -- Zur Klassifizierung der Wirtschaftswissenschaften. (ICAE Working Paper Series; Nr. 16).
	 \item Hirte, K. (2013). Zauberwort „Entkopplung“ – Ein Rückblick auf den Diskurs zur Agrarreform 2003 sowie die dortigen Diskrepanzen zwischen Versprechen und Wirkprinzipien der so genannten „Entkopplung“. (ICAE Working Paper Series; Nr. 12).
	 \item Nordmann, J. (2013). Das neoliberale Selbst. Zur Genese und Kritik neuer Subjektkonstruktionen. (ICAE Working Paper Series; Nr. 22).
	 \item Nordmann, J. (2013). Die neoliberale Gesellschaft. Ein theoretischer Umriss. (ICAE Working Paper Series; Nr. 24).
	 \item Plaimer, W., \& Pühringer, S. (2013). Der Fiskalpakt und seine Implementation in Österreich. (ICAE Working Paper Series; Band 2).
	 \item Pühringer, S. (2013). Aus den Vorhöfen der Macht in die Medien zur eigenen Partei. Formen der Einflussnahme von ÖkonomInnen auf Politik und Wirtschaft im Zuge der Finanz-­ und Wirtschaftskrisenpolitik. (ICAE Working Paper Series; Nr. 20).
	 \item Pühringer, S. (2013). The implementation of the European Fiscal Compact in Austria as a post-democratic phenomenon. (ICAE Working Paper Series; Nr. 15).
	 \item Pühringer, S., \& Hirte, K. (2013). The financial crisis as a tsunami. Discourse profiles of economists in the financial crisis. (ICAE Working Paper Series; Nr. 14).
	 \item Pühringer, S., \& Hirte, K. (2013). ÖkonomInnen in der Finanzkrise -- Netzwerkanalytische Sicht auf die deutschsprachigen ÖkonomInnen in der Finanzkrise. (S. 1-21). (ICAE Working Paper Series; Nr. 18).
	 \item Pühringer, S., \& Ötsch, W. (2013). Das Team Stronach. Eine österreichische Tea Party? (ICAE Working Paper Series; Nr. 19).
	 \item Ötsch, W. (2013). Populismus und Demagogie. Mit Beispielen von Jörg Haider, Heinz–Christian Strache und Frank Stronach. (ICAE Working Paper Series; Nr. 25).
\end{enumerate}
\subsection*{2012}
\begin{enumerate}
	 \item Kapeller, J., \& Pühringer, S. (2012). Democracy in liberalism and neoliberalism. The case of Popper and Hayek. (ICAE Working Paper Series; Nr. 10).
	 \item Pühringer, S., \& Hirte, K. (2012). Erdbeben, Fieber und zarte Pflänzchen. Chronologischer Verlauf des Finanzkrisen‐Diskurses deutschsprachiger Ökonomen. (ICAE Working Paper Series; Nr. 9).
	 \item Ötsch, W. (2012). Die Macht der Ratingagenturen. (ICAE Working Paper Series; Nr. 8).
	 \item Ötsch, W. (2012). Marktradikalität. Der Diskurs von \glqq dem Markt\grqq{}. (ICAE Working Paper Series; Nr. 7).
	 \item Ötsch, W. (2012). The Deep Meaning of “Market”. A Key to Understand the Neoliberal-­‐Market-­‐ Radical Society. (ICAE Working Paper Series; Nr. 11).
\end{enumerate}
\subsection*{2011}
\begin{enumerate}
	 \item Plaimer, W. (2011). Demokratiedefizit im Gesetzgebungsprozess budgetrelevanter Gesetze auf Bundesebene in Österreich. (ICAE Working Paper Series; Nr. 4).
	 \item Pühringer, M., \& Pühringer, S. (2011). Solidarität im Kapitalismus. Zur Unmöglichkeit einer Forderung. (ICAE Working Paper Series; Nr. 6).
	 \item Pühringer, S. (2011). Gleichheit versus Vielfalt. Ein konstruierter Widerspruch? (ICAE Working Paper Series; Nr. 3).
	 \item Ötsch, W. (2011). Politische Ökonomie und Gesellschaft. Eine theoriegeschichtliche Skizze. (ICAE Working Paper Series; Nr. 5).
\end{enumerate}
\subsection*{2010}
\begin{enumerate}
	 \item Ötsch, W. (2010). Die Finanz-­ und Wirtschaftskrise seit 2007: Ein Überblick. (ICAE Working Paper Series; Nr. 2).
	 \item Ötsch, W., Hirte, K., \& Nordmann, J. (2010). Die Evolution des ökonomischen Wissens und des Wissens über den Kapitalismus. Performativity als Analyseinstrument: das Beispiel der Fabian Society, der Mont Pèlerin Society und der Chicagoer Schule. (ICAE Working Paper Series; Nr. 1).
\end{enumerate}
