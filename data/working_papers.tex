\begin{enumerate}[leftmargin=*, labelsep=0.5cm]
\end{enumerate} 
 \subsection*{2025} 
 \begin{enumerate}[leftmargin=*, labelsep=0.5cm] 
	 \item Pühringer S., Aistleitner M., Cserjan L., Hieselmayr S.:  Idiosyncrasies of the oligarchic elite: On the political economy of wealth concentration in Austria  , Serie Working Paper Series, Nr. 157, 2025
\end{enumerate} 
 \subsection*{2024} 
 \begin{enumerate}[leftmargin=*, labelsep=0.5cm] 
	 \item Kapeller J., Hornykewycz A., Cserjan L., Weber J.:  Dekarbonisierung des Gebäudesektors als Teil einer sozial-ökologischen Transformation: ein Gestaltungsvorschlag  , Serie ICAE Working Paper Series, Nr. 156, 2024
	 \item Gräbner-Radkowitsch C., Kapeller J.:  The micro-macro link in heterodox economics  , Serie Working Paper Series, Nr. 154, 2024
	 \item Gräbner-Radkowitsch C., Kapeller J.:  Systemism  , Serie Working Paper Series, Nr. 155, 2024
	 \item Gräbner-Radkowitsch C., Kapeller J.:  Path Dependence  , Serie Working Paper Series, Nr. 154, 2024
\end{enumerate} 
 \subsection*{2023} 
 \begin{enumerate}[leftmargin=*, labelsep=0.5cm] 
	 \item Pühringer S., Wolfmayr G.:  Competitive Performativity of (Academic) Social Networks. The subjectivation of Competition on ResearchGate, Google Scholar und Twitter  , Serie ICAE Working Paper Series, Nr. 150, 2023
	 \item Pühringer S., Wolfmayer G.:  Organizers and promotors of academic competition? The role of (academic) social networks and platforms in the competitization of science  , Serie ICAE Working Paper Series, Nr. 152, Seite(n) 1-21, 2023
	 \item Pühringer S., Wolfmayer G.:  Competitive Performativity of Academic Social Networks. The Subjectification of Competition on Researchgate, Twitter and Google Scholar.  , Serie SPACE Working Paper Series, Nr. 19, Seite(n) 1-25, 2023
	 \item Pühringer S.:  Wie viel Wettbewerb wollen wir (uns leisten)? Zur Verwettbewerblichung der Universitäten in Österreich und darüber hinaus  , Serie ICAE Working Paper Series, Vol. 149, Seite(n) 1-19, 2023
	 \item Porak L., Reinke R.:  The charm of emission trading: Ideas of German public economists on economic policy in times of crises  , Serie ICAE Working Paper Series, Nr. 145, 2023
	 \item Kapeller J., Hubmann G.:  Dilemmata marktliberaler Globalisierung – Globale Freiheit durch globalen Wettbewerb?  , Serie ICAE Working Paper Series, Nr. 151, Seite(n) 1-18, 2023
	 \item Hager T., Pühringer S.:  Gendered Competitive Practices in Economics. A Multi-Layer Model of Women’s Underrepresentation  , Serie ICAE Working Paper Series, Vol. 148, Seite(n) 1-22, 2023
	 \item Hager T., Mellacher P., Rath M.:  Endogenous Heterogeneous Gender Norms and the Distribution of Paid and Unpaid Work in an Intra-Household Bargaining Model  , Serie ICAE Working Paper Series, Nr. 147, Seite(n) 1-32, 2023
	 \item Altreiter C., Gräbner-Radkowitsch C., Pühringer S., Rogojanu A., Wolfmayr G.:  The three faces of competitization: From marketization to a multiplicity of competition  , 2023
	 \item Altreiter C., Gräbner-Radkowitsch C., Pühringer S., Rogojanu A., Wolfmayr G.:  The three faces of competitization: From marketization to a multiplicity of competition  , Serie SPACE Working Paper Series, Nr. 18, 2023
\end{enumerate} 
 \subsection*{2022} 
 \begin{enumerate}[leftmargin=*, labelsep=0.5cm] 
	 \item Schütz B.:  Investment booms, diverging competitiveness and wage growth within a monetary union: An AB-SFC model  , Serie ICAE Working Paper Series, Nr. 138, 2022
	 \item Reiner C., Bellak C.:  Hat die ökonomische Macht von Unternehmen in Österreich zugenommen?  , 2022
	 \item Reiner C.:  It’s the End of Globalization as We Know It! Zeitgemäße Betrachtungen zur politischen Ökonomik der Globalisierungskrise  , 2022
	 \item Hager T., Heck I., Rath J.:  Polanyi and Schumpeter: Transitional Processes via Societal Spheres  , Serie SPACE Working Paper Series, Nr. 16, Seite(n) 1-25, 2022
	 \item Gräbner-Radkowitsch C., Strunk B.:  Degrowth and the global South: the twin problem of global dependencies  , Serie ICAE Working Paper Series, Nr. 142, 2022
	 \item Gräbner-Radkowitsch C., Strunk B.:  Degrowth and the Global South? How institutionalism can complement a timely discourse on ecologically sustainable development in an unequal world  , Serie ICAE Working Paper Series, Nr. 144, 2022
	 \item Gräbner-Radkowitsch C., Hornykewycz A., Schütz B.:  The emergence of debt and secular stagnation in an unequal society: a stockflow consistent agent-based approach  , 2022
	 \item Gräbner-Radkowitsch C., Hornykewycz A., Schütz B.:  The emergence of debt and secular stagnation in an unequal society: a stock- flow consistent agent-based approach  , 2022
	 \item Gräbner-Radkowitsch C., Hager T., Hornykewycz A.:  Competing for sustainability? An institutionalist analysis of the new development model of the European Union  , Serie SPACE Working Paper Series, Nr. 17, 2022
	 \item Gräbner-Radkowitsch C., Hager T., Hornykewycz A.:  Competing for sustainability? An institutionalist analysis of the new development model of the European Union  , 2022
	 \item Gräbner-Radkowitsch C.:  Elements of an evolutionary approach to comparative economic studies: complexity, systemism, and path dependent development  , 2022
	 \item Altreiter C., Azevedo S., Porak L., Pühringer S., Wolfmayr G.:  Winning urban competition with a social agenda. The competition imaginary in Viennese urban development plans  , 2022
	 \item Altreiter C., Azevedo S., Porak L., Pühringer S., Wolfmayr G.:  Winning urban competition with a social agenda. The competition imaginary in Viennese urban development  , Serie SPACE Working Paper Series, Nr. 13, 2022
	 \item Aistleitner M., Kapeller J., Kronberger D.:  The Authors of Economics Journals Revisited: Evidence from a Large-Scale Replication of Hodgson & Rothman (1999)  , 2022
	 \item Aistleitner M.:  Development and Interdisciplinarity: re-examining the “economics silo”  , 2022
	 \item Aistleitner M.:  Development and Interdisciplinarity: re-examining the “economic silo”  , Serie SPACE Working Paper Series, Nr. 14, 2022
\end{enumerate} 
 \subsection*{2021} 
 \begin{enumerate}[leftmargin=*, labelsep=0.5cm] 
	 \item Tamesberger D., Theurl S.:  Design and Take Up of Austria’s Coronavirus Short Time Work Model  , 2021
	 \item Pühringer S., Porak L., Rath J.:  Talking about competition? Discursive shifts in the economic imaginary of competition in public debates  , 2021
	 \item Pühringer S.:  Zur Pluralität der ökonomischen Politikberatung in Deutschland  , 2021
	 \item Porak L.:  Governing the Ungovernable. Recontextualizations of Competition in European Policy Discourse  , in ICAE Linz, Serie SPACE Working Paper Series, Nr. 8, 2021
	 \item Porak L.:  Governing the Ungovernable - Recontextualizations of ‘Competition’ in European Policy Discourse  , 2021
	 \item Kapeller J., Leitch S., Wildauer R.:  Is a 10 trillion euro European climate investment initiative fiscally sustainable?  , 2021
	 \item Kapeller J., Leitch S., Wildauer R.:  A European Wealth Tax for a Fair and Green Recovery  , 2021
	 \item Kapeller J., Gräbner-Radkowitsch C.:  Standortwettbewerb und Deindustrialisierung: Das Beispiel MAN als Lehrbuchfall  , 2021
	 \item Heimberger P., Gechert S.:  Do Corporate Tax Cuts Boost Economic Growth?  , Serie wiiw Working Papers, Nr. 201, forthcoming, doi/full/10.1080/09692290.2021.1904269, 2021
	 \item Heimberger P.:  Do Higher Public Debt Levels Reduce Economic Growth?  , Serie wiiw Working Papers, Nr. 211, 2021
	 \item Hager T., Heck I., Rath J.:  Competition in Transitional Processes: Polanyi & Schumpeter  , Serie ICAE Working Paper Series, Nr. 128, 2021
	 \item Gräbner-Radkowitsch C., Pühringer S.:  Competition universalism: Its historical origins and timely alternatives  , 2021
	 \item Gräbner-Radkowitsch C., Heimberger P., Kapeller J., Landesmann M., Schütz B.:  The evolution of debtor-creditor relationships within a monetary union: Trade imbalances, excess reserves and economic policy  , in Universität Duisburg, Serie Ifso working paper, Nr. 10, 2021
	 \item Gräbner-Radkowitsch C., Heimberger P., Kapeller J., Landesmann M., Schütz B.:  The evolution of debtor-creditor relationships within a monetary union: Trade imbalances, excess reserves and economic policy  , 2021
	 \item Gräbner C., Hager T.:  (Mis)Measuring Competitiveness: The Quantification of a Malleable Concept in the European Semester  , Serie ICAE Working Paper Series, Nr. 130, 2021
	 \item Gräbner C., Hager T.:  (Mis)Measuring Competitiveness: The Quantification of a Malleable Concept in the European Semester  , Serie SPACE Working Paper Series, Nr. 10, 2021
	 \item Eder J.:  Decreasing Dependency through Self-Reliance: Strengthening Local Economies through Community Wealth Building  , 2021
	 \item Altreiter C., Gräbner C., Pühringer S., Rogojanu A., Wolfmayr G.:  Theorizing Competition. An Interdisciplinary Framework  , Serie SPACE Working Paper Series, Nr. 4, 2021
	 \item Altreiter C., Gräbner C., Pühringer S., Rogojanu A., Wolfmayr G.:  Theorizing Competition - An interdisciplinary approach to the genesis of a contested concept  , Serie SPACE Working Paper Series, Nr. 3, 2021
	 \item Aistleitner M., Pühringer S.:  Exploring the trade (policy) narratives in economic elite discourse  , Serie SPACE Working Paper Series, Nr. 1, Seite(n) 1-16, 2021
\end{enumerate} 
 \subsection*{2020} 
 \begin{enumerate}[leftmargin=*, labelsep=0.5cm] 
	 \item Wildauer R., Leitch S., Kapeller J.:  How to boost the European Green Deal’s scale and ambition  , 2020
	 \item Schulmeister S.:  Fixing long-term price paths for fossil energy – the optimal incentive for limiting global warming  , 2020
	 \item Rath J.:  Substituting Trust by Technology: A Comparative Study  , Serie ICAE Working Paper Series, Nr. 107, 2020
	 \item Pühringer S., Rath J., Griesebner T.:  The political economy of academic publishing: On the commodification of a public good  , 2020
	 \item Pühringer S., Porak L., Rath J.:  Talking about Competition? Discursive Shifts in the Economic Imaginary of Competition in Public Debates  , Serie SPACE Working Paper Series, Nr. 6, 2020
	 \item Porak L.:  Der größte ‘Trumpf’ Europas: Eine Analyse des ‘economic imaginary’ der Europäischen Kommission  , 2020
	 \item Kapeller J., Gräbner-Radkowitsch C.:  Konzernmacht in globalen Güterketten  , 2020
	 \item Hornykewycz A., Gräbner-Radkowitsch C.:  Capability Accumulation and Product Innovation: Agent-Based Perspective  , Serie Rebuilding Macroeconomics Working Paper Series, Nr. 9, Seite(n) 1-22, 2020
	 \item Hirte K.:  Friedman’s Instrumentalismus und das Problem von Kopernikus  , Serie ICAE Working Paper Series, Nr. 114, Seite(n) 1-25, 2020
	 \item Hirte K.:  Das doppelte Relexionsproblem in der Ökonomik  , Serie ICAE Working Paper Series, Nr. 115, Seite(n) 1-19, 2020
	 \item Hager T.:  Special Interest Groups & Growth  , Serie ICAE Working Paper Series, Nr. 116, 2020
	 \item Gräbner-Radkowitsch C., Hornykewycz A.:  Capability accumulation and product innovation: an agent-based perspective  , 2020
	 \item Gräbner-Radkowitsch C., Heimberger P., Kapeller J.:  Pandemic pushes polarisation: The Corona crisis and macroeconomic divergence in the Eurozone  , 2020
	 \item Gräbner-Radkowitsch C., Heimberger P., Kapeller J.:  Do the “smart kids” catch up? Technological capabilities, globalisation and economic growth  , 2020
	 \item Gräbner-Radkowitsch C., Hafele J.:  The emergence of core-periphery structures in the European Union: a complexity perspective  , 2020
	 \item Gräbner-Radkowitsch C., Hafele J.:  Emergence of Core-Periphery Structures in the European Union: A Complexity Perspective  , Serie Rebuilding Macroeconomics Working Paper Series, Nr. 17, Seite(n) 1-21, 2020
	 \item Beyer K., Griesser M., Pühringer S.:  Zwischen Meritokratie und Wohlfahrtschauvinismus  , Serie ICAE Working Paper Series, Nr. 109, 2020
	 \item Altreiter C., Gräbner-Radkowitsch C., Pühringer S., Rogojanu A., Wolfmayr G.:  Theorizing competition: an interdisciplinary framework  , 2020
	 \item Altreiter C., Gräbner-Radkowitsch C., Pühringer S., Rogojanu A., Wolfmayr G.:  Theorizing Competition: an interdisciplinary approach  , 2020
	 \item Aistleitner M., Pühringer S.:  Exploring the trade (policy) narratives in economic elite discourse  , 2020
	 \item Aistleitner M., Gräbner-Radkowitsch C., Hornykewycz A.:  Theory and empirics of capability accumulation: implications for macroeconomics modelling  , Serie Rebuilding Macroeconomics Working Paper Series, Nr. 6, Seite(n) 1-29, 2020
	 \item Aistleitner M., Gräbner-Radkowitsch C., Hornykewycz A.:  Theory and Empirics of Capability Accumulation: Implications for Macroeconomic Modelling  , 2020
\end{enumerate} 
 \subsection*{2019} 
 \begin{enumerate}[leftmargin=*, labelsep=0.5cm] 
	 \item Ötsch W., Pühringer S.:  The anti-democratic logic of right-wing Populism and neoliberal market-fundamentalism  , in Institut für Ökonomie, Cusanus Hochschule, Serie Working Paper Serie ök, Nr. 48, 2019
	 \item Wildauer R., Kapeller J.:  Rank Correction: A New Approach to Differential Non-Response in Wealth Survey Data  , 2019
	 \item Wildauer R., Kapeller J.:  A Comment on Fitting Pareto Tails to Complex Survey Data  , Serie ICAE Working Paper Series, Nr. 102, 2019
	 \item Tamesberger D., Theurl S.:  Vorschlag für eine Jobgarantie für Langzeitarbeitslose in Österreich  , Serie ICAE Working Paper Series, Nr. 100, 2019
	 \item Raghavendra S., Piiroinen P.:  Confict as a closure: A Kaleckian model of growth and distribution under fnancialization  , Serie ICAE Working Paper Series, Nr. 96, 2019
	 \item Pühringer S., Ötsch W.:  Die Wirkmacht der „Liebe zum Markt”: Zum anhaltenden Einfluss ordoliberaler ÖkonomInnen-Netzwerke in Politik und Gesellschaft  , 2019
	 \item Pühringer S., Ötsch W.:  Die Wirkmacht der "Liebe zum Markt": Zum anhaltenden Einfluss ordoliberaler ÖkonomInnenNetzwerke in Politik und Gesellschaft.  , in Institut für Ökonomie, Cusanus Hochschule, Serie Working Paper Serie ök, Nr. 51, 2019
	 \item Kapeller J., Gräbner-Radkowitsch C., Heimberger P.:  Economic Polarisation in Europe: Causes and Policy Options  , 2019
	 \item Kapeller J., Gräbner C., Heimberger P.:  Wirtschaftliche Polarisierung in Europa: Ursachen und Handlungsoptionen  , Serie ICAE Working Paper Series, Nr. 98, 2019
	 \item Hirte K.:  Das dritte Gossensche Gesetz. Zur Überlieferungspraxis in der ökonomischen Dogmengeschichte  , Serie ICAE Working Paper Series, Nr. 93, Seite(n) 1-63, 2019
	 \item Heimberger P.:  The Impact of Labour Market Institutions and Capital Accumulation on Unemployment: Evidence for the OECD, 1985-2013  , Serie wiiw Working Papers, Nr. 164, 2019
	 \item Heimberger P.:  Does economic globalisation affect income inequality? A meta-analysis  , Serie wiiw Working Papers, Nr. 165, 2019
	 \item Gräbner-Radkowitsch C., Tamesberger D., Heimberger P., Kapelari T., Kapeller J.:  Trade Models in the European Union  , 2019
	 \item Gräbner C., Heimberger P., Kapeller J.:  Export performance, price comeptitiveness and technology: Revisiting the Kaldor paradox  , Serie ICAE Working Paper Series, Nr. 88, 2019
	 \item Gräbner C., Ghorbani A.:  Defining institutions - A review and a synthesis  , Serie ICAE Working Paper Series, Nr. 89, 2019
	 \item Gräbner C.:  Unrealistic models and how to identify them: on accounts of model realisticness  , Serie ICAE Working Paper Series, Nr. 90, 2019
	 \item Flechtner S., Gräbner-Radkowitsch C.:  The heterogeneous relationship between income and inequality: a panel co-integration approach  , 2019
	 \item Beyer K., Pühringer S.:  Divided we stand? Professional consensus and political conflict in academic economics  , 2019
	 \item Beyer K., Pühringer S.:  Divided we stand? Professional consensus and political conflict in academic economics  , in Institut für Ökonomie, Cusanus Hochschule, Serie Working Paper Serie ök, Nr. 51, 2019
	 \item Aistleitner M., Pühringer S.:  Exploring the trade narrative in top economics journals  , Serie ICAE Working Paper Series, Nr. 97, 2019
\end{enumerate} 
 \subsection*{2018} 
 \begin{enumerate}[leftmargin=*, labelsep=0.5cm] 
	 \item Ötsch W., Pühringer S.:  Marktfundamentalismus als Kollektivgedanke. Mises und die Ordoliberalen  , in Institut für Ökonomie, Cusanus Hochschule, Bernkastel-Kues, Serie Working Paper Serie ök, Nr. 41, 2018
	 \item Ötsch W., Graupe S.:  Der vergessene Lippmann: Politik, Propaganda und Markt  , in Institut für Ökonomie, Cusanus Hochschule, Bernkastel-Kues, Serie Working Paper Serie ök, Nr. 39, 2018
	 \item Ötsch W.:  Wissen und Nicht-Wissen angesichts "des Marktes": Das Konzept von Hayek  , in Institut für Ökonomie, Cusanus Hochschule, Bernkastel-Kues, Serie Working Paper Serie ök, Vol. 43, 2018
	 \item Ötsch W.:  Bilder in der Geschichte der Ökonomie: Das Beispiel der Metapher von der Wirtschaft als Maschine  , in Institut für Ökonomie, Cusanus Hochschule, Bernkastel-Kues., Serie Working Paper Serie ök, Vol. 42, 2018
	 \item Schütz B.:  Employment and the minimum wage: A pluralist approach  , Serie ICAE Working Paper Series, Nr. 81, 2018
	 \item Pühringer S., Bäuerle L.:  What economics education is missing: The real world  , in Institut für Ökonomie, Cusanus Hochschule, Serie Working Paper Serie ök, Nr. 37, 2018
	 \item Pühringer S.:  The “eternal character” of austerity measures in European crisis policies.  , 2018
	 \item Heimberger P.:  The dynamic effects of fiscal consolidation episodes on income inequality: Evidence for 17 OECD Countries over 1978-2013  , 2018
	 \item Gräbner-Radkowitsch C., Strunk B.:  Pluralism in economics: its critiques and their lessons  , 2018
	 \item Gräbner-Radkowitsch C., Heimberger P., Kapeller J., Springholz F.:  Measuring Economic Openness: A review of existing measures and empirical practices  , 2018
	 \item Gräbner-Radkowitsch C., Heimberger P., Kapeller J., Schütz B.:  Structural change in times of increasing openness: assessing path dependency in European economic integration  , 2018
	 \item Grimm C., Kapeller J., Pühringer S.:  Paradigms and Policies: The state of economics in the german-speaking countries  , 2018
	 \item Griesser M.:  Arbeitsmarktpolitik als Gesellschaftspolitik  , Serie ICAE Working Paper Series, Nr. 78, 2018
	 \item Foissner F.:  Folgen einer möglichen Abschaffung der Notstandshilfe in Oberösterreich  , Serie ICAE Working Paper Series, Nr. 87, 2018
	 \item Buyinza F., Tibaingana A., Mutenyo J.:  Factors Affecting Acces to Formal Credit by Micro and Small Enterprises in Ugande  , Serie ICAE Working Paper Series, Nr. 83, 2018
	 \item Buyinza F., Kapeller J.:  Household Electrification and Education Outcomes: Panel Evidence from Uganda  , Serie ICAE Working Paper Series, Nr. 85, 2018
	 \item Aistleitner M., Grimm C., Kapeller J.:  Auftragsvergabe, Leistungsqualität und Kostenintensität im Schienenpersonenverkehr  , Serie ICAE Working Paper Series, Nr. 86, 2018
	 \item Aigner E., Aistleitner M., Glötzl F., Kapeller J.:  The focus of academic economics: before and after the crisis  , Serie ICAE Working Paper Series, Nr. 75, 2018
	 \item Aigner E., Aistleitner M., Glötzl F., Kapeller J.:  The Focus of Academic Economics: Before and After the Crisis  , in Institute for New Economic Thinking (INET), Serie Institute for New Economic Thinking  Working Paper Series, 2018
\end{enumerate} 
 \subsection*{2017} 
 \begin{enumerate}[leftmargin=*, labelsep=0.5cm] 
	 \item Ötsch W., Pühringer S.:  Right-wing populism and market-fundamentalism  , Serie ICAE Working Paper Series, Vol. 59, 2017
	 \item Pühringer S., Ötsch W.:  Neoliberalism and Right-wing Populism: conceptual analogies  , in Institut für Ökonomie, Cusanus Hochschule, Bernkastel-Kues, Serie Working Paper Serie ök, Vol. 36, 2017
	 \item Pühringer S., Liedl B.:  Argumentationsstrategien einer neoliberalen Reformagenda: Zum Diskursprofil der Agenda Austria in medialen Debatten  , in Institut für Ökonomie, Cusanus Hochschule, Bernkastel-Kues, Serie Working Paper Serie ök, Vol. 27, 2017
	 \item Pühringer S., Griesser M.:  From the "planning euphoria" to the "bitter economic truth": The transmission of economic ideas into German labour market policies in the 1960s and 2000s  , in Institut für Ökonomie, Cusanus Hochschule, Bernkastel-Kues, Serie Working Paper Serie ök, Nr. 30, 2017
	 \item Pühringer S.:  Think tank networks of German neoliberalism power structures in economics and economic policies in post-war Germany  , in Institut für Ökonomie, Cusanus Hochschule, Bernkastel-Kues, Serie Working Paper Serie ök, Nr. 24, 2017
	 \item Pühringer S.:  The “eternal character” of austerity measures in European crisis policies. Evidences from the Fiscal Compact discourse in Austria.  , in Institut für Ökonomie, Cusanus Hochschule, Bernkastel-Kues, Serie Working Paper Serie ök, Nr. 32, 2017
	 \item Kapeller J., Steinerberger S.:  Stability, Fairness and Random Walks in the Bargaining Problem  , 2017
	 \item Kapeller J., Steinerberger S.:  Emergent Phenomena in Scientific Publishing: A Simulation Exercise  , Serie ICAE Working Paper Series, Nr. 66, 2017
	 \item Kapeller J.:  Pluralism in Economics: Epistemological Rationales and Pedagogical Implementation  , Serie ICAE Working Paper Series, Nr. 68, 2017
	 \item Kapeller J.:  Delayed by outsourcing? Zur Stabilität des Kapitalismus im 21. Jahrhundert  , 2017
	 \item Humer S., Moser M., Schnetzer M.:  Inheritances and the Accumulation of Wealth in the Eurozone  , Serie ICAE Working Paper Series, Nr. 73, 2017
	 \item Huber J., Heimberger P., Kapeller J.:  From paradigms to policies: Economic models in the EU’s fiscal regulation framework  , Serie ICAE Working Paper Series, Nr. 61, 2017
	 \item Gräbner-Radkowitsch C., Heimberger P., Kapeller J., Schütz B.:  Is Europe disintegrating? Macroeconomic divergence, structural polarization, trade and fragility  , 2017
	 \item Gräbner-Radkowitsch C., Elsner W., Lascaux A.:  Trust and Social Control. Sources of cooperation, performance, and stability in informal value transfer systems  , 2017
	 \item Gräbner-Radkowitsch C., Elsner W., Lascaux A.:  To trust or to control: Informal value transfer systems and computational analysis in institutional economics  , 2017
	 \item Gräbner-Radkowitsch C.:  The Complexity of Economies and Pluralism in Economics  , 2017
	 \item Gräbner-Radkowitsch C.:  How to relate models to reality? An epistemological framework for the validation and verification of computational models  , 2017
	 \item Grimm C., Kapeller J., Pühringer S.:  Zum Profil der deutschsprachigen Volkswirtschaftslehre  , Serie ICAE Working Paper Series, Nr. 70, 2017
	 \item Gerhartinger P., Haunschmid P., Tamesberger D.:  How to explain Wage Growth Slowdown in Austria?  , 2017
	 \item Ferschli B., Kapeller J., Schütz B., Wildauer R.:  Bestände und Konzentration privater Vermögen in Österreich  , Serie ICAE Working Paper Series, Nr. 72, 2017
	 \item Beyer K., Grimm C., Kapeller J., Pühringer S.:  Der deutsche Sonderweg im Fokus: Eine vergleichende Analyse der paradig¬matischen Struktur und der politischen Orientierung der deutschen und US-amerikanischen Ökonomie  , Serie ICAE Working Paper Series, Nr. 71, 2017
	 \item Aistleitner M., Kapeller J., Steinerberger S.:  Citation Patterns in Economics and Beyond: Assessing the Peculiarities of Economics from Two Scientometric Perspectives  , 2017
\end{enumerate} 
 \subsection*{2016} 
 \begin{enumerate}[leftmargin=*, labelsep=0.5cm] 
	 \item Ötsch W.:  Imaginierte Grundlagen bei Adam Smith  , in Institut für Ökonomie, Cusanus Hochschule, Bernkastel-Kues, Serie Working Paper Serie ök, Nr. 19, 2016
	 \item Pühringer S., Stelzer-Orthofer C.:  Neoliberale Think Tanks als (neue) Akteure in österreichischen gesellschafts- und sozialpolitischen Diskursen  , Serie ICAE Working Paper Series, Nr. 44, 2016
	 \item Pühringer S., Griesser M.:  Has economics returned to being the “dismal science”? The changing role of economic thought in German labour market reforms  , Serie ICAE Working Paper Series, Vol. 49, 2016
	 \item Pühringer S.:  Think Tank Networks of German Neoliberalism  , Serie ICAE Working Paper Series, Nr. 53, 2016
	 \item Pühringer S.:  Still the queen of the social sciences?  , Serie ICAE Working Paper Series, Nr. 52, 2016
	 \item Kapeller J., Schütz B., Springholz F.:  Internationale Tendenzen und Potentiale der Vermögensbesteuerung  , 2016
	 \item Kapeller J.:  Internationaler Freihandel: Theoretische Ausgangspunkte und empirische Folgen  , 2016
	 \item Heimberger P., Kapeller J., Schütz B.:  What’s ‘structural’ about unemployment in Europe: On the Determinants of the European Commission’s NAIRU Estimates  , 2016
	 \item Heimberger P., Kapeller J.:  The performativity of potential output: pro-cyclicality and path dependency in coordinating European fiscal policies  , in Institute for New Economic Thinking, Serie Institute for New Economic Thinking  Working Paper Series, Nr. 50, 2016
	 \item Heimberger P., Kapeller J.:  The performativity of potential output  , Serie ICAE Working Paper Series, Nr. 50, 2016
	 \item Heimberger P., Kapeller J.:  A model-based measurement device in European fiscal policy-making: The ontology and epistemology of potential output  , Serie ICAE Working Paper Series, Nr. 55, 2016
	 \item Heimberger P.:  Wirtschaftliche Stagnation als "neue Normalsituation"?  , Serie ICAE Working Paper Series, Nr. 48, 2016
	 \item Heimberger P.:  Die aktuelle Krise im wirtschaftshistorischen Vergleich mit der Großen Depression der 1930er-Jahre  , Serie ICAE Working Paper Series, Nr. 42, 2016
	 \item Heimberger P.:  Did Fiscal Consolidation Cause the Double Dip Recession in the Euro Area?  , Serie wiiw Working Papers, Nr. 130, 2016
	 \item Grimm C.:  Wirtschaftspolitische Positionen österreichischer Parteien im historischen Verlauf  , Serie ICAE Working Paper Series, Nr. 51, 2016
	 \item Grimm C.:  Postdemokratie, Machtverhältnisse und Ökonomik  , Serie ICAE Working Paper Series, Nr. 54, 2016
	 \item Dobusch L., Wandl S., Kapeller J.:  Monocular Accounting and its Discontents: The case of the stability and growth pact  , Serie ICAE Working Paper Series, Nr. 43, 2016
	 \item Beyer K., Bräutigam L.:  Das europäische Schattenbankensystem Typologisierung und die Bewertung regulatorischer Initiativen auf europäischer Ebene  , 2016
	 \item Aistleitner M., Kapeller J., Steinerberger S.:  The Power of Scientometrics and the Development of Economics  , Serie ICAE Working Paper Series, Nr. 46, 2016
\end{enumerate} 
 \subsection*{2015} 
 \begin{enumerate}[leftmargin=*, labelsep=0.5cm] 
	 \item Ötsch W., Pühringer S.:  Marktradikalismus als Politische Ökonomie. Wirtschaftswissenschaften und ihre Netzwerke in Deutschland ab 1945  , Serie ICAE Working Paper Series, Nr. 38, 2015
	 \item Ötsch W.:  Ökonomie und Moral  , Serie ICAE Working Paper Series, Nr. 39, 2015
	 \item Pühringer S.:  Wie wirken ÖkonomInnen und Ökonomik auf Politik und Gesellschaft? Darstellung des gesellschaftlichen und politischen Einflusspotenzials von ÖkonomInnen anhand eines „Performativen Fußabdrucks“  , Serie ICAE Working Paper Series, Nr. 35, 2015
	 \item Pühringer S.:  Markets as “ultimate judges” of economic policies - Angela Merkel´s discourse profile during the economic crisis and the European crisis policies.  , Serie ICAE Working Paper Series, Nr. 31, 2015
	 \item Kapeller J., Schütz B.:  Verteilungstendenzen im Kapitalismus: Nationale und Globale Perspektiven  , 2015
	 \item Kapeller J., Gräbner-Radkowitsch C.:  The Micro‐Macro Link in Heterodox Economics  , 2015
	 \item Heimberger P.:  Did Fiscal Consolidation Cause the Double-Dip Recession in the Euro Area?  , Serie ICAE Working Paper Series, Nr. 41, 2015
	 \item Beyer K., Bräutigam L.:  Shadow Banking and the Offshore Nexus - Some Considerations on the Systemic Linkages of Two Important Economic Phenomena  , Serie ICAE Working Paper Series, Nr. 40, 2015
	 \item Beyer K.:  Emanzipation bei Marx und seine Kritik an Proudhon  , Serie ICAE Working Paper Series, Nr. 34, 2015
	 \item Aistleitner M., Fölker M., Kapeller J., Mohr F., Pühringer S.:  Verteilung und Gerechtigkeit: Philosophische Perspektiven  , Serie ICAE Working Paper Series, Nr. 32, 2015
	 \item Aistleitner M., Fölker M., Kapeller J.:  Die Macht der Wissenschaftsstatistik und die Entwicklung der Ökonomie  , 2015
\end{enumerate} 
 \subsection*{2014} 
 \begin{enumerate}[leftmargin=*, labelsep=0.5cm] 
	 \item Ötsch W.:  Warum war Keynes so erfolgreich? Eine Darstellung anhand der Methode von Bruno Latour  , Serie ICAE Working Paper Series, Nr. 27, 2014
	 \item Ötsch W.:  Relationships are Constructed from Generalized Unconscious Social Images Kept in Steady Locations in Mental Space  , Serie ICAE Working Paper Series, Nr. 29, 2014
	 \item Pühringer S.:  Kontinuitäten neoliberaler Wirtschaftspolitik in der Krise  , Serie ICAE Working Paper Series, Nr. 30, 2014
	 \item Kapeller J., Schütz B., Tamesberger D.:  From Free to Civilized Markets: First Steps towards Eutopia  , Serie ICAE Working Paper Series, Nr. 28, 2014
	 \item Kapeller J.:  The return of the rentier. Review of: Piketty, Thomas (2014): Capital in the 21st century. Harvard University Press, 685 pages  , 2014
\end{enumerate} 
 \subsection*{2013} 
 \begin{enumerate}[leftmargin=*, labelsep=0.5cm] 
	 \item Ötsch W.:  Populismus und Demagogie. Mit Beispielen von Jörg Haider, Heinz–Christian Strache und Frank Stronach  , Serie ICAE Working Paper Series, Nr. 25, 2013
	 \item Thieme S., Hirte K.:  Mainstream, Orthodoxie und Heterodoxie – Zur Klassifizierung der Wirtschaftswissenschaften  , 2013
	 \item Pühringer S., Ötsch W.:  Das Team Stronach. Eine österreichische Tea Party?  , Serie ICAE Working Paper Series, Nr. 19, 2013
	 \item Pühringer S., Hirte K.:  ÖkonomInnen in der Finanzkrise - Netzwerkanalytische Sicht auf die deutschsprachigen ÖkonomInnen in der Finanzkrise  , Serie ICAE Working Paper Series, Nr. 18, Seite(n) 1-21, 2013
	 \item Pühringer S., Hirte K.:  The financial crisis as a tsunami. Discourse profiles of economists in the financial crisis.  , in ICAE Uni Linz, Serie ICAE Working Paper Series, Nr. 14, 2013
	 \item Pühringer S.:  The implementation of the European Fiscal Compact in Austria as a post-democratic phenomenon.  , Serie ICAE Working Paper Series, Nr. 15, 2013
	 \item Pühringer S.:  Aus den Vorhöfen der Macht in die Medien zur eigenen Partei. Formen der Einflussnahme von ÖkonomInnen auf Politik und Wirtschaft im Zuge der Finanz-­ und Wirtschaftskrisenpolitik.  , Serie ICAE Working Paper Series, Nr. 20, 2013
	 \item Plaimer W., Pühringer S.:  Der Fiskalpakt und seine Implementation in Österreich  , Serie ICAE Working Paper Series, Vol. 2, 2013
	 \item Nordmann J.:  Grenzen aktueller Krisendebatten Über Konstruktionen der öffentlichen Meinung und das Verhältnis von Sach‐ und Grundsatzdiskussionen in (neo)liberalen Demokratien  , 2013
	 \item Nordmann J.:  Die neoliberale Gesellschaft. Ein theoretischer Umriss.  , Serie ICAE Working Paper Series, Nr. 24, 2013
	 \item Nordmann J.:  Das neoliberale Selbst. Zur Genese und Kritik neuer Subjektkonstruktionen  , Serie ICAE Working Paper Series, Nr. 22, 2013
	 \item Hirte K.:  Zauberwort „Entkopplung“ – Ein Rückblick auf den Diskurs zur Agrarreform 2003 sowie die dortigen Diskrepanzen zwischen Versprechen und Wirkprinzipien der so genannten „Entkopplung“.  , Serie ICAE Working Paper Series, Nr. 12, 2013
	 \item Hirte K.:  Mainstream, Orthodoxie und Heterodoxie - Zur Klassifizierung der Wirtschaftswissenschaften.  , Serie ICAE Working Paper Series, Nr. 16, 2013
	 \item Beyer K., Ötsch W.:  Die Finanzkrise 2007-2009 als Krise von Schattenbanken. Eine einführende institutionelle Analyse.  , Serie ICAE Working Paper Series, Nr. 17, 2013
	 \item Beyer K., Bräutigam L.:  CDOs – A Critical Phenomenon of the Financial System in Crisis.  , Serie ICAE Working Paper Series, Nr. 21, 2013
\end{enumerate} 
 \subsection*{2012} 
 \begin{enumerate}[leftmargin=*, labelsep=0.5cm] 
	 \item Ötsch W.:  The Deep Meaning of “Market”. A Key to Understand the Neoliberal-­‐Market-­‐ Radical Society.  , Serie ICAE Working Paper Series, Nr. 11, 2012
	 \item Ötsch W.:  Marktradikalität. Der Diskurs von "dem Markt"  , Serie ICAE Working Paper Series, Nr. 7, 2012
	 \item Ötsch W.:  Die Macht der Ratingagenturen  , Serie ICAE Working Paper Series, Nr. 8, 2012
	 \item Pühringer S., Hirte K.:  Erdbeben, Fieber und zarte Pflänzchen. Chronologischer Verlauf des Finanzkrisen‐Diskurses deutschsprachiger Ökonomen  , in ICAE, JKU, Serie ICAE Working Paper Series, Nr. 9, Linz, 2012
	 \item Kapeller J., Schütz B.:  Conspicuous consumption, inequality and debt: The nature of consumption-driven profit-led regimes  , 2012
	 \item Kapeller J., Pühringer S.:  Democracy in liberalism and neoliberalism. The case of Popper and Hayek.  , Serie ICAE Working Paper Series, Nr. 10, 2012
\end{enumerate} 
 \subsection*{2011} 
 \begin{enumerate}[leftmargin=*, labelsep=0.5cm] 
	 \item Ötsch W.:  Politische Ökonomie und Gesellschaft. Eine theoriegeschichtliche Skizze  , Serie ICAE Working Paper Series, Nr. 5, 2011
	 \item Pühringer S.:  Gleichheit versus Vielfalt. Ein konstruierter Widerspruch?  , Serie ICAE Working Paper Series, Nr. 3, 2011
	 \item Pühringer M., Pühringer S.:  Solidarität im Kapitalismus. Zur Unmöglichkeit einer Forderung  , Serie ICAE Working Paper Series, Nr. 6, 2011
	 \item Plaimer W.:  Demokratiedefizit im Gesetzgebungsprozess budgetrelevanter Gesetze auf Bundesebene in Österreich  , Serie ICAE Working Paper Series, Nr. 4, 2011
\end{enumerate} 
 \subsection*{2010} 
 \begin{enumerate}[leftmargin=*, labelsep=0.5cm] 
	 \item Ötsch W., Hirte K., Nordmann J.:  Die Evolution des ökonomischen Wissens und des Wissens über den Kapitalismus. Performativity als Analyseinstrument: das Beispiel der Fabian Society, der Mont Pèlerin Society und der Chicagoer Schule  , in ICAE, JKU, Serie ICAE Working Paper Series, Nr. 1, Linz, 2010
	 \item Ötsch W.:  Die Finanz-­ und Wirtschaftskrise seit 2007: Ein Überblick  , Serie ICAE Working Paper Series, Nr. 2, 2010
\end{enumerate}