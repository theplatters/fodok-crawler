\begin{itemize}
\end{itemize} 
 \subsection{2023} 
 \begin{itemize} 
	 \item Organizers and promotors of academic competition? The role of (academic) social networks and platforms in the competitization of science: Pühringer S., Wolfmayer G. , Serie ICAE Working Paper Series, Nr. 152, Seite(n) 1-21, 2023 - Working Paper
	 \item Competitive Performativity of Academic Social Networks. The Subjectification of Competition on Researchgate, Twitter and Google Scholar.: Pühringer S., Wolfmayer G. , Serie SPACE Working Paper Series, Nr. 19, Seite(n) 1-25, 2023 - Working Paper
	 \item Wie viel Wettbewerb wollen wir (uns leisten)? Zur Verwettbewerblichung der Universitäten in Österreich und darüber hinaus: Pühringer S. , Serie ICAE Working Paper Series, Vol. 149, Seite(n) 1-19, 2023 - Working Paper
	 \item The charm of emission trading: Ideas of German public economists on economic policy in times of crises: Porak L., Reinke R. , Serie ICAE Working Paper Series, Nr. 145, 2023 - Working Paper
	 \item Dilemmata marktliberaler Globalisierung – Globale Freiheit durch globalen Wettbewerb?: Kapeller J., Hubmann G. , Serie ICAE Working Paper Series, Nr. 151, Seite(n) 1-18, 2023 - Working Paper
	 \item Gendered Competitive Practices in Economics. A Multi-Layer Model of Women’s Underrepresentation: Hager T., Pühringer S. , Serie ICAE Working Paper Series, Vol. 148, Seite(n) 1-22, 2023 - Working Paper
	 \item Endogenous Heterogeneous Gender Norms and the Distribution of Paid and Unpaid Work in an Intra-Household Bargaining Model: Hager T., Mellacher P., Rath M. , Serie ICAE Working Paper Series, Nr. 147, Seite(n) 1-32, 2023 - Working Paper
	 \item The three faces of competitization: From marketization to a multiplicity of competition: Altreiter C., Gräbner-Radkowitsch C., Pühringer S., Rogojanu A., Wolfmayr G. , Serie SPACE Working Paper Series, Nr. 18, 2023 - Working Paper
\end{itemize} 
 \subsection{2022} 
 \begin{itemize} 
	 \item Investment booms, diverging competitiveness and wage growth within a monetary union: An AB-SFC model: Schütz B. , Serie ICAE Working Paper Series, Nr. 138, 2022 - Working Paper
	 \item Polanyi and Schumpeter: Transitional Processes via Societal Spheres: Hager T., Heck I., Rath J. , Serie SPACE Working Paper Series, Nr. 16, Seite(n) 1-25, 2022 - Working Paper
	 \item Degrowth and the global South: the twin problem of global dependencies: Gräbner-Radkowitsch C., Strunk B. , Serie ICAE Working Paper Series, Nr. 142, 2022 - Working Paper
	 \item Degrowth and the Global South? How institutionalism can complement a timely discourse on ecologically sustainable development in an unequal world: Gräbner-Radkowitsch C., Strunk B. , Serie ICAE Working Paper Series, Nr. 144, 2022 - Working Paper
	 \item Competing for sustainability? An institutionalist analysis of the new development model of the European Union: Gräbner-Radkowitsch C., Hager T., Hornykewycz A. , Serie SPACE Working Paper Series, Nr. 17, 2022 - Working Paper
	 \item Winning urban competition with a social agenda. The competition imaginary in Viennese urban development: Altreiter C., Azevedo S., Porak L., Pühringer S., Wolfmayr G. , Serie SPACE Working Paper Series, Nr. 13, 2022 - Working Paper
	 \item Development and Interdisciplinarity: re-examining the “economic silo”: Aistleitner M. , Serie SPACE Working Paper Series, Nr. 14, 2022 - Working Paper
\end{itemize} 
 \subsection{2021} 
 \begin{itemize} 
	 \item Governing the Ungovernable. Recontextualizations of Competition in European Policy Discourse: Porak L. , in ICAE Linz, Serie SPACE Working Paper Series, Nr. 8, 2021 - Working Paper
	 \item Do Corporate Tax Cuts Boost Economic Growth?: Heimberger P., Gechert S. , Serie wiiw Working Papers, Nr. 201, forthcoming, doi/full/10.1080/09692290.2021.1904269, 2021 - Working Paper
	 \item Do Higher Public Debt Levels Reduce Economic Growth?: Heimberger P. , Serie wiiw Working Papers, Nr. 211, 2021 - Working Paper
	 \item Competition in Transitional Processes: Polanyi & Schumpeter: Hager T., Heck I., Rath J. , Serie ICAE Working Paper Series, Nr. 128, 2021 - Working Paper
	 \item (Mis)Measuring Competitiveness: The Quantification of a Malleable Concept in the European Semester: Gräbner C., Hager T. , Serie ICAE Working Paper Series, Nr. 130, 2021 - Working Paper
	 \item (Mis)Measuring Competitiveness: The Quantification of a Malleable Concept in the European Semester: Gräbner C., Hager T. , Serie SPACE Working Paper Series, Nr. 10, 2021 - Working Paper
	 \item Theorizing Competition. An Interdisciplinary Framework: Altreiter C., Gräbner C., Pühringer S., Rogojanu A., Wolfmayr G. , Serie SPACE Working Paper Series, Nr. 4, 2021 - Working Paper
	 \item Theorizing Competition - An interdisciplinary approach to the genesis of a contested concept: Altreiter C., Gräbner C., Pühringer S., Rogojanu A., Wolfmayr G. , Serie SPACE Working Paper Series, Nr. 3, 2021 - Working Paper
	 \item Exploring the trade (policy) narratives in economic elite discourse: Aistleitner M., Pühringer S. , Serie SPACE Working Paper Series, Nr. 1, Seite(n) 1-16, 2021 - Working Paper
\end{itemize} 
 \subsection{2020} 
 \begin{itemize} 
	 \item Substituting Trust by Technology: A Comparative Study: Rath J. , Serie ICAE Working Paper Series, Nr. 107, 2020 - Working Paper
	 \item Talking about Competition? Discursive Shifts in the Economic Imaginary of Competition in Public Debates: Pühringer S., Porak L., Rath J. , Serie SPACE Working Paper Series, Nr. 6, 2020 - Working Paper
	 \item Capability Accumulation and Product Innovation: Agent-Based Perspective: Hornykewycz A., Gräbner-Radkowitsch C. , Serie Rebuilding Macroeconomics Working Paper Series, Nr. 9, Seite(n) 1-22, 2020 - Working Paper
	 \item Friedman’s Instrumentalismus und das Problem von Kopernikus: Hirte K. , Serie ICAE Working Paper Series, Nr. 114, Seite(n) 1-25, 2020 - Working Paper
	 \item Das doppelte Relexionsproblem in der Ökonomik: Hirte K. , Serie ICAE Working Paper Series, Nr. 115, Seite(n) 1-19, 2020 - Working Paper
	 \item Special Interest Groups & Growth: Hager T. , Serie ICAE Working Paper Series, Nr. 116, 2020 - Working Paper
	 \item Emergence of Core-Periphery Structures in the European Union: A Complexity Perspective: Gräbner-Radkowitsch C., Hafele J. , Serie Rebuilding Macroeconomics Working Paper Series, Nr. 17, Seite(n) 1-21, 2020 - Working Paper
	 \item Zwischen Meritokratie und Wohlfahrtschauvinismus: Beyer K., Griesser M., Pühringer S. , Serie ICAE Working Paper Series, Nr. 109, 2020 - Working Paper
	 \item Theory and empirics of capability accumulation: implications for macroeconomics modelling: Aistleitner M., Gräbner-Radkowitsch C., Hornykewycz A. , Serie Rebuilding Macroeconomics Working Paper Series, Nr. 6, Seite(n) 1-29, 2020 - Working Paper
\end{itemize} 
 \subsection{2019} 
 \begin{itemize} 
	 \item The anti-democratic logic of right-wing Populism and neoliberal market-fundamentalism: Ötsch W., Pühringer S. , in Institut für Ökonomie, Cusanus Hochschule, Serie Working Paper Serie ök, Nr. 48, 2019 - Working Paper
	 \item A Comment on Fitting Pareto Tails to Complex Survey Data: Wildauer R., Kapeller J. , Serie ICAE Working Paper Series, Nr. 102, 2019 - Working Paper
	 \item Vorschlag für eine Jobgarantie für Langzeitarbeitslose in Österreich: Tamesberger D., Theurl S. , Serie ICAE Working Paper Series, Nr. 100, 2019 - Working Paper
	 \item Confict as a closure: A Kaleckian model of growth and distribution under fnancialization: Raghavendra S., Piiroinen P. , Serie ICAE Working Paper Series, Nr. 96, 2019 - Working Paper
	 \item Die Wirkmacht der "Liebe zum Markt": Zum anhaltenden Einfluss ordoliberaler ÖkonomInnenNetzwerke in Politik und Gesellschaft.: Pühringer S., Ötsch W. , in Institut für Ökonomie, Cusanus Hochschule, Serie Working Paper Serie ök, Nr. 51, 2019 - Working Paper
	 \item Wirtschaftliche Polarisierung in Europa: Ursachen und Handlungsoptionen: Kapeller J., Gräbner C., Heimberger P. , Serie ICAE Working Paper Series, Nr. 98, 2019 - Working Paper
	 \item Das dritte Gossensche Gesetz. Zur Überlieferungspraxis in der ökonomischen Dogmengeschichte: Hirte K. , Serie ICAE Working Paper Series, Nr. 93, Seite(n) 1-63, 2019 - Working Paper
	 \item The Impact of Labour Market Institutions and Capital Accumulation on Unemployment: Evidence for the OECD, 1985-2013: Heimberger P. , Serie wiiw Working Papers, Nr. 164, 2019 - Working Paper
	 \item Does economic globalisation affect income inequality? A meta-analysis: Heimberger P. , Serie wiiw Working Papers, Nr. 165, 2019 - Working Paper
	 \item Export performance, price comeptitiveness and technology: Revisiting the Kaldor paradox: Gräbner C., Heimberger P., Kapeller J. , Serie ICAE Working Paper Series, Nr. 88, 2019 - Working Paper
	 \item Defining institutions - A review and a synthesis: Gräbner C., Ghorbani A. , Serie ICAE Working Paper Series, Nr. 89, 2019 - Working Paper
	 \item Unrealistic models and how to identify them: on accounts of model realisticness: Gräbner C. , Serie ICAE Working Paper Series, Nr. 90, 2019 - Working Paper
	 \item Divided we stand? Professional consensus and political conflict in academic economics: Beyer K., Pühringer S. , in Institut für Ökonomie, Cusanus Hochschule, Serie Working Paper Serie ök, Nr. 51, 2019 - Working Paper
	 \item Exploring the trade narrative in top economics journals: Aistleitner M., Pühringer S. , Serie ICAE Working Paper Series, Nr. 97, 2019 - Working Paper
\end{itemize} 
 \subsection{2018} 
 \begin{itemize} 
	 \item Marktfundamentalismus als Kollektivgedanke. Mises und die Ordoliberalen: Ötsch W., Pühringer S. , in Institut für Ökonomie, Cusanus Hochschule, Bernkastel-Kues, Serie Working Paper Serie ök, Nr. 41, 2018 - Working Paper
	 \item Der vergessene Lippmann: Politik, Propaganda und Markt: Ötsch W., Graupe S. , in Institut für Ökonomie, Cusanus Hochschule, Bernkastel-Kues, Serie Working Paper Serie ök, Nr. 39, 2018 - Working Paper
	 \item Wissen und Nicht-Wissen angesichts "des Marktes": Das Konzept von Hayek: Ötsch W. , in Institut für Ökonomie, Cusanus Hochschule, Bernkastel-Kues, Serie Working Paper Serie ök, Vol. 43, 2018 - Working Paper
	 \item Bilder in der Geschichte der Ökonomie: Das Beispiel der Metapher von der Wirtschaft als Maschine: Ötsch W. , in Institut für Ökonomie, Cusanus Hochschule, Bernkastel-Kues., Serie Working Paper Serie ök, Vol. 42, 2018 - Working Paper
	 \item Employment and the minimum wage: A pluralist approach: Schütz B. , Serie ICAE Working Paper Series, Nr. 81, 2018 - Working Paper
	 \item What economics education is missing: The real world: Pühringer S., Bäuerle L. , in Institut für Ökonomie, Cusanus Hochschule, Serie Working Paper Serie ök, Nr. 37, 2018 - Working Paper
	 \item Arbeitsmarktpolitik als Gesellschaftspolitik: Griesser M. , Serie ICAE Working Paper Series, Nr. 78, 2018 - Working Paper
	 \item Folgen einer möglichen Abschaffung der Notstandshilfe in Oberösterreich: Foissner F. , Serie ICAE Working Paper Series, Nr. 87, 2018 - Working Paper
	 \item Factors Affecting Acces to Formal Credit by Micro and Small Enterprises in Ugande: Buyinza F., Tibaingana A., Mutenyo J. , Serie ICAE Working Paper Series, Nr. 83, 2018 - Working Paper
	 \item Household Electrification and Education Outcomes: Panel Evidence from Uganda: Buyinza F., Kapeller J. , Serie ICAE Working Paper Series, Nr. 85, 2018 - Working Paper
	 \item Auftragsvergabe, Leistungsqualität und Kostenintensität im Schienenpersonenverkehr: Aistleitner M., Grimm C., Kapeller J. , Serie ICAE Working Paper Series, Nr. 86, 2018 - Working Paper
	 \item The focus of academic economics: before and after the crisis: Aigner E., Aistleitner M., Glötzl F., Kapeller J. , Serie ICAE Working Paper Series, Nr. 75, 2018 - Working Paper
	 \item The Focus of Academic Economics: Before and After the Crisis: Aigner E., Aistleitner M., Glötzl F., Kapeller J. , in Institute for New Economic Thinking (INET), Serie Institute for New Economic Thinking  Working Paper Series, 2018 - Working Paper
\end{itemize} 
 \subsection{2017} 
 \begin{itemize} 
	 \item Right-wing populism and market-fundamentalism: Ötsch W., Pühringer S. , Serie ICAE Working Paper Series, Vol. 59, 2017 - Working Paper
	 \item Neoliberalism and Right-wing Populism: conceptual analogies: Pühringer S., Ötsch W. , in Institut für Ökonomie, Cusanus Hochschule, Bernkastel-Kues, Serie Working Paper Serie ök, Vol. 36, 2017 - Working Paper
	 \item Argumentationsstrategien einer neoliberalen Reformagenda: Zum Diskursprofil der Agenda Austria in medialen Debatten: Pühringer S., Liedl B. , in Institut für Ökonomie, Cusanus Hochschule, Bernkastel-Kues, Serie Working Paper Serie ök, Vol. 27, 2017 - Working Paper
	 \item From the "planning euphoria" to the "bitter economic truth": The transmission of economic ideas into German labour market policies in the 1960s and 2000s: Pühringer S., Griesser M. , in Institut für Ökonomie, Cusanus Hochschule, Bernkastel-Kues, Serie Working Paper Serie ök, Nr. 30, 2017 - Working Paper
	 \item Think tank networks of German neoliberalism power structures in economics and economic policies in post-war Germany: Pühringer S. , in Institut für Ökonomie, Cusanus Hochschule, Bernkastel-Kues, Serie Working Paper Serie ök, Nr. 24, 2017 - Working Paper
	 \item The “eternal character” of austerity measures in European crisis policies. Evidences from the Fiscal Compact discourse in Austria.: Pühringer S. , in Institut für Ökonomie, Cusanus Hochschule, Bernkastel-Kues, Serie Working Paper Serie ök, Nr. 32, 2017 - Working Paper
	 \item Emergent Phenomena in Scientific Publishing: A Simulation Exercise: Kapeller J., Steinerberger S. , Serie ICAE Working Paper Series, Nr. 66, 2017 - Working Paper
	 \item Pluralism in Economics: Epistemological Rationales and Pedagogical Implementation: Kapeller J. , Serie ICAE Working Paper Series, Nr. 68, 2017 - Working Paper
	 \item Inheritances and the Accumulation of Wealth in the Eurozone: Humer S., Moser M., Schnetzer M. , Serie ICAE Working Paper Series, Nr. 73, 2017 - Working Paper
	 \item From paradigms to policies: Economic models in the EU’s fiscal regulation framework: Huber J., Heimberger P., Kapeller J. , Serie ICAE Working Paper Series, Nr. 61, 2017 - Working Paper
	 \item Zum Profil der deutschsprachigen Volkswirtschaftslehre: Grimm C., Kapeller J., Pühringer S. , Serie ICAE Working Paper Series, Nr. 70, 2017 - Working Paper
	 \item Bestände und Konzentration privater Vermögen in Österreich: Ferschli B., Kapeller J., Schütz B., Wildauer R. , Serie ICAE Working Paper Series, Nr. 72, 2017 - Working Paper
	 \item Der deutsche Sonderweg im Fokus: Eine vergleichende Analyse der paradig¬matischen Struktur und der politischen Orientierung der deutschen und US-amerikanischen Ökonomie: Beyer K., Grimm C., Kapeller J., Pühringer S. , Serie ICAE Working Paper Series, Nr. 71, 2017 - Working Paper
\end{itemize} 
 \subsection{2016} 
 \begin{itemize} 
	 \item Imaginierte Grundlagen bei Adam Smith: Ötsch W. , in Institut für Ökonomie, Cusanus Hochschule, Bernkastel-Kues, Serie Working Paper Serie ök, Nr. 19, 2016 - Working Paper
	 \item Neoliberale Think Tanks als (neue) Akteure in österreichischen gesellschafts- und sozialpolitischen Diskursen: Pühringer S., Stelzer-Orthofer C. , Serie ICAE Working Paper Series, Nr. 44, 2016 - Working Paper
	 \item Has economics returned to being the “dismal science”? The changing role of economic thought in German labour market reforms: Pühringer S., Griesser M. , Serie ICAE Working Paper Series, Vol. 49, 2016 - Working Paper
	 \item Think Tank Networks of German Neoliberalism: Pühringer S. , Serie ICAE Working Paper Series, Nr. 53, 2016 - Working Paper
	 \item Still the queen of the social sciences?: Pühringer S. , Serie ICAE Working Paper Series, Nr. 52, 2016 - Working Paper
	 \item The performativity of potential output: pro-cyclicality and path dependency in coordinating European fiscal policies: Heimberger P., Kapeller J. , in Institute for New Economic Thinking, Serie Institute for New Economic Thinking  Working Paper Series, Nr. 50, 2016 - Working Paper
	 \item The performativity of potential output: Heimberger P., Kapeller J. , Serie ICAE Working Paper Series, Nr. 50, 2016 - Working Paper
	 \item A model-based measurement device in European fiscal policy-making: The ontology and epistemology of potential output: Heimberger P., Kapeller J. , Serie ICAE Working Paper Series, Nr. 55, 2016 - Working Paper
	 \item Wirtschaftliche Stagnation als "neue Normalsituation"?: Heimberger P. , Serie ICAE Working Paper Series, Nr. 48, 2016 - Working Paper
	 \item Die aktuelle Krise im wirtschaftshistorischen Vergleich mit der Großen Depression der 1930er-Jahre: Heimberger P. , Serie ICAE Working Paper Series, Nr. 42, 2016 - Working Paper
	 \item Did Fiscal Consolidation Cause the Double Dip Recession in the Euro Area?: Heimberger P. , Serie wiiw Working Papers, Nr. 130, 2016 - Working Paper
	 \item Wirtschaftspolitische Positionen österreichischer Parteien im historischen Verlauf: Grimm C. , Serie ICAE Working Paper Series, Nr. 51, 2016 - Working Paper
	 \item Postdemokratie, Machtverhältnisse und Ökonomik: Grimm C. , Serie ICAE Working Paper Series, Nr. 54, 2016 - Working Paper
	 \item Monocular Accounting and its Discontents: The case of the stability and growth pact: Dobusch L., Wandl S., Kapeller J. , Serie ICAE Working Paper Series, Nr. 43, 2016 - Working Paper
	 \item The Power of Scientometrics and the Development of Economics: Aistleitner M., Kapeller J., Steinerberger S. , Serie ICAE Working Paper Series, Nr. 46, 2016 - Working Paper
\end{itemize} 
 \subsection{2015} 
 \begin{itemize} 
	 \item Marktradikalismus als Politische Ökonomie. Wirtschaftswissenschaften und ihre Netzwerke in Deutschland ab 1945: Ötsch W., Pühringer S. , Serie ICAE Working Paper Series, Nr. 38, 2015 - Working Paper
	 \item Ökonomie und Moral: Ötsch W. , Serie ICAE Working Paper Series, Nr. 39, 2015 - Working Paper
	 \item Wie wirken ÖkonomInnen und Ökonomik auf Politik und Gesellschaft? Darstellung des gesellschaftlichen und politischen Einflusspotenzials von ÖkonomInnen anhand eines „Performativen Fußabdrucks“: Pühringer S. , Serie ICAE Working Paper Series, Nr. 35, 2015 - Working Paper
	 \item Markets as “ultimate judges” of economic policies - Angela Merkel´s discourse profile during the economic crisis and the European crisis policies.: Pühringer S. , Serie ICAE Working Paper Series, Nr. 31, 2015 - Working Paper
	 \item Did Fiscal Consolidation Cause the Double-Dip Recession in the Euro Area?: Heimberger P. , Serie ICAE Working Paper Series, Nr. 41, 2015 - Working Paper
	 \item Shadow Banking and the Offshore Nexus - Some Considerations on the Systemic Linkages of Two Important Economic Phenomena: Beyer K., Bräutigam L. , Serie ICAE Working Paper Series, Nr. 40, 2015 - Working Paper
	 \item Emanzipation bei Marx und seine Kritik an Proudhon: Beyer K. , Serie ICAE Working Paper Series, Nr. 34, 2015 - Working Paper
	 \item Verteilung und Gerechtigkeit: Philosophische Perspektiven: Aistleitner M., Fölker M., Kapeller J., Mohr F., Pühringer S. , Serie ICAE Working Paper Series, Nr. 32, 2015 - Working Paper
\end{itemize} 
 \subsection{2014} 
 \begin{itemize} 
	 \item Warum war Keynes so erfolgreich? Eine Darstellung anhand der Methode von Bruno Latour: Ötsch W. , Serie ICAE Working Paper Series, Nr. 27, 2014 - Working Paper
	 \item Relationships are Constructed from Generalized Unconscious Social Images Kept in Steady Locations in Mental Space: Ötsch W. , Serie ICAE Working Paper Series, Nr. 29, 2014 - Working Paper
	 \item Kontinuitäten neoliberaler Wirtschaftspolitik in der Krise: Pühringer S. , Serie ICAE Working Paper Series, Nr. 30, 2014 - Working Paper
	 \item From Free to Civilized Markets: First Steps towards Eutopia: Kapeller J., Schütz B., Tamesberger D. , Serie ICAE Working Paper Series, Nr. 28, 2014 - Working Paper
\end{itemize} 
 \subsection{2013} 
 \begin{itemize} 
	 \item Populismus und Demagogie. Mit Beispielen von Jörg Haider, Heinz–Christian Strache und Frank Stronach: Ötsch W. , Serie ICAE Working Paper Series, Nr. 25, 2013 - Working Paper
	 \item Das Team Stronach. Eine österreichische Tea Party?: Pühringer S., Ötsch W. , Serie ICAE Working Paper Series, Nr. 19, 2013 - Working Paper
	 \item ÖkonomInnen in der Finanzkrise - Netzwerkanalytische Sicht auf die deutschsprachigen ÖkonomInnen in der Finanzkrise: Pühringer S., Hirte K. , Serie ICAE Working Paper Series, Nr. 18, Seite(n) 1-21, 2013 - Working Paper
	 \item The financial crisis as a tsunami. Discourse profiles of economists in the financial crisis.: Pühringer S., Hirte K. , in ICAE Uni Linz, Serie ICAE Working Paper Series, Nr. 14, 2013 - Working Paper
	 \item The implementation of the European Fiscal Compact in Austria as a post-democratic phenomenon.: Pühringer S. , Serie ICAE Working Paper Series, Nr. 15, 2013 - Working Paper
	 \item Aus den Vorhöfen der Macht in die Medien zur eigenen Partei. Formen der Einflussnahme von ÖkonomInnen auf Politik und Wirtschaft im Zuge der Finanz-­ und Wirtschaftskrisenpolitik.: Pühringer S. , Serie ICAE Working Paper Series, Nr. 20, 2013 - Working Paper
	 \item Der Fiskalpakt und seine Implementation in Österreich: Plaimer W., Pühringer S. , Serie ICAE Working Paper Series, Vol. 2, 2013 - Working Paper
	 \item Die neoliberale Gesellschaft. Ein theoretischer Umriss.: Nordmann J. , Serie ICAE Working Paper Series, Nr. 24, 2013 - Working Paper
	 \item Das neoliberale Selbst. Zur Genese und Kritik neuer Subjektkonstruktionen: Nordmann J. , Serie ICAE Working Paper Series, Nr. 22, 2013 - Working Paper
	 \item Zauberwort „Entkopplung“ – Ein Rückblick auf den Diskurs zur Agrarreform 2003 sowie die dortigen Diskrepanzen zwischen Versprechen und Wirkprinzipien der so genannten „Entkopplung“.: Hirte K. , Serie ICAE Working Paper Series, Nr. 12, 2013 - Working Paper
	 \item Mainstream, Orthodoxie und Heterodoxie - Zur Klassifizierung der Wirtschaftswissenschaften.: Hirte K. , Serie ICAE Working Paper Series, Nr. 16, 2013 - Working Paper
	 \item Die Finanzkrise 2007-2009 als Krise von Schattenbanken. Eine einführende institutionelle Analyse.: Beyer K., Ötsch W. , Serie ICAE Working Paper Series, Nr. 17, 2013 - Working Paper
	 \item CDOs – A Critical Phenomenon of the Financial System in Crisis.: Beyer K., Bräutigam L. , Serie ICAE Working Paper Series, Nr. 21, 2013 - Working Paper
\end{itemize} 
 \subsection{2012} 
 \begin{itemize} 
	 \item The Deep Meaning of “Market”. A Key to Understand the Neoliberal-­‐Market-­‐ Radical Society.: Ötsch W. , Serie ICAE Working Paper Series, Nr. 11, 2012 - Working Paper
	 \item Marktradikalität. Der Diskurs von "dem Markt": Ötsch W. , Serie ICAE Working Paper Series, Nr. 7, 2012 - Working Paper
	 \item Die Macht der Ratingagenturen: Ötsch W. , Serie ICAE Working Paper Series, Nr. 8, 2012 - Working Paper
	 \item Erdbeben, Fieber und zarte Pflänzchen. Chronologischer Verlauf des Finanzkrisen‐Diskurses deutschsprachiger Ökonomen: Pühringer S., Hirte K. , in ICAE, JKU, Serie ICAE Working Paper Series, Nr. 9, Linz, 2012 - Working Paper
	 \item Democracy in liberalism and neoliberalism. The case of Popper and Hayek.: Kapeller J., Pühringer S. , Serie ICAE Working Paper Series, Nr. 10, 2012 - Working Paper
\end{itemize} 
 \subsection{2011} 
 \begin{itemize} 
	 \item Politische Ökonomie und Gesellschaft. Eine theoriegeschichtliche Skizze: Ötsch W. , Serie ICAE Working Paper Series, Nr. 5, 2011 - Working Paper
	 \item Gleichheit versus Vielfalt. Ein konstruierter Widerspruch?: Pühringer S. , Serie ICAE Working Paper Series, Nr. 3, 2011 - Working Paper
	 \item Solidarität im Kapitalismus. Zur Unmöglichkeit einer Forderung: Pühringer M., Pühringer S. , Serie ICAE Working Paper Series, Nr. 6, 2011 - Working Paper
	 \item Demokratiedefizit im Gesetzgebungsprozess budgetrelevanter Gesetze auf Bundesebene in Österreich: Plaimer W. , Serie ICAE Working Paper Series, Nr. 4, 2011 - Working Paper
\end{itemize} 
 \subsection{2010} 
 \begin{itemize} 
	 \item Die Evolution des ökonomischen Wissens und des Wissens über den Kapitalismus. Performativity als Analyseinstrument: das Beispiel der Fabian Society, der Mont Pèlerin Society und der Chicagoer Schule: Ötsch W., Hirte K., Nordmann J. , in ICAE, JKU, Serie ICAE Working Paper Series, Nr. 1, Linz, 2010 - Working Paper
	 \item Die Finanz-­ und Wirtschaftskrise seit 2007: Ein Überblick: Ötsch W. , Serie ICAE Working Paper Series, Nr. 2, 2010 - Working Paper
\end{itemize}